
\usepackage{cmap}						% поиск в PDF

\usepackage[T2A]{fontenc}				% русская кодировка
\usepackage[utf8]{inputenc}				% кодировка исходного текста
\usepackage[english,russian]{babel}		% локализация и переносы

\usepackage{indentfirst}				% красная строка для первого параграфа

\usepackage{amsmath, amsfonts, amssymb, amsthm, mathtools} 	% AMS
\usepackage{icomma} 										% "Умная" запятая

\usepackage{tabularx}

\usepackage{color, colortbl}
\definecolor{light-blue}{rgb}{0.8,0.85,1}
\definecolor{light-green}{rgb}{0.9,1,0.8}


%\usepackage{fancyhdr}		% колонтитулы


\usepackage[usenames,dvipsnames,svgnames,table,rgb]{xcolor}
\usepackage{hyperref}
\hypersetup{
	unicode = true, 	% русские буквы в разделе PDF
	colorlinks = true, 	% false - в рамке treur - без рамке
	linkcolor = red,	% внутри документа
	citecolor = green, 	% библиография
	urlcolor = cyan		% URL
}

%% Шрифты
%\usepackage{euscript}	% Шрифт Евклид
%\usepackage{mathrsfs} 	% Красивый матшрифт


\usepackage{tikz}



%%% Счетчики %%%
\newcounter{exercises}
\setcounter{exercises}{0}

\newcounter{informalpractices}
\setcounter{informalpractices}{0}

\newcounter{meditations}
\setcounter{meditations}{0}

\newcounter{reflection}
\setcounter{reflection}{0}



%%% Команды %%%
\newcommand{\Exercises}[1]{\refstepcounter{exercises}
	\vspace{4ex}~\\
	\noindent\colorbox{light-blue}{{\Large\textit{Упражнение~\arabic{exercises}}}}\nopagebreak\vspace{2ex}~\\
	{\Large\textbf{#1}}\addcontentsline{toc}{section}{\textit{Упр.\,\arabic{exercises}}\quad#1}
	\vspace{3ex}~\\
}

\newcommand{\InformalPractices}[1]{\refstepcounter{informalpractices}
	\vspace{4ex}~\\
	\noindent\colorbox{light-green}{{\Large\textit{Неформальная практика}}}\nopagebreak\vspace{2ex}~\\
	{\Large\textbf{#1}}\addcontentsline{toc}{section}{\textit{Неформальная практика}\quad#1}
	\vspace{3ex}~\\
}

