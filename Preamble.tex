
\usepackage{cmap}						% поиск в PDF

\usepackage[T2A]{fontenc}				% русская кодировка
\usepackage[utf8]{inputenc}				% кодировка исходного текста
\usepackage[english,russian]{babel}		% локализация и переносы

\usepackage{indentfirst}				% красная строка для первого параграфа

\usepackage{amsmath, amsfonts, amssymb, amsthm, mathtools} 	% AMS
\usepackage{icomma} 										% "Умная" запятая

\usepackage{tabularx}


%\usepackage{fancyhdr}		% колонтитулы


\usepackage[usenames,dvipsnames,svgnames,table,rgb]{xcolor}
\usepackage{hyperref}
\hypersetup{
	unicode = true, 	% русские буквы в разделе PDF
	colorlinks = true, 	% false - в рамке treur - без рамке
	linkcolor = red,	% внутри документа
	citecolor = green, 	% библиография
	urlcolor = cyan		% URL
}

%% Шрифты
%\usepackage{euscript}	% Шрифт Евклид
%\usepackage{mathrsfs} 	% Красивый матшрифт


\usepackage{tikz}
