
\usepackage{cmap}						% поиск в PDF

\usepackage[T2A]{fontenc}				% русская кодировка
\usepackage[utf8]{inputenc}				% кодировка исходного текста
\usepackage[english,russian]{babel}		% локализация и переносы

\usepackage{indentfirst}				% красная строка для первого параграфа


%%% Поля документа %%%
\usepackage{geometry}
\geometry{top=35mm}
\geometry{bottom=35mm}
\geometry{left=35mm}
\geometry{right=25mm}



%%% Математика и символы %%%
%\usepackage{amsmath, amsfonts, amssymb, amsthm, mathtools} % AMS
%\usepackage{icomma} 										% "Умная" запятая
\usepackage{fdsymbol}										% Сердечко
\usepackage{marvosym}										% Инь и Янь
%\usepackage{stix}											% Стрелочки, ромбик


%%% Таблицы %%%
\usepackage{array,tabularx,tabulary}		
\usepackage{makecell}		% Выравнивает ячейки по значениям l c r
\usepackage{multirow} 		% Слияние строк в таблице


%%% Работа с цветом %%%
\usepackage{color, colortbl}
%%% Мои цвета %%%
\definecolor{light-blue}{rgb}{0.8,0.85,1}
\definecolor{light-green}{RGB}{167,255,167}%{199,255,186}
\definecolor{light-yellow}{RGB}{254,214,107}%{255,255,62}%{246,255,170}
\definecolor{light-ping}{RGB}{252,161,164}%{235,199,223}

%%% Колонтитулы %%%
%\usepackage{fancyhdr}


%%% Гиперссылки %%%
\usepackage[usenames,dvipsnames,svgnames,table,rgb]{xcolor}
\usepackage{hyperref}
\hypersetup{
	unicode = true, 	% русские буквы в разделе PDF
	colorlinks = true, 	% false - в рамке treur - без рамке
	linkcolor = red,	% внутри документа
	citecolor = green, 	% библиография
	urlcolor = cyan		% URL
}


%%% Шрифты %%%
%\usepackage{euscript}	% Шрифт Евклид
%\usepackage{mathrsfs} 	% Красивый матшрифт


%%% Графика %%%
\usepackage{tikz}


%%% Счетчики %%%
\newcounter{exercises}
\setcounter{exercises}{0}

\newcounter{informalpractices}
\setcounter{informalpractices}{0}

\newcounter{meditations}
\setcounter{meditations}{0}

\newcounter{reflection}
\setcounter{reflection}{0}



%%% Команды %%%
\newcommand{\Exercises}[1]{\refstepcounter{exercises}
	\noindent\colorbox{light-blue}{{\Large\textit{Упражнение~\arabic{exercises}}}}\nopagebreak\vspace{2ex}~\\
	{\Large\textbf{#1}}\addcontentsline{toc}{section}{\textit{Упр.\,\arabic{exercises}}\quad#1}
	\vspace{1ex}~\\
}

\newcommand{\InformalPractices}[1]{\refstepcounter{informalpractices}
	\vspace{4ex}~\\
	\noindent\colorbox{light-green}{{\Large\textit{Неформальная практика}}}\nopagebreak\vspace{2ex}~\\
	{\Large\textbf{#1}}\addcontentsline{toc}{section}{\textit{Неформальная практика}\quad#1}
	\vspace{3ex}~\\
}

\newcommand{\Meditation}[1]{\refstepcounter{meditations}
	\vspace{4ex}~\\
	\noindent\colorbox{light-ping}{{\Large\textit{Медитация}}}\nopagebreak\vspace{2ex}~\\
	{\Large\textbf{#1}}\addcontentsline{toc}{section}{\textit{Медитация}\quad#1}
	\vspace{3ex}~\\
}

\newcommand{\Reflection}[1]{\refstepcounter{reflection}
	\vspace{4ex}~\\
	\noindent\colorbox{light-yellow}{{\Large\textit{Пища для размышлений}}}\nopagebreak\vspace{2ex}~\nopagebreak\\\nopagebreak
	\par\nopagebreak#1
	\vspace{3ex}~\\
}


% Сердечко в списке
\newcommand{\itemheart}{\item[\textcolor{red}{$\heartsuit$}]}
% Инь и Янь в списке
\newcommand{\itemyinyang}{\item[\Yinyang]}
% Ромбик
\newcommand{\itemdiamondsuit}{\item[$\diamondsuit$]}
% Звездочка
\newcommand{\itemast}{\item[$\ast$]}
% Рука с корондашом
\newcommand{\itemWritingHand}{\item[\WritingHand]}





%%% Переопределение команд %%%
\renewcommand{\thempfootnote}{\arabic{mpfootnote}}

