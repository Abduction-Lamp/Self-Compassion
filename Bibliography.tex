
\begin{thebibliography}{999}
	\addcontentsline{toc}{chapter}{\refname}
	
	%%% Itnoduction %%%
	%%%%%%%%%%%%%%%%%%%
	\bibitem{1} Quote retrieved from --- \href{www.bbc.co.uk/worldservice/learningenglish/movingwords/quotefeature/rumi.shtml}{go to}\\
	bbc.co.uk/worldservice/learningenglish/movingwords/quotefeature/rumi.shtml.
	
	\bibitem{2} Zessin, U., Dickhauser, O., \& Garbade, S. (2015). The relationship between self-compassion and well-being: A meta-analysis. \textit{Applied Psychology: Health and Well-Being}, 7(3), 340–364.
	\bibitem{3} Breines, J. G., \& Chen, S. (2012). Self-compassion increases self-improvement motivation. \textit{Personality and Social Psychology Bulletin}, 38(9), 1133–1143.
	\bibitem{4}	Neff, K. D., \& Beretvas, S. N. (2013). The role of self-compassion in romantic relationships. \textit{Self and Identity}, 12(1), 78–98.
	\bibitem{5} Dunne, S., Sheffield, D., \& Chilcot, J. (2016). Brief report: Self-compassion, physical health and the mediating role of health-promoting behaviours. \textit{Journal of Health Psychology}.
	\bibitem{6} MacBeth, A., \& Gumley, A. (2012). Exploring compassion: A meta-analysis of the association between self-compassion and psychopathology. \textit{Clinical Psychology Review}, 32, 545–552.
	
	\bibitem{7} Sbarra, D. A., Smith, H. L., \& Mehl, M. R. (2012). When leaving your ex, love yourself: Observational ratings of self-compassion predict the course of emotional recovery following marital separation. \textit{Psychological Science}, 23, 261–269.
	\bibitem{8} Brion, J. M., Leary, M. R., \& Drabkin, A. S. (2014). Self-compassion and reactions to serious illness: The case of HIV. \textit{Journal of Health Psychology}, 19(2), 218–229.
	\bibitem{9} Neff, K. D., Hseih, Y., \& Dejitthirat, K. (2005). Self-compassion, achievement goals, and coping with academic failure. \textit{Self and Identity}, 4, 263–287.
	\bibitem{10} Hiraoka, R., Meyer, E. C., Kimbrel, N. A., DeBeer, B. B., Gulliver, S. B., \& Morissette, S. B. (2015). Self-compassion as a prospective predictor of PTSD symptom severity among trauma-exposed U.S. Iraq and Afghanistan war veterans. \textit{Journal of Traumatic Stress}, 28, 1–7.
	
	\bibitem{11} Birnie, K., Speca, M., \& Carlson, L. E. (2010). Exploring self-compassion and empathy in the context of mindfulness-based stress reduction (MBSR). \textit{Stress and Health}, 26, 359–371.
	\bibitem{12} Kuyken, W., Watkins, E., Holden, E., White, K., Taylor, R. S., Byford, S., et al. (2010). How does mindfulnessbased cognitive therapy work? \textit{Behavior Research and Therapy}, 48, 1105–1112.
	\bibitem{13} Keng, S., Smoski, M. J., Robins, C. J., Ekblad, A. G., \& Brantley, J. G. (2012). Mechanisms of change in mindfulness-based stress reduction: Self-compassion and mindfulness as mediators of intervention outcomes. \textit{Journal of Cognitive Psychotherapy}, 26(3), 270–280.
	
	\bibitem{14} Neff, K. D., \& Germer, C. K. (2013). A pilot study and randomized controlled trial of the Mindful Self-Compassion program. \textit{Journal of Clinical Psychology}, 69(1), 28–44.
	\bibitem{15} Bluth, K., Gaylord, S. A., Campo, R. A., Mullarkey, M. C., \& Hobbs, L. (2016). Making friends with yourself: A mixed methods pilot study of a Mindful Self-Compassion program for adolescents. \textit{Mindfulness}, 7(2), 1–14.
	\bibitem{16} Friis, A. M., Johnson, M. H., Cutfield, R. G., \& Consedine, N. S. (2016). Kindness matters: A randomized controlled trial of a mindful self-compassion intervention improves depression, distress, and HbA1c among patients with diabetes. \textit{Diabetes Care}, 39(11), 1963–1971.
	
	\bibitem{17} Neff, K. D., \& Vonk, R. (2009). Self-compassion versus global self-esteem: Two different ways of relating to oneself. \textit{Journal of Personality}, 77, 23–50.
	
	\bibitem{18} Germer, C. K., Siegel, R., \& Fulton, P. (Eds.). (2013). \textit{Mindfulness and psychotherapy} (2nd ed.). New York: Guilford Press.
	
	\bibitem{19} Germer, C. K. (2009). \textit{The mindful path to self-compassion: Freeing yourself from destructive thoughts and emotions}. New York: Guilford Press.
	
	\bibitem{20} Neff, K. D. (2011). Self-compassion: \textit{The proven power of being kind to yourself}. New York: William Morrow.
	\bibitem{21} Germer, C. K., \& Neff, K. D. (in press). \textit{Teaching the Mindful Self-Compassion program: A guide for professionals}. New York: Guilford Press.
	
	
	%%% Chapter 1   %%%
	%%%%%%%%%%%%%%%%%%%
	\vspace{3ex}
	\textbf{Глава \ref{What_is_self-compassion}}
	
	\bibitem{22} Neff, K. D. (2003). Self- compassion: An alternative conceptualization of a healthy attitude toward one-self. \textit{Self and Identity}, 2, 85–102.
	
	\bibitem{23} Knox, M., Neff, K., \& Davidson, O. (2016, June). \textit{Comparing compassion for self and others: Impacts on personal and interper-sonal well-being.} Paper presented at the 14th annual meeting of the Association for Contextual Behavioral Science World Conference, Seattle, WA.
	
	
	%%% Chapter 2   %%%
	%%%%%%%%%%%%%%%%%%%	
	\vspace{3ex}
	\textbf{Глава \ref{What_Self-Compassion_Is_Not}}
	
	\bibitem{24} Neff, K. D., \& Pommier, E. (2013). The relationship between self-compassion and other-focused concern among college undergraduates, community adults, and practicing meditators. \textit{Self and Identit}y, 12(2), 160–176.
	\bibitem{25}Raes, F. (2010). Rumination and worry as mediators of the relationship between self-compassion and depression and anxiety. \textit{Personality and Individual Differences}, 48, 757–761.
		
	\bibitem{26} Sbarra, D. A., Smith, H. L., \& Mehl, M. R. (2012). When leaving your ex, love yourself: Observational ratings of self-compassion predict the course of emotional recovery following marital separation. \textit{Psychological Science}, 23, 261–269.
	\bibitem{27} Hiraoka, R., Meyer, E. C., Kimbrel, N. A., DeBeer, B. B., Gulliver, S. B., \& Morissette, S. B. (2015). Self-compassion as a prospective predictor of PTSD symptom severity among trauma-exposed U.S. Iraq and Afghanistan war veterans. \textit{Journal of Traumatic Stress}, 28, 1–7.
	\bibitem{28} Wren, A. A., Somers, T. J., Wright, M. A., Goetz, M. C., Leary, M. R., Fras, A. M., et al. (2012). Self-compassion in patients with persistent musculoskeletal pain: Relationship of self-compassion to adjustment to persistent pain. \textit{Journal of Pain and Symptom Management}, 43(4), 759–770. 184 
	
	\bibitem{29} Neff, K. D., \& Beretvas, S.N. (2013). The role of self-compassion in romantic relationships. \textit{Self and Identity}, 12(1), 78–98.
	\bibitem{30} Yarnell, L. M., \& Neff, K. D. (2013). Self-compassion, interpersonal conflict resolutions, and well-being. \textit{Self and Identity}, 2(2), 146–159.
	\bibitem{31} Neff, K. D., \& Pommier, E. (2013). The relationship between self-compassion and other-focused concern among college undergraduates, community adults, and practicing meditators. \textit{Self and Identity}, 12(2), 160–176.
	
	\bibitem{32} Magnus, C. M. R., Kowalski, K. C., \& McHugh, T. L. F. (2010). The role of self-compassion in women’s self-determined motives to exercise and exercise-related outcomes. \textit{Self and Identity}, 9, 363–382.
	\bibitem{33} Schoenefeld, S. J., \& Webb, J. B. (2013). Self-compassion and intuitive eating in college women: Examining the contributions of distress tolerance and body image acceptance and action. \textit{Eating Behaviors}, 14(4), 493–496.
	\bibitem{34} Brooks, M., Kay-Lambkin, F., Bowman, J., \& Childs, S. (2012). Self-compassion amongst clients with problematic alcohol use. \textit{Mindfulness}, 3(4), 308–317.
	\bibitem{35} Terry, M. L., Leary, M. R., Mehta, S., \& Henderson, K. (2013). Self-compassionate reactions to health threats. \textit{Personality and Social Psychology Bulletin}, 39(7), 911–926.
	
	\bibitem{36} Zhang, J. W., \& Chen, S. (2016). Self-compassion promotes personal improvement from regret experiences via acceptance. \textit{Personality and Social Psychology Bulletin}, 42(2), 244– 258.
	\bibitem{37} Howell, A. J., Dopko, R. L., Turowski, J. B., \& Buro, K. (2011). The disposition to apologize. \textit{Personality and Individual Differences}, 51(4), 509–514.
	
	\bibitem{38} Neff, K. D. (2003). Development and validation of a scale to measure self-compassion. \textit{Self and Identity}, 2, 223–250.
	\bibitem{39} Neff, K. D., Hseih, Y., \& Dejitthirat, K. (2005). Self-compassion, achievement goals, and coping with academic failure. \textit{Self and Identity}, 4, 263–287.
	\bibitem{40} Breines, J. G., \& Chen, S. (2012). Self-compassion increases self-improvement motivation. \textit{Personality and Social Psychology Bulletin}, 38(9), 1133–1143.
	
	\bibitem{41} Compared with self-esteem, self-compassion is less contingent: Neff, K. D., \& Vonk, R. (2009). Self-compassion versus global self-esteem: Two different ways of relating to oneself. \textit{Journal of Personality}, 77, 23–50.
	
	
	%%% Chapter 3   %%%
	%%%%%%%%%%%%%%%%%%%	
	\vspace{3ex}
	\textbf{Глава \ref{The_Benefits_of_Self-Compassion}}

	\bibitem{42} MacBeth, A., \& Gumley, A. (2012). Exploring compassion: A meta-analysis of the association between self-compassion and psychopathology. \textit{Clinical Psychology Review}, 32, 545–552.
	\bibitem{43} Zessin, U., Dickhauser, O., \& Garbade, S. (2015). The relationship between self-compassion and well-being: A meta-analysis. \textit{Applied Psychology: Health and Well-Being}, 7(3), 340–364.
	\bibitem{44} Neff, K. D., Long, P., Knox, M. C., Davidson, O., Kuchar, A., Costigan, A., et al. (in press). The forest and the trees: Examining the association of self-compassion and its positive and negative components with psychological functioning. \textit{Self and Identity}.
	\bibitem{45} Hall, C. W., Row, K. A., Wuensch, K. L., \& Godley, K. R. (2013). The role of self-compassion in physical and psychological well-being. \textit{Journal of Psychology}, 147(4), 311–323.
	
	\bibitem{46} Neff, K. D., \& Germer, C. K. (2013). A pilot study and randomized controlled trial of the Mindful Self-Compassion program. \textit{Journal of Clinical Psychology}, 69(1), 28–44.
	
	\bibitem{47} Neff, K. D. (2003). Development and validation of a scale to measure self-compassion. \textit{Self and Identity}, 2, 223–250.
	\bibitem{48} Raes, F., Pommier, E., Neff, K. D., \& Van Gucht, D. (2011). Construction and factorial validation of a short form of the Self-Compassion Scale. \textit{Clinical Psychology and Psychotherapy}, 18, 250–255.
	
	\bibitem{49} Ullrich, P. M., \& Lutgendorf, S. K. (2002). Journaling about stressful events: Effects of cognitive processing and emotional expression. \textit{Annals of Behavioral Medicine}, 24(3), 244–250.
	
	
	%%% Chapter 4   %%%
	%%%%%%%%%%%%%%%%%%%	
	\vspace{3ex}
	\textbf{Глава \ref{The_Physiology_of_Self-Criticism_and_Self-Compassion}}
	
	\bibitem{50} Gilbert, P. (2009). \textit{The compassionate mind}. London: Constable.
	
	\bibitem{51} LeDoux, J. E. (2003). \textit{Synaptic self: How our brains become who we are}. New York: Penguin.
		
	\bibitem{52} Solomon, J., \& George, C. (1996). Defining the caregiving system: Toward a theory of caregiving. \textit{Infant Mental Health Journa}l, 17(3), 183–197.
	
	\bibitem{53} Stellar, J. E., \& Keltner, D. (2014). Compassion. In M. Tugade, L. Shiota, \& L. Kirby (Eds.), \textit{Handbook of positive emotions} (pp. 329–341). New York: Guilford Press.
	
	\bibitem{54} Rockcliff, H., Gilbert, P., McEwan, K., Lightman, S., \& Glover, D. (2008). A pilot exploration of heart rate variability and salivary cortisol responses to compassion-focused imagery. \textit{Clinical Neuropsychiatry}, 5, 132–139.


 	%%% Chapter 5   %%%
 	%%%%%%%%%%%%%%%%%%%	
 	\vspace{3ex}
 	\textbf{Глава \ref{The_Yin_and_Yang_of_Self-Compassion}}

	\bibitem{55} Eagly, A. H. (1987). \textit{Sex differences in social behavior: A social-role interpretation}. Hillsdale, NJ: Erlbaum.
	
	
	\bibitem{56} 
	\bibitem{57} 
	\bibitem{58} 
	\bibitem{59}
	\bibitem{60}
\end{thebibliography}
