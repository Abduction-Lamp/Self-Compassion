% !TEX root = Self-Compassion.tex

\begin{thebibliography}{999}
	\addcontentsline{toc}{chapter}{\refname}
	
	%%% Itnoduction %%%
	%%%%%%%%%%%%%%%%%%%
	\bibitem{1} Quote retrieved from --- \url{www.bbc.co.uk/worldservice/learningenglish/movingwords/quotefeature/rumi.shtml}
	
	\bibitem{2} Zessin, U., Dickhauser, O., \& Garbade, S. (2015). The relationship between self-compassion and well-being: A meta-analysis. \textit{Applied Psychology: Health and Well-Being}, 7(3), 340–364.
	\bibitem{3} Breines, J. G., \& Chen, S. (2012). Self-compassion increases self-improvement motivation. \textit{Personality and Social Psychology Bulletin}, 38(9), 1133–1143.
	\bibitem{4}	Neff, K. D., \& Beretvas, S. N. (2013). The role of self-compassion in romantic relationships. \textit{Self and Identity}, 12(1), 78–98.
	\bibitem{5} Dunne, S., Sheffield, D., \& Chilcot, J. (2016). Brief report: Self-compassion, physical health and the mediating role of health-promoting behaviours. \textit{Journal of Health Psychology}.
	\bibitem{6} MacBeth, A., \& Gumley, A. (2012). Exploring compassion: A meta-analysis of the association between self-compassion and psychopathology. \textit{Clinical Psychology Review}, 32, 545–552.
	
	\bibitem{7} Sbarra, D. A., Smith, H. L., \& Mehl, M. R. (2012). When leaving your ex, love yourself: Observational ratings of self-compassion predict the course of emotional recovery following marital separation. \textit{Psychological Science}, 23, 261–269.
	\bibitem{8} Brion, J. M., Leary, M. R., \& Drabkin, A. S. (2014). Self-compassion and reactions to serious illness: The case of HIV. \textit{Journal of Health Psychology}, 19(2), 218–229.
	\bibitem{9} Neff, K. D., Hseih, Y., \& Dejitthirat, K. (2005). Self-compassion, achievement goals, and coping with academic failure. \textit{Self and Identity}, 4, 263–287.
	\bibitem{10} Hiraoka, R., Meyer, E. C., Kimbrel, N. A., DeBeer, B. B., Gulliver, S. B., \& Morissette, S. B. (2015). Self-compassion as a prospective predictor of PTSD symptom severity among trauma-exposed U.S. Iraq and Afghanistan war veterans. \textit{Journal of Traumatic Stress}, 28, 1–7.
	
	\bibitem{11} Birnie, K., Speca, M., \& Carlson, L. E. (2010). Exploring self-compassion and empathy in the context of mindfulness-based stress reduction (MBSR). \textit{Stress and Health}, 26, 359–371.
	\bibitem{12} Kuyken, W., Watkins, E., Holden, E., White, K., Taylor, R. S., Byford, S., et al. (2010). How does mindfulnessbased cognitive therapy work? \textit{Behavior Research and Therapy}, 48, 1105–1112.
	\bibitem{13} Keng, S., Smoski, M. J., Robins, C. J., Ekblad, A. G., \& Brantley, J. G. (2012). Mechanisms of change in mindfulness-based stress reduction: Self-compassion and mindfulness as mediators of intervention outcomes. \textit{Journal of Cognitive Psychotherapy}, 26(3), 270–280.
	
	\bibitem{14} Neff, K. D., \& Germer, C. K. (2013). A pilot study and randomized controlled trial of the Mindful Self-Compassion program. \textit{Journal of Clinical Psychology}, 69(1), 28–44.
	\bibitem{15} Bluth, K., Gaylord, S. A., Campo, R. A., Mullarkey, M. C., \& Hobbs, L. (2016). Making friends with yourself: A mixed methods pilot study of a Mindful Self-Compassion program for adolescents. \textit{Mindfulness}, 7(2), 1–14.
	\bibitem{16} Friis, A. M., Johnson, M. H., Cutfield, R. G., \& Consedine, N. S. (2016). Kindness matters: A randomized controlled trial of a mindful self-compassion intervention improves depression, distress, and HbA1c among patients with diabetes. \textit{Diabetes Care}, 39(11), 1963–1971.
	
	\bibitem{17} Neff, K. D., \& Vonk, R. (2009). Self-compassion versus global self-esteem: Two different ways of relating to oneself. \textit{Journal of Personality}, 77, 23–50.
	
	\bibitem{18} Germer, C. K., Siegel, R., \& Fulton, P. (Eds.). (2013). \textit{Mindfulness and psychotherapy} (2nd ed.). New York: Guilford Press.
	
	\bibitem{19} Germer, C. K. (2009). \textit{The mindful path to self-compassion: Freeing yourself from destructive thoughts and emotions}. New York: Guilford Press.
	
	\bibitem{20} Neff, K. D. (2011). Self-compassion: \textit{The proven power of being kind to yourself}. New York: William Morrow.
	\bibitem{21} Germer, C. K., \& Neff, K. D. (in press). \textit{Teaching the Mindful Self-Compassion program: A guide for professionals}. New York: Guilford Press.
	
	
	%%% Chapter 1   %%%
	%%%%%%%%%%%%%%%%%%%
	\vspace{3ex}
	\textbf{Глава \ref{What_is_self-compassion}}
	
	\bibitem{22} Neff, K. D. (2003). Self- compassion: An alternative conceptualization of a healthy attitude toward one-self. \textit{Self and Identity}, 2, 85–102.
	
	\bibitem{23} Knox, M., Neff, K., \& Davidson, O. (2016, June). \textit{Comparing compassion for self and others: Impacts on personal and interper-sonal well-being.} Paper presented at the 14th annual meeting of the Association for Contextual Behavioral Science World Conference, Seattle, WA.
	
	
	%%% Chapter 2   %%%
	%%%%%%%%%%%%%%%%%%%	
	\vspace{3ex}
	\textbf{Глава \ref{What_Self-Compassion_Is_Not}}
	
	\bibitem{24} Neff, K. D., \& Pommier, E. (2013). The relationship between self-compassion and other-focused concern among college undergraduates, community adults, and practicing meditators. \textit{Self and Identit}y, 12(2), 160–176.
	\bibitem{25}Raes, F. (2010). Rumination and worry as mediators of the relationship between self-compassion and depression and anxiety. \textit{Personality and Individual Differences}, 48, 757–761.
		
	\bibitem{26} Sbarra, D. A., Smith, H. L., \& Mehl, M. R. (2012). When leaving your ex, love yourself: Observational ratings of self-compassion predict the course of emotional recovery following marital separation. \textit{Psychological Science}, 23, 261–269.
	\bibitem{27} Hiraoka, R., Meyer, E. C., Kimbrel, N. A., DeBeer, B. B., Gulliver, S. B., \& Morissette, S. B. (2015). Self-compassion as a prospective predictor of PTSD symptom severity among trauma-exposed U.S. Iraq and Afghanistan war veterans. \textit{Journal of Traumatic Stress}, 28, 1–7.
	\bibitem{28} Wren, A. A., Somers, T. J., Wright, M. A., Goetz, M. C., Leary, M. R., Fras, A. M., et al. (2012). Self-compassion in patients with persistent musculoskeletal pain: Relationship of self-compassion to adjustment to persistent pain. \textit{Journal of Pain and Symptom Management}, 43(4), 759–770. 184 
	
	\bibitem{29} Neff, K. D., \& Beretvas, S.N. (2013). The role of self-compassion in romantic relationships. \textit{Self and Identity}, 12(1), 78–98.
	\bibitem{30} Yarnell, L. M., \& Neff, K. D. (2013). Self-compassion, interpersonal conflict resolutions, and well-being. \textit{Self and Identity}, 2(2), 146–159.
	\bibitem{31} Neff, K. D., \& Pommier, E. (2013). The relationship between self-compassion and other-focused concern among college undergraduates, community adults, and practicing meditators. \textit{Self and Identity}, 12(2), 160–176.
	
	\bibitem{32} Magnus, C. M. R., Kowalski, K. C., \& McHugh, T. L. F. (2010). The role of self-compassion in women’s self-determined motives to exercise and exercise-related outcomes. \textit{Self and Identity}, 9, 363–382.
	\bibitem{33} Schoenefeld, S. J., \& Webb, J. B. (2013). Self-compassion and intuitive eating in college women: Examining the contributions of distress tolerance and body image acceptance and action. \textit{Eating Behaviors}, 14(4), 493–496.
	\bibitem{34} Brooks, M., Kay-Lambkin, F., Bowman, J., \& Childs, S. (2012). Self-compassion amongst clients with problematic alcohol use. \textit{Mindfulness}, 3(4), 308–317.
	\bibitem{35} Terry, M. L., Leary, M. R., Mehta, S., \& Henderson, K. (2013). Self-compassionate reactions to health threats. \textit{Personality and Social Psychology Bulletin}, 39(7), 911–926.
	
	\bibitem{36} Zhang, J. W., \& Chen, S. (2016). Self-compassion promotes personal improvement from regret experiences via acceptance. \textit{Personality and Social Psychology Bulletin}, 42(2), 244– 258.
	\bibitem{37} Howell, A. J., Dopko, R. L., Turowski, J. B., \& Buro, K. (2011). The disposition to apologize. \textit{Personality and Individual Differences}, 51(4), 509–514.
	
	\bibitem{38} Neff, K. D. (2003). Development and validation of a scale to measure self-compassion. \textit{Self and Identity}, 2, 223–250.
	\bibitem{39} Neff, K. D., Hseih, Y., \& Dejitthirat, K. (2005). Self-compassion, achievement goals, and coping with academic failure. \textit{Self and Identity}, 4, 263–287.
	\bibitem{40} Breines, J. G., \& Chen, S. (2012). Self-compassion increases self-improvement motivation. \textit{Personality and Social Psychology Bulletin}, 38(9), 1133–1143.
	
	\bibitem{41} Compared with self-esteem, self-compassion is less contingent: Neff, K. D., \& Vonk, R. (2009). Self-compassion versus global self-esteem: Two different ways of relating to oneself. \textit{Journal of Personality}, 77, 23–50.
	
	
	%%% Chapter 3   %%%
	%%%%%%%%%%%%%%%%%%%	
	\vspace{3ex}
	\textbf{Глава \ref{The_Benefits_of_Self-Compassion}}

	\bibitem{42} MacBeth, A., \& Gumley, A. (2012). Exploring compassion: A meta-analysis of the association between self-compassion and psychopathology. \textit{Clinical Psychology Review}, 32, 545–552.
	\bibitem{43} Zessin, U., Dickhauser, O., \& Garbade, S. (2015). The relationship between self-compassion and well-being: A meta-analysis. \textit{Applied Psychology: Health and Well-Being}, 7(3), 340–364.
	\bibitem{44} Neff, K. D., Long, P., Knox, M. C., Davidson, O., Kuchar, A., Costigan, A., et al. (in press). The forest and the trees: Examining the association of self-compassion and its positive and negative components with psychological functioning. \textit{Self and Identity}.
	\bibitem{45} Hall, C. W., Row, K. A., Wuensch, K. L., \& Godley, K. R. (2013). The role of self-compassion in physical and psychological well-being. \textit{Journal of Psychology}, 147(4), 311–323.
	
	\bibitem{46} Neff, K. D., \& Germer, C. K. (2013). A pilot study and randomized controlled trial of the Mindful Self-Compassion program. \textit{Journal of Clinical Psychology}, 69(1), 28–44.
	
	\bibitem{47} Neff, K. D. (2003). Development and validation of a scale to measure self-compassion. \textit{Self and Identity}, 2, 223–250.
	\bibitem{48} Raes, F., Pommier, E., Neff, K. D., \& Van Gucht, D. (2011). Construction and factorial validation of a short form of the Self-Compassion Scale. \textit{Clinical Psychology and Psychotherapy}, 18, 250–255.
	
	\bibitem{49} Ullrich, P. M., \& Lutgendorf, S. K. (2002). Journaling about stressful events: Effects of cognitive processing and emotional expression. \textit{Annals of Behavioral Medicine}, 24(3), 244–250.
	
	
	%%% Chapter 4   %%%
	%%%%%%%%%%%%%%%%%%%	
	\vspace{3ex}
	\textbf{Глава \ref{The_Physiology_of_Self-Criticism_and_Self-Compassion}}
	
	\bibitem{50} Gilbert, P. (2009). \textit{The compassionate mind}. London: Constable.
	
	\bibitem{51} LeDoux, J. E. (2003). \textit{Synaptic self: How our brains become who we are}. New York: Penguin.
		
	\bibitem{52} Solomon, J., \& George, C. (1996). Defining the caregiving system: Toward a theory of caregiving. \textit{Infant Mental Health Journa}l, 17(3), 183–197.
	
	\bibitem{53} Stellar, J. E., \& Keltner, D. (2014). Compassion. In M. Tugade, L. Shiota, \& L. Kirby (Eds.), \textit{Handbook of positive emotions} (pp. 329–341). New York: Guilford Press.
	
	\bibitem{54} Rockcliff, H., Gilbert, P., McEwan, K., Lightman, S., \& Glover, D. (2008). A pilot exploration of heart rate variability and salivary cortisol responses to compassion-focused imagery. \textit{Clinical Neuropsychiatry}, 5, 132–139.


 	%%% Chapter 5   %%%
 	%%%%%%%%%%%%%%%%%%%	
 	\vspace{3ex}
 	\textbf{Глава \ref{The_Yin_and_Yang_of_Self-Compassion}}

	\bibitem{55} Eagly, A. H. (1987). \textit{Sex differences in social behavior: A social-role interpretation}. Hillsdale, NJ: Erlbaum.
	
	
    %%% Chapter 6   %%%
	%%%%%%%%%%%%%%%%%%%	
	\vspace{3ex}
	\textbf{Глава \ref{Mindfulness}}
	
	\bibitem{56} Bishop, S. R., Lau, M., Shapiro, S., Carlson, L., Anderson, N. D., Carmody, J., et al. (2004). Mindfulness: A proposed operational definition. \textit{Clinical Psychology Science and Practice}, 11, 191–206.
	
	\bibitem{57} Raichle, M. E., MacLeod, A. M., Snyder, A. Z., Powers, W. J., Gusnard, D. A., \& Shulman, G. L. (2001). A default mode of brain function. \textit{Proceedings of the National Academy of Sciences of the USA}, 98(2), 676–682.
	
	\bibitem{58} Brewer, J. A., Worhunsky, P. D., Gray, J. R., Tang, Y. Y., Weber, J., \& Kober, H. (2011). Meditation experience is associated with differences in default mode network activity and connectivity. \textit{Proceedings of the National Academy of Sciences of the USA}, 108(50), 20254– 20259.
	
	\bibitem{59} Taylor, V. A., Daneault, V., Grant, J., Scavone, G., Breton, E., Roffe-Vidal, S., et al. (2013). Impact of meditation training on the default mode network during a restful state. \textit{Social Cognitive and Affective Neuroscience}, 8(1), 4–14.
	 
	 
	%%% Chapter 7   %%%
	%%%%%%%%%%%%%%%%%%%
	\vspace{3ex}
	\textbf{Глава \ref{Letting_Go_of_Resistance}}
	
	\bibitem{60} Young, S. (2016). \textit{A pain processing algorithm}. Retrieved February 8, 2018, from \url{http://shinzen.org/wp-content/ uploads/2016/12/art_painprocessingalg.pdf}.
	
	\bibitem{61} McCracken, L. M., \& Eccleston, C. (2003). Coping or acceptance: What to do about chronic pain? \textit{Pain}, 105(1), 197–204.
	
	\bibitem{62} Wegner, D. M., Schneider, D. J., Carter, S. R., \& White, T. L. (1987). Paradoxical effects of thought suppression. \textit{Journal of Personality and Social Psychology}, 53(1), 5–13.
	
	
	%%% Chapter 8   %%%
	%%%%%%%%%%%%%%%%%%%
	\vspace{3ex}
	\textbf{Глава \ref{Backdraft}}
	
	\bibitem{63} Germer, C. K., \& Neff, K. D. (2013). Self-compassion in clinical practice. \textit{Journal of Clinical Psychology}, 69(8), 856–867.
	
	\bibitem{64} Singh, N. N., Wahler, R. G., Adkins, A. D., Myers, R. E., \& the Mindfulness Research Group. (2003). Soles of the feet: A mindfulness-based self-control intervention for aggression by an individual with mild mental retardation and mental illness. \textit{Research in Developmental Disabilities}, 24, 158–169.
	
	
	%%% Chapter 9   %%%
	%%%%%%%%%%%%%%%%%%%
	\vspace{3ex}
	\textbf{Глава \ref{Developing_Loving-Kindness}}
	
	\bibitem{65} Salzberg, S. (1997). \textit{Lovingkindness: The revolutionary art of happiness}. Boston: Shambhala.
	
	\bibitem{66} Goetz, J. L., Keltner, D., \& Simon-Thomas, E. (2010). Compassion: An evolutionary analysis and empirical review. \textit{Psychological Bulletin}, 136, 351–374.
	
	\bibitem{67} Dalai Lama. (2003). \textit{Lighting the path: The Dalai Lama teaches on wisdom and compassion.} South Melbourne, Australia: Thomas C. Lothian.
	
	\bibitem{68} Pace, T.\,W.\,W., Negi, L.\,T., Adame, D.\,D., Cole, S.\,P., Sivilli, T.\,I., Brown, T.\,D., et al. (2009). Effect of compassion meditation on neuroendocrine, innate immune and behavioral responses to psychosocial stress. \textit{Psychoneuroendocrinology}, 43(1), 87–98.
	
	\bibitem{69} Shonin, E., Van Gordon, W., Compare, A., Zangeneh, M., \& Griffiths, M. D. (2014). Buddhist-derived loving-kindness and compassion meditation for the treatment of psychopathology: A systematic review. \textit{Mindfulness}, 6, 1161–1180.
	
	\bibitem{70} Fredrickson, B. L., Cohn, M. A., Coffey, K. A., Pek, J., \& Finkel, S. M. (2008). Open hearts build lives: 188 Notes Positive emotions, induced through loving-kindness meditation, build consequential personal resources. \textit{Journal of Personal and Social Psychology}, 95, 1045–1062.
	
	\bibitem{71} Moyers, W., \& Ketcham, K. (2006). \textit{Broken: My story of addiction and redemption} (frontmatter, quoted from \textit{The Politics of the Brokenhearted} by Parker J. Palmer). New York: Viking Press.
	
	
	%%% Chapter 11  %%%
	%%%%%%%%%%%%%%%%%%%
	\vspace{3ex}
	\textbf{Глава \ref{Self-Compassionate_Motivation}}
	
	\bibitem{72} Gilbert, P. P., McEwan, K. K., Gibbons, L. L., Chotai, S. S., Duarte, J. J., \& Matos, M. M. (2012). Fears of compassion and happiness in relation to alexithymia, mindfulness, and self-criticism. \textit{Psychology and Psychotherapy: Theory, Research and Practice}, 85(4), 374–390.
	
	\bibitem{73} Neff, K. D., Hseih, Y., \& Dejitthirat, K. (2005). Self-compassion, achievement goals, and coping with academic failure. \textit{Self and Identity}, 4, 263–287.
	
	\bibitem{74} Neely, M. E., Schallert, D. L., Mohammed, S. S., Roberts, R. M., \& Chen, Y. (2009). Self-kindness when facing stress: The role of self-compassion, goal regulation, and support in college students well-being. \textit{Motivation and Emotion}, 33, 88–97.
	
	\bibitem{75} Breines, J. G., \& Chen, S. (2012). Self-compassion increases self-improvement motivation. \textit{Personality and Social Psychology Bulletin}, 38(9), 1133–1143.
	
	\bibitem{76} Schwartz, R. (1994). \textit{Internal family systems therapy}. New York: Guilford Press.
	
	
	%%% Chapter 12  %%%
	%%%%%%%%%%%%%%%%%%%
	\vspace{3ex}
	\textbf{Глава \ref{Self-Compassion_and_Our_Bodies}}
	
	\bibitem{77} Grogan, S. (2016). \textit{Body image: Understanding body dissatisfaction in men, women and children}. London: Taylor \& Francis.
	
	\bibitem{78} Braun, T. D., Park, C. L., \& Gorin, A. (2016). Self-compassion, body image, and disordered eating: A review of the literature. \textit{Body Image}, 17, 117–131.
	
	\bibitem{79} Albertson, E. R., Neff, K. D., \& Dill-Shackleford, K. E. (2014). Self-compassion and body dissatisfaction in women: A randomized controlled trial of a brief meditation intervention. \textit{Mindfulness}, 6(3), 444–454.
	
	
	%%% Chapter 13  %%%
	%%%%%%%%%%%%%%%%%%%
	\vspace{3ex}
	\textbf{Глава \ref{Stages_of_Progress}}
	
	\bibitem{80} Ch\"{o}dr\"{o}n, P. (1991/2001). \textit{The wisdom of no escape and the path of loving-kindness}. Boston: Shambhala, p. 4. 
	
	\bibitem{81} Nairn, R. (2009, September). Lecture (part of Foundation Training in Compassion), Kagyu Samye Ling Monastery, Dumfriesshire, Scotland.
	
	
	%%% Chapter 14  %%%
	%%%%%%%%%%%%%%%%%%%
	\vspace{3ex}
	\textbf{Глава \ref{Living_Deeply}}
	
	\bibitem{82} Hayes, S. C., Strosahl, K. D., \& Wilson, K. G. (2011). \textit{Acceptance and commitment therapy: The process and practice of mindful change} (2nd ed.). New York: Guilford Press.

	\bibitem{83} For a list of more than 50 common core values, see \url{http://jamesclear.com/core-values}.

	\bibitem{84} Nhat Hahn, T. (2014). \textit{No mud, no lotus: The art of transforming suffering}. Berkeley, CA: Parallax Press.


	%%% Chapter 15  %%%
	%%%%%%%%%%%%%%%%%%%
	\vspace{3ex}
	\textbf{Глава \ref{Being_There_for_Others_without_Losing_Ourselves}}
	
	\bibitem{85} Rizzolatti, G., Fogassi, L., \& Gallese, V. (2006). Mirrors in the mind. \textit{Scientific American}, 295(5), 54–61.

	\bibitem{86} Lloyd, D., Di Pellegrino, G., \& Roberts, N. (2004). Vicarious responses to pain in anterior cingulate cortex: Is empathy a multisensory issue? \textit{Cognitive, Affective, and Behavioral Neuroscience}, 4(2), 270–278.
	
	
	%%% Chapter 16  %%%
	%%%%%%%%%%%%%%%%%%%
	\vspace{3ex}
	\textbf{Глава \ref{Meeting_Difficult_Emotions}}
	
	\bibitem{87} Germer, C. K. (2009). \textit{The mindful path to self-compassion: Freeing yourself from destructive thoughts and emotions}. New York: Guilford Press.
	
	\bibitem{88} Creswell, J. D., Way, B. M., Eisenberger, N. I., \& Lieberman, M. D. (2007). Neural correlates of dispositional mindfulness during affect labeling. \textit{Psychosomatic Medicine}, 69, 560–565.
	
	
	%%% Chapter 17  %%%
	%%%%%%%%%%%%%%%%%%%
	\vspace{3ex}
	\textbf{Глава \ref{Self-Compassion_and_Shame}}
	
	\bibitem{89} Lieberman, M. D. (2014). \textit{Social: Why our brains are wired to connect.} Oxford, UK: Oxford University Press.
	
	\bibitem{90} Tangney, J. P., \& Dearing, R. L. (2003). \textit{Shame and guilt}. New York: Guilford Press.
	
	\bibitem{91} Johnson, E. A., \& O’Brien, K. A. (2013). Self-compassion soothes the savage ego-threat system: Effects on negative affect, shame, rumination, and depressive symptoms. \textit{Journal of Social and Clinical Psychology}, 32(9), 939–963.
	
	\bibitem{92} Dozois, D. J., \& Beck, A. T. (2008). Cognitive schemas, beliefs and assumptions. \textit{Risk Factors in Depression}, 1, 121–143.
	
	
	%%% Chapter 18  %%%
	%%%%%%%%%%%%%%%%%%%
	\vspace{3ex}
	\textbf{Глава \ref{Self-Compassion_in_Relationships}}
	
	\bibitem{93} Sartre, J. (1989). \textit{No exit and three other plays} (S. Gilbert, Trans.). New York: Vintage.
	
	\bibitem{94} Decety, J., \& Ickes, W. (2011). \textit{The social neuroscience of empathy.} Cambridge, MA: MIT Press.
	
	\bibitem{95} Garland, E. L., Fredrickson, B., Kring, A. M., Johnson, D. P., Meyer, P. S., \& Penn, D. L. (2010). Upward spirals of positive emotions counter downward spirals of negativity: Insights from the broaden-and-build theory and affective neuroscience on the treatment of emotion dysfunctions and deficits in psychopathology. \textit{Clinical Psychology Review}, 30(7), 849–864.
	
	\bibitem{96} Klimecki, O. M., Leiberg, S., Ricard, M., \& Singer, T. (2013). Differential pattern of functional brain plasticity after compassion and empathy training. \textit{Social Cognitive and Affective Neuroscience}, 9(6), 873–879.
	
	
	\bibitem{97} Neff, K. D., \& Beretvas, S. N. (2013). The role of self-compassion in romantic relationships. \textit{Self and Identity}, 12(1), 78–98.
	
	\bibitem{98} Gilbert, P. (2009). Introducing compassion-focused therapy. \textit{Advances in Psychiatric Treatment}, 15, 199–208.
	
	
	%%% Chapter 19  %%%
	%%%%%%%%%%%%%%%%%%%
	\vspace{3ex}
	\textbf{Глава \ref{Self-Compassion_for_Caregivers}}
	
	\bibitem{99} Lloyd, D., Di Pellegrino, G., \& Roberts, N. (2004). Vicarious responses to pain in anterior cingulate cortex: Is empathy a multisensory issue? \textit{Cognitive, Affective, and Behavioral Neuroscience}, 4(2), 270–278.
	
	\bibitem{100} Maslach, C. (2003). Job burnout: New directions in research and intervention. \textit{Current Directions in Psychological Science}, 12(5), 189–192.
	
	\bibitem{101} Williams, C. A. (1989). Empathy and burnout in male and female helping professionals. \textit{Research in Nursing and Health}, 12(3), 169–178.
	
	\bibitem{102} Singer, T., \& Klimecki, O. M. (2014). Empathy and compassion. \textit{Current Biology}, 24(18), R875–R878. 

	\bibitem{103} Rogers, C. (1961). \textit{On becoming a person: A therapist’s view of psychotherapy}. London: Constable, p. 248.
	
	\bibitem{104} Klimecki, O. M., Leiberg, S., Ricard, M., \& Singer, T. (2013). Differential pattern of functional brain plasticity after compassion and empathy training. \textit{Social Cognitive and Affective Neuroscience}, 9(6), 873–879.
	
	
	%%% Chapter 20  %%%
	%%%%%%%%%%%%%%%%%%%
	\vspace{3ex}
	\textbf{Глава \ref{Self-Compassion_and_Anger_in_Relationships}}
	
	\bibitem{105} Keltner, D., \& Haidt, J. (2001). Social functions of emotions. In T. J. Mayne \& G. A. Bonanno (Eds.), \textit{Emotions: Current issues and future directions} (pp. 192–213). New York: Guilford Press.
	
	\bibitem{106} Dimsdale, J. E., Pierce, C., Schoenfeld, D., Brown, A., Zusman, R., \& Graham, R. (1986). Suppressed anger and blood pressure: The effects of race, sex, social class, obesity, and age. \textit{Psychosomatic Medicine}, 48(6), 430–436.
	
	\bibitem{107} Blatt, S. J., Quinlan, D. M., Chevron, E. S., McDonald, C., \& Zuroff, D. (1982). Dependency and self-criticism: Psychological dimensions of depression. \textit{Journal of Consulting and Clinical Psychology}, 50(1), 113–124.
	
	\bibitem{108} Denson, T. F., Pedersen, W. C., Friese, M., Hahm, A., \& Roberts, L. (2011). Understanding impulsive aggression: Angry rumination and reduced self-control capacity are mechanisms underlying the provocation–aggression relationship. \textit{Personality and Social Psychology Bulletin}, 37(6), 850–862.
	
	\bibitem{109} Christensen, A., Doss, B., \& Jacobson, N. S. (2014).\textit{ Reconcilable differences: Rebuild your relationship by rediscovering the partner you love—without losing yourself} (2nd ed.). New York: Guilford Press.
	
	\bibitem{110} For the effects of stress on the body, see \url{http://apa.org/helpcenter/stress-body.aspx}. 
	
	\bibitem{111} Rosenberg, M. B. (2003). \textit{Nonviolent communication: A language of life.} Encinitas, CA: PuddleDancer Press.
	
	
	%%% Chapter 21  %%%
	%%%%%%%%%%%%%%%%%%%
	\vspace{3ex}
	\textbf{Глава \ref{Self-Compassion_and_Forgiveness}}
	
	\bibitem{112} Luskin, F. (2002). \textit{Forgive for good.} New York: HarperCollins.
	
	\bibitem{113} Breines, J. G., \& Chen, S. (2012). Self-compassion increases self-improvement motivation. \textit{Personality and Social Psychology Bulletin}, 38(9), 1133–1143.
	
	
	%%% Chapter 22  %%%
	%%%%%%%%%%%%%%%%%%%
	\vspace{7ex}
	\textbf{Глава \ref{Embracing_the_Good}}
	
	\bibitem{114} Singer, T., \& Klimecki, O. M. (2014). Empathy and compassion. \textit{Current Biology}, 24(18), R875–R878.
	
	\bibitem{115} Rozin, P., \& Royzman, E. B. (2001). Negativity bias, negativity dominance, and contagion. \textit{Personality and Social Psychology Review}, 5(4), 296–320.
	
	\bibitem{116} Hanson, R. (2009). \textit{Buddha’s brain: The practical neuroscience of happiness, love, and wisdom}. Oakland, CA: New Harbinger.
	
	\bibitem{117} Hanson, R. (2013). \textit{Hardwiring happiness: The practical science of reshaping your brain—and your life}. New York: Random House.
	
	\bibitem{118} Fredrickson, B. L. (2004). The broaden-andbuild theory of positive emotions. \textit{Philosophical Transactions of the Royal Society B: Biological Sciences}, 359(1449), 1367–1378.
	
	\bibitem{119} Keller, H. (2000). \textit{To love this life: Quotations by Helen Keller}. New York: AFB Press.
	
	\bibitem{120} Bryant, F. Notes 193 B., \& Veroff, J. (2007). \textit{Savoring: A new model of positive experience.} Hillsdale, NJ: Erlbaum.
	
	\bibitem{121} Jose, P. E., Lim, B. T., \& Bryant, F. B. (2012). Does savoring increase happiness?: A daily diary study. \textit{Journal of Positive Psychology}, 7(3), 176–187.
	
	\bibitem{122} Emmons, R. A. (2007). \textit{Thanks!: How the new science of gratitude can make you happier.} Boston: Houghton Mifflin Harcourt.
	
	\bibitem{123} Krejtz, I., Nezlek, J. B., Michnicka, A., Holas, P., \& Rusanowska, M. (2016). Counting one’s blessings can reduce the impact of daily stress. \textit{Journal of Happiness Studies}, 17(1), 25–39.
	
	\bibitem{124} Nepo, M. (2011). \textit{The book of awakening: Having the life you want by being present to the life you have.} Newburyport, MA: Conari Press, p. 23.
	
	\bibitem{125} Baraz, J., \& Alexander, S. (2010). Awakening joy: 10 steps that will put you on the road to real happiness. New York: Bantam. Also see the video of James Baraz’s mother, “Confessions of a Jewish Mother: How My Son Ruined My Life,” at \url{http://youtube.com/watch?v=FRbL46mWx9w}.
	
	\bibitem{126} This practice is based on an exercise developed by Bryant \& Veroff (2007), who found that walking in this way for one week significantly increased happiness.
	
	\bibitem{127} Dickinson, E. (1872). Dickinson–Higginson correspondence, late 1872. Dickinson Electronic Archives. Institute for Advanced Technology in the Humanities (IATH), University of Virginia. Retrieved February 8, 2018, from \url{http://archive.emilydickinson.org/correspondence/higginson/l381.html}.
	
	\bibitem{128} Godsey, J. (2013). The role of mindfulness-based interventions in the treatment of obesity and eating disorders: An integrative review. \textit{Complementary Therapies in Medicine}, 21(4), 430–439.
	
	\bibitem{129} For a review of this research, see Emmons, R. A. (2007). \textit{Thanks!: How the new science of gratitude can make you happier.} Boston: Houghton Mifflin Harcourt.
	
	
	
	%%% Chapter 23  %%%
	%%%%%%%%%%%%%%%%%%%
	\vspace{3ex}
	\textbf{Глава \ref{Self-Appreciation}}
	
	\bibitem{130} Neff, K. (2011). \textit{Self-compassion: The proven power of being kind to yourself.} New York: William Morrow.
	
	\bibitem{131} Williamson, M. (1996). \textit{A return to love: Reflections on the principles of “A course in miracles.”} San Francisco: Harper One.
	
\end{thebibliography}
