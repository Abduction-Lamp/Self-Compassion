% !TEX root = ../../Self-Compassion.tex

\chapter{Жить полной жизнью} \label{Living_Deeply}

Фундаментальный вопрос самосострадания~"--- это <<что мне нужно?>>. \textbf{Но мы не можем дать себе то, что нам нужно, если мы не знаем, что мы больше всего ценим в жизни}\cite{82}. Это наши основные (базовые) ценности, глубоко засевшие идеалы, которые ведут и направляют нас и наполняют жизнь смыслом. И потребности, и ценности отображают что-то неотъемлемое в человеческой природе. Потребности обычно связаны с выживанием (физическим и эмоциональным), как, например, потребность в здоровье и безопасности или потребность в любви и общении. А ценности содержат элемент выбора, например, выбор посвятить себя борьбе за равноправие или творчеству. 

Наши жизненные трудности очень зависят от наших основных ценностей. К примеру, если вы цените свободное время и новые приключения, то, что вас не повысили до должности, на которой вы должны были бы работать больше часов в неделю, может оказаться подарком судьбы, но, если ваша главная ценность~"--- обеспечение семьи, эта ситуация может вызвать чувство разочарования и опустошения.

\vspace{2ex}

Между \emph{целями} и \emph{основными ценностями} есть разница:

\begin{itemize}
	\itemdiamondsuit Целей можно \emph{достичь}. Основные ценности направляют нас даже \emph{\textbf{после}} достижения целей.

	\itemdiamondsuit Цель~"--- это \emph{пункт назначения}. Основные ценности~"--- это \emph{\textbf{указатели}} на вашем пути.

	\itemdiamondsuit Цель~"--- это что-то, что мы \emph{делаем}. Основные ценности~"--- это то, \emph{\textbf{кем мы являемся}}.

	\itemdiamondsuit Цели \emph{ставят} перед собой. Основные ценности \emph{\textbf{находят}}.

	\itemdiamondsuit Цели часто приходят \emph{снаружи}, из внешнего мира. Основные ценности зарождаются \emph{\textbf{глубоко внутри}}. 
\end{itemize}

\vspace{2ex}

К базовым ценностям могут относиться сострадание, щедрость, честность и служба. Многие из наших основных ценностей так или иначе связаны с отношениями с другими людьми~"--- с тем, как мы обращаемся с другими людьми и как нам бы хотелось, чтобы обращались с нами, а другие~"---  более личные: свобода, духовный рост, исследования или самовыражение через искусство \cite{83}. 

\begin{quotation}
	\textit{
		Марк работал в престижной юридической фирме, специализируясь на корпоративном праве, ездил на <<Лексусе>>, и все считали его успешным человеком. Его родители всегда хотели (и ожидали от него), чтобы он стал врачом или юристом, но, став наконец совладельцем фирмы, он понял, что чего-то ему не хватает.  Он не был по-настоящему счастлив и не знал, как так получилось, что его жизнь была максимально далека от того, чего ему хотелось. Марк обожал писать и с гораздо большим удовольствием бы работал над романом, чем исками против нарушителей авторского права.  Он часто мечтал о том, чтобы уйти из фирмы и стать писателем, но его постоянно останавливал страх, что родители это не одобрят. Более того, он страшно боялся потерпеть неудачу, боялся того, что может случиться, если он не сможет писательством зарабатывать на жизнь.
	} 
\end{quotation}

Когда мы живем в несоответствии со своими базовыми ценностями, мы страдаем. Поэтому очень важный для нас всех акт сострадания~"--- определить, какие наши ценности, понять, живем ли мы в соответствии с ними, и постараться обеспечить себя тем, что нам нужно. Если нам по-настоящему важно наше счастье и мы хотим облегчить свои страдания~"--- другими словами, если мы сострадаем себе~"--- мы, как правило, можем найти внутренние ресурсы на то, чтобы жить ближе к своим ценностям и вести более полную жизнь, в которой больше смысла. 

\begin{quotation}
	\textit{
		В итоге Марк впал в депрессию и начал заниматься с психотерапевтом, который рассказал ему о самосострадании. Марк понял, что для того, чтобы стать себе другом, нужно больше времени уделять тому, что ему действительно доставляет удовольствие. Он всегда рано вставал, поэтому он начал каждое утро посвящать написанию рассказа, сюжет которого крутился у него в голове уже пять лет. Это небольшое нововведение, которое Марк называл <<начнем с главного>>, сделало его счастливее и энергичнее, а также он заметил, что день офисной работы уже не казался ему таким обременительным. Он вступил в клуб писателей-любителей, который собирался по вечерам, после работы, нашел там друзей-единомышленников и начал ходить на публичное чтение книг в местном книжном магазине. Благодаря всему этому у него появилось чувство, что он наконец-то свернул на правильный путь. Внутреннее давление изменить профессию ушло, по крайней мере, на время.
	}
\end{quotation}


\Exercises{Поиск наших основных ценностей} \label{Ex:Discovering_Our_Core_Values}
\begin{itemize}
	\itemWritingHand Представьте себе, что вы уже пожилой. Вы сидите в красивом саду и размышляете о своей жизни. Оглядываясь назад на прожитое, вы чувствуете глубокое удовлетворение и радость. Хоть ваша жизнь и не всегда была легкой, вы все же смогли оставаться верным себе настолько, насколько могли. Какие основные ценности, которых вы придерживались, наполнили вашу жизнь смыслом? Это может быть помощь другим, путешествия и приключения или время, проведенное на природе. Запишите их ниже.
\end{itemize}

\setlength{\extrarowheight}{2mm}
\begin{tabularx}{\textwidth}{X}
	\\
	\arrayrulecolor{gray}\hline\\
	\hline\\
	\hline\\
	\hline\\
	\hline\\
	\hline\\	
	\hline\\
	\hline\\
\end{tabularx}
\setlength{\extrarowheight}{0mm}
\begin{itemize}
	\itemWritingHand Теперь напишите, что в вашей жизни противоречит вашим базовым ценностям. Например, вы могли быть слишком заняты, чтобы проводить время на природе, хотя это ваше самое любимое занятие.  Если у вас есть несколько ценностей, которых, как вам кажется, вы не придерживаетесь, выберите самую важную из них.
\end{itemize}

\setlength{\extrarowheight}{2mm}
\begin{tabularx}{\textwidth}{X}
	\\
	\arrayrulecolor{gray}\hline\\
	\hline\\
	\hline\\
	\hline\\
	\hline\\
	\hline\\	
	\hline\\
	\hline\\
	\hline\\
\end{tabularx}
\setlength{\extrarowheight}{0mm}
\begin{itemize}
	\itemWritingHand Конечно, часто мы сталкиваемся с препятствиями, которые мешают нам жить по своим основным ценностям. Некоторые из них~"--- внешние препятствия, как недостаток денег или времени. Например, ваша работа занимает столько времени, что у вас особенно нет времени на отдых на природе. Если в вашей жизни есть внешние препятствия, запишите их.
\end{itemize}

\setlength{\extrarowheight}{2mm}
\begin{tabularx}{\textwidth}{X}
	\\
	\arrayrulecolor{gray}\hline\\
	\hline\\
	\hline\\
	\hline\\
	\hline\\
	\hline\\	
	\hline\\	
	\hline\\	
	\hline\\
	\hline\\
	\hline\\
\end{tabularx}
\setlength{\extrarowheight}{0mm}
\begin{itemize}
	\itemWritingHand Могут быть и внутренние факторы, препятствующие жизни в согласии с вашими ценностями. Возможно, вы боитесь потерпеть неудачу, сомневаетесь в своих силах или вам мешает ваш внутренний критик? Например, может быть, вы считали, что не заслуживаете провести беззаботный день в лесу. Запишите все ваши внутренние препятствия.
\end{itemize}

\setlength{\extrarowheight}{2mm}
\begin{tabularx}{\textwidth}{X}
	\\
	\arrayrulecolor{gray}\hline\\
	\hline\\
	\hline\\
	\hline\\
	\hline\\
	\hline\\	
	\hline\\
	\hline\\
	\hline\\
	\hline\\
	\hline\\
\end{tabularx}
\setlength{\extrarowheight}{0mm}
\begin{itemize}
	\itemWritingHand Теперь поразмышляйте, могут ли вам помочь доброта к себе и самосострадание начать жить в соответствии с вашими ценностями~"--- например, помочь преодолеть внутренние препятствия вроде внутреннего критика. Может ли самосострадание помочь вам почувствовать себя в безопасности и набраться достаточно уверенности в себе, чтобы предпринять новые для вас действия, решиться на какой-то риск, прекратить делать что-то, на что вы зря тратите свое время? А, может быть, вы нашли какой-то новый способ больше претворять в жизнь свои ценности? Например, найти работу с гибким графиком, чтобы чаще ходить в походы? 
\end{itemize}

\setlength{\extrarowheight}{2mm}
\begin{tabularx}{\textwidth}{X}
	\\
	\arrayrulecolor{gray}\hline\\
	\hline\\
	\hline\\
	\hline\\
	\hline\\
	\hline\\	
	\hline\\
	\hline\\
\end{tabularx}
\setlength{\extrarowheight}{0mm}
\begin{itemize}
	\itemWritingHand Наконец, если есть какие-то непреодолимые препятствия, мешающие вам жить в соответствии с вашими ценностями, можете ли вы отнестись к себе с состраданием за эти трудности? Иными словами, не отказываться от своих ценностей, несмотря на обстоятельства? А если непреодолимая проблема состоит в том, что вы, как и все остальные люди, неидеальны, можете ли вы себя за это простить?
\end{itemize}

\setlength{\extrarowheight}{2mm}
\begin{tabularx}{\textwidth}{X}
	\\
	\arrayrulecolor{gray}\hline\\
	\hline\\
	\hline\\
	\hline\\
	\hline\\
	\hline\\	
	\hline\\
	\hline\\
	\hline\\
	\hline\\
\end{tabularx}
\setlength{\extrarowheight}{0mm}

\Reflection{
	Как все прошло? Наткнулись ли вы на что-то неожиданное.
	
	Некоторым людям сложно определить свои основные ценности во время выполнения упражнения. Возможно, причина в том, что многие из нас живут такой загруженной и расписанной по минутам жизнью, что у них даже не было времени сделать паузу и подумать, какие ценности для нас основные. В этом случае самосострадание выражается в форме одного вопроса: <<\textbf{Что для меня важно}?>> Такие ли уж личные ваши ценности, или все-таки это ценности, которые \textbf{\textit{навязаны}} другими людьми?
	
	Другим людям могут быть понятны свои основные ценности, но они чувствуют во время выполнения упражнения разочарование из-за того, что живут не в соответствии с ними. Хоть и полезно бывает практиковать самосострадание для того, чтобы отпустить что-то, что нам мешает, так же важно принять, что иногда мы просто не можем жить в соответствии со своими основными ценностями, как бы мы не старались.  Если вы оказались в такой же ситуации, попробуйте принять, что человеческая жизнь сложна, но в то же самое время поддерживать и не гасить пламя глубинных желаний в вашем сердце. Вы можете обнаружить, что выражение ваших ценностей даже в какой-то мелочи может изменить вашу жизнь. 
}

\newpage
\InformalPractices{Жизнь с клятвой} \label{IP:Living_with_a_Vow}

Часто чувства недовольства, разочарования и тревоги возникают от понимания, что мы живем не в соответствии с нашими базовыми ценностями. Когда мы делаем неприятное для нас открытие, что мы <<не в том месте, не в то время занимаемся не тем с не теми людьми>>, наступает время вспомнить наши основные ценности. 

Из базовых ценностей можно сделать клятвы, чтобы помочь нашей памяти. 

\vspace{5ex}

\noindent\textbf{Что такое клятва?}
 
\begin{itemize}
	\itemdiamondsuit \textbf{Клятва}~"--- это \emph{стремление}, к которому мы можем постоянно себя возвращать, когда мы как будто бы заблудились.
	
	\itemdiamondsuit \textbf{Клятва} \emph{бросает якорь} нашей жизни в важных для нас местах. Это не связывающий контракт. 
	
	\itemdiamondsuit \textbf{Клятва} работает по такому же \emph{принципу}, что и дыхание в дыхательной медитации~"--- это безопасное место, куда мы можем вернуться, когда нам кажется, что мы потеряны и течение повседневной жизни унесло нас не туда. 
\end{itemize}

\vspace{2ex}

Нужно относиться к себе с огромным состраданием, когда мы замечаем, что сбились с пути~"--- никакого стыда или обвинения себя!~"--- и снова сосредоточиться на наших основных ценностях.

\vspace{2ex}

\textbf{Выберите важную базовую ценность, которую вы открыли для себя в последнем упражнении и которую вам бы хотелось претворять в жизнь до конца ваших дней.}

\newpage
\begin{itemize}
	\itemWritingHand Теперь попробуйте записать ее в форме клятвы: <<Пусть я...>> или <<Я клянусь... как могу>>. 
\end{itemize}

\setlength{\extrarowheight}{2mm}
\begin{tabularx}{\textwidth}{X}
	\\
	\arrayrulecolor{gray}\hline\\
	\hline\\
	\hline\\
	\hline\\
	\hline\\
	\hline\\	
	\hline\\
	\hline\\
	\hline\\
	\hline\\
	\hline\\
	\hline\\
	\hline\\
\end{tabularx}
\setlength{\extrarowheight}{0mm}

\begin{center}
	{\large Закройте глаза и повторите вашу клятву про себя несколько раз.}
\end{center}

\vspace{3ex}

\Reflection{
	Удалось ли вам создать клятву, наполненную для вас глубоким смыслом? Что вы ощутили, когда задали своим намерениям это направление? 
	
	Многие люди говорят, что повторение клятвы каждый день помогает им не сбиваться с правильного пути~"--- как навигатор, которому вы задали маршрут до дома. Если хотите, утром, перед тем, как вставать, можете положить руку на сердце и повторить несколько раз свою клятву, а потом уже встать. То же самое можно делать и перед сном. Иногда бывает полезно придумать для себя небольшой ритуал~"--- например, зажигать свечу, когда произносите свою клятву.
}

\newpage
\Exercises{Нет худа без добра} \label{Ex:Silver_Linings}

Еще один важный аспект по-настоящему полной жизни~"--- это умение извлечь урок из неприятных ситуаций в нашей жизни. Большинство из нас боятся трудностей и неудач, но именно такой опыт преподает нам уроки, которые мы бы иначе не получили.

Тхить Нят Хань говорит: \textbf{<<Нет грязи, нет и лотоса>>}\cite{84}. Другими словами, если мы не повстречаемся с <<грязью>>~"--- неприятными и трудными ситуациями, мы не сможем <<расцвести>>~"--- достичь своего потенциала. Трудности могут нас заставить залезть глубоко внутрь себя за ответами и открыть для себя ресурсы и идеи, о существовании которых мы даже не подозревали. 

В пословице <<нет худа без добра>> речь идет именно об этой правде жизни. Один из подарков, которые нам приносит \textbf{самосострадание~"--- это способность побыть с нашим страданием без чувства подавленности и разбитости и этим самым дать себе поддержку, необходимую для личностного роста и развития}. 

До начала этого упражнения можете сделать два--три глубоких вдоха и выдоха и закрыть на несколько минут глаза, чтобы успокоиться и сосредоточиться. Положите руку на сердце или попробуйте любое другое успокаивающее прикосновение (смотри упрожнение на стр.\:\pageref{IP:Soothing_Touch}) в качестве жеста поддержки и доброты. 

\begin{itemize}
	\itemWritingHand Подумайте о каких-то трудностях из своего прошлого, которые тогда казались вам очень сложными или даже невыносимыми, но которые преподали вам важный урок. Выберите событие из достаточно далекого прошлого, чтобы оно имело ясную развязку и вы научились тому, чему должны были. Что это была за ситуация? Опишите ее ниже. 
\end{itemize}

\setlength{\extrarowheight}{2mm}
\begin{tabularx}{\textwidth}{X}
	\\
	\arrayrulecolor{gray}\hline\\
	\hline\\
	\hline\\
	\hline\\
	\hline\\
	\hline\\	
	\hline\\
	\hline\\
	\hline\\
	\hline\\
\end{tabularx}
\setlength{\extrarowheight}{0mm}
\begin{itemize}
	\itemWritingHand Какой урок, который вы без нее, скорее всего, не выучили бы, преподала вам эта трудность? Это тоже запишите.
\end{itemize}

\setlength{\extrarowheight}{2mm}
\begin{tabularx}{\textwidth}{X}
	\\
	\arrayrulecolor{gray}\hline\\
	\hline\\
	\hline\\
	\hline\\
	\hline\\
	\hline\\	
	\hline\\
	\hline\\
	\hline\\
	\hline\\
	\hline\\
	\hline\\
\end{tabularx}
\setlength{\extrarowheight}{0mm}
\begin{itemize}
	\itemWritingHand В качестве мысленного эксперимента подумайте, есть ли в вашей жизни  прямо сейчас трудность, в которой может быть что-то хорошее или полезное. Если да, какой скрытый урок может в ней содержаться?
\end{itemize}

\setlength{\extrarowheight}{2mm}
\begin{tabularx}{\textwidth}{X}
	\\
	\arrayrulecolor{gray}\hline\\
	\hline\\
	\hline\\
	\hline\\
	\hline\\
	\hline\\	
	\hline\\
	\hline\\
	\hline\\
	\hline\\
	\hline\\
	\hline\\
\end{tabularx}
\setlength{\extrarowheight}{0mm}
\begin{itemize}
	\itemWritingHand Как практика самосострадания может помочь вам почувствовать себя в безопасности и придать вам сил, чтобы вы смогли научиться тому, чему должны?
\end{itemize}

\setlength{\extrarowheight}{2mm}
\begin{tabularx}{\textwidth}{X}
	\\
	\arrayrulecolor{gray}\hline\\
	\hline\\
	\hline\\
	\hline\\
	\hline\\
	\hline\\	
	\hline\\
	\hline\\
	\hline\\
	\hline\\
	\hline\\
	\hline\\
\end{tabularx}
\setlength{\extrarowheight}{0mm}

\Reflection{
	Как для вас прошло это упражнение? Вышло ли у вас найти что-то хорошее в текущей ситуации или понять, как самосострадание может вам помочь это сделать?
	
	Иногда в трудных ситуациях нет ничего хорошего, и то, что вы их пережили~"--- уже достижение. Если у вас сейчас аналогичная ситуация, похвалите себя за это.
	
	Память об уроках, которые приходят к нам через трудности, помогает нам увидеть эти трудности в более позитивном свете. Конечно, это не значит, что вы должны отрицать сложность ситуации. Если вам тяжело было увидеть что-то хорошее в текущей ситуации, это вполне естественно и заставлять себя не нужно. Просто открывшись возможности, что в трудностях содержится личностный рост, мы можем легче отнестись ко всему.
}