% !TEX root = ../../Self-Compassion.tex

\chapter{Чем самосострадание не является} \label{What_Self-Compassion_Is_Not}

Часто у людей есть опасения и сомнения, а действительно ли самосострадание~"--- это хорошая идея и можно ли с ним переборщить. Западная культура, мягко говоря, не возводит самосострадание в ранг добродетели, поэтому у многих людей мысль о том, чтобы быть добрыми к себе, вызывает подозрение. Эти опасения могут помешать нам практиковать самосострадание, поэтому неплохо замечать их у себя и впоследствии от них избавляться.

\newpage

\Exercises{Мои опасения насчет самосострадания}

\begin{itemize}
	\itemWritingHand Запишите все опасения, которые у вас есть по поводу самосострадания~"--- страхи и беспокойство о его возможных минусах.
	
\end{itemize}

\setlength{\extrarowheight}{2mm}
\begin{tabularx}{\textwidth}{X}
	\\
	\arrayrulecolor{gray}\hline\\
	\hline\\
	\hline\\
	\hline\\
	\hline\\
	\hline\\
	\hline\\
	\hline\\
	\hline\\
	\hline\\
	\hline\\
	\hline\\
	\hline\\
	\hline\\
	\hline\\
	\hline\\
	\hline\\
	\hline\\	
\end{tabularx}
\setlength{\extrarowheight}{0mm}

Иногда на наше отношение к самосостраданию влияет то, что думают о нем другие люди в нашей жизни. Запишите все опасения по поводу самосострадания, которые есть у других людей или у общества в целом.

\Reflection{
	Если у вас получилось заметить какие-то из своих опасений по поводу самосострадания, это очень хорошо. \textbf{Эти опасения~"--- преграды на вашем пути к самосостраданию, и осознание этих преград~"--- первый шаг к их разрушению.} К счастью, все больше и больше исследований показывают, что самые частые опасения по поводу самосострадания~"--- на самом деле неправильные представления о нем. Другими словами, эти неправильные представления чаще всего необоснованны. Ниже мы приведем некоторые страхи, которыми с нами делились многие люди на наших курсах и объясним, почему эти страхи несостоятельны.
}

\textbf{\textit{<<Разве самосострадание – это не просто праздник жалости к себе, бедному и несчастному?>>}}

\vspace{2ex}

Многие боятся, что самосострадание~"--- это на самом деле всего лишь форма жалости к себе. В действительности,\textbf{ самосострадание~"--- противоядие от жалости к себе}. Жалость к себе внушает нам, что это мы такие бедные и несчастные, а с самосостраданием приходит понимание того, что жизнь бывает тяжела для всех. Исследования показывают, что люди, практикующие самосострадание, более склонны к тому, чтобы объективно взглянуть на вещи вместо того, чтобы зациклиться на своих собственных неприятных эмоциях. Они также менее склонны к тому, чтобы постоянно думать о том, как все плохо, и это одна из причин, почему у них, как правило, меньше проблем с психическим здоровьем. Когда мы практикуем самосострадание, мы помним, что \textbf{все время от времени страдают (человеческая общность) и не преувеличиваем наши трудности (осознанность)}. Самосострадание не имеет ничего общего с сокрушениями вроде <<о горе мне>>. 

\vspace{4ex}

\textbf{\textit{<<Самосострадание~"--- для нытиков, а мне, чтобы со всем в жизни справляться, нужно быть сильным>>.}}

\vspace{2ex}

Другой часто встречающийся страх~"--- \textbf{страх, что самосострадание сделает нас слабыми и уязвимыми}. На самом деле, самосострадание~"--- это надежный источник внутренней силы, которая дает нам храбрость и стабильность, когда мы встречаемся с трудностями. Исследование показывают, что люди, практикующие самосострадание, лучше справляются с тяжелыми ситуациями, такими, как разводы, психологическая травмы или хроническая боль.

\vspace{4ex}

\textbf{\textit{<<Мне нужно думать больше о других людях, а не о себе. Самосострадание сделает из меня эгоиста и эгоцентрика>>.}}

\vspace{2ex}

Некоторые беспокоятся, что, практикуя самосострадание, а не просто сострадание к другим людям, они станут эгоистичными или эгоцентричными. Тем не менее, сострадательное отношение к себе на самом деле позволяет наибольше «отдавать» в отношениях. Исследования показывают, что люди, практикующие самосострадание, часто более заботливы и готовы поддержать партнера в романтических отношениях, чаще идут на компромисс во время конфликтов и более великодушно и понимающе относятся к другим. 

\vspace{4ex}

\textbf{\textit{<<Самосострадание сделает меня ленивой. Я, наверное, буду просто отлынивать от работы, когда мне этого хочется, и целый день лежать в кровати и есть печенье!>>}}

\vspace{2ex}

Хотя многие люди боятся, что самосострадание означает потворство всем своим желаниям, на самом деле все с точностью до наоборот. \textbf{Самосострадание направляет нас в сторону наших долгосрочных интересов - например, здоровья и благополучия}~"--- а не краткосрочного удовольствия (так же, как заботливая мать не разрешает ребенку все время есть мороженое и дает ему овощи).  Исследования показывают, что люди, практикующие самосострадание, лучше заботятся о своем здоровье: едят здоровую пищу, занимаются физической активностью, реже пьют и более регулярно ходят ко врачу. 

\vspace{4ex}

\textbf{\textit{<<Если я буду относиться к себе с состраданием, я хоть убийство себе спущу с рук. Мне надо обходиться с собой строго, когда что-то идет не так, чтобы не причинить вреда другим людям>>.}}

\vspace{2ex}

Еще одно заблуждение~"--- это то, что самосострадание якобы служит, чтобы оправдать плохое поведение. На самом деле \textbf{самосострадание дает нам силу признать свои ошибки}, а не винить в них кого-то другого. Исследования показывают, что люди, практикующие самосострадание, чаще берут на себя ответственность за свои поступки и с б\'{о}льшей долей вероятности извинятся, если они кого-то обидели. 

\vspace{4ex}
\textbf{\textit{
<<Я никогда не достигну в жизни того, чего хочу, если хоть на минуту прекращу свою жесткую самокритику. Она мотивирует меня на успех. Самосострадание, может быть, кому-то и подходит, но у меня высокие требования к себе и амбициозные цели>>.}}

\vspace{2ex}
 
Самый частый страх~"--- это опасение, что самосострадание может подорвать мотивацию к достижениям. Большинство людей думают, что самокритика~"--- это эффективный мотиватор, но в реальности все не так. Самокритика убивает веру в себя и приводит к страху потерпеть неудачу. Если мы относимся к себе с состраданием, мы все равно мотивированы достичь наших целей~"--- не потому что с нами прямо сейчас что-то не так, а потому что мы заботимся о себе и хотим полностью реализовать свой потенциал (см. главу \ref{Self-Compassionate_Motivation} на стр. \pageref{Self-Compassionate_Motivation}). Исследования показывают, что у людей, практикующих самосострадание, \textbf{достаточно высокие личные стандарты}~"--- просто они не ругают себя, когда им что-то не удается. Это значит, что они меньше боятся ошибок и с б\'{о}льшей долей вероятности после неудачи будут пытаться дальше и доведут дело до конца. 

\vspace{4ex}

\begin{center}
	{\Large СВЕТ МОЙ, ЗЕРКАЛЬЦЕ, СКАЖИ...}	
\end{center}

\vspace{2ex}

Часто, рассказывая людям о самосострадание, мы получаем такую реакцию:

\textbf{\textit{<<Это же как Стюарт Смолли из передачи ,,Субботний вечер``, который любил смотреть в зеркало и говорить: ,,Я и хорош, и умен, и людям, черт побери, нравлюсь!`` Правда же?>> }}

\vspace{2ex}

Чтобы по-настоящему понять, что такое самосострадание, важно его отличать от близкой его родственницы~"--- самооценки. В западной культуре для высокой самооценки нужно выделяться из толпы~"--- другими словами, быть особенным и <<выше среднего>>. Проблема, конечно, заключается в том, что невозможно, чтобы \emph{абсолютно все} в одно и то же время были <<выше среднего>>. У нас может быть что-то, в чем мы особенно хороши, но всегда найдется кто-нибудь привлекательнее, успешнее и умнее нас, и это значит, что мы будем чувствовать себя неудачниками, сравнивая себя с теми, кто нас <<лучше>>. Желание быть <<выше среднего>> и сохранить такую неуловимую высокую самооценку может привести к очень некрасивому поведению. Почему подростки начинают  задирать других? Если я буду казаться «крутым парнем» по сравнению с мямлей-ботаником, на которого я нападаю, это мне повысит самооценку. Почему у нас столько предрассудков? Если я буду считать, что моя национальность, мой пол и мои политические взгляды лучше, чем все остальные, это мне повысит самооценку.
   
Но самосострадание и самооценка~"--- это не одно и то же. Хотя оба понятия тесно связаны с психологическим здоровьем, они сильно отличаются:

\begin{itemize}
	\item Высокая самооценка~"--- это положительная оценка нашей значимости. Самосострадание не является ни оценкой, ни суждением. Напротив, \textbf{самосострадание~"--- это отношение к самим себе с добротой, принятием (особенно когда мы терпим в чем-то неудачу) и признанием того, что мы постоянно меняемся}.
	\item Для высокой самооценки нужно чувствовать себя лучше других. Для самосострадание нужно принятие несовершенства каждого человека.
	\item Самооценка~"--- друг до первой беды~"--- она с нами, когда мы добиваемся успеха, но покидает нас именно тогда, когда мы больше всего в ней нуждаемся~"--- когда мы в чем-то терпим провал или выставляем себя на посмешище. Самосострадание~"--- надежный источник поддержки, который всегда с нами. Когда наша гордость разбивается вдребезги, нам больно, но мы можем быть добры к себе, \emph{потому что} нам больно. <<Боже мой, какое унижение. Мне так жаль, но все будет в порядке. Такие вещи случаются>>.
	\item По сравнению с самооценкой, самосострадание менее зависит от таких условий, как внешняя привлекательность или успех в работе, и дает более стабильное чувство собственной значимости. Оно также менее чем самооценка, вызывает сравнение себя с другими и нарциссизм.
\end{itemize}

\newpage

\Exercises{Как для вас работает самооценка?}

\begin{itemize}
	\itemWritingHand Как вы себя чувствуете, когда вам говорят, что вы середнячок в какой-то важной для вас области жизни (например, работа, родительские обязанности, дружба, личная жизнь)?
\end{itemize}

\setlength{\extrarowheight}{2mm}
\begin{tabularx}{\textwidth}{X}
	\\
	\arrayrulecolor{gray}\hline\\
	\hline\\
	\hline\\
	\hline\\
	\hline\\
	\hline\\
	\hline\\
	\hline\\	
\end{tabularx}
\setlength{\extrarowheight}{0mm}

\begin{itemize}
	\itemWritingHand Как вы себя чувствуете, когда кто-то лучше вас делает что-то, что для вас очень важно (например, успешнее поднимает продажи, печет более вкусное печенье для школьной вечеринки, лучше играет в баскетбол, лучше выглядит в купальнике)?
\end{itemize}

\setlength{\extrarowheight}{2mm}
\begin{tabularx}{\textwidth}{X}
	\\
	\arrayrulecolor{gray}\hline\\
	\hline\\
	\hline\\
	\hline\\
	\hline\\
	\hline\\
	\hline\\
	\hline\\
	
\end{tabularx}
\setlength{\extrarowheight}{0mm}

\begin{itemize}
	\itemWritingHand Как на вас влияет неудача в чем-то, что для вас важно (например, вас низко оценивают как учителя, ваш ребенок говорит, что вы ужасный отец, с вами прекращают общение после первого свидания?)
\end{itemize}

\setlength{\extrarowheight}{2mm}
\begin{tabularx}{\textwidth}{X}
	\\
	\arrayrulecolor{gray}\hline\\
	\hline\\
	\hline\\
	\hline\\
	\hline\\
	\hline\\
	\hline\\
	\hline\\
	\hline\\
	\hline\\
	\hline\\	
\end{tabularx}
\setlength{\extrarowheight}{0mm}

\Reflection{
	Наверняка вы обнаружили, что неприятно быть середнячком, что вам не нравится, когда кто-то лучше вас, и что в неудачах мало приятного. Такова человеческая природа. Но важно понимать, что все это~"--- крупные недостатки самооценки: она заставляет нас постоянно сравнивать себя с другими, и выходит, что наше ощущение собственной значимости прыгает вверх-вниз (в зависимости от нашего последнего успеха или неудачи), как шарик для пинг-понга. Когда мы замечаем, что наша нужда в высокой самооценке создает для нас проблемы, самое время попробовать относиться к себе по-новому~"--- с самосостраданием!
}
%%% end section %%%