% !TEX root = ../../Self-Compassion.tex

\chapter{Самосострадание в отношениях с другими} \label{Self-Compassion_in_Relationships}

Достаточно большая часть нашего страдания появляется в отношениях с другими людьми. Как написал однажды Сартр: <<Ад~"--- это другие>>\cite{93}. Но у нас есть для вас и хорошая новость: часто это страдание абсолютно ненужное и лишнее, и его можно предотвратить, построив любящие отношения с собой. 

Есть по крайней мере два вида боли в отношениях. Один из них~"--- это \textbf{боль} \emph{связи} (когда кто-то, кто нам важен, страдает~"--- смотри главу \ref{Self-Compassion_for_Caregivers} на стр.\:\pageref{Self-Compassion_for_Caregivers}). Другой~"--- \textbf{боль} \emph{разобщенности} (когда мы кого-то теряем или оказываемся отвергнутыми, при этом чувствуя боль, гнев и одиночество~"--- смотри главу \ref{Self-Compassion_and_Anger_in_Relationships} на стр.\:\pageref{Self-Compassion_and_Anger_in_Relationships}).

Наша способность к эмоциональному резонансу означает, что эмоции <<заразны>>\cite{94}. Это относится особенно к близким отношениям. Если, например, вы раздражены поведением своего молодого человека, но пытаетесь это скрыть, часто он все-таки заметит ваше раздражение. Он может сказать: <<Ты на меня сердишься?>> Даже если вы будете это отрицать, ваш молодой человек почувствует ваше раздражение: оно повлияет на его настроение, в результате чего его тон станет раздраженным. Вы, в свою очередь, это почувствуете, что только усилит ваше раздражение, а ваши реплики станут еще жестче, и так далее и тому подобное. Это происходит потому, что наши мозги передают друг другу эмоции вне зависимости от того, насколько осторожно мы выбираем выражения. 

В социальных взаимодействиях может возникнуть так называемая \emph{нисходящая спираль} негативных эмоций~"--- \textbf{когда у одного возникают негативные эмоции, они усиливают негативные эмоции другого человека, и так далее}\cite{95}. Это значит, что другие люди частично ответственны за наше эмоциональное состояние, но и мы сами тоже частично ответственны за их состояние. Здесь хорошая новость заключается в том, что эмоциональное <<заражение>> дает нам больше сил изменить общий эмоциональный тон наших отношений, чем мы думаем. Самосострадание может прервать нисходящую спираль и вместо нее начать восходящую. 

Сострадание~"--- положительная эмоция, которая, даже несмотря на то, что она возникает в присутствии страдания, активирует центры удовольствия мозга\cite{96}. Таким образом, очень полезный способ изменить направление негативного взаимодействия с кем-то~"--- отнестись с состраданием к боли, которую мы в настоящий момент испытываем. Положительное чувство сострадания, которое мы испытываем к себе, другие тоже почувствуют~"--- из нашего тона и почти незаметных изменений в выражении лица~"--- и это поможет прервать негативный цикл. Получается, что формирование в себе самосострадания~"--- это почти самое лучше, что мы можем сделать как для нас, так и для наших отношений с другими. 

Что неудивительно, исследования показали, что у людей, практикующих самосострадание, более счастливые и прочные романтические отношения\cite{97}. В одном из таких исследований, например, партнеры людей с более высоким уровнем самосострадания описывали их более принимающими и непредвзятыми, чем партнеры людей, которым не хватает самосострадания. Вместо того, чтобы пытаться изменить своих партнеров, люди, практикующие самосострадание, как правило, уважали их мнения и пытались понять их точку зрения. Партнеры также называли их более заботливыми, любящими, близкими и готовыми обсудить любые проблемы в отношениях, чем партнеры людей, не практикующих самосострадание. В то же время партнеры самосострадательных людей говорили, что они дают им больше свободы и независимости в отношениях, поддерживали их в принятии собственных решений и следовании собственным интересам. А партнеры людей, не практикующих самосострадание, наоборот, говорили, что они чаще их критикуют и пытаются контролировать. Также их называли более эгоцентричными, всегда желающими, чтобы все было так, как им нравится. 

\begin{quotation}
	\textit{
		Стив встретил Шейлу в университете и после 15 лет брака он все еще нежно ее любил. Тем не менее, как бы ему ни было неприятно себе в этом признаваться, она начинала сводить его с ума~"--- в плохом смысле. Шейла была ужасно неуверена в себе и постоянно нуждалась в том, чтобы Стив в очередной раз заверил ее в своей любви и привязанности. Как будто бы 15 лет жизни в браке недостаточно! Если он ей не говорил <<я тебя люблю>> каждый день, она начинала волноваться, а после нескольких дней серьезно обижалась. Стив чувствовал себя, как будто эта ее нужда его контролировала, а еще ему совсем не нравилось, что она не уважала его собственную потребность выражать свои мысли и чувства свободно. Их отношения начали страдать.
	}
\end{quotation}

\textbf{Чтобы иметь близкие и доверительные отношения с другими, нужно сначала выстроить такие с собой.} Поддерживая себя в трудные времена, мы приобретаем эмоциональные ресурсы, которые нужны для заботы о партнерах. Когда мы удовлетворяем собственную потребность в любви и принятии, мы можем  меньше запрашивать от своих партнеров, что позволяет им быть собой. \textbf{Самосострадание далеко от эгоизма. Оно дает нам возможность строить и поддерживать счастливые и здоровые отношения.} 

\begin{quotation}
	\textit{
		Со временем Шейла увидела, как ее постоянная нужда в заверении от Стива отталкивала его от нее. Она поняла, что стала эмоциональной черной дырой и что Стив никогда бы не смог полностью удовлетворить потребности, созданные ее неуверенностью к себе. Любое количество любви было бы <<недостаточным>>. Поэтому Шейла начала практиковать ведение дневника по вечерам, чтобы дать себе любовь и привязанность, которые она так хотела получить.  Она писала себе нежные слова вроде тех, которые она надеялась услышать от Стива, например: <<Я люблю тебя, дорогая. Я никогда тебя не покину>>. Потом утром, только проснувшись, она сразу же читала, что написала вечером, и пыталась это в себя вобрать. Она начала сама заверять себя в том, в чем она раньше ждала заверения от Стива, и отпустила мужа со своего крючка. Конечно, как ей пришлось признаться себе, это было не так приятно, но зато ей нравилась, что она больше так не зависела от Стива. Когда давления стало меньше, тот естественным образом стал более экспрессивным, и они с женой опять сблизились. Чем более уверенной она себя чувствовала в собственном принятии себя, тем легче ей становилось принять его любовь такой, какая она есть, а не такой, какой она бы хотела ее видеть. По иронии судьбы, удовлетворяя свои собственные потребности, она стала менее зацикленной на себе и начала ощущать новое и приятное чувство независимости.
	}
\end{quotation}

\newpage
\InformalPractices{Перерыв на самосострадание во время конфликтов в отношениях} \label{IP:Self-Compassion_Break_in_Relationship_Conflicts}

\begin{itemize}
	\item В следующий раз, когда обнаружите себя в конфликте с кем-то, попробуйте использовать <<Перерыв на самосострадание>> (смотри главу \ref{The_Physiology_of_Self-Criticism_and_Self-Compassion} на стр.\:\pageref{IP:Self-Compassion_Break}). Можете на минуту выйти или, если такой возможности нет, проделать практику про себя: <<Это момент страдания>>, <<Страдание~"--- это часть любых отношений>>, <<Пусть я буду добра к себе>>. Еще помогают поддерживающие прикосновения. Если вы одни, можете положить руку на сердце, но если с вами еще кто-то, стоит попробовать применить какую-то не очень заметную форму прикосновения, например, держать себя за руку.
	
	\item До продолжения общения с другим человеком попробуйте попрактиковать <<Давать и получать сострадание>> (смотри главу \ref{Being_There_for_Others_without_Losing_Ourselves} на стр.\:\pageref{M:Giving_and_Receiving_Compassion}), чтобы ваше заботливое отношение сохранилось. Вдыхайте за себя, признавая боль, которую вы сейчас чувствуете, а потом выдохните за другого человека.  Валидируйте свою собственную боль целиком и дайте себе то, что вам нужно, а также уважайте трудности другого человека.
	
	\item Заметьте, как, возможно, меняется душевное состояние другого человека по мере того, как меняется ваше собственное. 
\end{itemize}

\Reflection{
	Попробовав несколько раз сделать перерыв на самосострадание в своих отношениях, заметили ли вы какие-то перемены в ваших взаимодействиях?
	
	Эта практика может иметь особенно выраженный эффект, если второй человек в отношениях знаком с концепцией самосострадания и тоже его практикует. В этом случае, когда атмосфера накаляется, кто-то из вас должен выкрикнуть: <<Перерыв на самосострадание!>>, после чего вы оба можете взять паузу и одарить себя состраданием за испытываемую боль, а потом начать сначала.
}

\newpage
\Exercises{Удовлетворение наших эмоциональных потребностей} \label{Ex:Fulfilling_Our_Emotional_Needs}

Часто мы вносим в отношения лишнее давление, ожидая, что наш партнер магическим образом прочитает наши мысли, вычислит наши эмоциональные потребности и сразу же их все удовлетворит. Например, если вы возмущаетесь, что во время выполнения одного проекта по работе ваш партнер не догадался, что вас нужно обнять и подбодрить, но со следующим проектом вам оказалось нужно личное пространство и время, то ваш партнер будет страдать под бременем нечеловеческих ожиданий. Вы тоже будете страдать, потому что ваши потребности останутся неудовлетворенными. Вместо того, чтобы полагаться на то, что ваш партнер даст вам именно то, что вам нужно, вы можете попробовать сами напрямую удовлетворять свои эмоциональные потребности. Конечно, сами мы все свои потребности удовлетворить не можем и в чем-то придется полагаться на других людей, но мы не настолько от них зависим, как нам иногда кажется.
\begin{itemize}
	\itemWritingHand Возьмите лист бумаги и запишите все, что вас не удовлетворяет в ваших текущих отношениях. Например, может быть, вам кажется, что вы не получаете от своего партнера достаточно внимания, уважения, поддержки или валидации. Вместо того, чтобы сфокусироваться на мелочах или чересчур конкретных вещах (например, вы получаете от партнера меньше СМС--сообщений, чем вам бы хотелось), попробуйте определить именно ту потребность, которая у вас остается неудовлетворенной~"--- например, нужда в том, чтобы вас ценили, о вас заботились и т.\,д. 
\end{itemize}

\setlength{\extrarowheight}{2mm}
\begin{tabularx}{\textwidth}{X}
	\\
	\arrayrulecolor{gray}\hline\\
	\hline\\
	\hline\\
	\hline\\
	\hline\\
	\hline\\	
	\hline\\
	\hline\\
	\hline\\
	\hline\\
	\hline\\
\end{tabularx}
\setlength{\extrarowheight}{0mm}
\begin{itemize}
	\itemWritingHand Теперь запишите несколько идей, как вы можете попытаться сами удовлетворить свои потребности. Например, если вам нужно знать, что о вас заботятся, можете ли вы купить себе цветы? Если вам нужно больше нежных прикосновений, можете ли вы ходить раз в неделю на массаж или просто держать себя за руку? Можете ли вы показать себе, что вас любят, при помощи добрых слов? Сначала это может казаться глупым, но если удовлетворение собственных эмоциональных потребностей войдет у вас привычку, вы будете меньше зависеть от партнера в этом плане и у вас будет больше ресурсов, чтобы давать что-то другим. 
\end{itemize}

\setlength{\extrarowheight}{2mm}
\begin{tabularx}{\textwidth}{X}
	\\
	\arrayrulecolor{gray}\hline\\
	\hline\\
	\hline\\
	\hline\\
	\hline\\
	\hline\\	
	\hline\\
	\hline\\
	\hline\\
	\hline\\
	\hline\\
	\hline\\
	\hline\\
\end{tabularx}
\setlength{\extrarowheight}{0mm}

\Reflection{
	Для многих людей становится открытием, что они могут и сами удовлетворить некоторые из своих эмоциональных потребностей вместо того, чтобы полагаться полностью на кого-то еще. Тем не менее, некоторые чувствуют грусть, горе или гнев по поводу того, что их партнеры не могут нормально удовлетворить их эмоциональные потребности. Запомните, что то, что вы удовлетворяете часть своих эмоциональных потребностей, не означает, что ваш партнер не должен тоже стараться это сделать, особенно если вы это ясно проговорили. В здоровых отношениях обе стороны и дают, и получают, но этот обмен получается легче, когда каждый из партнеров эмоционально удовлетворены любовью, поддержкой и заботой, которые они дают сами себе.
}

\Meditation{Сострадающий друг} \label{M:Compassionate_Friend}

Эта медитация визуализации поможет вам установить связь с той частью себя, которая проявляет к вам сострадание, мысленно представив себе некий ее образ и завязав с этим образом беседу. Установление более близких отношений с сострадающей частью вас~"--- это важный ресурс для улучшения ваших отношений с другими людьми. Эта медитация, адаптированная из методики Пола Гилберта, будет особенно полезна людям, которым тяжело развить в себе самосострадание \cite{98}. У некоторых людей хорошо получается визуализировать, у других хуже. Выполняйте эту медитацию достаточно расслабленно, позволяя ей самой развиться и разрешая образам приходить и уходить. Если у вас не возникает никаких образов, в этом тоже нет ничего страшного: вы можете просто сосредоточиться на присутствующих в данный момент чувствах.

\begin{itemize}
	\item Найдите удобное положение, сидя или лежа. Сделайте несколько глубоких вдохов и выдохов, чтобы укорениться вниманием в своем теле. Положите одну или две руки на сердце в напоминание себе проявить к себе любящее внимание.
\end{itemize}

\vspace{3ex}

\noindent{\large \textbf{Безопасное место}}
\begin{itemize}
	\item Представьте себя в месте, где вам комфортно и безопасно~"--- это может быть уютная комната с горячим камином, спокойный пляж с теплым солнцем и прохладным ветерком или лесная поляна. Это также может быть полностью воображаемое место~"--- например, на облаках... в любом месте, где вам спокойно и безопасно. Остановитесь на чувстве комфорта в этом месте и насладитесь им.
\end{itemize}

\vspace{3ex}

\noindent{\large \textbf{Сострадающий друг}}
\begin{itemize}
	\item Скоро к вам придет гость, присутствие которого будет теплым и заботливым~"--- это ваш сострадающий друг, воображаемая личность, которая олицетворяет мудрость, силу и безусловную любовь.
	
	\item Это может быть духовная личность или мудрый, сострадающий вам учитель; ваш <<друг>> может воплощать в себе качества кого-то, кого вы знали раньше~"--- например, любящей бабушки~"--- или быть полностью продуктом вашего воображения. У этого существа может даже не быть определенной формы, возможно, вы ощущаете его по чьему-то присутствию или яркому свету.
	
	\item Для вашего сострадающего друга вы очень важны, и ему хочется, чтобы вы были счастливы и свободны от ненужных трудностей.
	
	\item Разрешите образу друга появиться у вас в воображении.
\end{itemize}

\vspace{3ex}

\noindent{\large \textbf{Прибытие}}
\begin{itemize}
	\item У вас есть выбор: выйти из своего безопасного места, чтобы встретиться с другом, или пригласить его к себе.
	
	\item Займите по отношению к другу удобное положение~"--- то, какое вам кажется подходящим в настоящий момент. Потом прочувствуйте, как это~"--- быть в компании этого существа. От вас не требуется ничего, кроме восприятия настоящего момента.
	
	\item Попробуйте разрешить себе получить всю безусловную любовь и сострадание к вам этого существа и впитать все это в себя. Если вы не можете впустить эти чувства, ничего страшного~"--- ваш гость все равно их чувствует.
\end{itemize}

\vspace{3ex}

\noindent{\large \textbf{Встреча}}
\begin{itemize}
	\item Ваш сострадающий друг мудрый и всезнающий, поэтому он знает, на каком этапе своей жизни вы находитесь. Ваш друг может захотеть что-то вам сказать, именно то, что вам нужно прямо сейчас услышать. Остановитесь на минуту и внимательно выслушайте то, что ваш сострадающий друг пришел сказать.
	
	\item Если не получается услышать никакие слова, тоже ничего страшного~"--- просто наслаждайтесь хорошей компанией. Это само по себе большая удача.
	
	\item А может быть, вы хотите что-то сказать своему сострадающему другу. Он выслушает вас внимательно и поймет вас на сто процентов. Есть ли у вас что-нибудь, чем вы хотели бы поделиться?
	
	\item Вашему другу может захотеться вам что-то подарить на прощание~"--- какой-то материальный объект. Этот объект может просто появиться у вас в руках~"--- или вы можете сами протянуть руки и его получить. Это что-то, что обладает для вас особенным смыслом.
	
	\item Теперь пару минут просто наслаждайтесь присутствием своего друга. По мере того, как у вас появляются приятные эмоции, разрешите себе понять, что на самом деле этот друг~"--- часть вас. Все эти сострадательные чувства, образы и слова проистекают из вашей личной внутренней мудрости и вашего же сострадания.
\end{itemize}

\vspace{3ex}

\noindent{\large \textbf{Возвращение}}
\begin{itemize}
	\item Наконец, когда будете готовы, дайте образам постепенно раствориться в вашем сознании, при этом помня, что сострадание и мудрость всегда находятся внутри вас, особенно тогда, когда они вам особенно нужны. Вы можете в любой момент навестить своего <<сострадающего друга>>.
	
	\item Теперь мысленно вернитесь в свое тело, растягивая удовольствие от того, что только что случилось, поразмышляйте о словах, которые вы услышали, или о предмете, который вам подарили.
	
	\item И напоследок отпустите медитацию и разрешите себе чувствовать то, что вы сейчас чувствуете, и быть именно таким, какой вы есть.
	
	\item Мягко откройте глаза.
\end{itemize}

\newpage
\Reflection{
	Получилось ли у вас визуализировать безопасное место и почувствовать его комфорт? Пришел ли вам в голову образ сострадающего друга или просто чье-то присутствие? Услышали ли вы от вашего друга  что-то очень значимое, что вам нужно было сейчас услышать? Каково было иметь возможность разговаривать с этим существом? Получили ли вы что-то с особым смыслом?
	
	Было ли в этой медитации что-то трудное? Что вы почувствовали, когда узнали, что этот друг на самом деле часть вас и что его сострадание и мудрость всегда для вас доступны?
	 
	Для людей~"--- <<визуалов>> эта медитация может быть особенно эффективной, особенно для того, чтобы услышать внутренний голос сострадания и взяться за практические, повседневные проблемы.  
	
	Иногда сострадающий друг принимает форму того, кто уже умер~"--- например, родители, бабушка или дедушка~"--- и от этого может возникнуть чувство горя. Если это горе мешает вам почувствовать сострадание со стороны этого человека, хорошим решением может быть переключение на полностью воображаемое существо, которое олицетворяет те же качества, или вообще неясное сострадающее присутствие. Но если горе вами не поглотило вас полностью, осознание того, что тот близкий человек, которого вы потеряли, все еще живет внутри вас в форме глубиной мудрости и сострадания может быть настоящим сокровищем.
}