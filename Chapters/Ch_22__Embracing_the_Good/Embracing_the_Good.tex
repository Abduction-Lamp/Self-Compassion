% !TEX root = ../../Self-Compassion.tex

\chapter{Принимать все хорошее} \label{Embracing_the_Good}

Одно из самых важных достоинств самосострадания~"--- это то, что оно не только помогает справиться с негативными эмоциями~"--- оно еще и \textbf{генерирует положительные эмоции}. Когда мы принимаем себя и свой опыт с любящим, общечеловеческим присутствием, мы потом чувствуем себя хорошо. Это ощущение не <<приторное>> и не пытается сопротивляться или избегать страдания. 

Скорее, самосострадание позволяет нам испытать всю гамму эмоций и событий~"--- от горьких до сладких. Но обычно мы больше фокусируемся на том, что в наших жизнях есть плохого, чем хорошего. Например, когда вы получаете отчет о своей работе за год, что вам больше запоминается~"--- похвала или критика? Или, допустим, если вы съездили в торговый центр купить одежду и встретили пятерых вежливых консультантов и одного грубого, какой из них вам больше всего запомнится?

Это называется психологическим термином \textbf{предвзятость негативного опыта}. Рик Хансон говорит, что мозг работает <<как застежка с негативным опытом и как антипригарная сковорода с позитивным>>. Если говорить об эволюционных причинах этого явления, во времена наших далеких предков у  людей, которые каждый вечер о чем-то волновались~"--- например, где стая гиен была вчера и где она может быть завтра~"--- были выше шансы на выживание, чем у тех, кто беззаботно отдыхал. Это эволюционное преимущество, когда нам угрожает физическая угроза. Но в связи с тем, что большая часть того, что в наши дни нам угрожает, угрожает нашему самоощущению, избавиться от предвзятости негативного опыта значит проявить самосострадание, так как она искажает реальность.  

Нужно намеренно \emph{замечать} и впитывать в себя положительный опыт, чтобы развить более реалистичную и уравновешенную осознанность, которая не перекошена в сторону негатива. Для этого нужны тренировки, как и для развития самосострадания и осознанности. Более того, тренировка сострадания включает в себя открытие боли, нам может понадобиться энергия, полученная в процессе концентрации на хорошем.  

У концентрации на хорошем есть и другие важные плюсы. Барбара Фредриксон, разработчица теории <<расширения и укрепления>>, считает, что эволюционная роль позитивных эмоций~"--- расширить поле нашего внимания. Другими словами, когда люди чувствуют себя довольными и в безопасности, они становятся любопытными и начинают исследовать окружающую их среду, подмечая все, что может нам дать питание, отдых или убежищем. Это позволяет нам воспользоваться возможностями, которые мы иначе бы не заметили.  

\begin{center}
	\textbf{
		Когда одна дверь счастья закрывается, открывается другая; но мы часто не замечаем ее, уставившись взглядом в закрытую дверь.
	}
	
	\textbf{"--- Хелен Келлер}
\end{center}


Недавно в психологии появилось направление, сосредоточенное на поиске наиболее эффективных способов помочь людям испытать положительные эмоции, и две особенно полезные практики, которые были найдены~"--- \emph{смакование} и \emph{благодарность}.

\vspace{3ex}

\noindent{\large \textbf{Смакование}}

\vspace{1ex}

Смаковать~"--- значит замечать и ценить положительные стороны жизни~"--- вбирать их в себя, позволять им задержаться, а потом отпускать. Это больше, чем просто \emph{удовольствие}~"--- смакование включает в себя осознанность к ощущению удовольствия. Другими словами, осознанность к тому, что что-то хорошее происходит, когда оно происходит. Учитывая естественную тенденцию нашего внимания пропускать хорошее и зацикливаться на плохом, нам нужно вложить определенное количество усилий, чтобы концентрироваться на том, что приносит нам удовольствие.

К счастью, смакование~"--- достаточно простая практика: требуется лишь замечать чуть терпкий и сочный вкус свежего яблока, ощущение легкого прохладного ветерка на щеке, тёплую улыбку коллеги, то, как нежно держит в своей руке вашу ваш партнёр. Исследования показывают, что, если даже просто выделять времени на то, чтобы заметить и прочувствовать эти положительные моменты, это может сделать нас гораздо более счастливыми.


\vspace{3ex}

\noindent{\large \textbf{Благодарность}}

\vspace{1ex}

Благодарность как практика~"--- это распознавание, признание и благодарность за все хорошее в жизни. Если мы будем постоянно думать о том, чего у нас \emph{нет}, наше душевное состояние так и останется подавленным. Но если сконцентрироваться на том, что у нас всё-таки \emph{есть}, и быть за это благодарным, все кардинально меняется.

Смакование делает упор в основном на чувствах, а благодарность~"--- это практика мудрости. Мудрость здесь~"--- это понимание взаимозависимости всего происходящего. Пересечение всех факторов, необходимых, чтобы случилось даже незначительное событие, захватывает дух и может пробудить в нас благоговение и уважение. В понятие благодарности так же входит признание мириадов людей и событий, которые вносят свой вклад во все хорошее в нашей жизни. Как заметил один раз один из участников программы ОСС, <<благодарность~"--- это фактура мудрости>>.

\textbf{Смакование~"--- это осознанность к положительным ощущениям и событиям.} Благодарность можно направить на большие и важные вещи в жизни~"--- например, на наше здоровье и нашу семью~"--- но ее эффект будет ещё сильнее, когда она направлена на мелочи, например, когда автобус прибывает вовремя или в жаркий летний день работает кондиционер. Исследования показывают, что между благодарностью и счастьем существует сильная связь. Как писал философ Марк Непо, <<ключ к радости~"--- это неприхотливость>>, т.е. для счастья нужно радоваться мелочам. Учитель медитации Джеймс Бараз рассказывает в своей книге <<Пробуждение радости>> восхитительную историю о силе благодарности, которую мы с его разрешения адаптировали.

\begin{quotation}
	\textit{
		Один раз я навещал мать, которой было на тот момент 89 лет, и принёс с собой журнал, в котором была статья о положительных эффектах благодарности. За ужином я рассказал ей кое-что из неё. Она была впечатлена, но призналась, что всю жизнь привыкла видеть стакан наполовину пустым. <<Я знаю, что мне очень повезло в жизни и что у меня так много вещей, за которые я могу быть благодарна, но меня выводят из себя даже мелочи>>. Ещё она сказала, что хотела бы отучиться от этой привычки, но сомневалась, что это возможно. <<Я просто привыкла видеть плохое>>, "--*изрекла она в заключение.
	}
	
	\textit{
		<<Мам, знаешь, ключ к благодарности содержится в нашем восприятии ситуации, "--*начал я. "--*Например, представь себе, что ни с того ни с того у тебя в телевизоре начались помехи>>.
	}
	
	\textit{<<Знакомая история>>, "--*согласилась она с понимающей улыбкой.}
	
	\textit{
		<<Описать эту ситуацию можно по-разному. Например, можно сказать: <<Как же меня это раздражает, кричать хочется!>>. А можно "--*<<Как же это меня раздражает... но моя жизнь в целом очень счастливая>>. <<Она согласилась, что разница большая.
	}
	
	\textit{<<Но я не думаю, что могу каждый раз об этом вспоминать>>, "--*вздохнула она.}
	
	\textit{
		Поэтому мы вместе придумали игру, чтобы ей напоминать. Каждый раз, когда она на что-то жаловалась, я просто говорил <<но...>>, и она продолжала: <<моя жизнь в целом очень счастливая>>. Я ликовал, что она согласилась это попробовать. Хоть это все и началось как просто веселая игра, через какое-то время проявились настоящие эффекты. По мере того как наши недели наполнялись благодарностью, ее настроение улучшалось. К моей радости и удивлению, моя мать продолжила выполнять эту практику, и радикальные изменения были налицо.
}
\end{quotation}

\InformalPractices{Прогулка <<Почувствуй и Смакуй>>} \label{IP:Sense_and_Savor_Walk}

Эта практика смакования особенно улучшает настроение, когда выполняется в окружении красивой природы, например, в саду или в лесу, но ее можно практиковать везде, где вам комфортно.

\begin{itemize}
	\item Выделите 15 минут на то, чтобы побродить на открытом воздухе. Цель прогулки~"--- заметить и насладиться всеми положительными ощущениями или красивыми объектами медленно, один за другим, используя все органы чувств~"--- зрения, обоняния, слуха, осязание... а, может быть, даже и вкуса.
	
	\item Цель не в том, чтобы <<постараться>> получить удовольствие или сделать так, чтобы что-то произошло. Просто позвольте себе заметить, погрузиться, задержаться и потом отпустить то, что приносит вам удовольствие~"--- это может быть все, что угодно.
	
	\item Сколько всего красивого, привлекательного, вдохновляющего вы замечаете, когда гуляете? Наслаждаетесь ли вы запахом еловых шишек, тёплым солнцем, красивым листком, формой камня, улыбающимся лицом, пением птиц, ощущением земли под ногами?
	
	\item Когда найдёте что-то приятное, погрузитесь в это. Растяните удовольствие. Понюхайте свежескошенную траву или попробуйте на ощупь палку, если хотите. Полностью отдайтесь происходящему и вашим ощущениям, как будто бы это единственное, что в мире есть.
	
	\item Когда вы потеряете интерес и вам захочется чего-то нового, отпустите тот объект и подождите, пока не найдёте что-то ещё, что вас привлекает. Представьте себе, что вы шмель, перелетающий между цветами. Когда вы насытились одним, летите к другому.
	
	\item Не спешите, двигайтесь медленно и посмотрите, что произойдёт.
\end{itemize}

\newpage
\Reflection{
	Что вы чувствовали, избирательно обращая внимание на позитивные ощущения и приятные окружающие вас объекты? Заметили ли вы что-то, что обычно бы прошло мимо вас? Получилось ли у вас задержаться на красоте и удовольствию и вобрать их в себя?
	
	Как вы чувствуете себя сейчас по сравнению с тем, как вы себя чувствовали до практики?
	
	Большинство людей обнаруживают, что, дав себе разрешение впитать в себя такой положительный опыт, они чувствуют себя счастливее. Это упражнение может показать нам, \textbf{как наш внутренний диалог мешает наслаждаться приятными вещами}. Когда мы вновь концентрируемся на том, что чувствуем напрямую, наша чувствительность повышается: цвета становятся ярче, звуки~"--- яснее, запахи~"--- более ароматными и так далее. Как писала Эмили Дикинсон, <<жизнь сама по себе так удивительна, что оставляет мало места для других занятий>>.
}

\newpage
\InformalPractices{Смакование еды} \label{IP:Savoring_Food}

Смакование еды~"--- это осознанный процесс приема пищи с наслаждением ее вкусом.

\begin{itemize}
	\item Выберите что-нибудь, чем вам хотелось бы перекусить или съесть на основной приём пиши.
	
	\item Сначала оцените внешний вид пищи. Потом насладитесь запахом и текстурой еды.
	
	\item Подумайте о многих парах рук, благодаря которым эта еда окажется у вас во рту~"--- о руках фермера, водителя грузовика, продавца...
	
	\item Теперь очень медленно начните есть.  Замечайте, как у вас текут слюни до еды, как вы подносите пищу ко рту, как вы начинаете жевать, как у вас во рту появляется вкус, как вы начинаете глотать...
	
	\item Продолжайте так есть, дав себе полное разрешение насладиться каждым моментом приема пищи, как будто бы он первый и последний в вашей жизни. 
	
	\item Когда закончите, обратите внимание на послевкусие~"--- на то, как вкусы задерживаются у вас во рту.
\end{itemize}

\Reflection{
	Поменялся ли вкус еды, когда вы разрешили себе не спешить и насладиться ей? Как вы себя чувствовали во время практики?
	
	Смакование еды обычно приносит моментальное удовлетворение и хорошее самочувствие. Как ни иронично, когда мы едим неосознанно, мы обычно не наслаждаемся едой и часто переедаем. Исследования показали, что ещё один плюс осознанного приема пищи~"--- это то, что он помогает поддерживать вес и остановиться, когда вы наелись.
}

\newpage
\Exercises{Благодарность за большие и маленькие вещи} \label{Ex:Gratitude_for_the_Big_and_Small_Things}


Запишите пять вещей в вашей жизни, которые очень важны для вас и за которые вы благодарны. Это может быть ваше здоровье, дети, карьера, друзья.

\vspace{1ex}

\noindent
\setlength{\extrarowheight}{3mm}
\begin{tabular*}{\textwidth}{rp{13.5cm}}
	\textbf{1.} & \\ \cline{2-2}
	\textbf{2.} & \\ \cline{2-2}
	\textbf{3.} & \\ \cline{2-2}
	\textbf{4.} & \\ \cline{2-2}
	\textbf{5.} & \\ \cline{2-2}
\end{tabular*}
\setlength{\extrarowheight}{0mm}

\vspace{4ex}

Теперь запишите пять небольших и не очень важных вещей в вашей жизни, на которые вы обычно не обращаете внимания, но за которые вы благодарны. Это могут быть пуговицы, насос для велосипеда, тёплая вода, искренняя улыбка, очки для чтения.

\vspace{1ex}

\noindent
\setlength{\extrarowheight}{3mm}
\begin{tabular*}{\textwidth}{rp{13.5cm}}
	\textbf{1.} & \\ \cline{2-2}
	\textbf{2.} & \\ \cline{2-2}
	\textbf{3.} & \\ \cline{2-2}
	\textbf{4.} & \\ \cline{2-2}
	\textbf{5.} & \\ \cline{2-2}
\end{tabular*}
\setlength{\extrarowheight}{0mm}

\Reflection{
	Появилось ли что-то в вашем списке, что вас удивило? За что было легче почувствовать благодарность: за важные вещи или мелочи? Как вы чувствуете себя сейчас сравнительно с тем, как чувствовали себя до практики?
	
	Вы можете выполнять эту практику, когда просыпаетесь утром до подъема или вечером после выключения света и до отхода ко сну.
	
	Попробуйте использовать пальцы на руках~"--- пять пальцев для больших вещей, а остальные пять~"--- для маленьких, за которые вы благодарны. Это занимает всего лишь несколько минут, но исследования показывают, что <<подсчёт своих радостей>> может иметь значительное положительное влияние на эмоциональное здоровье.
}


