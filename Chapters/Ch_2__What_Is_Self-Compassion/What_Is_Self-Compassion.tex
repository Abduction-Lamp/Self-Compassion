% !TEX root = ../../Self-Compassion.tex

\section{Что такое самосострадание?}

Понятие самосострадания включает в себя отношение к себе так, как вы бы отнеслись к другу, которому сейчас тяжело --- даже если этот друг сам в этом виноват,  чувствует себя из-за этого неполноценным или у него просто нелегкая ситуация в жизни. В западной культуре очень большая важность придается доброте по отношению к друзьям, членам семьи, соседям, у которых какие-то проблемы. Когда же речь заходит о нас самих, ситуация кардинально меняется. Самосострадание --- это практика, посредством которой мы учимся быть хорошим другом для самих себя в те моменты, когда это нам больше всего необходимо --- стать собственным союзником вместо собственного врага. Но, как правило, к себе мы относимся хуже, чем к своим друзьям.

Золотое правило морали гласит: <<Поступай с другими так, как ты хотел бы, чтобы поступили с тобой>>. Но вряд ли вы бы сочли разумным поступать с другими так же, как поступаете с самим собой! Представьте себе, что ваша лучшая подруга звонит вам после того, как ее бросил молодой человек, и ваш разговор проходит вот так:\\
"--*Привет, --- говорите вы, поднимая трубку, --- как дела?\\
"--*Ужасно, --- отвечает она, глотая слезы. --- Помнишь Майкла, с которым я встречалась? Он был первым мужчиной, который меня зацепил с тех пор, как я развелась с мужем. Вчера вечером он мне сказал, что я слишком много на него давлю и что он хочет остаться друзьями. Я просто опустошена.\\
Вы вздыхаете и говорите:\\
"--*Ну, честно говоря, это все наверняка, потому что ты страшная, скучная и для него старовата, не говоря уже о том, что ты слишком приставучая. И у тебя как минимум 10 кг лишнего веса. Я бы на твоем месте просто сдалась, потому что у тебя уже нет никакой надежды найти кого-то, кто тебя полюбит --- да и ты этого, прямо скажем, не заслуживаешь.

Стали бы вы разговаривать так с кем-то, кто вам дорог? Конечно нет. Но, как ни парадоксально, именно так (или даже хуже!) мы разговариваем сами с собой в подобных ситуациях. Через самосострадание мы учимся разговаривать с собой, как с хорошим другом. <<Мне так жаль! Ты в порядке? Я представляю себе, как ты расстроена. Но помни, пожалуйста, что я очень тебя ценю и всегда готова тебе помочь. Могу я сейчас что-нибудь для тебя сделать?>>

Если упростить, то самосострадание --- это отношение к себе как к хорошему другу, но более полное определение включает три основных компонента, которыми мы пользуемся, когда нам больно: доброта к себе, человеческая общность и осознанность.

\paragraph{Доброта к себе.} 
Когда мы совершаем ошибку или терпим в чем-то неудачу, мы наверняка будем себя винить, а не поддерживать. Подумайте обо всех знакомых вам щедрых и заботливых людях, которые постоянно ругают себя (это можете быть и вы сами). 

Если мы добры к себе, мы поступаем в точности наоборот: мы так же заботливо относимся к себе, как и к другим. Вместо того, чтобы жестко критиковать себя за все ошибки или несовершенства, мы себя поддерживаем, ободряем и пытаемся защитить самих себя от вреда. Вместо того чтобы ругать себя за нашу несостоятельность, мы дарим самим себе тепло и безусловное принятие. Когда внешние обстоятельства кажутся невыносимо тяжелыми, мы активно успокаиваем и утешаем себя. 

\begin{quote}
	\textit{Тереза была очень довольна. <<У меня получилось! Я просто не могу поверить! На прошлой неделе я была на корпоративе и случайно ляпнула  что-то неуместное в разговоре с коллегой. Вместо того, чтобы начать себя обзывать, как я обычно делаю, я попыталась быть доброй и понимающей. Я сказала себе: ,,Ну случилось и случилось, это не конец света. Я не имела в виду ничего плохого, даже если в итоге вышла не очень красивая ситуация``>>.}
\end{quote}

\paragraph{Человеческая общность.}
 Чувство взаимосвязанности с другими людьми лежит в основе самосострадания. Это признание того, что никто не безупречен, мы все ошибаемся, терпим в чем-то неудачи и встречаемся с жизненными трудностями. Самосострадание --- это уважение того неизбежного факта, что все без исключения в какие-то моменты жизни испытывают страдание. Это может показаться очевидным, но очень легко забывается. Мы ошибочно верим, что все <<должно>> быть хорошо и что, когда что-то не получается, это значит, что что-то не так. 

Конечно, очень вероятно (и, на самом деле, неизбежно), что мы будем совершать ошибки и встречаться с трудностями достаточно часто. Это абсолютно нормально и естественно. Но большинство из нас не думает об этих вещах рационально. Вместо этого мы не просто страдаем, а еще и чувствуем себя одинокими в своем страдании. Когда мы вспоминаем, что боль --- это часть жизни каждого человека, каждый момент страдания становится моментом общности с другими. Та боль, которую я испытываю в трудные моменты --- это та же боль, которую испытываете в трудные моменты вы. Обстоятельства бывают разные, как и уровень боли, но само по себе страдание --- это часть жизни каждого человека, и это нас всех объединяет. 

\begin{quote}
	\textit{Тереза продолжила: <<Я вспомнила, что у всех иногда случаются оговорки. Конечно, я не всегда говорю именно то, что нужно сказать. То, что такие вещи случаются --- очень естественн>>.} 
\end{quote}

\begin{figure}[h]
	\begin{center}
		% !TEX root = What_Is_Self-Compassion.tex


%%% TikZ %%%

%%% The Three Elements of Self-Compassion %%%

\begin{tikzpicture}

	\begin{scope}[blend group = soft light]
		\fill[red!30!white]   ( 90:2) circle (2.1);
		\fill[green!30!white] (210:2) circle (2.1);
		\fill[blue!30!white]  (330:2) circle (2.1);
	\end{scope}

	\node at (90:2.1)	{Доброта к себе};
	\node at (210:2.1)  {Человеческая};
	\node at (222:2.4)	{общность};
	\node at (330:2.1)  {Осознанность};

\end{tikzpicture}
		\caption{Три элемента самосострадания}
	\end{center}
\end{figure}

\paragraph{Осознанность.} Осознанность включает в себя ясное и уравновешенное внимание к тому, что происходит в каждый конкретный момент. Быть осознанным --- значит быть открытым к реальности настоящего момента, позволяя всем мыслям, эмоциям и ощущениям войти в наше сознание без сопротивления или избегания (об осознанности речь будет идти более подробно в 6 главе). 
Почему осознанность --- необходимый компонент самосострадания? Потому что мы должны уметь повернуться лицом к нашему страданию и признать, что мы страдаем, чтобы дать себе достаточно времени для того, чтобы ощутить свою боль и ответить на нее заботой и добротой.
 
Может показаться, что страдание всегда очевидно, но многие люди не понимают или не признают, какую сильную боль они чувствуют, особенно когда эта боль рождается из самокритики. Встречаясь с жизненными трудностями, мы часто настолько сосредоточены на решении проблемы, что мы не замечаем, как нам  это время тяжело. Осознанность помогает нам бросить привычку избегать болезненных мыслей и эмоций и позволяет нам встретиться лицом к лицу с правдой того, что мы переживаем, даже если это неприятно. В тоже время осознанность останавливает нас от того, чтобы утонуть с головой в негативных мыслях и эмоциях и начать идентифицировать себя с ними, от того, чтобы застрять в наших неприятных реакциях и позволить им нас контролировать. Долгие размышления и пережевывание одного и того же смещают центр нашего внимания и преувеличивают значимость того, что мы переживаем. Я не просто потерпел неудачу, я <<неудачник>>. Я не просто разочарован, <<моя жизнь --- сплошное разочарование>>. Когда мы осознанно наблюдаем за своей болью, мы можем признать наше страдание, не преувеличивая его, что позволяет нам увидеть себя и свою жизнь с более мудрой и более объективной перспективы. Осознанность --- это первый шаг к самосостраданию: чтобы отреагировать по-новому, нам нужно мысленно присутствовать в моменте.
 
Итак, сразу после ситуации на корпоративе Тереза, вместо того, чтобы, например, заесть свой стыд коробкой конфет, нашла в себе храбрость встретиться лицом к лицу с произошедшим.

\begin{quote} 
\textit{Тереза добавила: <<Я признала, как плохо я себя чувствовала в тот момент. Мне хотелось бы,  чтобы этого не произошло, но это уже случилось. Что меня поразило, так это то, что я смогла ощущать свое чувство стыда, свои покрасневшие щеки и кровь, прилившую к голове, не осуждая при этом себя. Я знала, что эти чувства меня не убьют и что в конце концов они пройдут. И они прошли! Я подбодрила себя, а на следующий день подошла к коллеге извиниться и объяснить ситуацию, и все встало на свои места>>.}
\end{quote}

Словосочетание, которым можно описать три необходимых элемента самосострадания --- <<\emph{любящее} (доброта к себе), \emph{общечеловеческое} (человеческая общность) \emph{присутствие} (осознанность)>>. Когда мы находимся в состоянии любящего, общечеловеческого присутствия, наши отношения с самими собой, другими людьми и миром в целом преображается.


\newpage


\subsection{Упражнение №~1}

\noindent{\large \textbf{\textit{\underline{Как я обращаюсь с другом?}}}} 

\vspace{3ex}

\textbf{\textit{Закройте глаза и подумайте над нижеприведенным вопросом: }}

Подумайте о разных ситуациях, когда кто-то из ваших близких друзей встречался с какими-то трудностями --- когда с ними случалось несчастье, они терпели в чем-то неудачу или чувствовали себя не в силах что-то исправить --- а вы чувствовали себя хорошо и были довольны собой. Как вы обычно обращаетесь к ним в таких ситуациях? 
Что вы говорите? Каким тоном? Какая у вас поза? Невербальные жесты? 

\vspace{2ex}

\textbf{\textit{Запишите то, что вы обнаружили.}}


\newpage


\textbf{\textit{Теперь опять закройте глаза и подумайте над следующим вопросом:}}

Подумайте о разных ситуациях, когда трудности были у вас: когда с вами случалось несчастье, вы терпели неудачу или чувствовали себя не в состоянии что-то исправить. Как вы обращаетесь к себе к в таких ситуациях? Что вы говорите? Каким тоном? Какая у вас поза? Невербальные жесты?

\vspace{2ex}

\textbf{\textit{Запишите то, что вы обнаружили.}}

\vfill

\textbf{\textit{Наконец, разницу между тем, как вы относитесь к своим друзьям, когда им трудно, и как вы относитесь к себе. Вы заметили какие-нибудь закономерности?}}

\vfill


\newpage


\noindent{\large \textbf{\textit{Пища для размышлений}}}

\vspace{3ex}

Что вы поняли и с чем вы встретились во время выполнения этой практики?
Во время выполнения этого упражнения многих людей приводит в шок то, как плохо они относятся к себе по сравнению со своими друзьями. Если вы тоже это испытали, вы далеко не одни. Предварительные данные психологических исследований показывают, что большинство людей проявляют больше сострадания по отношению к другим, чем к себе. Наша культура не поощряет доброту к себе, поэтому нам нужно сознательно и намеренно практиковать изменение отношений с самими собой, чтобы побороть привычки, развившиеся у нас на протяжении всей жизни.


\newpage


\subsection{Упражнение №~2}

\noindent{\large \textbf{\textit{\underline{Отношение к себе с самосостраданием}}}} 

\vspace{3ex}

Подумайте о какой-нибудь из трудностей, которые имеются в вашей жизни на настоящей момент --- желательно не слишком серьезной. Например, возможно, вы поссорились со своим молодым человеком или девушкой и сказали что-то, о чем жалеете. Или, может быть, вы осознали, что из рук вон плохо сделали какое-то задание на работе и боитесь, что ваш начальник сделает вам выговор.

\vspace{2ex}

\textbf{\textit{Опишите ситуацию.}}

\vfill

Напишите, как ваше восприятие ситуации может быть искажено и как вас могло на ней <<заклинит>>. Может быть, вы только о ней и думаете или раздуваете из мухи слона? Например, вам страшно, что вас уволят, хотя ваша ошибка была достаточно незначительной?

\vfill


\newpage


Теперь попробуйте осознанно признать и заметить боль, которую вам причиняет эта ситуация, не преувеличивая ее при этом. Запишите, какие болезненные или трудные чувства вы испытываете, пытаясь делать это нейтральным и объективным тоном. Валидируйте трудность ситуации. Например: <<Мне очень страшно, что после этого случая у меня будут проблемы с начальником.  Мне сложно это сейчас чувствоват>>.

\vfill

Потом напишите, как эта ситуация вас изолирует, заставляя вас думать, что такого не должно было произойти или что вы единственный человек на свете, с которым такое случалось. Например, вы считаете, что ваша работа должна быть идеальной, а совершать ошибки --- ненормально? Что никто из ваших коллег никогда так не ошибается?

\vfill


\newpage


Теперь постарайтесь напомнить себе об общечеловечности этой ситуации --- о том, что нормально себя так чувствовать и что многие люди наверняка чувствовали что-то похожее. Например: <<Мне кажется, вполне естественно чувствовать страх после какой-то ошибки на работе. Все когда-то ошибаются, и я уверена, что многие люди были в ситуации, похожей на мою>>.

\vfill

Потом запишите свои мысли, осуждающие вас за то, что случилось. Например, может быть, вы себя обзываете (<<тупая идиотк>>) или слишком строго с собой разговариваете (<<Ты постоянно все портишь. Когда ты уже научишься все нормально делать?>>).

\vfill


\newpage


Наконец, попытайтесь написать для себя несколько добрых слов в ответ на сложные эмоции, которые вы ощущаете. Используйте те же мягкие, ласковые слова, которыми вы бы обратились к другу, который важен для вас. Например: <<Мне очень жаль, что тебе сейчас страшно. Я уверена, что все будет хорошо, и я готова тебя поддержать в любом случае>>. Или: <<Ошибаться --- это нормально, бояться последствий этого --- тоже нормально. Я знаю, что ты старалась изо всех сил>>.


\newpage

\noindent{\large \textbf{\textit{Пища для размышлений}}}

\vspace{3ex}

Как для вас прошла эта практика? Возьмите паузу и попытайтесь полностью принять, как вы себя чувствуете в этот момент, позволяя себе чувствовать именно это. 

Некоторых люди ободряют и утешают слова осознанности, человеческой общности и доброты к себе после выполнения этого письменного упражнения. Если это ваш случай, можете ли вы позволить себе насладиться чувством заботы о себе?

Многие люди, напротив, чувствуют себя некомфортно или неловко во время этой практики. Если вы --- один (одна) из них, можете ли вы позволить себе учиться в собственном темпе, зная, что для формирования новых привычек нужно время?


%%% end section %%%