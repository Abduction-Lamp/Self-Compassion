% !TEX root = ../../Self-Compassion.tex

\chapter{Самосострадание и прощение} \label{Self-Compassion_and_Forgiveness}

Когда кто-то нанес нам вред и мы все еще чувствуем по отношению к нему гнев и горечь, иногда самое разумное, что можно сделать с точки зрения сострадания~"--- простить. Простить~"--- это значит отпустить гнев на кого-то, кто нам навредил. Но прощение обязательно включает в себя перед этим также и период <<горевания>>. Самое важное, что нужно знать о практике прощения~"--- это то, что мы не можем простить других людей, не открывшись сначала к той боли, которую они нам нанесли. Аналогично, чтобы простить \emph{себя}, мы должны открыться боли, сожалению и чувству вины за то, что мы сделали больно другому. 

Простить не значит смотреть сквозь пальцы на плохое поведение или продолжить вредные и токсичные отношения. Если какие-то отношения нам вредят, мы должны сначала защитить себя, чтобы потом простить других. Если мы сами вредим кому-то в отношениях, нельзя простить себя, если мы используем прощение как отговорку, чтобы и дальше продолжать совершать плохие поступки. Сначала мы должны прекратить вредоносное поведение, а потом признать тот ущерб, который мы нанесли, и взять на себя за него ответственность. 

В то же время бывает небесполезно помнить, что нанесенный вред~"--- это финальный продукт целой вселенной причин и условий, которые тянутся назад во времени на много лет. Мы частично унаследовали темперамент от родителей и бабушек с дедушками, а наши действия частично определены событиями нашего раннего детства, культурой, состоянием здоровья и текущими событиями. Таким образом, мы не всегда полностью контролируем то, что мы говорим или делаем. 

Иногда мы можем сделать кому-то больно нечаянно и потом долгое время испытывать чувства вины и сожаления~"--- например, когда мы переезжаем на другой конец страны, чтобы начать новую жизнь, покидая родителей и друзей, или когда из-за ситуации на работе мы не можем уделять престарелым родителям столько внимания, сколько им нужно. В боли, которая возникает после таких ситуаций, никто не виноват, но ее можно признать и залечить оставшиеся от нее раны с помощью самосострадания. 

Для того чтобы уметь прощать, нужно вспомнить и осознать человеческую общность. Все мы неидеальные люди, чьи действия произрастают из паутины взаимозависимых условий, которые гораздо сильнее нас. Другими словами, не стоит принимать свои ошибки очень близко к сердцу. Как ни парадоксально, это понимание помогает нам взять на себя больше ответственности за наши действия, потому что таким образом мы чувствуем б\'{о}льшую эмоциональную безопасность. В одном исследовании участников просили вспомнить недавний поступок, за который они испытывают чувство вины~"--- например, когда они списали на экзамене, солгали своему партнеру, сказали что-то вредное~"--- и из-за которого они до сих пор переживают. В результате оказалось, что у тех людей, которым помогали отнестись к своим действиям с самосостраданием, было больше желания и мотивации извиниться, а также более сильное намерение не повторять такое поведение, чем у тех, кому проявить самосострадание не помогали.  

\begin{quotation}
	\textit{
		Аннике было очень сложно простить себя после того, как она разозлилась на свою подругу и по совместительству коллегу Хильду и послала ее куда подальше. В тот момент на Аннику очень сильно давили на работе, чтобы она заключила удачную сделку с новыми клиентами, и на ужине, на который ее компания их пригласила, должно было все решиться. Клиенты были достаточно консервативными, поэтому Анника знала, что, чтобы войти к ним в доверие, нужно не опаздывать и выглядеть прилично. Хильда должна была за ней заехать и отвезти на ужин, но во время, на которое они договорились, ее все еще не было. Анника в спешке позвонила ей. <<Где ты?>> "--*нервно спросила она. Оказалось, что Хильда вообще забыла о мероприятии. <<Боже, мне так жаль>>, "--*нескладно сказала она. Анника ее послала, сказала еще несколько неприятных слов, а потом повесила трубку и вызвала такси. Сразу же после этого Анника почувствовала себя ужасно виноватой. Это же ее подруга! У Хильды не было никаких плохих намерений~"--- она просто забыла, а Анника была слишком занята, чтобы ей напомнить. На самом деле Анника так беспокоилась за закрытие сделки, что из-за этого она потеряла объективность и гипертрофированно отреагировала.
	}
\end{quotation}

\vspace{3ex}

\noindent{\large \textbf{Существует пять стадий прощения:}}

\begin{enumerate}
	\item \textbf{Открыться боли}~"--- как следует ощутить страдание или боль от того, что произошло.
	
	\item \textbf{Самосострадание}~"--- позволить нашим сердцам растаять в сочувствии нашей боли, неважно, какова ее причина.
	
	\item \textbf{Мудрость}~"--- начать осознавать, что вина в ситуации не полностью лежит на каком-то одном человеке и что эта ситуация была последствием многих взаимозависимых причин и условий.
	
	\item \textbf{Намерение простит}ь~"--- <<Пусть я начну прощать себя [другого] за то, что я
	[он/она] сделал, сознательно или нет, и что причинило этому человеку [мне] боль>>.
	
	\item \textbf{Ответственность за защиту}~"--- обязаться не повторять ту же ошибку и избегать вреда, насколько мы на это способны.
\end{enumerate}

\begin{quotation}
	\textit{
		Сначала Анника очень себя ругала за свое поведение, но она понимала, что это никому не поможет. Вместо этого ей нужно было простить себя за то, что она совершила ошибку, точно так же, как и все люди их совершают. Анника помнила пять стадий прощения со времен курса ОСС, поэтому она знала, что делать. Первым делом нужно было принять боль, которую она причинила Хильде. Это оказалось для нее очень трудным, особенно в свете того, что она так и не заключила с клиентами контракт, который она надеялась заключить. Ее мозгу хотелось свалить всю вину на Хильду. Она же сама во всем виновата! Но в глубине души Анника знала правду: у нее не было никаких оправданий разговаривать так с Хильдой. Это был недопустимый поступок.
	}
	
	\textit{
		Анника постаралась прочувствовать то, что наверняка чувствовала Хильда, когда услышала такие слова~"--- от кого-то, кого она считала своей подругой. Для этого потребовалась смелость, потому что ей стало безумно за это стыдно. Потом Анника проявила к себе сострадание за боль осознания того, что она причинила боль кому-то, кого она любила. <<Все совершают ошибки. Мне так жаль, что ты ранила подругу таким образом. Я знаю, что ты глубоко об этом жалеешь>>. Это помогло ей увидеть ситуацию в более широкой перспективе, и она смогла признать невероятный стресс, которому она подвергалась. Обстоятельства пробудили в ней ее худшие стороны. Тогда Анника попыталась себя простить, пусть даже и не окончательно, за свое поведение. <<Пусть я начну прощать себя за боль, которую я нечаянно причинила своей подруге Хильде>>. Более того, она дала себе обещание каждый раз делать как минимум один глубокий вдох перед тем, как что-то говорить, когда она злится.
	}
	
	\textit{
		Анника знала, что это может занять время, потому что она даже не всегда могла распознать, когда она злится, но она была твердо намерена научиться реагировать менее остро в стрессовых ситуациях.
	} 
\end{quotation}

Следующие две практики проведут вас через все пять стадий прощения~"--- прощения других  и прощения себя. Опять же, \textbf{центральная задача прощения~"--- открыться боли, которую мы испытали} или \textbf{причинили другим}. Важно правильно подобрать время, так как мы не очень любим чувствовать вину за причиненную другим боль или снова стать уязвимыми. Как говорится, сначала нам нужно <<оставить всю надежду на лучшее прошлое>>. 

\newpage
\InformalPractices{Прощение других} \label{IP:Forgiving_Others}

\begin{itemize}
	\itemWritingHand Сделайте два--три глубоких вдоха и выдоха и закройте на пару минут глаза, чтобы прочувствовать свое тело и настоящий момент. Положите руку на сердце в качестве жеста поддержки и доброты к себе. Теперь подумайте о каком-нибудь человеке, который причинил вам боль и которого вы готовы простить. Вспомните какой-то случай из ваших отношений с этим человеком, который доставил вам дискомфорт (примерно 3 на шкале от 1 до 10). Для этого упражнения важно выбрать человека и поступок, которых вы действительно готовы простить, потому что понимаете, что остающиеся ощущения гнева и вины причиняют вам лишнюю боль. Опишите не спеша тот случай, с которым вы хотели бы поработать. 
\end{itemize}
\setlength{\extrarowheight}{2mm}
\begin{tabularx}{\textwidth}{X}
	\\
	\arrayrulecolor{gray}\hline\\
	\hline\\
	\hline\\
	\hline\\
	\hline\\
	\hline\\	
	\hline\\
	\hline\\
	\hline\\
	\hline\\
	\hline\\
	\hline\\
	\hline\\
	\hline\\
\end{tabularx}
\setlength{\extrarowheight}{0mm}
\begin{itemize}
	\item Во время этого упражнения постарайтесь освободить в голове много места для того, что вы почувствуете, подходя к практике с любопытством и осознанностью к тому, что произошло, не дав вашим тогдашним эмоциям сбить вас в пути. Если вы чувствуете себя некомфортно, приостановите упражнение. Вы можете в любой момент к нему вернуться.
\end{itemize}

\vspace{3ex}

\noindent{\large \textbf{Откройтесь боли}}

\begin{itemize}
	\item Вспомните детали произошедшего так точно, как можете, вступая в контакт с болью, которую вам причинил этот человек, возможно, даже ощущая ее телесно.
	
	\item Вам нужно просто прикоснуться к боли, а не утонуть в ней с головой.
\end{itemize}

\vspace{3ex}

\noindent{\large \textbf{Самосострадание}}

\begin{itemize}
	\item Валидируйте боль, как будто говорите с любимой подругой: <<Конечно, ты так себя чувствуешь~"--- тебе причинили боль!>>, <<Это же очень больно!>>
	
	\item Продолжайте проявлять к себе сострадание~"--- можете, например, положить руку на сердце и представить себе, что доброта перетекает из вашей руки в ваше тело. Или сказать себе пару слов сострадания: <<Желаю себе быть в безопасности>>, <<Желаю себе быть сильной>>, <<Желаю себе быть к себе добрее>>.
	
	\item Теперь спросите себя: <<Действительно ли я готова простить этого человека?>> Если нет, продолжайте проявлять к себе сострадание.
\end{itemize}
 
\vspace{3ex}

\noindent{\large \textbf{Мудрость}}

\begin{itemize}
	\item Если вы по-настоящему готовы простить, попробуйте понять, что подтолкнуло этого человека на плохой поступок. Вспомните, что быть человеком~"--- значит периодически делать ошибки, подумайте о внешних факторах, которые повлияли на произошедшее~"--- факторы вне вашего или его/ее контроля, которые вы не приняли во внимание. 

	Например, может быть, этот человек подвергался в то время огромному количеству стресса? Или были сложности, которые повлияли на его характер? (Например, трудное детство, низкая самооценка, культурные факторы)? Большая часть людей просто стараются как можно лучше прожить жизнь. Но при этом, несмотря на то, какие другие факторы там присутствовали, вам все равно причинили боль.
\end{itemize}
	
\vspace{3ex}

\noindent{\large \textbf{Намерение простить}}

\begin{itemize}
	\item Теперь, если~"--- и только если~"--- вы чувствуете, что готовы простить, начните мысленно посылать прощение этому человеку. Можете при этом произносить: <<Пусть я начну прощать тебя за то, что ты сделал~"--- сознательно или неосознанно~"--- и то, что причинило мне боль.  

	\item Повторите эту фразу два или три раза. 
\end{itemize}
	
\vspace{3ex}

\noindent{\large \textbf{Ответственность за защиту}}

\begin{itemize}
	\item Наконец, если вы готовы, заключите с собой сделку: пообещайте себе, что, насколько возможно, вам больше не смогут причинить такую боль~"--- ни этот человек, ни кто-то другой. 
\end{itemize}

\Reflection{
	Каково вам было вновь вступить в контакт с болью, которую вы испытали? Смогли ли вы проявить к себе сострадание? Было ли внутри некое сопротивление?
	
	Получилось ли у вас идентифицировать факторы, о которых вы раньше не думали, которые могли привести к причинившему вам боль поведению этого человека? Что вы ощутили, произнеся фразу прощения? Смогли ли вы пообещать себе в будущем себя защищать? 
	
	Когда некоторые люди пытаются выполнить это упражнение, они понимают, что еще не готовы простить. Нежелание прощать~"--- это само по себе важный дидактический опыт. Если вы с этим встретились, попробуйте сосредоточиться на слове \textbf{начать} во фразе прощения~"--- эта формулировка уважает намерение, но не подразумевает, что нужно немедленно расшибиться в лепешку, чтобы это сделать. Мы понимаем, что простили, по ощущению свободы в сердце, но если прощение ощущается как тяжелая ноша, мы еще не готовы. 
}

\newpage
\InformalPractices{Прощение себя} \label{IP:Forgiving_Ourselves}

\begin{itemize}
	\item Сделайте два--три глубоких вдоха и выдоха и закройте на пару минут глаза, чтобы прочувствовать свое тело и настоящий момент. Положите руку на сердце в качестве жеста поддержки и доброты к себе.
	
	\itemWritingHand Теперь подумайте о каком-нибудь человеке, которому вы причинили боль. Вспомните какой-то конкретный случай из ваших отношений с этим человеком, о котором вы жалеете и за который вы бы хотели себя простить. Как и в предыдущей практике, для первого раза выберите достаточно несложную ситуацию~"--- где-то 3 на шкале от 1 до 10.  (примерно 3 на шкале от 1 до 10). Для этого упражнения важно выбрать человека и поступок, которых вы действительно готовы простить, потому что понимаете, что остающиеся ощущения гнева и вины причиняют вам лишнюю боль. Не спешите, найдите именно ту ситуацию, с которой вам нужно поработать.	
\end{itemize}
\setlength{\extrarowheight}{2mm}
\begin{tabularx}{\textwidth}{X}
	\\
	\arrayrulecolor{gray}\hline\\
	\hline\\
	\hline\\
	\hline\\
	\hline\\
	\hline\\	
	\hline\\
	\hline\\
	\hline\\
	\hline\\
	\hline\\
	\hline\\
	\hline\\
	\hline\\
	\hline\\
	\hline\\
\end{tabularx}
\setlength{\extrarowheight}{0mm}


\noindent{\large \textbf{Откройтесь боли}}

\begin{itemize}
	\item Для начала подумайте, как ваши действия повлияли на другого человека, и разрешите себе почувствовать вину и угрызения совести, которые естественным образом возникают, когда мы причиняем кому-то боль. Для этого может потребоваться смелость. 

	Может быть полезным прочувствовать связанные с чувством вины ощущения в теле.
	
	(Если вы понимаете, что чувствуете стыд, а не вину, можете повторить упражнение <<Работа со стыдом>> из главы \ref{Self-Compassion_and_Shame} на стр.\:\pageref{IP:Working_with_Shame})
\end{itemize}

\vspace{3ex}

\noindent{\large \textbf{Самосострадание}}

\begin{itemize}
	\item Если вы чувствуете, что поступили не так, как надо было, вспомните, что ошибки~"--- естественная часть человеческой жизни, как и чувство вины.

	\item Проявите к себе сострадание за то, как вы страдали. Можете сказать: <<Желаю себе быть к себе добрее. Желаю себе принять себя таким, какой я есть>>. Если хотите, положите руку на сердце и представьте себе, что по вашим пальцам в ваше тело течет доброта. 

	\item Если, по ощущениям, вам нужно какое-то время остаться здесь, делайте это без колебаний. Нет необходимости двигаться дальше.
\end{itemize}

\vspace{3ex}

\noindent{\large \textbf{Мудрость}}

\begin{itemize}
	\item Когда вы готовы, попробуйте понять, что привело к вашей ошибке. Подумайте, были ли какие-то внешние факторы, которые могли на вас повлиять. Например, может быть, вы подвергались большому количеству стресса? Или, может быть, некоторые стороны вашего характера иррационально отреагировали на что-то неприятное и знакомое~"--- вам нажали на <<старые кнопки>>? Посмотрите на ситуацию шире, не ограничивайтесь своей личной интерпретацией.

	\item А может быть, вы и вовсе не ошиблись, а просто пытались жить своей жизнью, как знали? 
\end{itemize}

\vspace{3ex}

\noindent{\large \textbf{Намерение простить}}

\begin{itemize}
	\item Теперь попробуйте себя простить, произнеся фразу: <<Пусть я начну прощать себя за то, что я сделал, сознательно или неосознанно, что причинило этому человеку боль>>. 
\end{itemize}

\vspace{3ex}

\noindent{\large \textbf{Ответственность за защиту}}

\begin{itemize}
	\item И, если вы в состоянии это сейчас сделать, обещайте себе больше не причинять никому боль таким образом, по крайней мере, насколько можете.
\end{itemize}


\Reflection{
	Что было легче, простить \emph{себя} или \emph{других}? Смогли ли вы открыться боли осознания, что вы причинили человеку страдание? 
	
	Получилось ли у вас проявить к себе сострадание, даже если вам казалось, что вы его не заслуживаете? Что вы почувствовали, произнося фразы прощения? Смогли ли вы пообещать себе больше не причинять кому-то боль таким способом?
	
	Нужна особенная смелость, чтобы открыться чувству вины и угрызениям совести, которые возникают, когда мы причиняем кому-то боль. Чем дольше мы можем остановиться на этих неудобных чувствах с состраданием, тем сильнее будет наша решимость избежать повторения нашей ошибки. Некоторые люди беспокоятся, что, если они простят себя, то сложат с себя всю ответственность за свои действия~"--- напротив, искреннее прощение себя как раз содействует эффективным изменениям.
}