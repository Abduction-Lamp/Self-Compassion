% !TEX root = ../../Self-Compassion.tex

\chapter{Инь~и~Янь самосострадания} \label{The_Yin_and_Yang_of_Self-Compassion}

На первый взгляд сострадание может показаться признаком мягкотелости или слабости, который нужен только для успокоения и утешения. Поскольку сострадание~"--- это необходимая часть воспитания детей, у некоторых из нас оно может ассоциироваться с традиционно женскими гендерными ролями. Значит ли это, что самосострадание может кому-то не подойти?
Спросите себя: разве не меньше сострадания в спасении кого-то из горящего здания или в работе с утра до вечера, чтобы обеспечить семью? А ведь оба этих поступка скорее относятся к гендерным нормам, которые предъявляются к мужчинам и которые вряд ли можно назвать признаком слабости. Неплохо было бы расширить определение сострадания и самосострадания в нашей культуре так, чтобы они включали все множество возможных форм этих качеств.

Среди качеств, которые необходимы для самосострадания, можно найти и традиционно женские, и традиционно мужские~"--- так же, как и у всех людей в характере есть и те, и те качества. В традиционной китайской философии эту двойственность олицетворяет символ \textit{инь} и \textit{янь}. Он демонстрирует, что все качества, кажущиеся противоположными~"--- например, мужское-женское, темнота-свет, активность-пассивность~"--- на самом деле неразрывно связаны и дополняют друг друга. Это значит, что и мужчинам, и женщинам нужно иметь некоторые качества, ассоциирующиеся с противоположным полом, чтобы находиться в балансе. Символически это выражается в том, что каждой половине инь-яня есть точка противоположного цвета. 

\newpage 
\begin{center}
	{\Huge \Yinyang}
\end{center}

\begin{itemize}
	\item \textbf{\textit{Инь}} самосострадания содержит аспекты <<бытия>> в гармонии с собой: \textit{поддержки, утешения, валидации}.
	\item \textbf{\textit{Янь}} самосострадания – активные <<действия>>: \textit{защита, обеспечение, мотивация}.
\end{itemize}

\begin{quotation}
	\textit{У Моник были сильные сомнения в эффективности самосострадания. Она выросла в неблагополучном районе и с гордостью рассказывала всем, что выжила благодаря твердости характера и житейской смекалке. С любой трудностью она сразу же разбиралась напрямую без малейших колебаний. Недавно у нее диагностировали рассеянный склероз, и тут-то ее обычный подход к решению проблем и дал осечку.  Каждый раз, когда Моник чувствовала себя уязвимой или испытывала страх по поводу своего диагноза и сомнения насчет прописанного врачом курса лечения (в который входило много отдыха), она устраивала своей семье, друзьям и даже врачам полнейший разнос. Обычно бурная активность защищала Моник от столкновения со своими эмоциями, но в случае с рассеянным склерозом это было бесполезно. Сама идея самосострадания и доброты к себе казалось Моник, которая считала себя сильной и жесткой, чуть ли не святотатством.}  
	
	\textit{У Ксавье была обратная проблема. Хотя у него тоже было непростое детство и он постоянно был свидетелем того, как отчим кричал на мать, он научился сбегать из реального мира в мир книг и оставаться как можно более незаметным до конца домашней <<бури>>. Он очень рано понял, что конфронтация бы сделала только хуже. Теперь Ксавье было уже за 20, он закончил университет, и ему нужно было уже выстраивать свою собственную жизнь и начать хотя бы зарабатывать достаточно, чтобы съехать от матери, но он крайне сомневался в своих силах. Он устроился в больницу работать санитаром, чтобы меньше времени проводить дома, но удовлетворения это ему не принесло. Ему нужно было, чтобы кто-то в него верил и поддерживал его, чтобы реализовать свой потенциал.} 
\end{quotation}

Курс ОСС содержит много различных практик и упражнений, которые каждый читатель может попробовать, чтобы найти то, что наиболее эффективно именно для него. Некоторые практики нацелены на развитие навыков <<инь>>, а некоторые~"--- на <<янь>>, но большая их часть включает элементы и того, и другого. В таблице ниже приведены примеры практик из этой книги, которые соответствуют аспектам <<инь>> и <<янь>>. Разумеется, инь и янь взаимосвязаны и активно взаимодействуют. Например, когда мы признаем и валидируем свои потребности, мы часто находим мотивацию их удовлетворить.

\begin{table}[!h]
	\begin{center}
		\setlength{\extrarowheight}{1mm}
		\begin{tabular}{ccc||l}
			\multicolumn{3}{r||}{{\large\textbf{Аспекты}}} &{\large\textbf{Практики}}\\
			\cline{3-4}
			\multirow{9}{*}{{\huge\textbf{Инь}}} &  & \multirow{3}{*}{\textbf{Поддержка}} & Перерыв на самосострадание (Глава~\ref{The_Physiology_of_Self-Criticism_and_Self-Compassion}, стр.\:\pageref{IP:Self-Compassion_Break})\\ 
			&  &  & Самосострадание в повседневной жизни (глава 8)\\ 
			& $\nearrow$ &  & Медитация любящей доброты к себе (глава 10)\\ \cline{3-4}
			& \multirow{3}{*}{\textbf{$\rightarrow$}} & \multirow{3}{*}{\textbf{Утешение}} & Успокаивающие прикосновения (Глава~\ref{The_Physiology_of_Self-Criticism_and_Self-Compassion}, стр.\:\pageref{IP:Soothing_Touch})\\
			&   &   & Медитация с любящим дыханием (глава 6)\\
			&   &   & Смягчите-утешьте-разрешите (глава 16)\\ \cline{3-4}
			& $\searrow$ & \multirow{3}{*}{\textbf{Валидация}} & Сострадание, когда все идет не так (глава 13)\\
			&   &   & Сортировка эмоций (глава 16)\\
			&   &   & Цените себя (глава 23)\\ \cline{3-4}
		\end{tabular}
	\end{center}
\end{table} 
\begin{table}[!h]
	\begin{center}
		\setlength{\extrarowheight}{1mm}
		\begin{tabular}{ccc||l}
			\multicolumn{3}{r||}{{\large\textbf{Аспекты}}} &{\large\textbf{Практики}}\\
			\cline{3-4}
			\multirow{9}{*}{{\huge\textbf{Янь}}} &  & \multirow{3}{*}{\textbf{Защита}} & Почувствуйте свои стопы (глава 8) \\ 
			&  &  & Спокойное сострадание (глава 19)\\ 
			& $\nearrow$ &  & Яростное сострадание (глава 20)\\ \cline{3-4}
			& \multirow{3}{*}{\textbf{$\rightarrow$}} & \multirow{3}{*}{\textbf{Обеспечение}} & Поиск наших основных ценностей (глава 14)\\
			&   &   & Удовлетворение эмоциональных потребностей (глава 18)\\
			&   &   & Удовлетворение неудовлетворенных потребностей (глава 20)\\ \cline{3-4}
			& $\searrow$ & \multirow{3}{*}{\textbf{Мотивация}} & Поиск вашего сострадательного голоса (глава 11)\\
			&   &   & Сострадательное письмо себе (глава 11)\\
			&   &   & Жизнь с клятвой (глава 14)\\ \cline{3-4}
		\end{tabular}
		\setlength{\extrarowheight}{0mm}
		\caption{Культивирование Инь и Ян самосострадания}\label{tab:Yin_and_Yang}
	\end{center}
\end{table} 

