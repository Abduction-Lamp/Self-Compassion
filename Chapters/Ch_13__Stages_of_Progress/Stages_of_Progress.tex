% !TEX root = ../../Self-Compassion.tex

\chapter{Стадии прогресса} \label{Stages_of_Progress}

Люди, практикующие самосострадание, обычно проходят через три этапа:
\begin{itemize}
	\item Усилия
	\item Разочарование
	\item Радикальное принятие
\end{itemize}

Когда мы только начинаем практиковать доброту к себе, мы переносим на этот процесс такое же отношение, с которым мы подходим к остальным жизненным задачам: мы прикладываем очень много (даже слишком много) \emph{усилий}, чтобы все правильно сделать. А когда это приносит плоды, и мы чувствуем сострадание к себе, мы чувствуем облегчение и даже повышенный энтузиазм к практике. Этот этап похож на начальный период любых романтических отношений~"--- страстную влюбленность. Мы часто очень довольны счастьем, которое испытываем, и поэтому привязываемся и к самому ощущению, и к человеку, который его приносит. По такому же принципу, когда мы видим, что можем сами удовлетворить свои потребности, радость этого открытия может быть похожей на влюбленность. Это часто приводит к определенному духовному подъему. 

\begin{quotation}
	\textit{
		Когда Джонатан в первый раз сделал <<перерыв на самосострадание>> (глава \ref{The_Physiology_of_Self-Criticism_and_Self-Compassion}, стр.\:\pageref{IP:Self-Compassion_Break}), он был в шоке от того, насколько сильным был эффект. Он думал о стрессовой ситуации на работе, но короткая практика мгновенно превратила его стресс в состояние покоя и умиротворения. <<То есть вы хотите сказать, что все, что мне нужно делать~"--- осознанно относиться к своей боли, принять то, что я человек и похож на других людей, и быть добрым к себе? "--*подумал он. "--*Это же замечательно!>>
	}
\end{quotation}

Но, как и происходит со всеми отношениями, изначальный глянец начинает стираться. Например, мы кладем руку себе на сердце, надеясь испытать то же чувство безопасности и связи с собой, которое мы испытали в начале, но ничего не происходит. Черт! Так мы вступаем в следующий этап практики~"--- \emph{разочарование}.

Когда самосострадание начинает нас подводить, мы думаем, что это еще одна вещь из списка того, что у нас не получается. Как сказал однажды один учитель медитации: <<Все приемы обречены на провал>>. Почему? Потому что когда наша практика превращается в <<прием>> для манипуляции реальностью~"--- чтобы нам стало лучше и боль ушла~"--- она становится скрытой формой сопротивления, а какой эффект производит сопротивление, мы знаем! 

\begin{quotation}
	\textit{
		Когда Джонатан серьезно поссорился с сыном и почувствовал сильную злость и разочарование, он подумал, что знает, что делать, чтобы успокоиться~"--- <<перерыв на самосострадание>>! К сожалению, это не помогло, поэтому он попробовал <<успокаивающее прикосновение>> (глава \ref{The_Physiology_of_Self-Criticism_and_Self-Compassion}, стр.\:\pageref{IP:Soothing_Touch}). Это тоже не сработало. Чувствую себя, как будто его предал друг, которому он доверял, Джонатан совсем пришел в уныние. <<Я-то думал, что всему научился, но сейчас я все еще чувствую себя отвратительно. Значит, самосострадание у меня получается плохо>>.  
	}
\end{quotation}

Когда отчаяние чувства разочарования ставит нас на колени и мы беспомощно сдаемся, наконец-то и начинается прогресс. На самом деле прогресс означает отпустить мысль о прогрессе. Мы прекращаем прилагать неимоверное количество усилий, чтобы продвинуться, достичь цели стать мастером самосострадания и окончательно избавиться от боли и начинаем потихоньку менять наши намерения. Вместо того, чтобы зацикливаться на результатах практики самосострадания, мы начинаем ей заниматься ради самого процесса. Так мы и переходим на этап \emph{радикального принятия}, которое лучше всего можно описать парадоксом, упомянутым в главе \ref{Letting_Go_of_Resistance} на странице \pageref{Letting_Go_of_Resistance}: 

\vspace{4ex}

\begin{center}
	{\Large Когда мы попадаем в трудную ситуацию, мы сострадаем себе не затем, чтобы нам было легче, а потому что нам плохо.}
\end{center}

\vspace{4ex}

Другими словами, в трудные моменты мы практикуем самострадание не с целью избавиться от боли~"--- мы практикуем самосострадание, потому что иногда быть человеком сложно.

\textbf{Радикальное принятие}~"--- это как родитель, утешающий ребенка, у которого двухдневный грипп. Он или она заботится о ребенке не затем, чтобы моментально избавить его от гриппа~"--- он сам пройдет в положенное время. Но, поскольку ребенка бросает в жар и ему плохо, родитель его утешает. Это естественная реакция на страдание во время процесса выздоровления.

То же самое происходит, когда мы пытаемся утешить самих себя. Когда мы окончательно принимаем, что реальность такова, что мы неидеальные человеческие существа, которые иногда делают ошибки и которым иногда трудно, наши сердца начинают естественным образом смягчаться. Мы все еще ощущаем боль, но мы ощущаем и любовь, держащую боль, и все становится более терпимым. Эта реакция <<радикальна>>, потому что она противоположна нашей привычной реакции на боль. Само по себе превращение бывает радикальным. 

\begin{quotation}
	\textit{
		Поговорив с учителем медитации, Джонатан понял, что намерение, которое приводило его к  	практике самосострадания, незаметно для него самого изменилось. Ему приносило такое облегчение то, что он лучше чувствовал себя после практики самосострадания, что у него появилась привычка использовать это, чтобы избавиться от боли, когда ему плохо. В конце концов Джонатан осознал, что в жизни не может не быть боли. После того как это понимание укрепилось в его сердце, Джонатан начал замечать, как в нем возникало чувство тихой нежности, когда ему было плохо. Он даже начал видеть боль как напоминание себе, чтобы он открыл свое сердце и что с жизнь раскрытым сердцем~"--- это то, чего ему больше хотелось.
	}
\end{quotation}

Как говорит учитель медитации Пема Чодрон: <<После всех этих лет мы все равно можем быть немного сумасшедшими. Мы все равно можем злиться. Мы все равно можем быть застенчивыми, ревнивыми, считать себя недостойными чего-то. Смысл не в том, чтобы распрощаться с собой настоящим и стать чем-то лучшим. Он в том, чтобы подружиться с тем, кем мы уже являемся>> \cite{80}.

Учитель медитации Роб Наирн выразился еще более кратко: <<Цель практики~"--- быть в раздрае, но относиться к себе с состраданием>>\cite{81}. Это значит быть человеком во всех смыслах этого слова~"--- человеком, которому часто нелегко, который часто неуверенный и растерянный~"--- с большим состраданием.  А самое лучшее~"--- это то, что эта цель действительно достижима. Вне зависимости от того, как стремительно мы падаем, насколько мучительна наша боль, насколько наши жизни или мы сами неидеальны, мы можем быть осознанными к своему страданию, помнить о человеческой общности и быть добрыми к себе.

Стадии прогресса часто нелинейны и не следуют в строгом порядке друг за другом. Они скорее похожи на спираль, которая постоянно закручивается, а еще иногда мы мечемся туда-обратно между двумя стадиями. Но со временем периоды усилий и разочарования становятся короче, а радикальное принятие все чаще и чаще сопровождает нас в превратностях жизни. Мы начинаем верить, что, что бы ни происходило, мы всегда сможем заключить себя в объятия любящего, общечеловеческого присутствия.

\newpage
\Exercises{На каком этапе я нахожусь в своей практике?} \label{Where_Am_I_in_My_Self-Compassion
_Practice}

В пустом пространстве запишите мысли, которые приходят вам в голову, когда вы думаете над тремя вопросами:

\begin{itemize}
	\itemWritingHand Помня о том, что в своей практике самосострадания мы циклически передвигаемся между стадиями прогресса, подумайте, в какой части цикла вы можете сейчас находиться: усилия, разочарование или радикальное принятие? 
\end{itemize}
\setlength{\extrarowheight}{2mm}
\begin{tabularx}{\textwidth}{X}
	\\
	\arrayrulecolor{gray}\hline\\
	\hline\\
	\hline\\
	\hline\\
	\hline\\
	\hline\\
	\hline\\
\end{tabularx}

\setlength{\extrarowheight}{0mm}
\begin{itemize}
	\itemWritingHand Если в каких-то аспектах практики у вас есть трудности, можете ли вы как-то эти трудности уменьшить? Есть ли что-то, для чего вы бы хотели освободить больше места, что бы вы хотели отпустить или, наоборот, прочувствовать более сильно?
\end{itemize}
\setlength{\extrarowheight}{2mm}
\begin{tabularx}{\textwidth}{X}
	\\
	\arrayrulecolor{gray}\hline\\
	\hline\\
	\hline\\
	\hline\\
	\hline\\
	\hline\\	
	\hline\\
	\hline\\
	\hline\\	
	\hline\\
	\hline\\
\end{tabularx}
\setlength{\extrarowheight}{0mm}

\begin{itemize}
	\itemWritingHand Вы можете как-нибудь проявить страдание к себе на вашем пути? Можете ли вы быть нежным и справедливым по мере продвижения вашей практики, может быть, произнести несколько слов доброты, понимания, поддержки или признательности?
\end{itemize}
\setlength{\extrarowheight}{2mm}
\begin{tabularx}{\textwidth}{X}
	\\
	\arrayrulecolor{gray}\hline\\
	\hline\\
	\hline\\
	\hline\\
	\hline\\
	\hline\\	
	\hline\\
	\hline\\
	\hline\\
	\hline\\
	\hline\\
	\hline\\
	\hline\\
\end{tabularx}
\setlength{\extrarowheight}{0mm}

\Reflection{
	Когда люди слышат слово прогресс, они думают, что чем больше прогресса, тем лучше. Другими словами, люди часто себя осуждают за то, что они пока не добрались до стадии радикального принятия. Важно понимать, что самосострадание~"--- это процесс и образ жизни, а не конечная точка. У нас будут моменты радикального принятия, но будет и много моментов усилий и разочарования. Все это~"--- равнозначные по важности аспекты нашего пути. Если вы вынесли какое-то суждение (положительное и отрицательное) по поводу того, на какой стадии прогресса вы сейчас находитесь, попробуйте отпустить это суждение, а также привычку постоянно оценивать себя и откройтесь реальности настоящего момента с теплым сердцем.
}

\newpage
\InformalPractices{Быть в раздрае, но относиться к себе с состраданием} \label{Being_a_Compassionate_Mess}

Когда вы замечаете, что используете самосострадание для того, чтобы избавиться от боли и <<стать лучше>>, попробуйте перевести внимание с этой еле уловимой формы сопротивления на что-то другое \textbf{и практиковать сострадание просто потому, что мы все неидеальные люди, живущие неидеальными жизнями}. А жизнь на самом деле тяжела. Другими словами, практикуйте <<пребывание в раздрае\footnote{Раздр\'{a}й~"--- отсутствие ясности, порядка, согласованности; путаница, неразбериха, сумбур.}, но с состраданием>>. Эту практику можно выполнять в повседневной жизни, когда вам тяжело. 

\begin{itemize}
	\item Подумайте о какой-то ситуации в вашей жизни, которая приносит вам эмоциональную боль, потому что вы чувствуете себя из-за нее несостоявшимся. Может быть, вы сделали что-то, о чем вы теперь жалеете, или потерпели неудачу в чем-то, что для вас важно. Выберите не слишком сложную проблему, так как нам нужно постепенно наращивать ресурс самосострадания.
	
	\item Чувствуете ли вы в своем теле дискомфорт, когда вспоминаете об этой ситуации? Если нет, выберите ситуацию посложнее, а если дискомфорт очень сильный~"--- ситуацию полегче.
	
	\item Ощущая эмоциональный дискомфорт, попробуйте полностью принять эту боль, позволяя своему сердцу растаять, чтобы утешить себя и позаботиться о тебе в связи тем, что вам трудно. Вы можете пройти через этот процесс в компании вашего же любящего и общечеловеческого присутствия?
	
	\item Сделайте два-три глубоких вдоха и выдоха и закройте на минуту глаза. Положите руку на сердце или используйте какое-нибудь другое успокаивающее прикосновение в качестве жеста поддержки и доброты.
	
	\item Попробуйте поговорить с собой (вслух или про себя) теплыми, поддерживающими, сострадательными словами. Например:
	
	\begin{itemize}
		\item <<\textbf{Мне так жаль, что ты сейчас так плохо о себе думаешь, но эти чувства не навсегда.} Я с тобой, все будет хорошо>>.
		
		\item <<Боль от неудачи почти нестерпимая, но я постараюсь просто ,,побыть`` с ней со смелостью, терпением и открытым сердце>>.
	\end{itemize}

	\item Можете ли вы позволить себе быть таким, какой вы есть~"--- человеком во всех смыслах? Можете ли вы начать отпускать попытки достичь совершенства и признать, что вы стараетесь, как можете? Попробуйте поговорить с собой словами, которые признают ваше несовершенство, но в то же время создают ощущение безусловной поддержки~"--- словами, которыми вы бы поговорили с другом или с кем-то, кто для вас очень важен. Например:
	
	\begin{itemize}
		\item <<Нет ничего плохого в том, чтобы быть в раздрае, быть неидеальным>>.
		
		\item <<Да уж, вот тут я действительно налажал. Хотелось бы мне, чтобы это не произошло, но оно произошло. Очень тяжело себя так чувствовать, но я никак не могу изменить тот факт, что я неидеальный человек, который иногда совершает ошибки. Желаю себе принять себя целиком с пониманием и добротой>>.   
	\end{itemize}
\end{itemize}

\Reflection{
	Сопротивляться тому, чтобы отпустить желание делать все идеально или принять наши несовершенства~"--- вполне нормально и естественно. Мы хотим чувствовать себя в безопасности, а, делая ошибки, мы из нашей зоны безопасности и комфорта выходим. Но не нужно сыпать соль на рану, осуждая себя за то, что мы хотим быть другими, нежели мы есть. Надо просто понять, каким образом и в чем эта внутренняя борьба причиняет нам лишние страдания, попытаться хотя бы начать принимать себя и наши человеческие недостатки и посмотреть, что случится.
}
