% !TEX root = ../../Self-Compassion.tex

\chapter{Инь~и~Янь самосострадания} \label{The_Yin_and_Yang_of_Self-Compassion}

На первый взгляд сострадание может показаться признаком мягкотелости или слабости, который нужен только для успокоения и утешения. Поскольку сострадание~"--- это необходимая часть воспитания детей, у некоторых из нас оно может ассоциироваться с традиционно женскими гендерными ролями\cite{55}. Значит ли это, что самосострадание может кому-то не подойти?
Спросите себя: разве не меньше сострадания в спасении кого-то из горящего здания или в работе с утра до вечера, чтобы обеспечить семью? А ведь оба этих поступка скорее относятся к гендерным нормам, которые предъявляются к мужчинам и которые вряд ли можно назвать признаком слабости. Неплохо было бы расширить определение сострадания и самосострадания в нашей культуре так, чтобы они включали все множество возможных форм этих качеств.

Среди качеств, которые необходимы для самосострадания, можно найти и традиционно женские, и традиционно мужские~"--- так же, как и у всех людей в характере есть и те, и те качества. В традиционной китайской философии эту двойственность олицетворяет символ \textit{инь} и \textit{янь}. Он демонстрирует, что все качества, кажущиеся противоположными~"--- например, мужское-женское, темнота-свет, активность-пассивность~"--- на самом деле неразрывно связаны и дополняют друг друга. Это значит, что и мужчинам, и женщинам нужно иметь некоторые качества, ассоциирующиеся с противоположным полом, чтобы находиться в балансе. Символически это выражается в том, что каждой половине инь-яня есть точка противоположного цвета. 

\vspace{3ex}

\begin{center}
	{\Huge\Yinyang}
\end{center}

\vspace{1ex}

\begin{itemize}
	\item \textbf{\textit{Инь}} самосострадания содержит аспекты <<бытия>> в гармонии с собой: \textit{поддержки, утешения, валидации}.
	\item \textbf{\textit{Янь}} самосострадания~"--- активные <<действия>>: \textit{защита, обеспечение, мотивация}.
\end{itemize}

\begin{quotation}
	\textit{У Моник были сильные сомнения в эффективности самосострадания. Она выросла в неблагополучном районе и с гордостью рассказывала всем, что выжила благодаря твердости характера и житейской смекалке. С любой трудностью она сразу же разбиралась напрямую без малейших колебаний. Недавно у нее диагностировали рассеянный склероз, и тут-то ее обычный подход к решению проблем и дал осечку.  Каждый раз, когда Моник чувствовала себя уязвимой или испытывала страх по поводу своего диагноза и сомнения насчет прописанного врачом курса лечения (в который входило много отдыха), она устраивала своей семье, друзьям и даже врачам полнейший разнос. Обычно бурная активность защищала Моник от столкновения со своими эмоциями, но в случае с рассеянным склерозом это было бесполезно. Сама идея самосострадания и доброты к себе казалось Моник, которая считала себя сильной и жесткой, чуть ли не святотатством.}  
	
	\textit{У Ксавье была обратная проблема. Хотя у него тоже было непростое детство и он постоянно был свидетелем того, как отчим кричал на мать, он научился сбегать из реального мира в мир книг и оставаться как можно более незаметным до конца домашней <<бури>>. Он очень рано понял, что конфронтация бы сделала только хуже. Теперь Ксавье было уже за 20, он закончил университет, и ему нужно было уже выстраивать свою собственную жизнь и начать хотя бы зарабатывать достаточно, чтобы съехать от матери, но он крайне сомневался в своих силах. Он устроился в больницу работать санитаром, чтобы меньше времени проводить дома, но удовлетворения это ему не принесло. Ему нужно было, чтобы кто-то в него верил и поддерживал его, чтобы реализовать свой потенциал.} 
\end{quotation}

Курс ОСС содержит много различных практик и упражнений, которые каждый читатель может попробовать, чтобы найти то, что наиболее эффективно именно для него. Некоторые практики нацелены на развитие навыков <<инь>>, а некоторые~"--- на <<янь>>, но большая их часть включает элементы и того, и другого. В таблице ниже приведены примеры практик из этой книги, которые соответствуют аспектам <<инь>> и <<янь>>. Разумеется, инь и янь взаимосвязаны и активно взаимодействуют. Например, когда мы признаем и валидируем свои потребности, мы часто находим мотивацию их удовлетворить.

\begin{table}[!h]
	\begin{center}
		\caption{Культивирование Инь и Янь самосострадания}\label{tab:Yin_and_Yang}
		\setlength{\extrarowheight}{1mm}
		\begin{tabular}{p{1.2cm}p{3mm}p{2.6cm}||p{9.9cm}}
			\multicolumn{3}{r}{ } & {\large \textbf{Практики}}\\[1mm]
			\cline{3-4}
			\multirow{9}{*}{{\LARGE\textbf{Инь}}} &  & \multirow{3}{*}{\textbf{Поддержка}} & {\small Перерыв на самосострадание (глава~\ref{The_Physiology_of_Self-Criticism_and_Self-Compassion}, стр.\:\pageref{IP:Self-Compassion_Break})}\\ 
			&  &  & {\small Самосострадание в повседневной жизни (глава~\ref{Backdraft}, стр.\:\pageref{IP:Self-Compassion_in_Daily_Life})}\\ 
			& $\nearrow$ &  &{\small  Медитация любящей доброты к себе (глава \ref{Loving-Kindness_for_Ourselves}, стр.\:\pageref{M:Loving-Kindness_for_Ourselves})}\\ \cline{3-4}
			& \multirow{3}{*}{\textbf{$\rightarrow$}} & \multirow{3}{*}{\textbf{Утешение}} & {\small Успокаивающие прикосновения (глава~\ref{The_Physiology_of_Self-Criticism_and_Self-Compassion}, стр.\:\pageref{IP:Soothing_Touch})}\\
			&   &   & {\small Медитация с любящим дыханием (глава~\ref{Mindfulness}, стр.\:\pageref{medit:Affectionate_Breathing})}\\
			&   &   & {\small Смягчите--утешьте--разрешите (глава \ref{Meeting_Difficult_Emotions}, стр.\:\pageref{IP:Soften–Soothe–Allow})}\\ \cline{3-4}
			& $\searrow$ & \multirow{3}{*}{\textbf{Валидация}} &{\small  Сострадание, когда все идет не так (глава \ref{Stages_of_Progress}, стр.\:\pageref{Being_a_Compassionate_Mess})}\\
			&   &   & {\small Сортировка эмоций (глава \ref{Meeting_Difficult_Emotions}, стр.\:\pageref{IP:Working_with_Difficult_Emotions})}\\
			&   &   & {\small Цените себя (глава 23)}\\ \cline{3-4}
		\end{tabular}
		\setlength{\extrarowheight}{0mm}
	\end{center}
\end{table} 
\begin{table}[!h]
	\begin{center}
		\setlength{\extrarowheight}{1mm}
		\begin{tabular}{p{1.2cm}p{3mm}p{2.6cm}||p{9.9cm}}
			\cline{3-4}
			\multirow{9}{*}{{\LARGE\textbf{Янь}}} &  & \multirow{3}{*}{\textbf{Защита}} & {\small Почувствуйте свои стопы (глава \ref{Backdraft}, стр.\:\pageref{IP:Feeling_the_Soles_of_Your_Feet})} \\ 
			&  &  & {\small Спокойное сострадание (глава 19)}\\ 
			& $\nearrow$ &  & {\small Яростное сострадание (глава 20)}\\ \cline{3-4}
			& \multirow{3}{*}{\textbf{$\rightarrow$}} & \multirow{3}{*}{\textbf{Обеспечение}} & {\small Поиск наших основных ценностей (глава \ref{Living_Deeply}, стр.\:\pageref{Ex:Discovering_Our_Core_Values})}\\
			&   &   & {\small Эмоциональных потребностей} (глава 18)\\
			&   &   & {\small Неудовлетворенных потребностей (глава 20)}\\ \cline{3-4}
			& $\searrow$ & \multirow{3}{*}{\textbf{Мотивация}} & {\small Поиск вашего сострадательного голоса (глава \ref{Self-Compassionate_Motivation}, стр.\:\pageref{EX:Finding_Your_Compassionate_Voice})}\\
			&   &   & {\small Сострадательное письмо себе (глава \ref{Self-Compassionate_Motivation}, стр.\:\pageref{IP:Compassionate_Letter_to_Myself})}\\
			&   &   & {\small Жизнь с клятвой (глава \ref{Living_Deeply}, стр.\:\pageref{IP:Living_with_a_Vow})}\\ \cline{3-4}
		\end{tabular}
		\setlength{\extrarowheight}{0mm}
	\end{center}
\end{table} 

То, что объединяет все эти практики~"--- дружеское, заботливое отношение. Иногда самосострадательная забота о себе принимает форму успокоения и мягкого обхождения со сложными эмоциями (поддержка), а иногда она диктует сказать твердое <<нет>> и отвернуться от опасности (защита). Иногда нужно дать телу знать, что все хорошо, через тепло и нежность (утешение), а иногда~"--- выяснить, что нам нужно, и дать себе это (обеспечение). Иногда самосострадание~"--- это о том, чтобы принять и открыться тому, что происходит (валидация), а иногда~"--- о том, чтобы начать что-то активно делать (мотивация). 

\begin{quotation}
	\textit{У Моник были хорошо развиты качества <<янь>>: сила, решимость и готовность к действию~"--- и она умело их использовала, когда ее безопасности и благополучию что-то угрожало. Она умела защищать и обеспечивать себя. Но более восприимчивая сторона <<инь>> у нее хромала~"--- возможно, потому что в детстве было небезопасно просто быть восприимчивой и принимать происходящее. Диагноз Моник означал, что ей нужно было освоить новые навыки, чтобы справиться. Подруга Моник рассказала ей об упражнении <<Перерыв на самосострадание>> (глава~\ref{The_Physiology_of_Self-Criticism_and_Self-Compassion}, стр.\:\pageref{IP:Self-Compassion_Break}), которое представляет собой комбинацию разных элементов самосострадания, делая при этом особенный упор на валидацию (<<очень страшно получить диагноз рассеянного склероза>>), понимание, что она не одна (<<обнаружив, что у них серьезное заболевание, почти все чувствуют себя уязвимыми и одинокими>>) и последующее утешение себя: <<Все будет хорошо. Давай сначала попробуем просто жить одним днем>>. <<Перерыв на самосострадание>> открыл для Моник двери самосострадания. Это был нелегкий путь из-за огромного количества боли, которую ей причинили в разных отношениях, когда она была моложе и уязвимее, но у Моник было достаточно смелости. У рассеянного склероза были и свои плюсы~"--- так как ей пришлось принять свое заболевание, Моник стала испытывать внутреннее спокойствие от принятия, которое она раньше не считала возможным.}
		
	\textit{Ксавье, наоборот, был не так напорист, но зато у него было нежное сердце. Его напористость подавил в нем сердитый отчим, которому всегда нужно было, чтобы его слово было последним, и он научился умело избегать конфликтов, оставаясь в тени. Но теперь ему необходимы были сила и смелость, чтобы выйти в большой мир. Практически случайно он наткнулся у себя в больнице на листовку о коротком тренинге по самосостраданию для работников здравоохранения. Тогда-то он и понял, что тот самый внутренний голос, который говорил ему оставаться в безопасности, делая себя незаметным, теперь говорил ему выйти из своего <<панциря>> Лучшей практикой самосострадания для Ксавье оказалось написание сострадательного письма (глава \ref{Self-Compassionate_Motivation}, стр.\:\pageref{IP:Compassionate_Letter_to_Myself}), похожее на то, какое он бы написал близкому другу в похожей ситуации. Он начал писать себе письма каждую неделю, заостряя внимание на трудностях, с которыми он встречался. Понемногу Ксавье стал замечать в себе новый голос~"--- своего собственного внутреннего тренера, болеющего за него со скамейки запасных. Со временем он смог найти в себе то, что ему было нужно, чтобы наполнить свою жизнь смыслом~"--- свои основные ценности, и предпринять практические шаги, чтобы воплотить их в жизнь.}
\end{quotation}

\newpage
\Exercises{Какие аспекты самосострадания мне сейчас нужны?}

Наверняка самосострадание включает в себя больше различных аспектов, чем вы изначально думали. Внизу приведен список некоторых <<инь>> и <<янь>>~"--- свойств самосострадания. Проглядите его и подумайте, какими из них вам нужно воспользоваться прямо сейчас. Это поможет вам понять, как самосострадание может быть для вас полезным.

\vspace{3ex}

\noindent{\Large Инь}
\begin{itemize}
	\itemyinyang \textit{Поддержка}. Поддержка~"--- это что-то, что мы можем дать близкому другу, который попал в трудную ситуацию. Это означает помочь кому-то, кто страдает, почувствовать себя лучше, особенно внимательно относясь к его эмоциональным потребностям. Нужно ли вам что-то такое сейчас? Как вы думаете, вам бы помогло научиться поддерживать себя, когда вы расстроены?
\end{itemize}

\setlength{\extrarowheight}{2mm}
\begin{tabularx}{\textwidth}{X}
	\\
	\arrayrulecolor{gray}\hline\\
	\hline\\
	\hline\\
	\hline\\
	\hline\\
	\hline\\
	\hline\\
	\hline\\
	\hline\\
	\hline\\
	\hline\\
	\hline\\
	\hline\\
	\hline\\
	\hline\\
	\hline\\
	\hline\\	
\end{tabularx}
\setlength{\extrarowheight}{0mm}

\begin{itemize}
	\itemyinyang \textit{Утешение}. Утешение~"--- это еще один способ помочь человеку почувствовать себя лучше, особенно физически его успокоив. Может быть, это что-то, в чем вы нуждаетесь? Хотелось ли бы вам чувствовать себя комфортнее и более расслабленно в своем теле? 
\end{itemize}

\setlength{\extrarowheight}{2mm}
\begin{tabularx}{\textwidth}{X}
	\\
	\arrayrulecolor{gray}\hline\\
	\hline\\
	\hline\\
	\hline\\
	\hline\\
	\hline\\
	\hline\\
	\hline\\
	\hline\\
	\hline\\
	\hline\\
\end{tabularx}
\setlength{\extrarowheight}{0mm}

\begin{itemize}
	\itemyinyang \textit{Валидация}. Можно также помочь кому-нибудь почувствовать себя лучше, понимая, что она сейчас испытывает, и произнося это добрым и нежным тоном. Возможно, вы чувствуете себя одинокими или непонятыми и нуждаетесь в такой валидации? Как вы думаете, было ли бы для вас полезно научиться валидировать свои чувства?
\end{itemize}

\setlength{\extrarowheight}{2mm}
\begin{tabularx}{\textwidth}{X}
	\\
	\arrayrulecolor{gray}\hline\\
	\hline\\
	\hline\\
	\hline\\
	\hline\\
	\hline\\
	\hline\\
	\hline\\
	\hline\\
	\hline\\
\end{tabularx}
\setlength{\extrarowheight}{0mm}


\noindent{\Large Янь}
\begin{itemize}
	\itemyinyang \textit{Защита}. Первый шаг к самосостраданию~"--- это чувство безопасности от вреда. Защищать себя значит говорить <<нет>> другим людям, которые приносят нам боль, или тому вреду, который мы часто себе неосознанно наносите. Возможно, вам каким-то образом кто-то или что-то вредит и вам хотелось бы найти в себе силы это остановить? 
\end{itemize}

\setlength{\extrarowheight}{2mm}
\begin{tabularx}{\textwidth}{X}
	\\
	\arrayrulecolor{gray}\hline\\
	\hline\\
	\hline\\
	\hline\\
	\hline\\
	\hline\\
	\hline\\
	\hline\\
	\hline\\
	\hline\\
\end{tabularx}
\setlength{\extrarowheight}{0mm}

\begin{itemize}
	\itemyinyang \textit{Обеспечение}. Это значит давать себе то, что нам действительно нужно. Сначала нам нужно узнать, в чем мы нуждаемся, потом~"--- убедить себя в том, что мы этого заслуживаем, а потом попытаться воплотить это в жизнь. Никто не может сделать это для нас лучше, чем мы сами. Хотелось бы вам научиться обеспечивать свои потребности более эффективно?
\end{itemize}

\setlength{\extrarowheight}{2mm}
\begin{tabularx}{\textwidth}{X}
	\\
	\arrayrulecolor{gray}\hline\\
	\hline\\
	\hline\\
	\hline\\
	\hline\\
	\hline\\
	\hline\\
	\hline\\
	\hline\\
	\hline\\
\end{tabularx}
\setlength{\extrarowheight}{0mm}

\begin{itemize}
	\itemyinyang \textit{Мотивация}. У большинства из нас есть мечты и стремления, которые мы бы хотели реализовать в течение жизни. У нас также есть менее значимые краткосрочные цели. Самосострадание, как хороший тренер, мотивирует добротой, поддержкой и пониманием, а не жесткой критикой. Было ли бы для вас полезным научиться мотивировать себя любовью, а не страхом? 
\end{itemize}

\setlength{\extrarowheight}{2mm}
\begin{tabularx}{\textwidth}{X}
	\\
	\arrayrulecolor{gray}\hline\\
	\hline\\
	\hline\\
	\hline\\
	\hline\\
	\hline\\
	\hline\\
	\hline\\
	\hline\\
	\hline\\
	\hline\\
	\hline\\
	\hline\\
\end{tabularx}
\setlength{\extrarowheight}{0mm}


\Reflection{
	Мы надеемся, что теперь вопрос <<\textbf{В чем я сейчас нуждаюсь?}>> будет постоянно возникать в вашей голове на протяжении всей работы с этим пособием. Просто задавая себе этот вопрос, вы позволяете себе момент самосострадания, даже если вы не можете на него ответить или вы не в состоянии удовлетворить свои потребности в какой-то момент времени.
}

%%% end section %%%