% !TEX root = ../../Self-Compassion.tex

\chapter{Самосострадание и мотивация} \label{Self-Compassionate_Motivation}

Одно из самых больших препятствий на пути к самосостраданию~"--- это убежденность, что она разрушит нашу мотивацию. Обычно мысли людей по этому поводу звучат примерно так: <<Если я переборщу с самосостраданием, я же наверняка только и буду делать, что весь день буду сидеть в Интернете и есть вредную еду?>>\cite{72}. Проведем аналогию: разве хорошая мать, которая заботится о своем чаде и относится к нему с состраданием, разве она позволит ему делать все, что ему вздумается (например, целый день сидеть в Интернете и есть вредную еду)? Нет, конечно. Она напоминает ему пойти в школу, сделать домашнее задание, вовремя лечь спать. Так почему же вы думаете, что самосострадание работает по-другому? 

А что если мать хочет \emph{мотивировать} ребенка на какие-то изменения? Скажем, сын-подросток приходит домой из школы с очень плохими оценками по математике. Перед матерью стоит выбор, как лучше помочь ему подтянуться. Можно прибегнуть к жесткой критике: <<Мне за тебя стыдно. Ты ничего не стоящий неудачник>>. Вас даже коробит от этих слов, не правда ли? (И, тем не менее, разве мы не говорим себе похожие ужасные вещи, когда у нас что-то не получается?) Работает ли этот метод? Временно~"--- возможно. Мальчик какое-то время будет заниматься усиленно, чтобы не сердить мать, но в долгосрочной перспективе он наверняка потеряет веру в свои способности к математике, начнет бояться неудач и вряд ли ему захочется когда-нибудь заниматься продвинутой математикой. 

\begin{quotation}
	\textit{
		Билл был очень успешным компьютерным инженером в Силиконовой долине. Он был лучше всех в своем потоке в Калифорнийском университете в Беркли, а сейчас задумывался о том, чтобы открыть собственный бизнес и создавать интересные, передовые операционные системы. Всю жизнь Билл мотивировал себя неустанной критикой. Когда, например, в университете он получал пятерки с минусом за экзамены, он начинал безжалостно себя унижать. <<Ну вот что ты за неудачник? Если ты не написал экзамен лучше всех в потоке~"--- это, считай, провал. Тебе должно быть стыдно, что ты не получил пять>>. Во взрослой жизни он продолжал использовать этот подход, чтобы себя замотивировать, и искренно верил, что, если не держать себя в ежовых рукавицах, он быстро начнет отлынивать от работы.
	}

	\textit{	
		В итоге у Билла возникала сильная тревожность, когда он пытался продвигаться дальше со своим бизнесом. Что если у него не получится? Что если этот новый проект просто докажет всем, что он, Билл, идиот и неудачник? Что он самозванец? Билл так замучил себя мыслями о цене неудачи и его жизнь стала такой несчастной, что ему становилось легче только тогда, когда он подумывал о том, чтобы сдаться и не исполнять свою мечту.
	}
\end{quotation}


Но у матери есть и другой способ мотивировать сына, помочь ему оправиться от неудачи и достичь успеха~"--- \textbf{проявить к нему сострадание}. Например: <<О господи, дорогой, ты, наверное, так расстроился, Иди сюда, обниму тебя. Ты знаешь, что я тебя люблю и буду любить всегда несмотря ни на что>>. Тем самым она дает сыну понять, что его могут принимать даже тогда, когда у него что-то не получается. Но мать, которая заботится о сыне и относится к нему с состраданием, на этом не остановится. Есть еще и элемент действия. Она наверняка добавит что-то вроде: <<Я знаю, что ты хочешь поступить в университет, и для этого тебе, конечно, нужно хорошо написать вступительные экзамены. Чем я могу тебе помочь? Я знаю, что, если ты постараешься, у тебя все получится. Я в тебя верю>>.
		
Такая поддержка и ободрение будут гораздо более эффективны в долгосрочной перспективе. Исследования показали, что у людей, проявляющих к себе сострадание, не только больше веры в себя\cite{73}, они еще и не боятся неудач\cite{74} и, когда у них что-то не получается, они попробуют начать заново и продолжать прикладывать усилия\cite{75}.

Важно понимать, зачем и \emph{почему} мы себя критикуем. Это ужасно больно и неприятно, зачем мы тогда так делаем? 

Как было упомянуто в главе \ref{The_Physiology_of_Self-Criticism_and_Self-Compassion} на стр.\:\pageref{The_Physiology_of_Self-Criticism_and_Self-Compassion}, самокритика прочно закреплена в системе защиты от угроз. На каком-то уровне наш внутренний критик пытается заставить нас что-то изменить, чтобы быть в \emph{безопасности}. Например, зачем мы себя ругаем за то, что мы не в форме? Потому что нам страшно, что наш организм развалится и перестанет нормально работать. Почему мы критикуем себя за то, что мы постоянно откладываем выполнение важного задания по работе? Потому что нам страшно потерпеть неудачу, потерять работу и оказаться на улице. Наш внутренний критик постоянно пытается помочь нам избежать каких-то опасных ситуаций, которые могут нам навредить. Конечно, этот внутренний критик нам особо не помогает~"--- потому что его подход контрпродуктивен~"--- но у него хорошие намерения. Понимая это, мы можем постараться превратить голос критика внутри нас во что-то менее жесткое и непрощающее. Мы можем научиться мотивировать себя \emph{новым} голосом~"--- голосом того <<я>>, которое вам \emph{сострадает}. 

\begin{quotation}
	\textit{
		Сначала Биллу было сложно проявлять к себе сострадание из-за страха, что, если он даст себе передышку от самокритики, он станет меньше работать и забросит свои цели. Как ни парадоксально, в реальности все оказалось с точностью до наоборот. Внутренний критик Билла был таким жестким, что он до смерти боялся неудачи и ему было трудно справляться даже с пустяковыми проблемами. Это привело к тому, что он постоянно откладывал работу над своим бизнесом и не предпринимал шагов к исполнению своей мечты. Билл понимал, что в этом виноват его безжалостный внутренний голос, и решил, что для достижения прогресса нужно что-то менять.
	}
	
	\textit{
		В этот период у Билла был тренер в спортзале примерно его же возраста, который его очень поддерживал. Например, когда Билл свалился на пол, делая отжимания, его тренер сказал: <<Отлично! Работа до мышечной усталости~"--- это как раз то, что нам нужно>>. Когда Билл пытался поднимать гантели, которые для него были слишком тяжелыми и наверняка нанесли бы какое-то повреждение, он сказал: <<Билл, давай вот эти оставим на потом. Их время придет раньше, чем ты думаешь>>. Вспомнив о нем, Билл решил перенести такое же отношение на свой новый бизнес-проект. <<Просто попробуй, "--*~сказал он себе. "--*~Я знаю, что ты можешь>>. Он представил себе, что бы сказал его тренер при любой неудаче: <<Держись, бро. Мы это сделаем>>. Бил начал потихоньку открывать для себя свой внутренний голос сострадания и научился поддерживать себя, а не ставить себе палки в колеса. В конце концов он уволился с работы, получил начальный капитал для своего проекта и начал наконец жить той жизнью, которая ему была нужны~"--- жизнью, которая делала его счастливым.
	}
\end{quotation}

\vspace{3ex}

\begin{center}
	{\LARGE Мотивация самосострадания рождается из любви, а мотивация самокритики~"--- из страха.
	\\
	\vspace{1ex}
	Любовь сильнее страха.}
\end{center}

\newpage
\Exercises{Поиск своего голоса сострадания} \label{EX:Finding_Your_Compassionate_Voice}

Это упражнение поможет вам услышать голос критика внутри вас, понять, как и чем он пытается вам помочь, и научиться мотивировать себя новым голосом~"--- голосом вашего внутреннего самосострадания. 

Иногда кажется, что внутреннему критику на самом деле нет дела до ваших интересов и до того, что лучше всего для вас. Это происходит особенно часто, если наш внутренний критик дублирует голос кого-то из нашего прошлого, кто применял к нам насилие (физическое или психологическое). Делая это упражнение, пожалуйста, отнеситесь к себе с состраданием. Если вы чувствуете, что вам некомфортно, оставьте его и вернитесь к нему только тогда, когда почувствуете себя сильным и готовым. До выполнения этого упражнения может быть полезным перечитать секцию <<советы для практик>> из введения этого пособия.

\begin{itemize}
	\itemWritingHand В пустом пространстве ниже запишите какую-нибудь свою привычку или поведение, которое вам хотелось бы изменить и за которое вы часто себя ругаете. Выберите что-то, что вам в жизни мешает и доставляет вам недовольство и дискомфорт, но постарайтесь, чтобы это было что-то не слишком сложное. Также смотрите, чтобы привычку или поведение можно было бы изменить (не выбирайте постоянную характеристику вроде <<у меня слишком большой размер ноги>>). Например, <<я нетерпелив>>, <<у меня мало физической активности>>, <<я постоянно откладываю всю работу на потом>>.
\end{itemize}

\setlength{\extrarowheight}{2mm}
\begin{tabularx}{0.96\textwidth}{X}
	\\
	\arrayrulecolor{gray}\hline\\
	\hline\\
	\hline\\
	\hline\\
	\hline\\
	\hline\\	
	\hline\\
	\hline\\
	\hline\\
	\hline\\
	\hline\\
	\hline\\
\end{tabularx}
\setlength{\extrarowheight}{0mm}

\noindent{\large \textbf{Идентификация вашего самокритичного голоса}}
\begin{itemize}
	\itemWritingHand Запишите, что вы обычно говорите себе, когда замечаете себя за этим поведением. Иногда внутренний критик оказывается очень жестким, а иногда он проявляет себя скорее через чувство уныния и упадка духа или еще что-то другое. Какие слова он использует, и, что еще более важно, каким \emph{тоном} говорит? А может быть так, что он вообще обходится без слов, просто в голове появляется какое-то изображение. Как ваш внутренний критик выражает себя? 
\end{itemize}

\setlength{\extrarowheight}{2mm}
\begin{tabularx}{0.96\textwidth}{X}
	\\
	\arrayrulecolor{gray}\hline\\
	\hline\\
	\hline\\
	\hline\\
	\hline\\
	\hline\\	
	\hline\\
	\hline\\
	\hline\\
\end{tabularx}
\setlength{\extrarowheight}{0mm}
\begin{itemize}
	\itemWritingHand Теперь понаблюдайте за собой и попытайтесь заметить, что вы чувствуете, критикуя себя. Подумайте о том, сколько проблем и страдания принес вам голос самокритики. Если хотите, можете проявить к себе сострадание, валидируя свои чувства, признавая, что слышать такие жесткие слова очень тяжело: <<это действительно трудно>>, <<мне так жаль, я знаю, как больно такое слышать>>.
\end{itemize}

\setlength{\extrarowheight}{2mm}
\begin{tabularx}{0.96\textwidth}{X}
	\\
	\arrayrulecolor{gray}\hline\\
	\hline\\
	\hline\\
	\hline\\
	\hline\\
	\hline\\	
	\hline\\
	\hline\\
	\hline\\
\end{tabularx}
\setlength{\extrarowheight}{0mm}
\begin{itemize}
	\itemWritingHand Задумайтесь, \emph{почему} эта критика продолжается так долго. Может быть, ваш внутренний критик пытается вас как-то защитить, оградить от опасности, помочь вам, даже если конечный результат был совсем не таким? Если да, напишите, что, как вы думаете, может мотивировать вашего внутреннего критика.
\end{itemize}

\setlength{\extrarowheight}{2mm}
\begin{tabularx}{0.96\textwidth}{X}
	\\
	\arrayrulecolor{gray}\hline\\
	\hline\\
	\hline\\
	\hline\\
	\hline\\
	\hline\\	
	\hline\\
	\hline\\
	\hline\\
\end{tabularx}
\setlength{\extrarowheight}{0mm}
\begin{itemize}
	\itemWritingHand Если вы не можете найти вообще ничего, с чем ваш внутренний критик мог бы пытаться вам помочь~"--- иногда самокритика действительно не представляет никакой ценности~"--- не идите дальше. Просто продолжайте сострадать себе за то, как вам было больно из-за самокритики в прошлом. Если вы все-таки идентифицировали что-то, с чем ваш внутренний критик пытается вас помочь или что-то, от чего хочет вас защитить, попробуйте признать, как он старался, может быть, даже запишите несколько слов благодарности. Дайте своему внутреннему критику знать, что, хоть сейчас он и не приносит вам особой пользы, у него были хорошие намерения и он старался как мог.
\end{itemize}

\setlength{\extrarowheight}{2mm}
\begin{tabularx}{0.96\textwidth}{X}
	\\
	\arrayrulecolor{gray}\hline\\
	\hline\\
	\hline\\
	\hline\\
	\hline\\
	\hline\\	
	\hline\\
	\hline\\
	\hline\\
\end{tabularx}
\setlength{\extrarowheight}{0mm}

\noindent{\large \textbf{Поиск голоса сострадания}}
\begin{itemize}
	\item Теперь, после того, как вы выслушали свой голос самокритики, попробуйте освободить место для другого голоса~"--- вашего внутреннего голоса сострадания. Он происходит из какой-то очень мудрой части вас, которая понимает, как выбранное вами поведение вам вредит. Он тоже хочет, чтобы вы что-то изменили, но по абсолютно отличным от голоса самокритики причинам.
	
	\item Положите руки на сердце и почувствуйте их тепло. Теперь опять подумайте о поведении, с которым вы пытаетесь бороться. Начните повторять следующую фразу, которая выражает всю суть вашего голоса самосострадания:
	\begin{itemize}
		\item <<Я люблю тебя и не хочу, чтобы ты страдал>>. 
		
		\item Или, если это для вас звучит естественнее и искреннее, скажите что-нибудь вроде <<я очень о тебе беспокоюсь, поэтому я хотел бы помочь тебе что-то изменить>> или <<я всегда с тобой и всегда тебя поддержу>>.	
	\end{itemize}

	\itemWritingHand Когда вы готовы, напишите себе послание как бы под диктовку вашего внутреннего \emph{голоса сострадания}. Пишите свободно и спонтанно, затрагивая поведение, которое вы бы хотели изменить. Что исходит из глубокого чувства и пожелания <<я люблю тебя и не хочу, чтобы ты страдал>>? Что вам нужно услышать, чтобы решиться на шаг к изменениям? Если вам сложно найти слова, пишите то, что вы бы сказали близкому другу в похожей ситуации.
\end{itemize}

\setlength{\extrarowheight}{2mm}
\begin{tabularx}{0.96\textwidth}{X}
	\\
	\arrayrulecolor{gray}\hline\\
	\hline\\
	\hline\\
	\hline\\
	\hline\\
	\hline\\	
	\hline\\
	\hline\\
	\hline\\
	\hline\\
	\hline\\
	\hline\\
	\hline\\
\end{tabularx}
\setlength{\extrarowheight}{0mm}

\Reflection{
	Как для вас прошло это упражнение? Смогли ли вы идентифицировать внутренний голос самокритики? Поняли ли вы, как и чем этот критик внутри вас пытается вам помочь? Было ли правильным благодарить внутреннего критика за его старания? 
	
	Какой эффект произвели слова <<я тебя люблю и не хочу, чтобы ты страдал>>? Смогли ли вы установить контакт с вашим внутренним голосом сострадания? Получилось ли у вас писать от его лица? 
	
	Если вы нашли слова, пришедшие от вашего внутреннего голоса сострадания, позвольте себе \emph{насладиться} чувством, что вас поддерживают. Если вам \emph{трудно} было найти слова любви, в этом тоже нет ничего плохого. Для этого нужно время. Самое главное~"--- искреннее намерение проявлять больше сострадания к себе, а новые привычки со временем сформируются. 
	
	Для многих людей это очень сильное и полезное упражнение. Открытие, что наш внутренний критик на самом деле пытается нам помочь, помогает нам перестать осуждать себя за то, что мы осуждаем себя. Когда мы видим, что внутренний критик пытается обеспечить нашу безопасность, крича <<Опасность! Опасность!>>, валидируем его усилия и благодарим критика за его хорошие намерения, он обычно успокаивается и уступает место другому голосу~"--- голосу самосострадания. (Если хотите узнать больше об этой методике, можете изучить модели внутренних семейных систем Ричарда Шварца \cite{76}).
	
	Многие люди находят удивительным, что и наш внутренний критик, и наш внутренний голос сострадания на самом деле пытаются добиться одного и того же изменения поведения~"--- просто качество и тон этого послания очень различаются. В качестве юмористического отвлечения вспомним, что один участник курса ОСС однажды нам сказал: <<Это просто удивительно. Мой внутренний критик на меня раньше всегда кричал: <<Ты тварь>>! А мой внутренний голос сострадания просто говорит: <<Ого, Тигр>>... 
	
	Некоторые читатели после этого упражнения могут испытать обратную тягу. Если с вами это происходит, обратитесь к главе \ref{Backdraft} на стр.\:\pageref{Backdraft} за советами о том, как с ней справляться: например, назвать эмоцию своим именем, выйти на прогулку и прочувствовать ощущения в стопах или просто заняться чем-то привычным и приятным. Иногда сострадание к себе может выражаться в том, чтобы поговорить с друзьями или немного отдохнуть от практики самосострадания.
}

\newpage
\InformalPractices{Сострадательное письмо себе} \label{IP:Compassionate_Letter_to_Myself}

Вы можете продолжать слушать свой внутренний голос сострадания, если каждый раз, когда сталкиваетесь с трудностями или когда хотите мотивировать себя на изменения, будете писать письмо себе. Для написания письма есть три основных метода;

\begin{itemize}
	\item Придумайте для себя воображаемого друга, который независимо от обстоятельств всегда мудрый, любящий и проявляет сострадание, и напишите письмо себе от лица этого друга;
	\item Напишите письмо, как будто вы разговариваете с любимым другом, у кого такие же трудности, как у вас;
	\item Напишите письмо от сострадательной части вас той части, которая испытывает трудности.
	После написания письма можете его отложить и перечитывать позже, когда вам больше всего нужно, чтобы эти слова вас утешили и успокоили. 
\end{itemize}

Чтобы вы почувствовали себя комфортно при написании себя письма голосом хорошего друга, может потребоваться немного времени, но с опытом это становится легче. Ниже приведен образец письма. Это письмо написала себе Карен, подающий надежды графический дизайнер, о том, что она проводит не так много времени, как ей хотелось бы, со своими двумя детьми, одному из которых 13 лет, а другой 8. Она писала его как будто от лица лучшей подруги, с которой у нее очень близкие отношения.

\begin{quotation}
	\textit{
		Дорогая Карен, я знаю, что ты винишь себя за то, что проводишь с детьми недостаточно времени. Тебе пришлось пропустить балетную репетицию маленькой Софи, а Бену на прошлой неделе пришлось два раза подогреть свой ужин в микроволновке, когда ты задержалась на работе. Но не ругай себя за это, пожалуйста. Мне больно, когда ты так делаешь. Ты хорошая мать, и время, которое ты проводишь с детьми, им ужасно нравится. Очень тяжело найти баланс между карьерой и семьей, и тебе нужно дать себе отдохнуть. Ты стараешься, как только можешь, и, как я это вижу, у тебя все прекрасно получается. Твои дети очень тебя любят. Я тоже очень тебя люблю. Я знаю, что тебе бы хотелось не работать допоздна, чтобы проводить больше времени с Софи и Беном~"--- может быть, имеет смысл обсудить это с начальником и рассказать ему о твоем беспокойстве. Ты уже семь лет работаешь в этой компании и доказала, что ты хороший работник~"--- тебе можно и нужно просить то, в чем ты нуждаешься. Самое плохое, что может случиться~"--- он скажет <<нет>>. И даже если ничего не поменяется, ты все равно хорошая любящая мать. Не забывай об этом, пожалуйста.
	}
\end{quotation}
