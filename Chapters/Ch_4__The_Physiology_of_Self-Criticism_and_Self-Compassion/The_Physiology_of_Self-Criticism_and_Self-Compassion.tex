% !TEX root = ../../Self-Compassion.tex

\chapter{Физиология самокритики и самосострадания} \label{The_Physiology_of_Self-Criticism_and_Self-Compassion}

По словам Пола Гилберта, создателя \textbf{методики CFT (психотерапия, основанная на сострадании)}, когда мы критикуем себя, мы задействуем мозговые функции, отвечающие за распознавание угроз и защиту (или, другими словами, систему защиты против угроз, которую еще часто называют <<рептильным мозгом>>). Из всех регионов мозга, которыми мы можем реагировать на что-то, кажущееся опасным, этот легче и быстрее всего активируется. Это значит, что часто, когда что-то идет не так, наша рефлекторная реакция~"--- самокритика. Система защиты против угроз так запрограммирована, что, когда мы чувствуем угрозу, наше миндалевидное тело (часть мозга, отвечающая за восприятие опасности) активируется, наш организм производит больше кортизола и адреналина, и мы готовимся либо начать схватку, либо убежать, либо замереть на месте. Эта система хорошо выполняет свою функцию, защищая нас от физических угроз нашему организму, но в наше время большинство угроз, с которыми мы встречаемся~"--- это те, которые ставят под сомнение нашу самооценку, самомнение, видение себя. Чувство, что нам что-то угрожает~"--- стресс для мозга и для всего организма, а хронический стресс может привести к депрессии и тревожным расстройствам. Именно поэтому самокритика так плохо влияет на физическое и эмоциональное здоровье. В случае самокритики мы сами в одно и то же время хищник и жертва. 

К счастью, в нашем мозге есть что-то не только от рептилий, но и от млекопитающих. Эволюционное превосходство млекопитающих обусловлено тем, что их детеныши рождаются беспомощными, и им нужен более долгий период развития, чтобы приспособиться к окружающей их среде. Чтобы обеспечить им безопасность в этот период уязвимости, механизм заботы о потомстве у млекопитающих эволюционировал, способствуя более близким отношениям родителей и детенышей. Когда этот механизм активирован, организм детеныша производит окситоцин (гормон любви) и эндорфины (природные <<опиаты>>, улучшающие эмоциональное состояние), что помогает уменьшить стресс и напряжение и  создать чувство безопасности. Два надежных способа активации этого механизма~"--- это успокаивающие прикосновения и мягкий голос (представьте себе кошку, которая мурлыкает и вылизывает своих котят).  

Сострадание, включая самосострадание, связано с механизмом заботы млекопитающих. Поэтому практика самосострадания, когда мы чувствуем себя несостоявшимися или неполноценными, дает нам ощущение безопасности и заботы, как у ребенка, которого нежно обнимает мать. Самосострадание помогает отрегулировать и смягчить реакцию на <<угрозу>>. Когда угроза нашей самооценке провоцирует рефлекторную реакцию (бей-беги-замри), эта <<порочная троица>> реакций с большой долей вероятности обратится против нас самих. Мы вступаем в борьбу с собой (самокритика) и убегаем от других (самоизоляция), а наши мысли как бы замирают на месте (мы углубляемся в негативные раздумья по поводу происходящего). Эти три реакции~"--- полная противоположность трех компонентов самосострадания (доброты к себе, человеческой общности и осознанности). Таблица ниже это демонстрирует более наглядно.

\vspace{4ex}

\noindent
\begin{minipage}{\textwidth}
	\begin{center}
		\setlength{\extrarowheight}{3mm}
		\begin{tabular}{p{2.5cm}p{5.5cm}c}
			\textbf{\makecell[c]{Реакция на}\linebreak\makecell[c]{стресс}}
			& 
			\textbf{\makecell[c]{Реакция на стресс,}\linebreak\makecell[c]{направленная}\linebreak\makecell[c]{внутрь себя}}
			& 
			\textbf{Самосострадание}\\[12mm]
			\hline \hline 
			\makecell[c]{Бей}	& \makecell[c]{Самокритика} & Доброта к себе \\
			\makecell[c]{Беги} & \makecell[c]{Самоизоляция}	& Человеческая общность \\
			\makecell[c]{Замри} & \makecell[c]{Руминации\footnote{Зацикленность на негативных мыслях}} & Осознанность \\[4mm]
		\end{tabular}
		\setlength{\extrarowheight}{0mm}
	\end{center}
\end{minipage}

\vspace{4ex}

\textbf{Практикуя самосострадание, мы выключаем систему защиты против угроз и активируем механизм заботы.} В одном исследовании, например, исследователи попросили участников эксперимента представить себе, что к ним относятся с состраданием и что они чувствуют его на телесном уровне. Каждую минуту им говорили что-то вроде <<разрешите себе почувствовать, что вы получаете сочувствие и сострадание; позвольте себе ощутить исходящую от них любящую доброту>>. Результаты исследования показали, что у участников, которым были даны такие инструкции, был ниже уровень кортизола в организме, чем у контрольной группы. После эксперимента у участников также увеличилась вариативность сердечного ритма. Чем больше люди чувствуют себя в безопасности, тем более открытыми и гибкими они могут быть по отношению к окружающей их среде, и это отражается в том, насколько их пульс меняется в ответ на внешние стимулы. Таким образом, можно сказать, что, когда участники относились к себе с состраданием, их сердца буквально открылись миру.

\vspace{4ex}

\begin{quotation}
	\textit{Томас был хорошим, сознательным человеком, занимался волонтерской работой в церкви, на него всегда можно было положиться и обратиться к нему за помощью. А еще он был неутомимым самокритиком. Он критиковал себя почти за все~"--- он считал себя недостаточно успешным, недостаточно умным, недостаточно щедрым. Словом, он был слишком самокритичен! Как только Томас замечал что-то, что ему в себе не нравилось, или делал что-то не так, он сразу же начинал себя оскорблять. <<Дебил. Тупой дурак. Неудачник>>. Постоянная самокритика его утомляла, и в конце концов он впал в депрессию. Узнав, что самокритика связана с чувством угрозы, Томас задумался, чего же он так боится, что этот страх делает его таким самокритичным. Ему сразу стало ясно, что он боится быть отвергнутым. В детстве его ужасно дразнили и задирали за то, что у него были проблемы с учебой, и он никогда не чувствовал себя частью коллектива. Внутри него было что-то, внушающее ему веру в то, что если он будет постоянно критиковать себя за свои недостатки, это каким-то волшебным образом его смотивирует на работу над собой, чтобы другие его приняли, а также защитит его от страха осуждения: никто не мог быть к нему строже, чем он сам. Разумеется, самокритика ему не помогала~"--- она только вогнала его в депрессию.}
	
	\textit{Томас узнал также, что можно почувствовать себя в безопасности, активируя механизм заботы. Для этого нужны были совсем простые вещи, например, разговаривать с собой дружелюбным и понимающим тоном. И Томас решил попробовать. Когда начиналась привычная череда оскорблений, он ловил себя на этом: <<Я вижу, что тебе страшно и ты пытаешься защититься>>. Потом он начал добавлять вещи вроде <<Все в порядке. Ты не идеален, но ты стараешься, как можешь>>. Хотя привычка критиковать себя и была сильной, понимание того, в чем ее причины, помогло Томасу не погрязнуть окончательно в этом болоте и дало ему надежду, что когда-нибудь он сможет относиться к себе с добротой и принятием, которых ему так не хватало в детстве.}
\end{quotation}

\newpage
\InformalPractices{Успокаивающие прикосновения}\label{IP:Soothing_Touch}

С непривычки это может показаться похожим на какие-то <<розовые сопли>>, но на самом деле очень полезно уметь использовать силу физического прикосновения, чтобы вызвать реакцию сострадания. Тепло, заботливо и ласково положив одну или обе руки на ваше тело, вы поможете себе почувствовать себя в безопасности. Важно понимать, что разные физические жесты вызывают разные эмоциональные реакции у разных людей. Ваша цель~"--- найти то самое прикосновение, которое действительно дает вам ощущение поддержки, и в дальнейшем использовать его, чтобы заботиться о себе в стрессовых ситуациях. 

Найдите уединенное место, где вы можете не волноваться о том, что вас кто-то увидит. Внизу приведен список разных вариантов прикосновений. Попробуйте проделать их все, а также придумать ваши собственные и поэкспериментировать с ними. Можно закрыть глаза, чтобы сосредоточиться на тактильных ощущениях. 

\begin{itemize}
	\itemheart Одна ладонь на сердце
	\itemheart Две ладони на сердце
	\itemheart Мягкое поглаживание груди
	\itemheart Ладонь <<чашечкой>> на кулаке другой ладони на сердце
	\itemheart Одна ладонь на сердце, другая на животе
	\itemheart Обе ладони на животе 
	\itemheart Лицо в ладонях
	\itemheart Нежное поглаживание рук
	\itemheart Перекрещенные руки, как бы обнимаете самого себя 
	\itemheart Одна рука нежно держит другую
\end{itemize}

\vspace{2ex}

\textit{Продолжайте поиски, пока не найдете прикосновение, которое действительно вас успокаивает~"--- у всех оно разное.}

\newpage
\Reflection{Как для вас прошла эта практика? Удалось ли вам найти прикосновение, которое вас утешило и успокоило? Если да, попробуйте обращаться к нему каждый раз, когда вы сталкиваетесь со стрессом или эмоциональной болью. Если ваше тело чувствует заботу и ощущает себя в безопасности, вашим уму и сердцу будет легче почувствовать себя так же.

Иногда успокаивающие прикосновения могут быть неловкими или некомфортными. Часто возникает <<обратная тяга>>~"--- понятие, о котором мы расскажем подробнее в главе \ref{Backdraft} на стр.\pageref{Backdraft}. Обратная тяга~"--- это старая боль, которая всплывает на поверхность, когда мы даем себе доброту: например, воспоминания о временах, когда к вам относились отнюдь не добро. Вот почему успокаивающие прикосновения могут совсем не успокоить. Если с вами такое случается, можно попробовать вместо себя прикасаться к теплому и мягкому объекту: например, гладить кошку или собаку или держать в руках подушку. Или, может быть, вам подойдет более твердое прикосновение, такое, как постукивание по груди кулаком. Конечная цель~"--- выразить заботу и доброту так, чтобы удовлетворить ваши потребности.
}

\newpage
\InformalPractices{Перерыв на самосострадание}\label{IP:Self-Compassion_Break}

Эта практика~"--- способ напомнить себе применить три основных компонента самосострадания: осознанность, человеческую общность и доброту~"--- когда в жизни возникают трудности. Она задействует силу успокаивающего прикосновения для создания ощущения заботы и безопасности. Важно найти слова, которые для вас лично будут эффективными~"--- нет смысла произносить слова, которые, как вам кажется, не несут в себе смысла. Например, некоторые люди предпочитают слово \textit{борьба} слову \textit{страдание}, а другие заменяют слово \textit{доброта} на слово \textit{защита} или \textit{поддержка}. Попробуйте несколько разных вариантов, а потом практикуйте то, что вам больше всего понравилось. 

После прочтения этих инструкций можете попробовать их выполнить с закрытыми глазами, чтобы ничто вас не отвлекало. В интернете можно найти полную аудиозапись этой практики.

\begin{itemize}
	\item Подумайте о какой-то жизненной ситуации, которая причиняет вам стресс: это могут быть какие-то проблемы со здоровьем, на работе, в отношениях и~т.\:д. Не выбирайте для начала слишком большую проблему, лучше мелкую или средних масштабов, поскольку учиться черпать ресурсы в самосострадании нужно постепенно.
	\item Четко визуализируйте ситуацию в своем воображении. Какая обстановка? Кто что кому говорит? Что происходит? Что \textit{может} произойти? Чувствуете ли вы дискомфорт на телесном уровне, когда представляете себе эту сложность? Если нет, выберите проблему немного посерьезнее.
	\item Теперь скажите себе: <<Сейчас я испытываю страдание>>. Это осознанность. Возможно, вам лучше подойдет другая формулировка.
	\begin{itemize}
		\item Вот несколько вариантов:
		\begin{itemize}
			\item \textit{Это причиняет мне боль.}
			\item \textit{Ай! Больно.}
			\item \textit{Это очень стрессовая ситуация.}
		\end{itemize}
	\end{itemize}
	\item После этого скажите себе: <<Страдание~"--- это часть жизни>>. Это человеческая общность.
	\begin{itemize}
		\item Несколько других вариантов:
		\begin{itemize}
			\item \textit{Я не один.}
			\item \textit{У всех людей, включая меня, такое бывает.}
			\item \textit{Именно такие чувства человек испытывает, сталкиваясь с такими трудностями.}
		\end{itemize}
	\end{itemize}
	\item Затем используйте то успокаивающее прикосновение, которое вы нашли для себя в предыдущем упражнении. 
	\item Скажите себе: <<Я желаю себе быть добрым к себе>> или <<Я желаю себе, чтобы у меня получилось дать себе то, что мне нужно>>.

	У вас могут быть какие-то определенные слова доброты и поддержки, которые вам нужно услышать прямо сейчас именно в этой сложной ситуации. Возможные варианты: 
	\begin{itemize}
		\item \textit{Я желаю себе принять себя таким, какой я есть.}
		\item \textit{Я желаю себе начать принимать себя таким, какой я есть.}
		\item \textit{Я желаю себе, чтобы я смог себя простить.}
		\item \textit{Я желаю себе быть сильным.}
		\item \textit{Я желаю себе быть терпеливым.}
	\end{itemize}
	\item Если вам трудно найти нужные слова, представьте себе, что ваш близкий друг (или подруга) оказался в такой же ситуации. Что бы вы сказали этому человеку? Что бы вы хотели до него донести?
\end{itemize}

{\large \textbf{Теперь попробуйте сказать то же самое себе.}}

\Reflection{Подумайте о том, какое влияние на вас произвело это упражнение. Заметили ли вы что-нибудь, вызвав осознанность фразой <<сейчас я испытываю страдание>>? Возможно, какие-то перемены в сознании? 
	
Что насчет второй фразы, напомнившей вам о человеческой общности, или третьей, принесшей доброту к себе? Смогли ли вы найти добрые слова, которые вы бы сказали другу, и, если да, каково вам было сказать их же самим себе? Легко? Тяжелее? 
	
Иногда нужно немного времени, чтобы найти слова, которые для вас лично эффективны и звучат искренне. Разрешите себе учиться медленно~"--- в итоге вы точно найдете правильные слова. 

Эту неформальную практику можно выполнять медленно, как мини-медитацию, или использовать эти слова как мантру, когда вы встречаетесь с трудностями в повседневной жизни.}

\newpage
\InformalPractices{Сострадательное движение}

Эту неформальную практику можно использовать каждый раз, когда вам нужен перерыв на небольшую растяжку. Ее можно выполнять с открытыми или закрытыми глазами. Цель практики в том, чтобы двигаться с состраданием к себе, не обязательно в точности так, как здесь написано. 

\vspace{3ex}

\textbf{Встатьне}
\begin{itemize}
	\item Встаньте и ощутите, как ваши подошвы касаются пола. Немного раскачайтесь вперед, назад и по сторонам. Сделайте небольшие круговые движения коленями, замечая, как изменяются ощущения в ваших стопах. Направьте внимание на какое-то время на стопы. 
\end{itemize}

\textbf{Откройтесь}
\begin{itemize}
	\item Теперь расширьте поле вашего внимания и просканируйте все тело в поиске других ощущений, замечая места легкости и места напряжения.
\end{itemize}

\textbf{Ответьте с состраданием}
\begin{itemize}
	\item Теперь сфокусируйтесь на местах, где вы испытываете дискомфорт. Постепенно начните двигаться так, как вам этого хочется, относясь к себе с состраданием. Например, немного перекрутите плечи, закатите голову, сделайте поворот торса, наклонитесь вперед... именно то, что вам сейчас нужно.
	\item Дайте своему тело то движение, которое ему нужно, позволяя ему вести себя. Иногда наши тела нас разочаровывают или нам не нравится их внешний вид, ощущения или движение. Если с вами такое бывает, просто побудьте минуту вместе с собой и своим нежным сердцем. Ваше тело старается, как может. Что вам нужно прямо сейчас?
\end{itemize}

\textbf{Переход в состояние покоя}
\begin{itemize}
	\item Наконец, перестаньте двигаться. Встаньте и прочувствуйте ощущения во всем теле, отмечая любые изменения. Разрешите себе просто быть такими, какие вы есть.
\end{itemize}

\newpage
\Reflection{Остановитесь и подумайте о том, как для вас прошло это упражнение. Почувствовали ли вы себя по-другому, когда использовали движение как намеренный и заботливый ответ на дискомфорт? Смогли ли вы найти те движения, которые дали вашему телу то, что ему нужно?
	
Эту практику можно повторять несколько раз в течение дня. Конечный ее результат на самом деле не так важен, как само намерение заметить, где в вашем теле вы чувствуете напряжение, и ответить на него с заботой. Мы часто игнорируем незаметные сигналы нашего тела о том, что что-то не так. Привить себе привычку проверять, как мы себя чувствуем, и намеренно давать себе то, что нам нужно, очень важно, чтобы построить более здоровые отношения с собой.
}
%%% end section %%%