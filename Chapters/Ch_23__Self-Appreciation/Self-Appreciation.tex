% !TEX root = ../../Self-Compassion.tex

\chapter{Ценить себя} \label{Self-Appreciation}

Большинство людей понимают важность выражения благодарности и признательности другим. А что касается нас самих? Это отнюдь не так просто.

Негативная предвзятость особенно сильна по отношению к нам самим. Ценить себя не просто кажется неестественным~"--- это может казаться и неправильным. Так как у нас есть тенденция зацикливаться на своих слабых сторонах вместо того, чтобы ценить сильные, у нас часто искаженное видение того, кто мы есть. Подумайте об этом. Когда вы получаете комплимент, вы его принимаете и осознаете или он сразу же от вас <<отскакивает>>? Часто даже \emph{мысли} о наших положительных качествах приносят нам дискомфорт. В голове сразу же появляются контраргументы, например: <<Я не всегда такой>> или <<У меня много и плохих качеств>>. Эта реакция вновь демонстрирует негативную предвзятость, потому что, когда нам говорят что-то неприятное, вряд ли первым делом появляется в голове что-то вроде <<да, но я не \emph{всегда} такой>> или <<а ты знаешь обо \emph{всех} моих хороших качествах?>> 

Многие из нас боятся признать свои сильные стороны. Вот несколько часто встречающихся причин:

\begin{itemize}
	\item Я не хочу распугать друзей излишней самоуверенностью. 
	\item Мои положительные качества~"--- это не проблема, которую нужно решить, поэтому бесполезно на них сосредотачиваться. 
	\item Я боюсь, что я воздвигаю себе пьедестал, с которого сразу же и упаду. 
	\item Из-за этого у меня появится чувство превосходства, и я отделюсь от других людей.
\end{itemize}

Конечно, есть большая разница между тем, чтобы просто признать объективную реальность~"--- то, что у нас есть хорошие и не очень качества~"--- и тем, чтобы говорить, что мы идеальны или лучше других. Важно ценить наши сильные стороны и с состраданием относиться к слабым, чтобы мы могли принять себя целиком такими, какие мы есть. 

Мы можем применять три компонента самосострадания~"--- \textbf{доброту к себе}, \textbf{человеческую общность} и \textbf{осознанность}~"--- к своим как положительным, так и отрицательным качествам. Эти три фактора помогают нам начать ценить себя, здор\'{о}во и сбалансированно \cite{130}. 

\vspace{5ex}

\begin{center}
	{\Large Ценить себя}
\end{center}

\textbf{Доброта к себе}: часть доброты к себе~"--- это выражение признательности за наши хорошие качества, как мы бы сделали с хорошим другом. 

\vspace{2ex}

\textbf{Человеческая общность}: когда мы вспоминаем, что хорошие качества есть у каждого человека, мы можем признать свои, при этом не изолируя себя и не чувствуя себя лучше других. 

\vspace{2ex}

\textbf{Осознанность}: чтобы ценить себя, нужно обращать внимание на наши хорошие качества, а не принимать их как данность. 

\vspace{5ex}

Важно осознавать, что ценить себя~"--- это не эгоистично и не эгоцентрично. Это просто признание того, что хорошие качества~"--- часть каждого человека. Хотя, возможно, некоторых детей воспитывают так, что они считают, что скромность~"--- это непризнание наших хороших качеств, такое воспитание может навредить их восприятию себя и препятствовать тому, чтобы они себя как следует узнали.

Ценить себя~"--- это способ исправить нашу негативную предвзятость по отношению к себе и увидеть себя более отчетливо и целостно\cite{131}. Это также дает эмоциональную гибкость и уверенность в себе, необходимые, чтобы помогать другим. Духовная учительница и автор бестселлеров Марианна Уильямсон пишет: <<Нам всем нужно засиять~"--- как свободно сияют дети... И, позволив себе засиять, ты позволишь засиять и другим людям. Отринув свой собственный страх, ты поможешь так же сделать другим>>.

Мудрость и благодарность здесь играют центральную роль. Эти качества, о которых речь шла в предыдущей главе, помогают нам увидеть наши хорошие качества в более широком контексте. Когда мы ценим себя, мы также ценим все причины, обстоятельства и людей~"--- друзей, родителей, учителей~"--- которые помогли нам эти качества развить. Это значит, что мы можем не воспринимать свои хорошие качества слишком лично! 

\begin{quotation}
	\textit{
		Алиса выросла в строгой протестантской семье, где скромность, граничившая с самоуничижением, была ожидаемой нормой. Когда Алисе было восемь лет и она пришла домой из школы с призом за победу в конкурсе по правописанию, ее мать только подняла брови и сказала: <<Только не зазнайся>>. Каждое свое достижение Алисе нужно было преуменьшать, чтобы не встретиться с осуждением со стороны своей семьи.
	} 
	
	\textit{
		Позже Алиса начала встречаться с мужчиной по имени Тео, который считал ее красивой, доброй, умной и чудесной и которому нравилось ей об этом говорить. В таких случаях Алису настигали не только смущение и стыд, но и тревога. Что произойдет, если Тео узнает, что она не идеальна? Что если она его подведет? Она постоянно отмахивалась от его комплиментов, что озадачивало Тео и из-за чего он чувствовал себя как бы по другую сторону невидимой стены.
	}
	
	\textit{
		Алиса решила научиться самосостраданию, и у нее отлично получалось~"--- особенно в том, что касается взгляда на свои недостатки как на признак того, что мы люди. Ценить себя казалось Алисе разумной идеей концептуально, но она понимала, что на практике ей предстоит долгий путь. Она начала с запоминания всего хорошего, что она делала в течение дня~"--- момента доброты, успеха, небольшого достижения. Потом она попробовала говорить об этом что-то с благодарностью, например, <<это было отлично, Алиса>>. Когда она с собой так разговаривала, ей казалось, что она нарушает невидимый контракт, заключенный в детстве, но она упорно продолжала работу. <<Этим я не хочу сказать, что я лучше кого-то или что я идеальна. Я просто признаю, что во мне есть что-то хорошее>>. Наконец Алиса поставила себе задачу принимать и наслаждаться комплиментами Тео. Он был настолько рад такому повороту событий, что подарил Алисе браслет, на котором внутри было написано: <<Я не идеальна, но что-то во мне есть отличное!>>
	} 
\end{quotation}

\newpage
\Exercises{Как я отношусь к моим хорошим качествам?} \label{Ex:How_Do_I_Relate_to_My_Good_Qualities?}

Задумайтесь над вопросами ниже, отвечая так честно и искренне, как можете.

\begin{itemize}
	\itemWritingHand Как вы реагируете на комплименты? Вы принимаете их с счастьем и благодарностью или напрягаетесь и отмахиваетесь от них?
\end{itemize}
\setlength{\extrarowheight}{2mm}
\begin{tabularx}{\textwidth}{X}
	\\
	\arrayrulecolor{gray}\hline\\
	\hline\\
	\hline\\
	\hline\\
	\hline\\
	\hline\\	
	\hline\\
	\hline\\
	\hline\\
	\hline\\
	\hline\\
\end{tabularx}
\setlength{\extrarowheight}{0mm}
\begin{itemize}
	\itemWritingHand Наедине с собой вам комфортно или некомфортно ценить свои хорошие качества?
\end{itemize}
\setlength{\extrarowheight}{2mm}
\begin{tabularx}{\textwidth}{X}
	\\
	\arrayrulecolor{gray}\hline\\
	\hline\\
	\hline\\
	\hline\\
	\hline\\
	\hline\\	
	\hline\\
	\hline\\
	\hline\\
	\hline\\
\end{tabularx}
\setlength{\extrarowheight}{0mm}
\begin{itemize}
	\itemWritingHand Если вам некомфортно, подумайте, почему это может быть. Может быть, вы боитесь стать чересчур самоуверенным, упасть с пьедестала, стать самодовольным или почувствовать себя отделенным от других, или есть какая-то другая причина?
\end{itemize}
\setlength{\extrarowheight}{2mm}
\begin{tabularx}{\textwidth}{X}
	\\
	\arrayrulecolor{gray}\hline\\
	\hline\\
	\hline\\
	\hline\\
	\hline\\
	\hline\\	
	\hline\\
	\hline\\
	\hline\\
	\hline\\
	\hline\\
	\hline\\
	\hline\\
	\hline\\
\end{tabularx}
\setlength{\extrarowheight}{0mm}

\Reflection{
	Многие люди находят, что ценить себя~"--- грань, которую они проводят в своей практике самосострадания. Почем-то можно принимать свои недостатки и изъяны, но свои достоинства? Ужас какой! Если у вас такая же ситуация, это значит, что для вас может быть полезным ввести в арсенал практику того, чтобы ценить себя, и выполнять ее сознательно в повседневной жизни.
}

\newpage
\Exercises{Ценить себя} \label{Ex:Self-Appreciation}

Это упражнение поможет вам найти качества, которые вы в себе цените, особенно путем признания того, что помогло вам развить в себе эти хорошие качества. 

Если вам вдруг станет некомфортно во время выполнения этого упражнения, освободите в голове место для того, что вы чувствуете, и дайте себе разрешение быть такими, какие вы есть.

\begin{itemize}
	\item Сделайте два-три глубоких вдоха и выдоха и закройте на пару минут глаза, чтобы прочувствовать свое тело и настоящий момент. Положите руку на сердце в качестве жеста поддержки и доброты к себе.
	
	\itemWritingHand Теперь подумайте о пяти вещах, которые вам в себе нравятся. Первое, что приходит на ум, может быть достаточно поверхностным. Попытайтесь открыться тому, что вы \emph{на самом деле} в глубине себя в себе цените. Не спешите и будьте честными. 
\end{itemize}
\setlength{\extrarowheight}{2mm}
\begin{tabularx}{\textwidth}{X}
	\\
	\arrayrulecolor{gray}\hline\\
	\hline\\
	\hline\\
	\hline\\
	\hline\\
	\hline\\	
	\hline\\
	\hline\\
	\hline\\
	\hline\\
	\hline\\
	\hline\\
	\hline\\
	\hline\\
	\hline\\
	\hline\\
\end{tabularx}
\setlength{\extrarowheight}{0mm}


\begin{itemize}
	\item Подумайте по очереди  о каждом из этих положительных качеств и внутренне себе кивните в знак признательности за них. 

	\item Заметьте, чувствуете ли вы какой-то дискомфорт, думая о своих положительных чертах. Если да, освободите для этого ощущения место в голове, позволяя вашим ощущениям быть такими, какие они есть. Помните, что этим вы не хотите сказать, что всегда проявляете эти хорошие качества или что вы лучше других. Вы просто признаете объективную реальность. 

	\itemWritingHand Теперь подумайте, есть ли кто-то, кто помог вам развить в себе эти качества? Может быть, друзья, родители, учителя, даже авторы книг, которые произвели на вас впечатление?
\end{itemize}
\setlength{\extrarowheight}{2mm}
\begin{tabularx}{\textwidth}{X}
	\\
	\arrayrulecolor{gray}\hline\\
	\hline\\
	\hline\\
	\hline\\
	\hline\\
	\hline\\	
	\hline\\
	\hline\\
	\hline\\
	\hline\\
	\hline\\
	\hline\\
	\hline\\
	\hline\\
	\hline\\
	\hline\\
	\hline\\
\end{tabularx}
\setlength{\extrarowheight}{0mm}
\begin{itemize}
	\item Подумайте о каждом из этих людей и мысленно отправьте им благодарность и признательность.

	\item Позвольте себе насладиться чувством гордости за себя хотя бы на минуту~"--- впитайте в себя это ощущение.
\end{itemize}

\Reflection{
Получилось ли у вас найти в себе хорошие качества? Как вы чувствовали себя, проявив к себе признательность? Стало ли ценить себя легче, когда вы подключили благодарность и признательность другим людям? 

Для многих людей самая интересная часть этого упражнения~"--- это то, насколько легко принять наши хорошие качества, когда мы осознаем, что они тесно связаны с жизнями и вкладом других людей. Ценить себя кажется менее эгоцентрично, и мы чувствуем себя менее одинокими, когда включаем других в круг признательности.

Многие находят эту практику сложной, особенно люди с детскими травмами или те, кто вырос в семье, где гордиться своими достижениями считалось <<плохим>>. Иногда, когда мы предпринимаем попытку ценить наши хорошие качества, мы вспоминаем случаи, когда кто-то их не ценил, или для нас становятся более очевидными наши не такие уж и хорошие качества.

Это обратная тяга (смотри главу \ref{Backdraft} на стр.\pageref{Backdraft}). Если это с вами произойдет, вспомните, что обратная тяга~"--- часть процесса трансформации, и проявите к себе нежность и сострадание. Обратная тяга может также означать, что эта практика может стать для вас особенно полезной и нужно ее выполнять медленно и терпеливо. Когда вы даете себе \textbf{разрешение принять себя целиком}~"--- и хорошее, и плохое~"--- вы открываете дверь в более полную и настоящую жизнь.
}