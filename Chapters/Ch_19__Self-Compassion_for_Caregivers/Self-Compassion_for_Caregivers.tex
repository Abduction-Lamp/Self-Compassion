% !TEX root = ../../Self-Compassion.tex

\chapter{Самосострадание и забота о других} \label{Self-Compassion_for_Caregivers}

К среднему возрасту большинство людей уже заботятся о ком-то. Для некоторых это профессия~"--- например, для врачей, медсестёр, психотерапевтов, социальных работников, учителей. Для других же забота не выходит за границы личной жизни~"--- например, они заботятся о детях, супругах, престарелых родителях, друзьях и так далее.

Когда мы заботимся о людях, которые страдают, в связи с процессом эмпатического резонанса мы чувствуем их боль как нашу собственную (смотри главу~\ref{Being_There_for_Others_without_Losing_Ourselves} на стр.\:\pageref{Being_There_for_Others_without_Losing_Ourselves}). Когда мы становимся свидетелями чьей-то боли, у нас в мозге тоже активируются центры боли. Эмпатическую боль бывает сложно перетерпеть, поэтому вполне естественная реакция~"--- попытаться ее заблокировать или избавиться от неё совсем, как мы бы поступили с любой другой болью. Но постоянная борьба с этим чувством изнуряет и приводит к появлению ‘усталости от заботы’ и выгорания.

Как понять, что мы достигли точки выгорания? Обычно об этом говорят такие симптомы, как рассеянность, гнев или раздражение, неугомонность, избегание других людей, проблемы со сном или неприятные навязчивые мысли. Усталость от заботы~"--- это не признак слабости, а признак искреннего сопереживания страдающему. Чем больше люди, заботящиеся о ком-то, способны к эмпатическому резонансу (это, по факту, и приводит многих к профессиям, связанным с заботой), тем более они уязвимы перед усталостью от заботы. У людей есть ограничение на количество чужого страдания, которого они могут взять на себя, при этом не перегружаясь эмоционально.

Обычно для борьбы с выгоранием дают два основных типа советов. Один из них~"--- провести эмоциональные \emph{границы} между собой и тем, о ком мы заботимся. Проблема этого подхода в том, что, если вы профессионально заботитесь о людях, эмоциональная восприимчивость необходима для эффективной работы, а если вы заботитесь о близком человеке~"--- например, ребёнке или родителе~"--- обозначение границ может повредить отношениям. 

Другой подход~"--- заботиться не только о других, но и о \emph{себе}. Здесь чаще всего речь идёт о физической нагрузке, здоровом питании, проведении времени с друзьями, путешествиях и подобных вещах. Хотя заботиться о себе очень важно, эффективность таких стратегий ограничена тем, что, как правило, мы заботимся о себе в \emph{свободное} время, поэтому это нам особо не помогает в процессе заботы о других. Например,  психотерапевт не может сказать пациенту, который ему только что рассказал душераздирающую историю: <<Ого, офигеть, ты меня прямо шокировал. Пойду сейчас на массаж снять стресс!>>

Какую роль здесь может сыграть сострадание? Многие думают, что именно сострадание утомляет тех, кто о ком-то заботится~"--- вот почему этот феномен часто называется ‘усталостью от сострадания’. Некоторые исследователи считают, что это неправильное название и что более точно было говорить ‘усталость от эмпатии’.

В чем разница между эмпатией и состраданием? Кард Роджерс давал \emph{эмпатии} такое определение: \textbf{<<точное понимание внутреннего мира [пациента], в связи с которым вы чувствуете внутренний мир пациента, как свой собственный>>}. Если мы просто будем резонировать со страданием других, не имея при этом эмоциональных ресурсов, мы очень быстро изведёмся. Сострадание представляет собой чувство нежности и заботы, которое принимает страдание других, а не пытается с ним бороться.

Сострадание~"--- это положительная, заряжающая энергией эмоция. В одном исследовании некоторых участников тренировали эмпатии, а некоторых~"--- состраданию и потом показали им короткий фильм, запечатлевающий страдание других людей. Фильм активировал у двух групп в корне различающиеся области мозга, и области, связанные с положительными эмоциями, были активированы только у группы, которую тренировали состраданию.

Очень важно дарить себе сострадание, когда вы испытываете эмпатическую боль~"--- как и дарить сострадание тем, о ком мы заботимся. Как нам говорят каждый раз в самолете, в случае внезапного падения в кабине давления нужно сначала надеть кислородную маску на себя, а потом уже помогать другим. Некоторые люди считают, что они должны заботиться только о потребностях других людей и часто себя критикуют, когда им кажется, что их отдача недостаточна. Но если вы не удовлетворите собственные эмоциональные потребности с помощью сострадания, вы достигнете состояния эмоционального истощения и не сможете давать другим то, что им нужно. Что важно, когда вы успокаиваете и утешаете себя, человек, о котором вы заботитесь, тоже почувствует себя спокойнее через собственный эмпатический резонанс с вами. Другими словами, когда мы находим мир и покой внутри себя, мы помогаем всем, с кем контактируем, тоже их найти.  

Мы поняли важность самосострадания при заботе о других через собственный опыт. Это помогло нам преуспеть в заботе о других (одной~"--- о ребёнке--аутисте, другой~"--- о своих пациентах), при этом не выгорев.

Я (Кристин) летела вместе со своим сыном, Роуэном, из США в Европу. По какой-то непонятной причине как раз в тот момент, когда в кабине приглушили свет и некоторые пассажиры надеялись заснуть, у Роуэна случилась вспышек гнева. Он устроил настоящую истерику, кричал и бил все вокруг. Ему в то время было лет пять. Я помню, как себя тогда чувствовала. Мне казалось, что все люди в самолете смотрели на нас и думали: <<Что с этим ребёнком не так? Он должен уже перерасти такое поведение. А с матерью что не так? Почему она не может контролировать своего ребёнка?>> Не зная, что мне делать, я решила закрыться с Роуэном в туалете, чтобы он кричал там и другие пассажиры его хотя бы хуже слышали. Но нам не повезло~"--- туалет был занят. 

Сидя вместе с Роуэном снаружи туалета в том небольшом пространстве, которое у нас было, я понимала, что у меня нет выбора, кроме как посострадать себе. Я вдохнула сострадание к себе, положила руку на сердце и про себя сказала несколько слов поддержки. <<Дорогая, это так для тебя тяжело. Мне жаль, что это происходит, но все в конце концов будет в порядке. Это пройдёт>>. Я, конечно, отвлекалась, чтобы убедиться, что Роуэн в безопасности, но 95\% моего внимания было сосредоточено на успокоении и утешении себя. Потом случилась одна вещь, которую я часто замечала у Роуэна. По мере того, как я успокаивалась, успокаивался и он. Из опыта я поняла, что когда я забывала про практику самосострадания и поддавалась возбуждению, Роуэн тоже возбуждался, но, когда я практиковала самосострадание, Роуэн утихомиривался. Он просто вступал в резонанс с моими эмоциями так же, как и я с его. С тех пор я начала заботиться в первую очередь о своих эмоциях, и таким образом я достигала стабильности, которая мне необходима, чтобы быть с сыном и безусловно любить и поддерживать его, несмотря на трудности. Я быстро поняла, что практика самосострадания~"--- вход в состояние любящего, общечеловеческого присутствия~"--- это один из самых эффективных способов помочь как и Роуэну, так и самой себе.

Я (Крис) согласился провести с пациентом сеанс психотерапии, несмотря на то, что мое расписание было практически полностью забито. Когда этот пациент, Франко, переступил порог моего кабинета, у меня сразу создалось впечатление, что он выглядит более депрессивным, чем мне показалось по телефону. Мне бросились в глаза его сгорбленные плечи и измученное лицо. В самом начале нашей сессии Франко сказал мне, что он поставил все свои лекарства на прикроватную тумбочку и что его утешала мысль о том, что он может в любой момент покончить с собой. Его жена недавно его бросила, его работа была мало чем лучше безработицы, а тем утром он получил сообщение от своего арендодателя о том, что тот его выселяет.

Когда Франко только пришёл, я не чувствовал ничего, кроме любопытства и сострадания к этому новому человеку. Но, когда он упомянул суицид, я почувствовал страх, проходящий через все мое тело, и пожалел о том, что согласился принять Франко. То, что я узнал о сложной ситуации, в которой находился Франко, только усилило мой страх, что Франко может попробовать наложить на себя руки.

Зная, что искренняя эмоциональная связь часто помогает людям оставаться в живых, пока в их душе чёрная ночь, я понял, что, несмотря на свой страх, мне нужно было попробовать установить и поддержать связь с Франко, несмотря на страх. Я сделал за себя глубокий вдох, напоминая себе, что такие ситуации~"--- часть работы психотерапевтом, и медленно выдохнул уже за Франко. Я делал это снова и снова, пока не почувствовал в себе силы выслушать Франко с открытым сердцем и без страха. Я также напомнил себе, что я не могу быть ответственным за спасение жизни Франко, но что, как психотерапевт, я сделаю для этого все, что могу. Дыхание и напоминание себе о моей ограниченной возможности контролировать ситуацию дали мне возможность почувствовать отчаяние Франко в своём теле. Когда я поделился с Франко тем, как глубоко на меня повлияла его ситуация, он смягчился и начал рассказывать мне про смелые шаги, которые он предпринимал, чтобы спасти свою жизнь и пережить кризис. К тому времени, когда Франко покинул мой кабинет, у обоих из нас был лучик надежды.

\newpage
\Exercises{Уменьшение стресса при заботе о ком-то} \label{Ex:Reducing_Stress_for_Caregivers}

Если вы о ком-то заботитесь, важно тщательно выбирать занятия, чтобы себя не перегрузить. Хотя невозможно избавиться от стресса совсем, сделать можно всё-таки многое. В каждой сфере жизни из приведённых ниже определите полезное поведение, которое вы используете, чтобы справиться со стрессом, вредное поведение, которое только усиливает стресс, и запишите несколько идей изменений, которые вы можете внести в свою жизнь, чтобы лучше заботиться о себе.

\vspace{5ex}

\noindent\textbf{\emph{Физические занятия}} (например, питание, физическая нагрузка, сон)

\vspace{2ex}

\noindent
\setlength{\extrarowheight}{2mm}
\begin{tabular*}{\textwidth}{rp{11.7cm}}
	Полезные? & \\ \cline{2-2}
	  & \\ \cline{2-2}
	  \\
	Вредные? & \\ \cline{2-2}
	  & \\ \cline{2-2}
	  \\
	Изменения? & \\ \cline{2-2}
	  & \\ \cline{2-2}
\end{tabular*}
\setlength{\extrarowheight}{0mm}

\vspace{5ex}

\noindent\textbf{\emph{Психологические занятия}} (например, психотерапия, книги, музыка)

\vspace{2ex}

\noindent
\setlength{\extrarowheight}{2mm}
\begin{tabular*}{\textwidth}{rp{11.7cm}}
	Полезные? & \\ \cline{2-2}
	& \\ \cline{2-2}
	\\
	Вредные? & \\ \cline{2-2}
	& \\ \cline{2-2}
	\\
	Изменения? & \\ \cline{2-2}
	& \\ \cline{2-2}
\end{tabular*}
\setlength{\extrarowheight}{0mm}

\newpage
\noindent\textbf{\emph{Занятия, связанные с отношениями}} (например, семья, группы, ваш партнер)

\vspace{2ex}

\noindent
\setlength{\extrarowheight}{2mm}
\begin{tabular*}{\textwidth}{rp{11.7cm}}
	Полезные? & \\ \cline{2-2}
	& \\ \cline{2-2}
	\\
	Вредные? & \\ \cline{2-2}
	& \\ \cline{2-2}
	\\
	Изменения? & \\ \cline{2-2}
	& \\ \cline{2-2}
\end{tabular*}
\setlength{\extrarowheight}{0mm}

\vspace{5ex}

\noindent\textbf{\emph{Занятия на работе }} (например, сколько часов в неделю вы работаете, время, проведённое за экраном, перерывы)

\vspace{2ex}

\noindent
\setlength{\extrarowheight}{2mm}
\begin{tabular*}{\textwidth}{rp{11.7cm}}
	Полезные? & \\ \cline{2-2}
	& \\ \cline{2-2}
	\\
	Вредные? & \\ \cline{2-2}
	& \\ \cline{2-2}
	\\
	Изменения? & \\ \cline{2-2}
	& \\ \cline{2-2}
\end{tabular*}
\setlength{\extrarowheight}{0mm}


\newpage
\InformalPractices{Сострадание с хладнокровием} \label{IP:Compassion_with_Equanimity}

В этой практике соединены элементы <<Давать и получать сострадание>> с практикой хладнокровия, или, иными словами, сохранения равновесия в трудные времена. Развитие хладнокровия особенно важно и полезно в ситуациях, возникающих при заботе о ком-то, потому что оно напоминает нам, что наш контроль над страданиями других ограничен, и позволяет нам посмотреть на вещи в перспективе, чтобы внутри нас могло родиться сострадание. Эту практику можно применять при любых сложных взаимодействиях с людьми, но она особенно эффективна для заботы о других.

\begin{itemize}
	\item Найдите удобную для себя позицию и сделайте несколько глубоких вдохов и выдохов, чтобы почувствовать своё тело и настоящий момент. Можете положить руку на сердце в напоминание себе подойти к этому опыту с любящей осознанностью к происходящему и к себе.
	
	\item Вспомните о ком-то, о ком вы заботитесь, который вас изматывает или расстраивает~"--- кто-то, кто важен для вас и кто сейчас страдает.
	
	\item Для этого вводного упражнения выберите кого угодно, кроме своих детей, так как там может присутствовать более сложная динамика. Отчетливо визуализируйте этого человека и ситуацию, которая приносит вам дискомфорт, и почувствуйте этот дискомфорт в своём теле.
	
	\item Теперь прочитайте слова ниже, позволяя им плавно течь по вашему сознанию, как волны:
	\begin{quotation}
		\begin{center}
			\noindent\textit{
				Каждый из нас находится на своём жизненном пути.\\Я не причина страданий этого человека и не в моих силах это страдание полностью прекратить, хотя мне и хотелось бы. Такие моменты тяжело перенести, но я все равно пытаюсь помочь, как могу.
			}
		\end{center}
	\end{quotation}
	
	\item Чувствуя напряжение в своём теле, вдохните глубоко, как будто вдыхаете сострадание, и заполните этим состраданием каждую клетку своего тела. Позвольте глубокому дыханию и получению сострадания, в котором вы нуждаетесь, успокоить вам.
	
	\item Выдыхая, отправьте сострадание человеку, который связан с вашим дискомфортом.
	
	\item Продолжайте вдыхать и выдыхать сострадание, позволяя своему телу постепенно найти естественный ритм дыхания~"--- разрешая ему дышать без вашего вмешательства.
	
	\item <<Один за меня, один за тебя>>. <<Входит за меня, выходит за тебя>>.
	
	\item Периодически сканируйте своё сознание на предмет любых отголосков страдания и отреагируйте на них, вдыхая сострадание за себя и выдыхая за другого человека. Если вы считаете, что кому-то из вас больше нужно сострадание, сосредоточьте своё внимание на дыхание в этом направлении.
	
	\item Представьте себе, что вы качаетесь на волнах океана сострадания~"--- безграничного океана, который охватывает все страдание.
	
	\item Ещё раз прочувствуйте эти слова:
	\begin{quotation}
		\begin{center}
			\noindent\textit{
				Каждый из нас находится на своём жизненном пути.\\Я не причина страданий этого человека и не в моих силах это страдание полностью прекратить, хотя мне и хотелось бы. Такие моменты тяжело перенести, но я все равно пытаюсь помочь, как могу.
			}
		\end{center}
	\end{quotation}

	\item Теперь отпустите практику и разрешите себе быть именно таким, какой вы в настоящем моменте есть.
	
	\item Осторожно откройте глаза.
\end{itemize}

\Reflection{
	Что вы заметили или почувствовали во время выполнения этой практики? Произошёл ли какой-то внутренний сдвиг, когда вы произносили фразы хладнокровия? Получилось ли у вас направить <<поток>> сострадания в ту сторону, в какую вам было нужно?
	
	Дыхание <<входит за меня, выходит за тебя>> помогает тем, кто заботится о других, не забывать о сострадании к себе. Вместе с хладнокровием эта практика~"--- обманчиво легкий способ в одно и то же время сохранить связь и держать эмоциональную дистанцию. Фразы хладнокровия приносят особенное облегчение тем, кто берет на себя слишком много ответственности за страдания тех, о ком они заботятся. Наша способность помочь ограничена тем, что у нас разные тела и разные жизни. Мы стараемся, как можем. То, что менее ограничено, так это наша способность испытывать сострадание. Дарить сострадание себе не значит отнимать что-то у других~"--- напротив, это только даёт нашему состраданию дополнительную <<вместимость>>.
	
	Хладнокровие труднее даётся родителям, особенно родителям маленьких детей, но в конце концов родители понимают, что даже у их детей есть отдельные от них, уникальные жизни и свои жизненные пути. На одном занятии ОСС кормящая мать сказала, что, когда она вдыхала <<за себя>>, она чувствовала тошноту~"--- как будто она лишала дочь чуть ли не жизни. Ещё одна из участниц группы сделала саркастическое замечание: <<Ну, я мать четырёх детей, которые уже все покинули родной дом, и я вдыхала один раз за себя и...один раз за всех четверых!>>
}