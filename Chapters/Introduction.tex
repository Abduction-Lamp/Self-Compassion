% !TEX root = ../Self-Compassion.tex

\section*{Введение}\addcontentsline{toc}{section}{Введение} \label{Introduction}

	\begin{quotation}
		\textit{
			Наша задача состоит не в том, чтобы искать любовь, а всего лишь в том, чтобы найти внутри себя все барьеры, которые мы возвели против нее.~"--- Руми 
		}\cite{1}
	\end{quotation}
	
	Мы все строили барьеры против любви~"--- нам приходилось, чтобы защититься от суровых реалий человеческой жизни. Но есть и другой способ чувствовать себя защищенными и в полной безопасности. Когда мы осознаем свои трудности и относимся к себе с состраданием, добротой и поддержкой в тяжелые времена, все начинает меняться. Мы можем научиться принимать себя и свою жизнь, несмотря на внутренние и внешние несовершенства, и черпать внутри себя силу для преуспевания в жизни. За последние десять лет множество исследований показало пользу самосострадания для благосостояния. Люди, которые относятся к самим себе с большим состраданием, счастливее, мотивированнее и более довольны жизнью, у них лучше физическое здоровье и отношения, а также они менее подвержены тревожности и депрессии~\cite{2,3,4,5,6}. Они обладают эмоциональной устойчивостью, необходимой, чтобы справляться со стрессовыми ситуациями: разводами, проблемами со здоровьем, неудачами в учебе и даже серьезными психологическими травмами~\cite{7,8,9,10}. 
	
	Тем не менее, когда мы встречаемся с трудностями~"--- когда мы страдаем, терпим провалы или чувствуем себя неполноценными~"--- тяжело осознанно относиться к тому, что происходит: вместо этого хочется кричать и бить кулаками по столу. Нам не просто не нравится происходящее, но мы еще и думаем, что с нами что-то не так, раз это происходит. За считанные секунды мы строим мыслительную цепочку: <<Мне \emph{не нравится} это чувство>>~"--- <<Я \emph{не хочу} так себя чувствовать>>~"--- <<Я \emph{не должен} так себя чувствовать>>~"--- <<Что-то со мной \emph{не так}, раз я себя так чувствую>>~"--- <<Я \emph{плохой}!>> Именно здесь нужно подключить самосострадание. Иногда нам нужно утешить и пожалеть себя за то, как сложно быть человеком, чтобы потом подходить к жизни более осознанно. 
	
	Самосострадание рождается из осознанности, когда мы встречаемся с болью и страданием в жизни. Осознанность приглашает нас открыться страданию с любящим осознанием. Самосострадание добавляет: <<\emph{Будь добр к себе}, когда ты страдаешь>>. Вместе осознанность и самосострадание формируют состояние любящего и общечеловеческого присутствия (см. главу \ref{What_is_self-compassion} на стр. \pageref{What_is_self-compassion}) в тяжелые моменты нашей жизни. 
	
	
	\subsection*{Осознанное самосострадание}\addcontentsline{toc}{subsection}{Осознанное самосострадание}
	
	<<Осознанное самосострадание>> (ОСС) было первой учебной программой, направленной на тренировку самосострадания. Учебные программы, основанные на осознанности (например, снижение стресса через осознанность и когнитивная терапия на основе осознанности) также тренируют самосострадание, но не напрямую~"--- скорее, самосострадание там является <<побочным эффектом>> осознанности~\cite{11,12,13}. ОСС было создано, чтобы целенаправленно научить людей навыкам, необходимым для практики самосострадания в повседневной жизни. ОСС~"--- это курс продолжительностью в восемь недель, где наставники руководят группой от 8 до 25 участников. Каждую неделю программа включает 2 часа 45 минут занятий, а также полудневный сеанс медитации. Исследования показали, что участники ОСС делают большой прогресс в самосострадании и осознанности, а сама программа уменьшает тревожность и депрессию~\cite{14}, улучшает здоровье~\cite{15} и даже стабилизирует уровень глюкозы в крови у диабетиков~\cite{16}.
	
	Идея ОСС родилась в 2008, когда мы, авторы программы, встретились на сеансе  медитации для ученых. Одна из нас (Кристин) занимается психологией развития, стояла у истоков изучения самосострадания. Другой соавтор (Крис)~"--- клинический психолог, который с середины 90-х активно работает над интеграцией осознанности в психотерапию. Мы вместе ехали в аэропорт после сеанса, когда к нам пришла идея <<объединить силы>> для создания программы, обучающей самосостраданию.
	
	Я (Кристин) впервые встретилась с идеей самосострадания в 1997 году на последнем курсе аспирантуры, когда в моей жизни царил полный бардак. Я только-только развелась и испытывала очень много стресса из-за учебы. Мне пришла в голову идея научиться буддистской медитации, чтобы справиться с этим стрессом. К моему большому удивлению, преподавательница на курсах медитации много говорила о том, как важно выработать самосострадание. Хоть я и знала, что в буддизме уделяется много внимания важности сострадания к другим, я никогда не думала, что сострадание к самой себе может быть таким же важным. Моя первая реакция была: <<Что? Это значит, что я могу быть добра к себе? А это разве не эгоистично?>> Но мне так хотелось обрести хоть немного душевного спокойствия, что я решила попробовать. Скоро я поняла, насколько ценным ресурсом может быть самосострадание. Я научилась быть хорошей подругой для самой себя в трудные времена. Когда я стала относиться к себе добрее и меньше себя осуждать, моя жизнь кардинально изменилась. 
	
	После получения докторской степени я провела два года в постдокторантуре под руководством ведущего исследователя в области самооценки и узнала о некоторых негативных сторонах ее повышения. Хотя гордиться собой и полезно, нужда быть <<особенным>> и <<выше среднего>> часто приводит к нарциссизму, постоянному сравнению себя с другими, агрессивным защитным механизмам, предрассудкам~\cite{17}... Другой недостаток опоры на самооценку заключается в том, что она часто зависит от обстоятельств: она присутствует, когда мы преуспеваем, но покидает нас, когда мы терпим провалы~"--- именно тогда, когда она нам больше всего нужна! Я поняла, что самосострадание~"--- идеальная альтернатива самоуважению, потому что оно дает чувство самоуважения, для которого не нужно быть идеальным или лучше. Получив позицию помощника профессора в Университете Техаса в Остине, я решила провести несколько исследований, чтобы изучить самосострадание. До меня никто не изучал самосострадание с научной перспективы, поэтому я попыталась дать самосостраданию точное определение и создала шкалу для его измерения. Это спровоцировала целую лавину исследований на эту тему.  
	
	Тем не менее, я \emph{убеждена} в эффективности самосострадания в первую очередь благодаря собственному опыту. У моего сына, Роуэна, в 2007 году диагностировали аутизм, и это было самым сложным опытом в моей жизни. Я не знаю, как бы я с этим справилась, не практикуй я самосострадание. Я помню, что в день, когда мне сказали диагноз, я собиралась на практику медитации. Я сказала мужу, что отменю сеанс, чтобы мы смогли вместе все осознать, а он ответил: <<Нет, иди и займись этим своим самосостраданием, а потом вернешься и мне поможешь>>. Итак, пока я была на сеансе, я искупала саму себя в сострадании. Я разрешила себе чувствовать все, что я в тот момент чувствовала, без осуждения~"--- даже те чувства, которых я, по моему мнению, <<не должна>> была испытывать. Чувство разочарования и даже иррационального стыда. Как я могла чувствовать такое по отношению к сыну, которого я люблю больше всех на свете? Но я знала, что я должна открыть свое сердце и все это впустить. Я впустила печаль, горе, страх. И достаточно скоро я поняла, что я обрела необходимую мне стабильность~"--- что самосострадание не только поможет мне пройти через все это, но и быть самой лучшей, самой любящей вне зависимости от обстоятельств матерью для Роуэна. И это все изменило!
	
	Из-за сенсорных проблем, которые испытывают дети с аутизмом, у них часто бывают бурные вспышки гнева. Единственное, что в этих случаях могут сделать родители~"--- это попытаться максимально обезопасить ребенка и ждать, пока он утихомирится. Когда мой сын кричал и бился руками и ногами в продуктовом магазине без каких-либо очевидных причин, а незнакомые люди недоброжелательно смотрели на меня, потому что думали, что это я избаловала ребенка и не смогла привить ему дисциплину, я практиковала самосострадание. Я утешала себя за то, что испытываю чувства потерянности, стыда и беспомощности, таким образом получая от самой себя ту эмоциональную поддержку, в которой я так нуждалась. Самосострадание помогло мне избежать злости и жалости к себе, позволило мне остаться терпеливой и любящей по отношению к Роуэну, несмотря на стресс и отчаяние, которые я испытывала. Я не хочу этим сказать, что у меня не было моментов, когда я выходила из себя. Наоборот, у меня их было много. Но благодаря самосостраданию я гораздо быстрее опоминалась после своих ошибок и вновь фокусировалась на том, чтобы давать Роуэну любовь и поддержку.
	
	Я (Крис) тоже научился самосостраданию в основном по личным причинам. Я практиковал медитацию с конца 70-х, стал клиническим психологом в начале 80-х и присоединился к исследовательской группе, изучающей осознанность и психотерапию. Моя страсть к этим двум областям  кульминировалась публикацией книги <<\emph{Осознанность и психотерапия}>>~\cite{18}. По мере того, как осознанность набирала популярность, меня все чаще приглашали выступать с речами. Проблема была в том, что выступления перед публикой вызывали у меня ужасную тревогу. Несмотря на то, что я регулярно практиковал медитацию в течение всей своей взрослой жизни и пробовал все клинические приемы, чтобы справиться с тревогой, каждый раз перед любым публичным выступлением у меня учащалось сердцебиение, потели руки и терялась способность ясно мыслить. Переломный момент наступил, когда я готовился дать речь на предстоящей конференции в Гарвардской школе медицины, которую я помогал организовать. До этого я успешно прятался в тени, но теперь я должен был произнести речь и разоблачить свой секрет перед всеми моими уважаемыми коллегами.
	
	Примерно в это же время один опытный преподаватель медитации посоветовал мне сосредоточить свою медитацию на любящей доброте и просто повторять фразы: <<Я желаю себе быть в безопасности>>, <<Я желаю себе счастья>>, <<Я желаю себе здоровья>>, <<Я желаю себе, чтобы моя жизнь была легкой>>. Я решил попробовать последовать его совету. За все годы, которые я занимался медитацией и рефлексией, я до этого ни разу не говорил с собой нежным, утешающим тоном. Я сразу почувствовал себя лучше, а мое сознание прояснилось. Я сделал любящую доброту своей основной медитативной практикой. 
	
	Когда бы у меня ни возникала тревога по поводу предстоящей конференции, я просто говорил себе фразы, наполненные любящей добротой~"--- день за днем, неделю за неделей. Я делал это не с целью успокоиться, а просто потому что я ничего больше не мог сделать. В конце концов настал день конференции. Когда подошла моя очередь говорить, во мне появился привычный страх. Но в этот раз внутри меня было и что-то новое~"--- тихий шепот на заднем плане, который говорил: <<Я желаю тебе быть в безопасности. Я желаю тебе быть счастливым>>. И в этот момент в первый раз во мне родилось что-то, что заняло место страха~"--- \emph{самосострадание}.
	
	Размышляя об этом впоследствии, я понял, что не мог осознанно принять свою тревогу, потому что тревога перед публичными выступлениями~"--- это не \emph{тревожное} расстройство, это расстройство \emph{стыда}~"--- и стыд был слишком сильным, чтобы его преодолеть. Представьте себе, что вы не можете дать речь на тему осознанность из-за тревоги! Я чувствовал себя некомпетентным и глупым. В тот судьбоносный день я сделал для себя следующее открытие: иногда, особенно когда мы погружены в интенсивные эмоции (например, стыд), нам нужно поддержать себя перед тем, как контролировать свои действия. Я начал учиться самосостраданию и испытал его силу на себе.
	
	В 2009 году я опубликовал <<\emph{Осознанный путь к самосостраданию}>>~\cite{19}, чтобы поделиться с миром тем, чему я научился, и собственным опытом того, как самосострадание помогало моим клиентам в клинической практике. В следующем году Кристин опубликовала <<\emph{Самосострадание}>>~\cite{20}, где она рассказала свою личную историю, изложила теорию и результаты исследований на тему самосострадания и описала много техник, помогающих ему научиться. Вместе мы провели первую программу ОСС в 2010 году. С тех пор мы, наряду с сообществом психологов по всему миру, посвятили очень много времени и энергии на дальнейшую разработку ОСС и на то, чтобы сделать нашу программу безопасной, приятной и эффективной для всех. Польза программы была показана множеством исследований, и в ОСС приняли и продолжают принимать участие десятки тысяч людей. 
	
	
	\subsection*{Как работать с этим пособием}\addcontentsline{toc}{subsection}{Как работать с этим пособием}
	
	В этой рабочей тетради заключено почти все содержимое программы ОСС в легкодоступном формате, благодаря которому вы сразу можете начать развивать в себе самосострадание. Некоторые из пользователей этой рабочей тетради сейчас проходят курс ОСС, а некоторые хотят освежить в памяти то, чему они учились раньше, но для многих она будет их первым опытом ОСС. Пособие специально продумано так, чтобы быть самодостаточным путем к развитию навыков, которые необходимы для того, чтобы быть более самосострадательными в повседневной жизни. Оно следует той же структуре, что и курс ОСС, и главы организованы в четкой последовательности, так что новые навыки в каждой главе требуют использования изученных ранее. В каждой главе приводится базовая информация о какой-то теме, а также упражнения, которые помогут вам применить навыки на практике. Большинство глав содержит примеры из личного опыта участников программы ОСС, чтобы проиллюстрировать, каких жизненных сценариев можно ожидать. Каждый пример составлен из нескольких реальных историй и не нарушает ничье право на неразглашение конфиденциальной информации, а имена в них выдуманные. В этой книге мы чередуем мужские и женские местоимения, когда говорим о человеке как об одном любом индивиде. Мы сделали этот выбор для удобства чтения, так как язык постоянно эволюционирует, а не из неуважения к людям, использующим другие местоимения. Мы искренне надеемся, что никто не будет чувствовать себя исключенным. 
	
	Мы рекомендуем проходить главы по очереди, оставляя между ними временные интервалы на то, чтобы сделать практики несколько раз. Оптимальный вариант~"--- уделять практикам около 30 минут в день и проходить одну-две главы в неделю. Тем не менее, продвигайтесь в вашем собственном темпе. Если вам нужно двигаться медленнее или уделить побольше времени определенной теме, обязательно это сделайте. Подстройте программу под себя. Если вы заинтересованы в том, чтобы пройти курс ОСС вживую с опытным тренером, вы можете найти программу недалеко от вас на сайте \url{http://centerformsc.org}. Также существуют онлайн-тренинги. 
	
	Идеи и практики в этой книге основаны на научных исследованиях, но также и на нашем опыте преподавания самосострадания тысячам людей. Сама программа ОСС продолжает эволюционировать по мере того как мы и наши клиенты учимся и вместе растем. 
	
	ОСС~"--- не замена психотерапии, но сама по себе программа очень терапевтична~"--- она поможет вам использовать ресурс самосострадания, чтобы преодолеть трудности, с которыми мы непременно сталкиваемся в жизни.  Тем не менее, практика самосострадания может иногда задевать старые раны, так что если у вас есть история психологических травм или трудности с психическим здоровьем в настоящий момент, мы рекомендуем пройти это пособие под наблюдением психотерапевта.
	
	
	\subsubsection*{Советы для практики}\addcontentsline{toc}{subsection}{Советы для практики}
	
	Есть некоторые вещи, о которых нужно помнить, чтобы извлечь максимальную пользу из пособия.  
	\begin{itemize}
		\item ОСС~"--- это приключение, которое приведет вас на неисследованную территорию, и вы обязательно столкнетесь с неожиданными ситуациями.  Попробуйте смотреть на это пособие как на эксперимент с целью открытия \emph{своего внутреннего мира} и \emph{трансформации себя}. Вы будете работать в лаборатории собственного жизненного опыта~"--- посмотрите, что получится.
		\item Вы будете учиться различным техникам и принципам осознанности и самосострадания~"--- не стесняйтесь их приспосабливать и адаптировать так, чтобы они работали лично для вас. Ваша конечная цель~"--- \emph{стать лучшим учителем для самих себя}.
		\item Знайте, что по мере того, как вы начнете относиться к своим трудностям по-новому, будут возникать тяжелые моменты. Вы, скорее всего, встретитесь со сложными эмоциями или болезненными суждениями о самих себе. К счастью, эта книга о том, как обрести эмоциональные ресурсы, навыки, силы и способность справиться с этими трудностями. 
		\item В то время как работа над самосостраданием может быть нелегкой, цель в том, чтобы найти способ его практиковать так, чтобы это было просто и приятно. В идеале,  каждый момент самосострадания подразумевает \emph{меньше стресса} и \emph{меньше работы}, а не больше. 
		\item Учиться медленно~"--- это хорошо! Некоторые люди препятствуют достижению цели самосострадания, заставляя себя слишком много работать, чтобы стать сострадательными сами к себе. Разрешите себе двигаться в собственном темпе. 
		\item Сама по себе рабочая тетрадь~"--- тренировка самосострадания. В процессе обучения стоит проявлять самосострадание. Другими словами, средства и конечная цель~"--- это, по сути, одно и то же.
		\item Важно разрешить себе \emph{открываться} и \emph{закрываться} по мере работы с этим пособием. Как наши легкие, которые расширяются и сужаются, наши сердца тоже естественным образом открываются и закрываются. Разрешать себе закрываться, когда нужно, и открываться, когда это происходит органично~"--- это и есть самосострадание. Знаки, что вы открываетесь~"--- это смех, слезы или более яркие мысли и ощущения. Знаки, что закрываетесь~"--- рассеянность, сонливость, раздражительность или самокритика.
		\item Попробуйте найти правильный баланс между открытием и закрытием. Как у крана в душе есть много различных состояний между отсутствием воды и полным напором, так и у вас есть множество различных состояний с разной степенью открытости, которую вы можете регулировать. Ваши потребности будут меняться: иногда вы можете быть не в настроении делать какую-то практику, а в другое время она будет именно тем, что вам нужно. \emph{Возьмите на себя ответственность за вашу эмоциональную безопасность и не заставляйте себя что-то делать, если вам в тот конкретный момент того не хочется.} Вы всегда можете к этому вернуться или сделать практику с помощью и под руководством друга или терапевта. Квинтэссенция самосострадания~"--- в вопросе: <<Что мне нужно?>> Эта тема будет развита глубже в пособии. 
	\end{itemize}
	
	
	\subsubsection*{Устройство этой рабочей тетради}\addcontentsline{toc}{subsection}{Устройство этой рабочей тетради}
	
	Вы обнаружите, что эта рабочая тетрадь содержит разные элементы, у каждого из которых своя функция. Главы, как правило, начинаются с общей информации и определения понятий~"--- эту часть нужно просто прочитать и осмыслить. 
	
	В пособии есть много письменных \emph{упражнений}, которые в основном выполняются только один раз, хотя иногда бывает полезно сделать их заново через какое-то время, чтобы увидеть свой прогресс. \emph{Неформальные практики} следует выполнять ежедневно в повседневной жизни~"--- например, в очереди в кассу в супермаркете~"--- когда они нужны. Некоторые практики, например, ведение дневника, требуют выполнения в специально отведенное им время. \emph{Медитации}~"--- это более формальные практики, которые мы рекомендуем делать регулярно для максимального эффекта в каком-то месте, где нет никаких возможных источников отвлечения. 
	
	Большинство практик в этой книге содержат секцию <<пища для размышлений>>, которая поможет вам осмыслить свой опыт. Там могут быть какие-то вопросы и короткое пояснение того,  с чем вы могли встретиться, выполняя практику.  Это включает в себя потенциально сложные реакции и советы, как работать с вашими реакциями конструктивно. Кто-то может просто обдумать это все в тишине, а кому-то может понадобиться отдельная тетрадь для записи своих мыслей. Эта тетрадь может также пригодиться, если вам будет нужно больше места, чтобы написать ваши ответы к упражнениям, чем дается в книге (или если вы не хотите, чтобы другие читали ваши ответы в этой рабочей тетради и предпочитаете использовать личную тетрадь для всех упражнений). Самое важное~"--- это делать именно те практики, которые вам кажутся наиболее приятными или наиболее эффективными, так как это их вы, скорее всего, будете в основном использовать. 
	
	Пока вы работаете с пособием, мы рекомендуем посвящать примерно 30 минут в день медитации и неформальной практике. Исследования показали, что то, насколько вам удастся развить самосострадание, зависит от количества времени, которое вы уделяете практике, но не имеет большого значения, формальная практика или неформальная.  
	
	\newpage
	
	
	\colorbox{light-blue}{\Large\textit{Упражнение}}
	
	\vspace{2ex}
	
	Упражнения обычно делаются один раз, но их можно выполнять повторно.
	
	\vspace{5ex}
	
	\colorbox{light-green}{\Large\textit{Неформальная практика}}
	
	\vspace{2ex}
	
	Неформальные практики делаются постоянно, обычно в естественном течении повседневной жизни.
	
	\vspace{5ex}
	
	\colorbox{light-ping}{\Large\textit{Медитация}}
		
	\vspace{2ex}
		
	Медитации~"--- это формальные практики, которые выполняются регулярно в специально отведенное время. 
	
	%%% end section %%%