% !TEX root = ../../Self-Compassion.tex

\chapter{Отпустите сопротивление}\label{Letting_Go_of_Resistance}

Осознанность~"--- это не просто внимание к тому, что происходит в настоящем моменте. Она также подразумевает некоторое \emph{качество} этого внимания~"--- \emph{принятие} того, что происходит, не клея на это ярлык <<хорошо>> или <<плохо>>. Такое отношение часто называют \emph{несопротивление}. Сопротивление~"--- это внутренняя борьба, которая происходит, когда мы думаем, что наш настоящий момент должен быть другим.

Сопротивление пробкам на дорогах в час пик, например, может выглядеть так: \emph{Черт побери! Опять на автостраде все стоит. Я опять опоздаю на ужин! Еще и какой-то идиот только что попытался меня подрезать. Так меня это задолбало, что хочется кричать!!!}

Принятие означает что, хотя нам может не нравиться происходящее, мы признаем, что оно все равно происходит и можем отпустить мысли о том, что что-то идет не так, как нам хотелось бы.

\textbf{Принятие может выглядеть так:} \emph{Опять застряла в пробке. Ну, учитывая, что сейчас почти час пик, этого следовало ожидать. Если я буду злиться, домой я от этого быстрее не доеду.}

Как \emph{понять}, что мы чему-то сопротивляемся? Некоторые из признаков этого~"--- физическое напряжение, постоянные отвлечения, беспокойство или мысли об одном и том же, переедание, работа до переутомления, злость или раздражение, онемение чувств.  Это способы, которыми мы пытаемся сопротивляться чему-то неприятному. Сопротивление~"--- это не всегда плохо. Без него мы бы совсем потерялись в наполненной неприятными ситуациями жизни. Сопротивление может нам помочь в краткосрочной перспективе, но оно может иметь негативные долгосрочные последствия. 

\begin{center}
	{\large \textbf{То, чему мы сопротивляемся, остается.}}
\end{center}
  
К сожалению, когда мы сопротивляемся неприятным ситуациям, они от этого не исчезают, а становятся только хуже. Бывало ли у вас когда-нибудь так, что вам сложно было заснуть, когда вы знали, что вам нужно хорошо выспаться перед важной встречей? Если вы пытаетесь мысленно бороться с бессонницей, это помогает вам сразу заснуть? Вряд ли. Когда мы пытаемся бороться с непростыми чувствами, мы просто подливаем масла в огонь. \textit{Сопротивление бесполезно} (как пытались предупредить нас пришельцы в одном известном сериале).

\begin{quotation}
	\textit{
		У Рафаэллы были проблемы с повышенной тревожностью, и она себя из-за этого ненавидела. Когда ее накрывала тревога, она пыталась заставить себя продолжать напролом, при этом говоря себе: <<Ну не будь ты как ребенок, пора уже вырасти>>. Через некоторое время, несмотря на все ее усилия, тревожность ее одолела и у нее начались панические атаки.
	}
\end{quotation}
	
Преподаватель медитации Шинзен Янг объяснил этот феномен простой формулой \cite{60}:
{\large \textbf{\[\text{Страдание} = \text{Боль} \times \text{Сопротивление}\]}}

Другими словами, боль~"--- от потери, волнения, разбитого сердца, трудностей~"--- в жизни неизбежна, но, пытаясь противостоять боли, мы ее еще и усиливаем\cite{61}. Между этой дополнительной болью и страданием можно поставить знак равенства. Мы страдаем не только потому, что испытываем что-то болезненное, а еще и потому, что мы как будто бьемся головой о стену реальности~"--- чувствуем злость и разочарование из-за того, что все идет не так, как нам хочется.

Очень распространенная форма сопротивления~"--- \textit{отрицание}. Мы надеемся, что, если не будем думать о проблеме, она исчезнет. Но это не так: психологические исследования показывают, что, когда мы пытаемся подавить нежелательные мысли и эмоции, они только усиливаются\cite{62}. Более того, когда мы избегаем или пытаемся подавить болезненные мысли или эмоции, мы не можем четко их увидеть и прочувствовать и отреагировать на них с состраданием. 

\begin{center}
	{\large \textbf{Только то, что мы чувствуем, мы можем <<исцелить>>.}}
\end{center}

Осознанность и самосострадание~"--- это ресурсы, которые помогают нам встретиться с тяжелыми ситуациями с меньшим сопротивлением. Просто представьте себе, что вы бы чувствовали, когда вы подавлены и разбиты, если бы в комнату вошел друг, обнял бы вас, сел рядом, выслушал ваш рассказ, а потом помог бы выработать план действий. Этим внимательным и заботливым другом можете быть вы сами! Начать нужно с того, чтобы открыться происходящему без сопротивления. 
Учитывая, что осознанность~"--- один из основных компонентов самосострадания, стоит задать вопрос: <<А как они связаны?>> Это одно и то же или разные вещи?
С точки зрения авторов, хотя эти два понятия очень тесно переплетаются, разница меду ними все же есть:
\begin{itemize}
	\item Осознанность в основном направлена на принятие происходящего. Самосострадание~"--- на заботу о том, с кем это происходит.
	\item Осознанность спрашивает: <<Что я сейчас \emph{испытываю}?>> Самосострадание спрашивает: <<Что мне сейчас \emph{нужно}?>> 
	\item Осознанность говорит, <<\emph{Прочувствуйте} свое страдание>>.  Самосострадание говорит: <<\emph{Будьте добры к себе}, когда вы страдаете>>.
\end{itemize}

Несмотря на их отличия, и осознанность, и самосострадание могут помочь нам жить с \emph{меньшим сопротивлением}, направленным на нас самих и нашу жизнь. Главный парадокс осознанного самосострадания можно резюмировать так:

\begin{center}
	{\large \textbf{Когда нам тяжело, мы относимся к себе с состраданием не чтобы нам стало лучше, а \emph{потому что} нам плохо.}}
\end{center}

Другими словами, нельзя использовать самосострадание как панацею, которая магическим образом избавит нас от боли. Это скрытая форма сопротивления, которая в итоге сделает только хуже. Но если мы можем полностью принять то, что происходящее вызывает у нас боль, нам будет легче эту боль пережить. Нам нужна осознанность, чтобы убедиться, что мы не используем самосострадание как орудие сопротивления, а самосострадание нам нужно, чтобы чувствовать себя в достаточной безопасности, чтобы осознанно открыться сложным ситуациям. Вместе это похоже на красивый танец.

\begin{quotation}
	\textit{
		Попрактиковавшись в течение нескольких месяцев разговаривать с собой с состраданием, Рафаэлла научилась  просто <<быть>> с собой и своей тревогой с осознанностью и состраданием, а не сопротивляться происходящему. Когда ей было тревожно или она немного паниковала, она говорила себе голосом внутреннего самосострадания: <<Я знаю, что сейчас тебе очень страшно. Хотелось бы, чтобы все было не так трудно, но так сложилась жизнь. Я знаю, что у тебя сейчас ком в горле и кружится голова. Но я с тобой, несмотря ни на что. Ты не одна. Мы это выдержим>>. С появлением нового, более сострадательного внутреннего голоса панические атаки отступили, а Рафаэлла нашла, что она гораздо лучше может работать со своей тревогой, чем она думала.
	}
\end{quotation}

\newpage
\Exercises{Кубик льда} \label{Ex:The_Ice_Cube}

Это упражнение~"--- шанс увидеть сопротивление в действии, а также посмотреть, что будет, когда мы применим к ней осознанность и сострадание. Прочитайте инструкции и решите, хочется ли вам выполнить его сейчас или позже.

\begin{itemize}
	\item Упражнение необходимо делать вне дома или в комнате с водонепроницаемым полом (людям с болезнью Рейно вообще не рекомендуется его выполнять).

	\item Достаньте из морозилки один или два кубика льда и держите в сжатом кулаке столько, сколько можете. Потом продолжайте его держать.
	
	\item Через несколько минут обратите внимание на свои мысли (например, <<я же могу повредить себе руку от этого>>, <<я больше не могу>>, <<люди, которые придумали это упражнение~"--- изверги>>). Это \emph{сопротивление}.  
	
	\item Теперь заострите внимание на том, что вы чувствуете в моменте. Например, ощутите чувство холода безоценочно как просто холод. Если у вас в руках возникла пульсирующая боль, почувствуйте эту пульсацию. Заметьте свои эмоции (такие, как страх) как просто эмоции. Заметьте, появляются ли у вас какие-то импульсы~"--- например, бросить лед и разжать кулак. Воспринимайте свои импульсы как просто импульсы. Это \emph{осознанность}. 
	
	\item Теперь добавим немного доброты. Например, утешьте себя мыслью о том, что это упражнение делать больно, но не опасно. Можете глубоко выдохнуть от облегчения. Если вы ощущаете дискомфорт в руках, можно добавить какой-нибудь нежный звук. Подумайте о том, что ваша рука предупреждает вас о чувстве боли, и постарайтесь это оценить. Кивните себе с уважением и восхищением за то, что вы вытерпели это упражнение, чтобы научиться чему-то новому. Для этого нужно много смелости.
	
	\item Вы наконец можете оставить кубик!
\end{itemize}

\newpage
\Reflection{
	Что вы заметили, делая эту практику? Как она для вас прошла? Поменялось ли что-то с приходом осознанности и доброты к себе?

	Для многих людей это упражнение служит ярким примером того, как сопротивление может усилить боль. Оно также показывает, что, когда мы осознанно принимаем боль и относимся к себе с добротой, страдание может уменьшиться. Если у вас не получилось отпустить (вполне естественное) сопротивление, не вините себя. Оно исходит из инстинкта самосохранения и природного желания быть в безопасности. Но у вас есть силы и возможность благодаря собственной заботе, поддержке и утешению почувствовать себя в безопасности. Вам может потребоваться просто немного терпения, чтобы научиться регулировать ваши автоматические реакции.
}

\newpage
\Exercises{Как я причиняю себе ненужные страдания?} \label{Ex:How_Do_I_Cause_Myself_Unnecessary_Suffering?}

\begin{itemize}
	\itemWritingHand Подумайте о какой-нибудь ситуации в вашей жизни в настоящий момент, в которой, как вам кажется, сопротивление реальности причиняет вам ненужные страдания и делает все только хуже (например, прокрастинация и пренебрежение важным проектом, недовольство чем-то на текущем месте работы, злость на лающую собаку соседа). Запишите ее.
\end{itemize}	

\setlength{\extrarowheight}{2mm}
\begin{tabularx}{\textwidth}{X}
	\\
	\arrayrulecolor{gray}\hline\\
	\hline\\
	\hline\\
	\hline\\
	\hline\\
	\hline\\
	\hline\\
	\hline\\
	\hline\\
	\hline\\
\end{tabularx}
\setlength{\extrarowheight}{0mm}
\begin{itemize}
	\itemWritingHand \emph{Как вы поняли, что вы сопротивляетесь?} Вы чувствуете какой-то телесный или моральный дискомфорт? Можете его описать?
\end{itemize}	

\setlength{\extrarowheight}{2mm}
\begin{tabularx}{\textwidth}{X}
	\\
	\arrayrulecolor{gray}\hline\\
	\hline\\
	\hline\\
	\hline\\
	\hline\\
	\hline\\
	\hline\\
	\hline\\
	\hline\\
	\hline\\
\end{tabularx}
\setlength{\extrarowheight}{0mm}
\begin{itemize}	
	\itemWritingHand \emph{К каким последствиям приводит сопротивление?} Например, стала ли бы ваша жизнь легче, если бы вы прекратили сопротивляться или, по крайней мере, делали это немного меньше?
\end{itemize}

\setlength{\extrarowheight}{2mm}
\begin{tabularx}{\textwidth}{X}
	\\
	\arrayrulecolor{gray}\hline\\
	\hline\\
	\hline\\
	\hline\\
	\hline\\
	\hline\\
	\hline\\
	\hline\\
	\hline\\
	\hline\\
	\hline\\
	\hline\\
\end{tabularx}
\setlength{\extrarowheight}{0mm}
\begin{itemize}	
	\itemWritingHand \emph{Видите ли вы, что сопротивление вам чем-то помогает?} Возможно, благодаря ей вы можете не ощущать чувства, которые могут быть слишком подавляющими? Если у вас при этом возникают трудные эмоции, будьте добры к себе. Уважайте свое сопротивление, вспоминая, что иногда оно помогает вам нормально жить.
\end{itemize}
	
\setlength{\extrarowheight}{2mm}
\begin{tabularx}{\textwidth}{X}
	\\
	\arrayrulecolor{gray}\hline\\
	\hline\\
	\hline\\
	\hline\\
	\hline\\
	\hline\\
	\hline\\
	\hline\\
	\hline\\
	\hline\\
	\hline\\
\end{tabularx}
\setlength{\extrarowheight}{0mm}
\begin{itemize}
	\itemWritingHand \emph{Теперь подумайте, как с помощью осознанности или самосострадания вы можете уменьшить свое сопротивление в этой ситуации.} Может быть, если вы валидируете боль (<<Это действительно тяжело>>), вам будет легче (или, наоборот, сложнее)? Или, например, если вы отнесетесь к себе с пониманием (<<Это не твоя вина>>) или вспомните о человеческой общности (<<Другие люди так же себя чувствуют в таких ситуациях>>), это принесет вам облегчение?
\end{itemize}
	
\setlength{\extrarowheight}{2mm}
\begin{tabularx}{\textwidth}{X}
	\\
	\arrayrulecolor{gray}\hline\\
	\hline\\
	\hline\\
	\hline\\
	\hline\\
	\hline\\
	\hline\\
	\hline\\
	\hline\\
	\hline\\
	\hline\\
	\hline\\
\end{tabularx}
\setlength{\extrarowheight}{0mm} 

\Reflection{
	Некоторые люди чувствуют себя уязвимыми после этого упражнения. Отпустить сопротивление значит открыться боли, а открываться боли сложно. Для этого может понадобиться признать, что часто у нас меньше контроля над происходящим, чем нам того бы хотелось. Вот здесь нужно щедро одарить себя добротой и состраданием. Если, проделав это упражнение, вы расстроитесь, положите руку себе на сердце и скажите себе пару слов поддержки. Что бы вы сказали другу, который себя чувствует так же, как вы сейчас? Можете ли вы сказать себе что-то похожее?
}

\InformalPractices{Замечайте сопротивление} \label{IP:Noticing_Resistance}

Так как сопротивление боли~"--- это очень естественный и автоматический процесс (даже амеба в чашке Петри отодвинется от токсина), большую часть нашего сопротивления мы и не замечаем. Поэтому очень полезная практика~"---  замечать, когда мы пытаемся чему-то сопротивляться, и называть вещи своими именами, когда это происходит. 

Всю следующую неделю попробуйте замечать даже небольшие или короткие моменты сопротивления (вы не хотите идти вечером во вторник на занятие аэробики, лифт на работе сломался и вам опять приходится подниматься по лестнице, ваш сын оставил свою немытую посуду на столе и так далее). Когда вы замечаете, что сопротивляетесь, признайте это и скажите нейтральным тоном, как констатацию факта: <<Сопротивляюсь>>. <<Это момент сопротивления>>. Чем больше мы замечаем, когда мы сопротивляемся, тем меньше ненужного напряжения и стресса мы впускаем в свою жизнь и тем выше шансы, что в сложных ситуациях мы будем вести себя мудро и обдуманно.