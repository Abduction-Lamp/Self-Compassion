% !TEX root = ../../Self-Compassion.tex

\chapter{Любящая доброта к себе} \label{Loving-Kindness_for_Ourselves}

Чтобы получить пользу от медитации любящей доброты, иногда необходимо персонализировать практику. Именно поэтому эта глава посвящена поиску фраз любящей доброты, которые подходят лично вам~"--- единственный в своем роде ключ от двери вашего сердца. 

\begin{quotation}
	\textit{
		Аши очень серьезно занималась медитацией и выполняла практику любящей доброты на протяжении нескольких лет после того, как один из ее любимых учителей ее ей научил. С этим был связан ее небольшой секрет: когда она произносила свои фразы любящей доброты, она ничего при этом не чувствовала, как робот, механически повторяющий слова. Она подозревала, что, может быть, у нее не тот темперамент для любящей доброты.
	}
\end{quotation}

Многие из фраз любящей доброты, которые по традиции используются во время медитации, переходили из уст в уста в течение нескольких столетий, поэтому нет ничего удивительного в том, что с ними бывает сложно себя отождествить. По этой причине очень важно найти фразы любящей доброты, которые вызывают лично у вас эмоциональный отклик~"--- а особенно это важно, когда вы хотите испытать любящую доброту к себе самим. Чтобы добиться эффекта, фразы нужно выбрать такие, какие вам покажутся естественными и искренними. Поиск правильных фраз чем-то похож на написание стихов: и то, и то~"--- это поиск слов для выражения чего-то, что нельзя описать словами. Наша цель~"--- найти именно те слова, которые вызывают у вас чувство любящей доброты и сострадания. 

Так же, как наблюдение за нашим дыханием может помогать во время медитации, фразы любящей доброты могут разбудить нашу осознанность. Успокаивающий эффект медитации связано с концентрацией и сосредоточенностью, поэтому найти две--четыре фразы, которые вы готовы повторять раз за разом, важно для достижения этого эффекта. Фразы любящей доброты можно использовать и в повседневной жизни, как мы увидели в практике любящей доброты в главе \ref{Developing_Loving-Kindness} на стр\:\pageref{Developing_Loving-Kindness}. С фразами, которые вы используете в обычной жизни, могут быть более гибкими~"--- их можно изменять в зависимости от того, что вы чувствуете в настоящий момент.

Есть несколько правил, которых стоит придерживаться, чтобы найти те самые фразы, которые для вас наполнены глубоким смыслом:

\begin{itemize}
	\item Фразы должны быть \emph{простыми}, \emph{ясными}, \emph{естественными} и \emph{доброжелательными}. Во время их произнесения в голове не должно быть никаких сомнений и внутреннего спора, а только благодарность: <<О, спасибо! Спасибо>>.
	
	\item Нет никакой необходимости использовать именно формулировку <<я желаю>>, если это для вас звучит странно. Это просто приглашение \emph{направить сердце} в положительном направлении, которое означает <<если все обстоятельства будут благоприятно, то...>> Фразы любящей доброты похожи на благословения.
	
	\item Эти фразы~"--- это не положительные аффирмации\footnote{Утвердительное (положительное) суждение} (например, <<каждый день я становлюсь здоровее>>). \textbf{Мы просто пробуждаем в себе благие намерения}, а не пытаемся убедить себя, что реальность не такая, какая она есть.
	
	\item Фразы должны вызывать в первую очередь расположение и \emph{благожелательность, а не позитивные} эмоции. Часто у людей возникают проблемы с медитацией любящей доброты из-за того, что у них есть какие-то неправильные представления насчет того, как мы должны себя после нее чувствовать. Сама по себе любящая доброта не изменяет наши эмоции напрямую, но \emph{позитивные} эмоции и чувства~"--- это неизбежный побочный эффект \emph{расположения и благожелательности}.
	
	\item \emph{Фразы должны быть общими, а не конкретными}. Например, фраза <<я желаю себе быть здоровым>> лучше, чем фраза <<я желаю себе вылечиться от диабета>>.
	
	\item Фразы нужно \emph{произносить медленно}: не надо никуда спешить, это не гонка и медаль за рекордное время вам никто не даст.
	
	\item Фразы нужно произносить с теплотой, как будто вы шепчете их на ухо кому-то, кого по-настоящему любите. Самое главное~"--- это \emph{отношение}.
	
	\item Наконец, к себе можно обращаться по-разному: <<я>>, <<ты>> или по имени (<<Джордж>>). Можно придумать для себя какое-нибудь ласковое обращение вроде <<солнце>> или <<дорогая>>~"--- это поддерживает атмосферу доброты и сострадания.
\end{itemize}

\newpage
{\large \textbf{Что мне нужно?}} \label{What_Do_I_Need} \addcontentsline{toc}{subsection}{Что мне нужно?}

\vspace{2ex}

Чтобы найти правильные и значимые для вас фразы, нужно сосредоточиться на главном вопросе самосострадания: <<\textbf{Что мне \textit{нужно}?}>>

\vspace{2ex}

\textbf{Что такое \emph{потребности} и чем они отличаются о \emph{желаний}? }

\vspace{2ex}

\emph{Желания}, как правило, индивидуальны и возникают в голове. Они бесконечны: от дорогой машины до кофе определенной марки.

\emph{Потребности} более универсальны и возникают (фигурально говоря) ниже головы. Некоторые из человеческих нужд~"--- быть принятыми, увиденными, услышанными, защищенными, любимыми, связанными с другими, уважаемыми; нам нужна валидация, чувство, что нами дорожат, и так далее. Существуют и другие потребности, которые тоже универсальны, но не так связаны с отношениями с другими людьми: здоровье, личностный рост, свобода, юмор, безопасность. Найти того, в чем мы на самом деле нуждаемся, значит заложить основу для того, чтобы найти по-настоящему наполненные смыслом для себя фразы любящей доброты.

\begin{quotation}
	\textit{
		Когда Аши наконец придумала свои собственные фразы любящей доброты~"--- пожелания, связанные с ее самыми глубокими потребностями~"--- это все изменило. В итоге она остановилась на трех фразах: <<Я желаю себе быть смелой. Я желаю себе, чтобы люди видели во мне мою истинную сущность, а не поверхностные детали. Я желаю себе жить наполненной любви жизнью>>. Вместо механического повторения слов она впервые начала произносить фразы, каждая из которых вызывала у нее эмоциональный отклик. Теперь практически каждый раз во время медитации любящей доброты Аши чувствует себя так, как будто она дарит себе очень ценный подарок и его же принимает с открытым и благодарным сердцем.
	}
\end{quotation}

\newpage
\Exercises{Найдите свои фразы любящей доброты} \label{Ex:Finding_Loving-Kindness_Phrases}

Цель этого упражнения~"--- помочь вам открыть для себя фразы любящей доброты и сострадания, которые для вас имеют глубокий смысл. Если у вас уже есть свои фразы, которые вы хотите продолжать использовать, попробуйте выполнить задание в качестве эксперимента, но менять свои фразы не нужно, если вы того не хотите.

\vspace{3ex}

\noindent{\large \textbf{Что мне нужно?}}
\begin{itemize}
	\item Для начала положите руку на сердце или куда-то еще, куда вам комфортно, и почувствуйте дыхание вашего тела.
	
	\item Теперь дайте своему сердцу время постепенно раскрыться~"--- стать восприимчивым~"--- как цветок раскрывается под теплыми лучами солнца.
	
	\item Потом задайте себе следующий вопрос и подождите, пока ответ естественным образом возникнет где-то в глубине вас:
	
	\begin{itemize}
		\item <<Что мне нужно?>>, <<Что мне на самом деле нужно?>>
		
		\item Если в какой-то день эта потребность не удовлетворена, день словно бы прожит зря.
		
		\item Пусть ответ будет \emph{универсальной} человеческой потребностью, такой как потребность быть любимым, свободным, потребность в отношениях с другими людьми или во внутреннем спокойствии и мире.
	\end{itemize}

	\item Когда вы готовы, запишите то, что у вас получилось.
	
	\item Слова, которые вы открыли для себя, можно использовать во время медитации в исходном виде или переформулировать в пожелания себе:
	
	\begin{itemize}
		\item \emph{Желаю себе быть доброй к себе}
		\item \emph{Желаю себе начать быть доброй к себе}
		\item \emph{Желаю себе чувствовать себя на своем месте}
		\item \emph{Желаю себе жить в мире}
		\item \emph{Желаю себе всю жизнь быть любимой} 
	\end{itemize}
\end{itemize}


\newpage
\noindent{\large \textbf{Что мне нужно услышать?}}
\begin{itemize}
	\item Теперь подумайте над следующим вопросом:
	
	\begin{itemize}
		\item \emph{Что мне нужно услышать от других?} Каких слов мне не хватает? В каких словах я как человек нуждаюсь? Откройте двери своего сердца и ждите, когда к вам придут эти слова.
		
		\item Если бы это было возможно, \emph{какие слова я бы хотела, чтобы мне шептали на ухо} каждый день до конца жизни~"--- слова, на которые я всегда бы искренне отвечала <<о, спасибо, спасибо>>? Разрешите себе быть уязвимой и смело откройтесь этой возможности. Слушайте себя.
	\end{itemize}

	\itemWritingHand Когда будете готовы, запишите, что вы услышали.
\end{itemize}

\setlength{\extrarowheight}{2mm}
\begin{tabularx}{0.96\textwidth}{X}
	\\
	\arrayrulecolor{gray}\hline\\
	\hline\\
	\hline\\
	\hline\\
	\hline\\
	\hline\\
	\hline\\
	\hline\\
	\hline\\
\end{tabularx}
\setlength{\extrarowheight}{0mm}
\begin{itemize}
	\itemWritingHand Если вы услышали много слов, попробуйте составить из них короткую фразу~"--- \emph{послание себе}. 
\end{itemize}

\setlength{\extrarowheight}{2mm}
\begin{tabularx}{0.96\textwidth}{X}
	\\
	\arrayrulecolor{gray}\hline\\
	\hline\\
	\hline\\
	\hline\\
	\hline\\
	\hline\\
	\hline\\
	\hline\\
	\hline\\
\end{tabularx}
\setlength{\extrarowheight}{0mm}
\begin{itemize}
	\item То, что вы записали, можно, как и в предыдущей части, оставить в исходном виде или переформулировать в пожелания себе. На самом деле, слова, которые мы хотим снова и снова слышать  от других~"--- \emph{это качества, которые мы хотели бы иметь в жизни}. Например, потребность услышать <<я люблю тебя>> указывает на то, что человек хочет знать, что он или она заслуживает любви. Поэтому нам и нужно слышать эти слова снова и снова.
\end{itemize}

\vspace{4ex}

\noindent{\large \textbf{Что вы хотите знать наверняка?}}
\begin{itemize}
	\itemWritingHand Если хотите, можете переформулировать слова в пожелания себе.
	\begin{itemize}
		\item <<Я люблю тебя>> можно перевести в <<Я желаю себе любить себя такой, какая я есть>>.
		
		\item <<Я с тобой и помогу в любой ситуации>>~"--- <<Желаю себе чувствовать себя защищенной и в безопасности>>.
		
		\item <<Ты хороший человек>>~"--- <<Желаю себе осознавать собственные достоинства и ценить себя>>.
	\end{itemize}
\end{itemize}

\setlength{\extrarowheight}{2mm}
\begin{tabularx}{0.96\textwidth}{X}
	\\
	\arrayrulecolor{gray}\hline\\
	\hline\\
	\hline\\
	\hline\\
	\hline\\
	\hline\\
	\hline\\
	\hline\\
\end{tabularx}
\setlength{\extrarowheight}{0mm}
\begin{itemize}
	\itemWritingHand Теперь перечитайте то, что вы записали, и выберите 2--4 фразы, которые хотите использовать для медитации. Запишите их отдельно. Эти слова или фразы~"--- подарки, которые вы будете дарить себе снова и снова.
\end{itemize}

\setlength{\extrarowheight}{2mm}
\begin{tabularx}{0.96\textwidth}{X}
	\\
	\arrayrulecolor{gray}\hline\\
	\hline\\
	\hline\\
	\hline\\
	\hline\\
	\hline\\
\end{tabularx}
\setlength{\extrarowheight}{0mm}

\begin{itemize}
	\item Постарайтесь \emph{запомнить} эти слова и фразы.
	
	\item Наконец, попробуйте их в действии, чтобы увидеть, какой эффект они производят. Начните медленно и нежно произносить свои фразы, повторяя их раз за разом и представляя себе, что вы их шепчете в собственное ухо, как будто на ухо близкого человека. Может быть, вы услышите эти слова где-то внутри, наблюдая за тем, как они создают резонанс внутри вас. Разрешите словам занять много места и заполнить все ваше <<я>>.
	
	\item Потом осторожно и мягко \emph{отпустите} фразы и позвольте себе остаться в вашем опыте, принимая практику такой, какая она была, и разрешая себе быть такой, какая вы есть.
	
	\item Считайте это упражнение всего лишь началом поиска правильных для вас фраз. Поиск фраз любящей доброты~"--- это душевное и поэтическое путешествия. Надеемся, что вы будете возвращаться к вопросам <<что мне нужно?>> и <<что я хочу услышать?>> по мере того, как вы будете практиковать медитацию любящей доброты.
\end{itemize}

\Reflection{
	Что вы заметили, делая это упражнение? Удивились ли вы тому, что вам нужно? Как вы относитесь к фразам, которые вы обнаружили?
	
	Как узнать, что вы нашли подходящую фразу? Благодарность! С благодарностью больше нет отчаянного желания и тоски. Мы чувствуем завершенность и целостность. Сердце отдыхает. Для того, чтобы найти фразы, которые действуют на вас именно так, может потребоваться время, но это стоит вех усилий.
}

\newpage
\Meditation{Любящая доброта к себе} \label{M:Loving-Kindness_for_Ourselves}

В ходе этой медитации вы будете использовать фразы, которые вы ранее записали. Вновь их прочитайте и выберите, какими из них вы будете пользоваться, чтобы во время медитации не тратить время на поиск новых. 

Медитация любящей доброты состоит из множества элементов, и практикующие обычно слишком сильно напрягаются, чтобы все сделать правильно. Чтобы избежать этой ошибки, попробуйте отпустить желание испытать какие-то определенные чувства во время медитации. Оставьте всю работу словам, как будто вы ложитесь в теплую ванну и ждете, когда вода сделает свое дело. 

\begin{itemize}
	\itemdiamondsuit Найдите удобное сидячее или лежачее положение. Закройте глаза, полностью или частично. Сделайте несколько глубоких вдохов и выдохов, чтобы ощутить свое тело и настоящий момент. Положите руку на сердце в напоминание того, что вам нужно не просто осознание, а \emph{любящее} осознание процесса и самого себя.
	
	\itemdiamondsuit Проведите так какое-то время, а потом постарайтесь ощутить, как воздух движется по вашему телу (там, где вам легче всего это почувствовать), и как ваше тело движется при дыхании. Прочувствуйте ритм своего дыхания, а, когда внимание ускользнет, верните его к дыханию.
	
	\itemdiamondsuit Теперь отпустите свое дыхание, чтобы оно постепенно ушло на задний план вашей осознанности, и начните шептать себе те фразы, которые больше всего для вас значат.
	
	\itemdiamondsuit Повторяйте эти слова снова и снова, позволяя им окружить вас~"--- окружите себя словами любви и сострадания.
	
	\itemdiamondsuit Если вам так хочется, вберите в себя слова, заполняя ими себя целиком. Позвольте им прозвучать в каждой клетке вашего тела.
	
	\itemdiamondsuit Вам не нужно ничего делать и никуда идти. Просто искупайте себя в добрых словах, впитывая их~"--- слова, которые вам нужно услышать.
	
	\itemdiamondsuit Когда вы замечаете, что ваше внимание куда-то улетело, напомните себе о своей цели, используя успокаивающее прикосновение или просто прочувствовав ощущения в вашем теле. А потом снова подарите себе эти слова. Почувствуйте себя, как будто возвращаетесь домой, где все пропитано любовью.
	
	\itemdiamondsuit В завершение, отпустите фразы и просто тихо наблюдайте за своим телом.
\end{itemize}

\Reflection{
	Что вы заметили во время этой медитации? Смогли ли вы почувствовать себя ближе к ней, используя персонализированные фразы? Как вы сейчас себя чувствуете? 
	
	Многим людям намного легче прочувствовать смысл своих фраз после того, как они нашли правильные слова. Если практика все равно была какой-то неловкой, попробуйте сократить свои фразы. Может быть, вам больше подойдет произносить несколько отдельных слов, например, <<любовь>>, <<принятие>>, <<поддержка>>, и покажется это более естественным. Экспериментируйте, пока не найдете то, что эффективно для вас.
	
	Это вторая основная медитативная практика в курсе ОСС, поэтому имеет смысл проделывать ее минут по 20 несколько дней подряд, пока у вас не начнет все получаться. Как было упомянуто раньше, что мы рекомендуем выполнять комбинацию формальных (медитация) и неформальных (повседневная жизнь) практик примерно 30 минут в день.
	
	А если вы пробовали много раз и вам кажется, что медитация любящей доброты не вызывает у вас никакого отклика, в этом нет ничего плохого. В этом пособии есть множество различных практик и медитаций, чтобы помочь вам выработать сострадательное отношение к себе. Главное~"--- стремиться привнести больше доброты в свою жизнь так, как вам удобнее всего это сделать. 
}