% !TEX root = ../../Self-Compassion.tex

\chapter{Осознанность} \label{Mindfulness}

\textbf{Осознанность~"--- это основа самосострадания.} Чтобы ответить на боль с добротой, нужно абстрагироваться от своего страдания и осознанно на него посмотреть. Осознанности можно дать следующее определение: <<сосредоточенность на реальности и ощущениях настоящего момента и их принятие>>. Тем не менее, определения, которое бы точно описывало природу осознанности, не существует, потому что когда мы осознанны, мы воспринимаем мир напрямую, а не через линзу мыслей. 

\textbf{Мысли~"--- это изображения, символы, которые описывают реальность, а не сама эта реальность.} Слово <<яблоко>> нельзя понюхать, попробовать на вкус или съесть. Когда мы <<спускаемся>> ниже мысленного уровня восприятия и вступаем в прямой контакт с реальностью, мы можем уловить ее изменчивую природу. Мы можем отпустить свои представления о том, какой реальность <<должна>> быть, и открыться ей настоящей. Это значит, что когда мы страдаем, мы можем не зацикливаться на том, что произошло, а просто <<быть>> и ощущать. 

\begin{quotation}
	\textit{Террелл поднял руку и начал рассказывать о своем опыте практики осознанности дома. <<Моего кота недавно пришлось усыпить, и это разбило мне сердце. Мы с моим партнером Ламаром завели его 12 лет назад~"--- он для нас был как любимый ребенок. После возвращения домой меня накрыло волной разных мыслей и чувств, но я вспомнил, что нам говорят \textbf{признать существование страдания и просто пытаться фокусироваться на телесных ощущениях}. Я сказал себе: ,,Сейчас это все очень тяжело``. Я почувствовал боль в животе, как будто меня пнули. Чувство горя меня переполняло и обуревало, но я просто пытался сохранить внимание на ощущениях. Честно говоря, я до сих пор чувствую это горе, но боль меня не поглотила~"--- она терпимая>>.}
		
	\textit{Во многих аспектах осознанность~"--- это простой навык, поскольку она требует просто замечать то, что происходит, когда оно происходит, задействовав все пять чувств. Например, вы можете попробовать сконцентрироваться на том, что вы воспринимаете через каждое из чувств по очереди.}	
\end{quotation}

Во многих аспектах осознанность~"--- это простой навык, поскольку она требует просто замечать то, что происходит, когда оно происходит, задействовав все пять чувств. Например, вы можете попробовать сконцентрироваться на том, что вы воспринимаете через каждое из чувств по очереди.

\begin{itemize}
	\itemdiamondsuit \textit{Слух}~"--- закройте глаза и прислушайтесь к окружающим вас звукам. Позвольте звукам самим вас достичь. Замечайте, что вы слышите~"--- один звук за другим~"--- с внутренним кивком узнавания. Не нужно называть или описывать то, что вы слышите.
	
	\itemdiamondsuit \textit{Зрение}~"--- широко откройте глаза. Опять замечайте все, что видите~"--- одно визуальное ощущение за другим.
	
	\itemdiamondsuit \textit{Осязание}~"--- снова закройте глаза и попробуйте ощутить, как ваше тело касается стула или как ваши ноги касаются пола.
	
	\itemdiamondsuit \textit{Обоняние}~"--- поднесите ладонь к носу и прочувствуйте запах (или запахи) своей кожи. 
	
	\itemdiamondsuit \textit{Вкус}~"--- обратите внимание на то, чувствуете ли вы прямо сейчас во рту вкус чего-то~"--- возможно, это послевкусие, оставшееся от последнего приема пищи. 
\end{itemize}

Легко быть осознанным пару секунд, но очень сложно оставаться в таком состоянии на протяжении более длительного времени, поскольку это противоречит естественным тенденциям в работе мозга. Нейробиологи смогли идентифицировать взаимосвязанную сеть областей мозга, которая активна, когда сознание отдыхает, и, наоборот, не активна, когда разум занят какой-то задачей~"--- она называется \emph{сетью пассивного режима работы мозга}. Она включает в себя части мозга, расположенные на линии, пересекающей мозг спереди назад и делящей его пополам. Эти части становятся очень активными, когда наше внимание ничего не захватывает и сознание блуждает. 

У этой сети три базовые функции: (1) она создает чувство своей индивидуальности, самоощущение, (2) проецирует созданное ощущение себя в прошлое или будущее и (3) ищет вопросы или задачи для выполнения. Например, бывало ли с вами, что вы сели за обед или за ужин и через несколько минут уже сидели перед пустой тарелкой до того, как вы успели это осознать? Где было ваше внимание? Пока вы (точнее, ваше тело) ели, ваше сознание было где угодно, только не там~"--- потерялось в сети пассивного режима работы мозга. Мозг использует свое <<свободное время>>, чтобы сфокусироваться на потенциальных задачах, которые нужно выполнить. Это очень ценное преимущество с точки зрения эволюции, потому что позволяет нам предчувствовать угрозы нашему существованию, но прожить так всю жизнь не очень-то приятно. Когда мозг работает в пассивно режиме, часто возникают трудности, но у нас не хватает присутствия сознания, чтобы понять, что у нас трудности. Когда мы осознанны, мы замечаем свой внутренний монолог и не так в нем теряемся. 

Для объяснения этого феномена часто используется следующая аналогия: вы сидите в кинотеатре и настолько увлеклись происходящим на экране, что цепляетесь за подлокотники, когда героя вот сбросят со скалы. Вдруг зритель рядом с вами чихает, и вы понимаете: <<О, да я же просто фильм смотрю!>>

Осознанность дает нам свободное мысленное пространство, а с  ним приходит свобода выбирать, как нам бы хотелось \emph{отреагировать} на какую-то ситуацию. Осознанность особенно нужна для выработки самосострадания, потому что, когда мы страдаем, она открывает дверь состраданию. Например, мы можем спросить себя: <<Что мне сейчас нужно?>> и попытаться утешить и поддержать себя, как хорошего друга. Исследования оказали, что один из плюсов регулярной практики осознанности~"--- это то, что она снижает активность сетью пассивного режима работы мозга как во время медитации, так и во время обычных повседневных занятий. Это значит, что, чем больше мы практикуем осознанность, тем больше у нас возможностей принять правильные для себя решения~"--- в том числе решение практиковать самосострадание. 

\newpage
\Meditation{Любящее дыхание}\label{medit:Affectionate_Breathing}

Эта медитация тренирует концентрацию внимания и расслабление разума. Дыхательная медитация~"--- это одна из самых распространенных разновидностей медитации. Здесь мы дополняем ее рекомендациями, как привнести в процесс любовь и нежность.

Большинство медитаций в этой книге выполняются с закрытыми глазами, но читать так, конечно, трудновато, поэтому мы советуем прочитать инструкции несколько раз до выполнения практики или просто открывать глаза по ходу медитации и читать следующий пункт, а потом сразу же закрывать. Какой бы подход вы бы ни выбрали, важно относиться к медитации максимально легко и помнить, что вы не обязаны все сделать идеально (особенно когда цель мотивации~"--- самосострадание!).

\begin{itemize}
	\item Примите позу, в которой вашему телу комфортно и будет комфортно в течение всей медитации. Потом мягко, плавно закройте глаза (частично или полностью). Несколько раз медленно вдохните и выдохните, избавьтесь от любого напряжения в теле.

	\item Если хотите, можете положить руку на сердце в напоминание того, что тренируем мы не просто осознанность, а любящую осознанность, по отношению к нашему дыханию и нам самим. Вы можете оставить руку там или убрать ее в любое время.
	
	\item Начните чувствовать ощущения в теле при дыхании; ощущайте, как вы вдыхаете и как выдыхаете.
	
	\item Заметьте, как ваше тело напитывается кислородом при вдохе и расслабляется при выдохе.
	
	\item Попробуйте позволить своему телу просто дышать, никаких активных действий с вашей стороны не требуется.
	
	\item Теперь обратите внимание на ритм своего дыхания, чередование вдохов и выдохов. Почувствуйте свой естественный ритм дыхания.
	
	\item Ощутите, как все ваше тело едва уловимо двигается при дыхании, как море.
	
	\item Ваше сознание будет блуждать, как любопытный ребенок или маленький щенок. Когда это происходит, просто спокойно вернитесь к наблюдению за ритмом дыхания.
	
	\item Разрешите всем вашему телу слегка покачиваться и чувствовать ласку, исходящую изнутри, при дыхании.
	
	\item Если вам того хочется, вы можете полностью отдаться дыханию; как будто оно~"--- все, что существует. Дышите. Будьте своим дыханием. 
	
	\item А теперь освободите свое внимание, тихо сидя, и позвольте себе чувствовать именно то, что вы чувствуете, быть точно таким, какой вы есть.
	\item Медленно и плавно откройте глаза. 
\end{itemize} 

\Reflection{Уделите минуту на то, чтобы \textbf{осмыслить} все, что вы испытали. <<Что я заметил?>> <<Как я себя почувствовал?>> <<Как я себя чувствую сейчас?>> 
	
	Если вы занимались дыхательной медитацией до этого, то каковы ваши ощущения от попытки привнести любовь и уважение в практику, разрешить вашему дыханию успокоить вас? 
	
	Вы заметили, что ваше внимание к процессу увеличивалось, когда вы дышали с удовольствием?
	
	Была ли для вас разница между тем, когда вы были дыханием и тем, когда вы пытались сфокусироваться на дыхании?
	
	Возможно, вы заметили, что много отвлекались во время медитации. Так работает наше сознание~"--- это сеть пассивного режима работы мозга в действии. Ни в коем случае не осуждайте себя за то, что у вас, как и у всех людей, сознанию свойственно блуждать, но даже если у вас не получится этого сделать, отнеситесь с состраданием к себе и этой общечеловеческой тенденции.
	
	Иногда, когда люди занимаются дыхательной медитацией, они концентрируются на ощущениях в каком-то одном месте, например, в ноздрях, что может сужать сознание. Если для вас это так, то попробуйте больше внимания отвести движению своего тела во время дыхания. Другими словами, фокусируйтесь не на самом дыхании, а на легком покачивании вашего тела, которое оно создает. 
	
	Это\textit{ одна из трех основных медитативных практик} в курсе ОСС, поэтому имеет смысл проделывать ее минут по 20 несколько дней подряд, пока у вас не начнет все получаться. Если эта практика вас успокаивает, можете выполнять ее регулярно. Помните, что мы рекомендуем выполнять комбинацию формальных (медитация) и неформальных (повседневная жизнь) практик примерно 30 минут в день.
}

\newpage
\InformalPractices{Камень <<Здесь и сейчас>>} \label{IP:Here-and-now_stone}

Найдите небольшой камушек, который вам нравится. Потом попробуйте следующее упражнение:

\begin{itemize}
	\item Для начала внимательно разглядите ваш камень. Заметьте его цвета, углы и то, как свет играет на его поверхности. Разрешите себе насладиться созерцанием.
	
	\item Теперь исследуйте камень с помощью осязания. Он гладкий или шершавый? Какая его температура?
	
	\item Позвольте себе полностью погрузиться в настоящий момент и полностью сфокусироваться на камне.
	
	\item Разрешите себе ощутить камень всеми органами чувств, восхищаясь его уникальностью.
	
	\item Заметьте, что когда ваше внимание сконцентрировано на камне, там остается мало места для беспокойства или сожаления о прошлом или будущем. В настоящий момент вы находитесь <<дома>>.
\end{itemize}

\Reflection{Что вы заметили, когда закрепили свое внимание на своем камне - «здесь и сейчас»? 
	Когда вы были заняты камнем, возможно, ваше сознание меньше блуждало и сеть пассивного режима работы мозга была менее активна? Если да, вы можете представить себе, что этот камень~"--- «волшебный», потому что он помогает вашему мозгу выйти из пассивного режима.
	В будущем, может быть, будет полезно держать этот камень в кармане. Когда вас захлестывают эмоции, потрите его пальцами. Прислушайтесь к ощущению прикосновения в камне. Насладитесь этим. Вернитесь в настоящий момент.}

\newpage
\InformalPractices{Осознанность в повседневной жизни}

\begin{itemize}
	\item Осознанность можно практиковать в любой момент вашего дня: пока чистите зубы, идете из гаража на работу, едите завтрак или когда у вас звонит телефон.
	
	\item \textit{Выберите привычное занятие.} Вы можете выбрать утренний кофе, принятие душа или переодевание. Если хотите, выберите что-то, что вы делаете в достаточно раннее время, пока ваше внимание ни на что не отвлечено.
	
	\item \textit{Выберите одно из пяти чувств}, чтобы исследовать это занятие~"--- например, вкус, который вы ощущаете, пока пьете кофе, или ощущение прикосновение воды к вашему телу в душе.
	
	\item\textit{ Погрузитесь в свои ощущения}, насладитесь ими по полной. Когда вы замечаете, что отвлеклись, снова и снова возвращайте внимание к тому, что вы чувствуете.
	
	\item\textit{ Продолжайте занятие с ласковой и дружелюбной осознанностью}, пока не закончите.
	
	\item Постарайтесь привнести осознанность в это занятие каждый день в течение недели.
\end{itemize}

\Reflection{Заметили ли вы какие-то перемены после привнесения осознанности в свою повседневную жизнь? Если вам трудно заниматься медитацией регулярно, посвящение нескольких минут в день неформальной практике осознанности тоже помогает выработать привычку осознанно находиться в настоящем моменте. Это не <<менее полезная>> практика, потому что наша цель~"--- проявлять осознанность как можно больше времени.}

