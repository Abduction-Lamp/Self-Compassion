% !TEX root = ../../Self-Compassion.tex

\chapter{Двигаясь дальше} \label{Taking_It_Forward}

Это пособие подошло к концу, и вы научились огромному множеству практик и подходов для развития самосострадание. Возможно, сейчас вы немного озадачены тем, как бы ввести все это в свою повседневную жизнь и как продолжать практиковаться в ближайшие несколько месяцев или даже лет. 

Иногда у людей возникает вопрос: <<Какая из практик мне подходит?>> На этот ответ лучше всего ответила учительница медитации  Шерон Зальцберг: <<Та, которой вы больше всего будете себя посвящать!>> Конечно, это мы узнаем только ретроспективно, но хорошее начало~"--- это практики, которые для вас были самыми легкими и приятными. Что это за практики? Чуть позже у вас будет возможность задуматься над этим.

Также неплохо знать, какие практики были для вас самыми полезными или значимыми. Может быть, вы встретились при их выполнении с обратной тягой, но чувствуете, что до свободы всего несколько шагов. Можете отметить это и вернуться к этой практике, когда будете готовы, в то же время подходя к своим практикам с самосостраданием.  

\vspace{2ex}

Вот несколько советов о том, как продолжать практику: 

\begin{itemize}
	\itemdiamondsuit Сделайте свою практику как можно более приятным, чтобы это усиливало ее эффект.
	\itemdiamondsuit Начинайте с малого~"--- даже короткие практики могут принести значительные изменения.
	\itemdiamondsuit Практикуйтесь в ходе повседневной жизни, когда вы больше всего в этом нуждаетесь.
	\itemdiamondsuit Если пропускаете практику или что-то не получается, проявите к себе сострадание и начните заново.
	\itemdiamondsuit Отпустите ненужные усилия делать все правильно~"--- просто относитесь к себе тепло и дружественно.
	\itemdiamondsuit Практикуйтесь каждый день в одно время.
	\itemdiamondsuit Определите, что препятствует практике.
	\itemdiamondsuit Читайте книги об осознанности и самосострадании.
	\itemdiamondsuit Ведите дневник своей практики.
	\itemdiamondsuit Установите связь с другими людьми~"--- практикуйтесь группой.
	\itemdiamondsuit Пройдите курс ОСС. На сайте Центра Осознанного Самосострадания\\ (\url{http:\\centerformsc.org}) есть список курсов ОСС во всем мире, а также онлайн-тренинги.
\end{itemize}

\newpage
\Exercises{Что бы мне хотелось запомнить?} \label{Ex:What_Would_I_Like_to_Remember?}

До того, как распрощаться с этой книгой, вам бы, возможно, хотелось поразмыслить над тем, чему вы научились. Может быть, ваш мозг перегружен огромным количеством новой информации. Так и возникает вопрос: <<Что бы мне хотелось запомнить?>> Ответьте на два следующих вопроса: один из них~"--- <<вопрос сердцу>>, другой~"--- практический вопрос.

\vspace{2ex}

\noindent{\large \textbf{Вопрос сердцу}}

\begin{itemize}
	\item Закройте глаза на минуту и позвольте себе обдумать весь опыт пользования этим пособием. Сканируя <<территорию>> вашего сердца, спрашивайте себя: <<Что меня затронуло, сдвинуло что-то внутри меня?>> Чтобы помочь своей памяти, можете пройтись по записям, которые вы сделали здесь или в отдельной тетради.

	Это может быть что угодно~"--- возможно, некий сюрприз или внезапное понимание.
	А может быть, это что-то, что успокаивало вас, поднимало вам настроение, вызвало у вас трудности или изменило вас? 

	\itemWritingHand Не спешите и записывайте все, что приходит в голову~"--- что вы хотели бы запомнить.
\end{itemize}

\setlength{\extrarowheight}{2mm}
\begin{tabularx}{0.96\textwidth}{X}
	\\
	\arrayrulecolor{gray}\hline\\
	\hline\\
	\hline\\
	\hline\\
	\hline\\
	\hline\\	
	\hline\\
	\hline\\
	\hline\\
	\hline\\
	\hline\\
	\hline\\
	\hline\\
	\hline\\
\end{tabularx}
\setlength{\extrarowheight}{0mm}


\noindent{\large \textbf{Практический вопрос}}

\begin{itemize}
	\itemWritingHand Теперь запишите практики, которые вам хотелось бы помнить и повторять, двигаясь дальше. Подумайте~"--- может быть, есть какие-то формальные медитации, которые вам помогли, и неформальные практики для повседневной жизни. Чтобы помочь себе вспомнить практики, пролистайте это пособие, отмечая практики, вызвавшие у вас особый отклик или произвели на вас сильный эффект.
\end{itemize}

\setlength{\extrarowheight}{2mm}
\begin{tabularx}{0.96\textwidth}{X}
	\\
	\arrayrulecolor{gray}\hline\\
	\hline\\
	\hline\\
	\hline\\
	\hline\\
	\hline\\	
	\hline\\
	\hline\\
	\hline\\
	\hline\\
	\hline\\
	\hline\\
	\hline\\
	\hline\\
	\hline\\
	\hline\\
	\hline\\
	\hline\\
	\hline\\
	\hline\\
	\hline\\
	\hline\\
	\hline\\
	\hline\\
\end{tabularx}
\setlength{\extrarowheight}{0mm}

\begin{center}
	{\huge ЗАКЛЮЧИТЕЛЬНОЕ СЛОВО}
\end{center}

\vspace{2ex}

Мы искренне благодарны вас, нашим читателям, за то, что вы присоединились к нам на нашем пути осознанности и самосострадания. Мы знаем, что, чтобы открыться полноте человеческой жизни, нужны смелость и упорство. Надеемся, что ваши старания уже принесли плоды~"--- возможно, в виде легкого и счастливого сердца. В этом практика парадоксальна~"--- чем глубже мы ныряем в страдание с осознанностью и состраданием, тем больше это освобождает наши сердца. 

Но нужно быть терпеливыми. Практика осознанности и самосострадания~"--- это путь длиною в жизнь. Это хорошо, поскольку это делает каждый момент нашей жизни более ценным через понимание того, что каждый момент~"--- это возможность попрактиковаться. Особенно мы ценим занятия вместе~"--- в сообществе. Мы надеемся, что вы считаете себя частью этого растущего сообщества самосострадания. В заключение, пусть плоды наших совместных усилий будут направлены на все живые существа, и пусть мы никогда не забудем включить себя в большой круг сострадания. 
