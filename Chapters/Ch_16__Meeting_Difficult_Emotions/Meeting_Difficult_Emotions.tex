% !TEX root = ../../Self-Compassion.tex

\chapter{Столкновение с трудными эмоциями} \label{Meeting_Difficult_Emotions}

Жизнь редко когда бывает легкой. Она часто приносит затруднительные ситуации, а вместе с ними~"--- \textbf{сложные эмоции} вроде \emph{гнева}, \emph{страха}, \emph{беспокойства} или \emph{горя}. С возрастом к нам приходит понимание, что пытаться убежать от проблем бесполезно~"--- с ними нужно разбираться напрямую. 

Когда мы поворачиваемся к трудным эмоциям лицом, даже если мы осознанны и сострадательны к себе, сначала наша боль часто только увеличивается, и у нас появляется естественный инстинкт от нее отвернуться. Но для того, чтобы душевные раны зажили, нужно с ними опять столкнуться~"--- выход из такой ситуации только один и ведет через них. \textbf{Чтобы жить настоящей и здоровой жизнью, нужно иметь смелость побыть какое-то время со своей эмоциональной болью.} Значит ли это, что мы должны столкнуться со своими эмоциями, когда их интенсивность максимальна? К счастью, нет. Однажды кто-то спросил учителя медитации Тхить Нят Ханя, сколько эмоционального дискомфорта и боли нам нужно оставить в нашей практике. Он ответил: <<Не очень много!>> 

Прочувствовать дискомфорт важно, чтобы подготовить почву для самосострадания, но для этой задачи нам нужно всего лишь \emph{дотронуться} до эмоциональной боли, и можно это делать медленно, чтобы не перегрузить себя. Искусство самосострадания включает в себя умение \emph{постепенно} поворачиваться навстречу эмоциональному дискомфорту, когда он возникает. 

\vspace{2ex}

Существуют \textbf{пять стадий принятия при столкновении с трудными эмоциями}, и на каждой стадии мы постепенно отпускаем все больше и больше эмоционального сопротивления\cite{87}. 
\begin{enumerate}
	\item \textbf{Сопротивление:} пытаемся бороться с происходящим~"--- <<Уходи!>>
	\item \textbf{Исследование:} смотрим на дискомфорт с любопытством~"--- <<Что я сейчас чувствую?>>
	\item \textbf{Терпение:} стойко переносим, держимся~"--- <<Мне это не нравится, но я могу это вытерпеть>>.
	\item \textbf{Разрешение:} позволяем чувствам приходить и уходить~"--- <<Все в порядке, я могу освободить для них место.>>
	\item \textbf{Дружба:} видим ценность в эмоционально сложных ситуациях~"--- <<Чему я могу научиться? Что я могу вынести из этого опыта?>>	
\end{enumerate}

Вы можете использовать эти стадии принятия, чтобы понять, как обезопасить себя, выполняя упражнения из этой книги. Если вы чувствуете, что вы перегружены, мудрым решением будет сделать шаг назад~"--- возможно, просто остаться любопытным, не открываясь при этом до конца трудным эмоциям. Отступление ради собственной безопасности~"--- потенциально самое полезное умение, которому может научить самосострадание.

Ресурсы осознанности и самосострадания помогают нам работать со сложными эмоциями, не избегая их и не пытаясь им сопротивляться, но и не доходя до перегрузки. 

\vspace{2ex}

Есть \textbf{три} особенно полезные стратегии работы с трудными эмоциями:
\begin{enumerate}
	\item Давать эмоциям название
	\item Быть осознанным к телесным проявлениям эмоций
	\item Смягчить--успокоить--разрешить
\end{enumerate} 

Первые два подхода построены на осознанности, а третий ориентирован на сострадание. 

\vspace{3ex}

\noindent{\large \textbf{Давать эмоциям название}}

\vspace{1ex}

<<Дайте этому название, и вы его приручите>>. Давать сложным эмоциям названия или хотя бы определять их категорию~"--- это практика, помогающая нам <<выпутаться>> из них. Исследования показали, что, когда мы называем сложные эмоции своими именами, миндалевидное тело~"--- область мозга, отвечающая за выявление опасности~"--- становится менее активным, и меньше вероятность, что оно спровоцирует телесную реакцию на стресс\cite{88}.

Когда мы мягко говорим <<это гнев>> или <<во мне возникает страх>>, мы обычно чувствуем некую эмоциональную свободу~"--- как будто вокруг этого ощущения освободилось место. Вместо того, чтобы потеряться в какой-то эмоции, мы можем признать, что мы ее испытываем, и таким образом дать себе больше вариантов реакции на нее.

\vspace{3ex}

\noindent{\large \textbf{Осознанность к телесным проявлениям эмоций}}

\vspace{1ex}

<<Прочувствуйте это, и оно заживет>>. У эмоций есть и умственные, и физические проявления. Например, когда мы злимся, мы замыкаемся в своей голове, мысленно доказывая свою точку зрения и планируя, что мы скажем (или что мы должны были бы раньше сказать). Мы также чувствуем физическое напряжение в брюшной полости, так как организм готовится к схватке. Регулировать сложные эмоции через мысли труднее, так мы легко теряемся в них. Работать с физическими ощущениями часто бывает легче. Мысли сменяют друг друга так быстро, что тяжело задержаться на них достаточно надолго, чтобы их изменить. А изменения в теле, наоборот, происходят достаточно медленно. Когда мы находим расположение наших эмоций в теле~"--- обнаруживаем физические ощущения, вызванные эмоциями, и осознанно наблюдаем за ними~"--- часто трудные эмоции начинают меняться сами по себе.

\begin{quotation}
	\textit{
		Когда Кейле, матери-одиночке, пришел счет из университетского книжного магазина, она была в полнейшем шоке от суммы. Она дала своей дочери Дине, когда она поехала в университет, кредитную карточку, чтобы та смогла закупиться учебниками и канцелярскими принадлежностями, но не имела никакого представления о том, как дорого стоят учебники. Кейла очень расстроилась, начала потеть и заламывать руки. После оплаты учебы Дины за первый семестр у нее и так был негативный баланс. Как она это все оплатит? Может, ей придется работать сверхурочно? Но ей врач уже говорил, что у нее слишком высокое давление, чтобы так напрягаться. Попросить бывшего мужа помочь? Еще чего! У него уже давно была новая семья, и он не раз говорил, что оставит Дину без какой-либо материальной поддержки, когда ей исполнится 18. Вот урод. Ей оставалось только позвонить дочери и сказать ей вернуть книги в магазин в надежде на то, что ей удастся занять их у друзей. А может, просто перевести Дину в университет подешевле?
	}

	\textit{
		Кейла понимала, что сначала ей надо успокоиться и попробовать применить некоторые из техник осознанности, которым она научилась. Она сделала себе чашку чая и в конце концов смогла освободить место в своей голове, чтобы спросить себя, что она чувствует. <<Страх? Стоп, не то... грусть!>> Было бы очень грустно, если бы ей пришлось забрать Дину из университета, ради поступления в который она так упорно работала. Да, счет высоковат, но он не сделает ее банкротом. Скоро ей выплатят премию на работе, которая может покрыть <<минус>> на счету. Просто называя и валидируя свои эмоции, Кейла смогла увидеть ситуацию в перспективе и посмотреть на вещи боле ясным взглядом. Затем она попыталась выяснить, где эта грусть в ее теле находится. Она в основном чувствовала ее в районе сердца~"--- как какую-то пустоту и тяжесть. Когда Кейла направила свое внимание и осознанность на ощущения в этом месте, интенсивность ее грусти упала еще ниже.
	}
\end{quotation}

\newpage
\noindent{\large \textbf{Смягчить--Успокоить--Разрешить}} \label{IP:Soften–Soothe–Allow}

\vspace{1ex}

Сложные эмоции становятся даже более проходящими~"--- они проще приходят и уходят из нас~"--- когда мы создаем в себе любящее и принимающее к ним отношение. Когда в нашей осознанности есть примесь страха, мы менее открыты своим эмоциям и еле-еле можем перенести такой опыт. Но когда наша осознанность нежная и теплая, у нас находятся силы прочувствовать происходящее внутри нас и обеспечить себя тем, в чем мы нуждаемся.

Смягчить--успокоить--разрешить~"--- это комбинация сострадательных реакций на сложные эмоции. 

\vspace{2ex}

\textbf{Мы можем привести себя в чувство тремя способами}:
\begin{itemize}
	\item \textbf{Смягчение}~"--- физическое сострадание
	\item \textbf{Успокоение}~"--- эмоциональное сострадание
	\item \textbf{Разрешение}~"--- умственное сострадание
\end{itemize}

Этот метод добавляет сострадания к двум предыдущим осознанным подходам. Вместо того, чтобы просто осознавать свой трудный опыт, мы \textbf{относимся к нему с теплотой}.  Самосострадание приносит еще большее чувство эмоциональной защищенности, так что в голове остается место для того, чтобы работать с эмоциями и учиться на них. 

\begin{quotation}
	\textit{
		Когда Кейла тем вечером легла спать, она долго не могла заснуть. Она все еще была расстроена, так что решила использовать практику <<смягчить-успокоить-разрешить>>, которой она недавно научилась. Сначала она вернулась к называнию того, что она чувствовала~"--- все еще грусть, смешанную со страхом -  и почувствовала в сердце, как и раньше, сильную боль. Потом она подключила сострадание. Она <<смягчила>>, как могла, свое тело, чтобы ощущение в грудной клетке чувствовалось не так сильно. Затем она положила руку на сердце и стала нежно его поглаживать, нежно и осторожно описывая ладонью небольшие круги, и сказала себе, как будто бы говорила с хорошей подругой: <<Мне так жаль, что у тебя сейчас материальные проблемы, дорогая, это очень несправедливо. Конечно, они вгоняют тебя в грусть~"--- ты же хочешь самого лучшего для своей дочери. Мы как-нибудь отсюда выберемся>>.
	}
	
	\textit{
		Когда Кейла проявила к себе понимание и поддержку, ее грусть уже не была такой тяжелой. Она могла позволить ей остаться там, где она была, и держать ее с нежностью. Она также поняла, что в этой ситуации был урок для нее: часто Кейла доставляла себе ненужные страдания, всегда думаю, что все пойдет по худшему варианту развития событий, что только повышало ее давление. Но на самом деле ей же абсолютно необязательно проходить через все эти мучения. Когда Кейла оставалась наедине со своим страхом и грустью (и самой собой!) со смелостью и добротой, эти чувства не брали над ней верх Это открытие дало Кейле уверенность в том, что она справится со всеми трудностями (особенно с трудностями, с которыми обычно сталкиваются матери-одиночки).
	}	
\end{quotation}


\newpage
\InformalPractices{Работа с трудными эмоциями} \label{IP:Working_with_Difficult_Emotions}

Описанные выше подходы к сложным эмоциям можно использовать отдельно или в комбинации друг с другом, и больше всего они полезны в повседневной жизни, когда вы больше всего в них нуждаетесь. Вы можете следовать инструкциям ниже или послушать доступную в Интернете аудиозапись.

\begin{itemize}
	\item Найдите удобное положение сидя или лежа, закройте глаза и сделайте три глубоких, успокаивающих вдоха и выдоха.
	
	\item Положите на минуту руку на сердце в напоминание того, что вы находитесь здесь и сейчас в этой комнате и что вы тоже достойны доброты.
	
	\item Вспомните какую-нибудь ситуацию \emph{среднего} или \emph{низкого} уровня сложности, в которой вы сейчас находитесь: это могут быть проблема со здоровьем, напряженные отношения с кем-то или страдания близкого человека. Не выбирайте слишком сложную проблему или какой-то пустяк~"--- выбранная вами проблема должна вызывать легкое напряжение в теле, когда вы о ней думаете.
	
	\item Четко визуализируйте эту ситуацию. \emph{Кто} в ней фигурирует? \emph{Что происходит}?
\end{itemize}

\vspace{2ex}

\noindent{\large \textbf{Дайте название эмоциям}}
\begin{itemize}
	\item Вновь переживая эту ситуацию, заметьте, возникают ли у вас внутри какие-нибудь эмоции. Если да, посмотрите, можете ли вы дать им \emph{названия}. Например:	
	\begin{itemize}
		\item Гнев
		\item Грусть
		\item Горе
		\item Замешательство
		\item Страх
		\item Тоска
		\item Отчаяние
	\end{itemize}

	\item Если вы чувствуете много разных эмоций, попробуйте дать название самой сильной эмоции, которую у вас вызывает эта ситуация. 
	
	\item Теперь повторите про себя название этой эмоции мягким, понимающим тоном, как будто вы валидируете чувства подруги: <<Это тоска>>. <<Это горе>>.
\end{itemize}

\vspace{2ex}

\noindent{\large \textbf{Осознанность к телесным проявлениям эмоций}}
\begin{itemize}
	\item Теперь включите все свое тело в поле вашего внимания и осознанности.
	
	\item Опять вспомните трудную ситуацию и просканируйте свое тело, чтобы найти то место, где вам легче всего ее почувствовать. Пробегитесь мысленным взглядом по всему телу с головы до ног, останавливаясь там, где вы чувствуете небольшое напряжение или дискомфорт.
	
	\item Просто прочувствуйте то, что вы можете прямо сейчас почувствовать в своем теле. Ничего большего.
	
	\item Теперь, если можете, выберите какое-то одно место, где чувство наиболее выражено (например, через мышечное напряжение, чувство пустоты или даже сердечную боль).
	Мысленно мягко  поверните свое внимание к этой точке. Разрешите своей осознанности полностью обосноваться в чувстве, которое эмоция вызвала в вашем теле.	
\end{itemize}

\vspace{2ex}

\noindent{\large \textbf{Смягчить--Успокоить--Разрешить}}
\begin{itemize}
	\item Теперь попробуйте \emph{смягчить} то место, где вы чувствуете трудную эмоцию. Позвольте своим мышцам смягчиться, расслабиться, как в теплой воде. Смягчение... смягчение... смягчение... Помните, что мы не пытаемся изменить это чувство~"--- мы просто поддерживаем его, но нежно, мягко и заботливо.
	
	\item Потом \emph{успокойте} себя в связи с этой трудной ситуацией
	
	\item Если хотите, положите руку на ту часть тела, в которой вы чувствуете больше всего дискомфорта. Можете представить себе, что тепло и доброта перетекают через вашу руку в это место. Можно даже попробовать вообразить, что ваше тело~"--- это тело любимого ребенка.
	
	\item Есть ли какие-то утешающие слова поддержки, которые вам бы нужно было сейчас услышать? Если есть, представьте себе, что у вас есть друг, у которого аналогичные трудности. Что бы вы этому другу сказали? (<<Мне так жаль, что ты так себя чувствуешь>>, <<Ты и твое состояние для меня очень важны>>.) Можете ли вы сказать себе что-то подобное? (<<Так тяжело чувствовать что-то подобное>>, <<Пусть я буду добр к себе>>.)
	
	\item Если вам нужно, открывайте глаза, когда хотите, или приостановите упражнение и просто прочувствуйте свое дыхание.
	
	\item Наконец, \emph{\textbf{разрешите}} дискомфорту остаться. Освободите для него место и отпустите нужду от него избавиться. Разрешите \emph{себе} быть точно таким, какой вы есть, хотя бы на один момент.
	
	\item Если хотите, можете повторить этот цикл с вашей эмоцией, каждый раз заходя немного глубже, не спуская внимания с нее, даже если она перейдет в другое место в вашем теле или вообще превратится в другую эмоцию. Смягчите... успокойте... разрешите. Смягчите... успокойте... разрешите. 
	
	\item Теперь отпустите практику и сфокусируйтесь на своем теле. Разрешите себе чувствовать то, что вы чувствуете, и быть точно таким, какой вы сейчас есть.
\end{itemize}

\Reflection{
	Заметили ли вы какие-то изменения, когда вы \emph{назвали} эмоцию? Что вы заметили, \emph{исследуя свое тело} на предмет физического ощущения, связанного с ней? Что произошло, когда вы \emph{смягчили} эту часть своего тела, \emph{успокоили} себя и \emph{разрешили} эмоции остаться? Поменялась ли эта эмоция за время выполнения упражнения, а может быть, физические ощущения перекочевали в другое место? Возникли ли у вас какие-то трудности во время практики?
	
	Некоторым людям сложно найти точку в своем теле, которая соответствует эмоции. Одна из причин этого~"--- то, что некоторые люди более чувствительны и осознаны к своим телесным ощущениям (это называется интероцепция). Другая причина заключается в том, что, если эмоция слишком сильная, мы можем как бы онеметь. В любом случае, вы можете сфокусироваться на любых проявлениях эмоции, например, на каком-то неясном ощущении напряжения в теле, или даже на онемении, с сострадающей осознанностью.
	
	Иногда изначальная эмоция переходит в какую-то другую или меняет расположение в вашем теле. Например, что-то, что может изначально показаться страхом, выражающемся в напряжении за глазами, может превратиться в горе, сидящее в глубине живота. По мере того как мы учимся определять и чувствовать эмоции и с состраданием разрешаем себе их ощущать, мы часто открываем новые, более глубокие слои эмоций.
	
	Если вы чувствуете перегрузку во время выполнения любых упражнений из этого пособия, приостановите практику, пока снова не почувствуете себя комфортно и в безопасности. Процесс излечения требует времени, а наш предел возможностей нужно уважать. Тише едешь, дальше будешь.
}