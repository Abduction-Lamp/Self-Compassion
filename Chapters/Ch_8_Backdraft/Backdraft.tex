% !TEX root = ../../Self-Compassion.tex

\chapter{Обратная тяга}\label{Backdraft}

<<Обратная тяга>>~"--- это боль (часто очень старая), которая может выйти на поверхность, когда мы начинаем относиться к себе с добротой и состраданием. Она может вызвать непонимание и негативные эмоции, но это необходимый, ключевой этап личностной трансформации.

\emph{Обратная тяга} в буквальном смысле~"--- это термин, который используют пожарные, чтобы описать ситуацию, когда пламя уже использовало весь кислород в помещении и через открытое окно или дверь проникает новый. Воздух стремится в помещение, а огонь~"--- на улицу.  Похожий эффект можно наблюдать, когда мы с помощью самосострадания открываем двери своего сердца. Сердца большинства из нас <<горят>> от страдания, накопившегося за все годы жизни. Чтобы нормально функционировать, нам было нужно забывать стрессовые или болезненные ситуации, защититься от них. Это значит, что, когда мы открываем двери своего сердца и впускаем туда свежий воздух самосострадания, оттуда часто выходят старая боль и страдание. В этом и состоит обратная тяга.

\begin{quotation}
	\textit{Чарльз был доволен первыми двумя занятиями по самосостраданию, но потом он начал сомневаться, правильно ли он все делает. Кладя руку на сердце и пытаясь говорить с собой с добротой, он каждый раз ощущал слабость и тревогу и начинал задыхаться. <<Что со мной не так?>>~"--- думал Чарльз. <<Вроде бы от этого я, наоборот, должен \textbf{лучше} себя чувствовать>>.}
\end{quotation}

Важно понимать, что дискомфорт от обратной тяги \emph{создает} не практика самосострадания. Если с нами это случается, это не значит, что мы что-то не так делаем. Наоборот, это знак, что у нас все получается и мы начинаем открывать двери своего сердца. Но сначала мы можем заново пережить старую боль, так как мы начинаем ее отпускать. Это естественный процесс, из-за которого не стоит волноваться.  

\section*{Как распознать обратную тягу?}\addcontentsline{toc}{subsection}{Как распознать обратную тягу?} \label{How_Do_We_Recognize_Backdraft?}

Обратная тяга может принимать форму любого эмоционального, умственного или физического неудобства или беспокойства. 

Например:
\begin{itemize}
	\item \emph{Эмоционально}~"--- стыд, горе, страх, грусть
	\item \emph{Умственно}~"--- мысли вроде <<Я совсем один>>, <<Я неудачник, у меня ничего никогда не получается>>, <<Я ломаного гроша не стою>>
	\item \emph{Физически}~"--- память тела, различные боли 
\end{itemize}

Часто неудобство возникает как будто ниоткуда, и мы не понимаем, почему это случилось. Во время медитации могут потечь слезы, появиться чувство гнева или страха и уязвимости. Попытка не чувствовать этой обратной тяги может запустить целую цепочку реакций. Мы можем, например, уйти в свои мысли, стать возбужденными, прекратить занятие, отключиться от реальности или начать критиковать себя и других людей. На все эти реакции можно (и нужно) отвечать с добротой и состраданием. Самое важное~"--- не дать чувствам, вызванными обратной тягой, полностью собой овладеть, но и разрешить себе медленно открывать двери своего сердца. Когда вы чувствуете обратную тягу, важно отнестись к себе с состраданием и позволить себе продвигаться в собственном темпе.

\section*{Что можно сделать?}\addcontentsline{toc}{subsection}{Что можно сделать?} \label{What_Can_We_do_about_backdraft?}

Для начала можете спросить себя: <<Что мне прямо сейчас нужно?>> или, что даже более важно: <<Что мне нужно прямо сейчас, чтобы почувствовать себя в \emph{безопасности}?>> В зависимости от вашего ответа вы можете попробовать приглянувшиеся вам стратегии, приведенные ниже. 

\vspace{2ex}

\textbf{Практикуйте осознанность для регуляции внимания}
\begin{itemize}
	\itemdiamondsuit Говоря тоном, каким вы бы говорили с близким другом, назовите вещи своими именами: <<Это называется обратная тяга>>.
	
	\itemdiamondsuit Назовите свою самую сильную эмоцию и валидируйте ее с состраданием: <<А то, что я сейчас чувствую~"--- горе>>.
	
	\itemdiamondsuit Попробуйте понять, где в вашем теле проявляется эта эмоция~"--- это может быть напряжение в голове или ощущение пустоты сердца. Прикоснитесь к этой части тела нежно и с поддержкой.
	
	\itemdiamondsuit Перенаправьте свое внимание на что-то нейтральное в своем теле (например, дыхание) или к чему-то во внешнем мире (это могут быть окружающие вас звуки или ваш камень <<здесь-и-сейчас>>~"--- глава \ref{Mindfulness}, стр.\:\pageref{IP:Here-and-now_stone}). Чем дальше от своего тела вы направите внимание, тем легче будет.
	
	\itemdiamondsuit Ощутите свои стопы (см. далее~--- стр.\:\pageref{IP:Feeling_the_Soles_of_Your_Feet}).
\end{itemize}

\vspace{2ex}

\textbf{Найдите себе убежище в привычных занятиях}
\begin{itemize}
	\item Возможно, вам понадобится направить свою осознанность на какое-то повседневное занятие, такое как мытье посуды, прогулка, душ или физические упражнения. Если это будет занятие, которое приятно для ваших органов чувств, разрешите себе им насладиться.

	\item А может быть так, что вам понадобится успокоить, утешить или поддержать себя с помощью чего-то практического: можете выпить чашку чая, принять теплую ванну, послушать музыку или погладьте собаку.

	\item Если вам после этого все еще нужна будет помощь, обратитесь к своей <<группе поддержки>> (друзья, семья, психотерапевты, учителя) и попросите у них то, что вам нужно.
\end{itemize}

\begin{quotation}
	\textit{Когда Чарльз узнал об обратной тяге, он больше не расстраивался так сильно, когда с ним это происходило. Когда тревога начинала расти, он говорил себе: <<А, ну это же обратная тяга. Это нормально>>. Он даже понял причину своей обратной тяги. В его детстве его мать частенько выпивала, и, хотя обычно она была очень ласковой и любящей, иногда она на него огрызалась и злилась без очевидных на то причин. Он рано понял, что не может всегда рассчитывать на ее любовь и поддержку~"--- это зависело от того, сколько вина она выпила. Теперь Чарльз осознал, что, давая себе любовь и поддержку, он бередил старые раны и будил детское чувство небезопасности. Иногда просто одно понимание спасало его от повышения тревожности и одышки. В других случаях обратная тяга была сильнее, и он знал, что самое полезное, что он мог для себя сделать~"--- отступить на шаг назад, немного отвлечься. «Попробую ощутить свои стопы, это дает мне ощущение устойчивости и надежности». Иногда Чарльза обуревали более интенсивные эмоции~"--- страх и отвращение~"--- и он знал, что нужно на время прервать практику и вместо этого сделать что-нибудь привычное и приятное, например, прокатиться на велосипеде по набережной. Потом, когда ему стало легче, он вернулся к намеренной практике самосострадания (например, класть ладони на сердце) из любопытства, особенно никак изменений не ожидая.}
\end{quotation}

\newpage
\InformalPractices{Почувствуйте свои стопы}\label{IP:Feeling_the_Soles_of_Your_Feet}
Цель этой практики~"--- стабилизировать ваше состояние и дать вам чувство опоры, когда вы испытываете очень интенсивные эмоции или обратную тягу. Исследования показали, что эта практика может помочь отрегулировать сильные эмоции, такие как гнев.

\begin{itemize}
	\item Встаньте и почувствуйте, как ваши стопы стоят на полу. Это можно делать в обуви или без нее.
	
	\item Заметьте ощущение прикосновения ваших стоп к полу.
	
	\item Чтобы яснее чувствовать ощущения в стопах, плавно пораскачивайтесь на ступнях вперед---назад и в стороны. Попробуйте описывать коленями небольшие круги, чувствуя, как при этом меняются ощущения в стопах.
	
	\item Если ваше внимание отвлечется, просто направьте его обратно на ощущения в стопах.
	
	\item Теперь начните идти~"--- медленно, замечая изменение ощущений в стопах. Обращайте внимание на ощущения, когда поднимаете стопу, делаете шаг и снова ставите ее на пол. Теперь сделайте то же самое с другой стопой. Потом чередуйте.
	
	\item Пока вы идете, думайте о том, какая маленькая площадь поверхности ваших стоп и о том, как, несмотря на это, они поддерживают все ваше тело. Если хотите, можете сделать <<минутку благодарности>> за то, как много работают ваши стопы~"--- обычно мы это принимаем как данность.
	
	\item Продолжайте идти~"--- медленно, ощущая свои стопы.
	\item После этого опять встаньте и расширьте область своей осознанности, наблюдая за всем телом и позволяя себе чувствовать то, что вы чувствуете, и быть таким, какой вы есть.
\end{itemize}

\newpage
\Reflection{Что вы заметили во время выполнения этой практики?

Есть много факторов, которые делают эту практику эффективной для случаев, когда вас переполняют сильные эмоции. Во-первых, ваше внимание сконцентрировано на стопах~"--- чем дальше от вашей головы (где разворачивается история, которая привела к переживаниям), тем лучше. Во-вторых, ощущение контакта с землей может дать вам чувство поддержки и (в буквальном смысле) опоры. Можете снять обувь и проделать эту практику на траве, когда будет такая возможность, чтобы связь с землей ощущалась еще сильнее.

Делайте эту практику в любом месте и в любое время когда у вас возникают сложные эмоции~"--- в аэропорту перед проверкой безопасности, в коридоре на работе и~т.\,д.}

\newpage
\InformalPractices{Самосострадание в повседневной жизни}\label{IP:Self-Compassion_in_Daily_Life}
\begin{itemize}
	\item Важно помнить, что вы уже умеете применять самосострадание. Если бы вы не умели заботиться о себе, вы бы столько не прожили. Забота о себе~"--- это и есть самосострадание, добрый ответ на боль. Таким образом, самосостраданию может научиться каждый.
	
	\item Самосострадание~"--- это гораздо больше, чем просто тренировка мозга. Поведенческое самосострадание~"--- это безопасный и эффективный способ практики самосострадания. Он вносит самосострадание в обычные занятия повседневной жизни.
	
	\item Если вы часто сталкиваетесь с обратной тягой, когда практикуете самосострадание напрямую (например, через успокаивающее прикосновение), поищите более привычных методы, которые помогли бы вам чувствовать себя в безопасности.
	
	\item Занесите в этот список все способы и приемы, которые вы уже используете, чтобы заботиться о себе, и придумайте какие-нибудь новые, которые вы можете добавить в свой репертуар.
	
	\item Попробуйте выполнять любые из этих рекомендаций, проявляя заботу к себе в трудный момент.  
\end{itemize}

\vspace{5ex}

{\large \textbf{Физически~"--- успокойте тело}}

\begin{itemize}
	\itemWritingHand Как вы заботитесь о своем теле? (например, физические упражнения, массаж, теплая ванна, чашка чая?)
\end{itemize}

\setlength{\extrarowheight}{2mm}
\begin{tabularx}{\textwidth}{X}
	\\
	\arrayrulecolor{gray}\hline\\
	\hline\\
	\hline\\
	\hline\\
	\hline\\
	\hline\\
	\hline\\
\end{tabularx}
\setlength{\extrarowheight}{0mm}

\begin{itemize}
	\itemWritingHand Можете ли вы придумать какие-то другие способы снять стресс и напряжение?
\end{itemize}

\setlength{\extrarowheight}{2mm}
\begin{tabularx}{\textwidth}{X}
	\\
	\arrayrulecolor{gray}\hline\\
	\hline\\
	\hline\\
	\hline\\
	\hline\\
	\hline\\
\end{tabularx}
\setlength{\extrarowheight}{0mm}

\vspace{5ex}

{\large \textbf{Умственно~"--- уменьшите возбуждение}}

\begin{itemize}
	\itemWritingHand Как вы заботитесь о своем разуме, особенно в стрессовых ситуациях (медитация, просмотр фильма, прочтение вдохновляющей книги)?
\end{itemize}

\setlength{\extrarowheight}{2mm}
\begin{tabularx}{\textwidth}{X}
	\\
	\arrayrulecolor{gray}\hline\\
	\hline\\
	\hline\\
	\hline\\
	\hline\\
	\hline\\
	\hline\\
\end{tabularx}
\setlength{\extrarowheight}{0mm}
	
\begin{itemize}
	\itemWritingHand Есть ли какая-то новая стратегия, которой вы хотели бы воспользоваться, чтобы научиться быстрее отпускать свои мысли?
\end{itemize}

\setlength{\extrarowheight}{2mm}
\begin{tabularx}{\textwidth}{X}
	\\
	\arrayrulecolor{gray}\hline\\
	\hline\\
	\hline\\
	\hline\\
	\hline\\
	\hline\\
\end{tabularx}
\setlength{\extrarowheight}{0mm}

\vspace{5ex}

{\large \textbf{Эмоционально~"--- утешьте и успокойте себя}}

\begin{itemize}
	\itemWritingHand Как вы заботитесь о своем эмоциональном состоянии (например, гуляете с собакой, ведете дневник, готовите)?
\end{itemize}

\setlength{\extrarowheight}{2mm}
\begin{tabularx}{\textwidth}{X}
	\\
	\arrayrulecolor{gray}\hline\\
	\hline\\
	\hline\\
	\hline\\
	\hline\\
	\hline\\
\end{tabularx}
\setlength{\extrarowheight}{0mm}

\begin{itemize}
	\itemWritingHand Хотели бы вы попробовать что-нибудь новое?
\end{itemize}

\setlength{\extrarowheight}{2mm}
\begin{tabularx}{\textwidth}{X}
	\\
	\arrayrulecolor{gray}\hline\\
	\hline\\
	\hline\\
	\hline\\
	\hline\\
	\hline\\
\end{tabularx}
\setlength{\extrarowheight}{0mm}

\vspace{5ex}

{\large \textbf{В отношениях~"--- установите связь с другими}}

\begin{itemize}
	\itemWritingHand Как или когда взаимодействия с людьми приносят вам искреннюю радость (встреча с друзьями, настольная игра, написание открытки на день рождения)?
\end{itemize}

\setlength{\extrarowheight}{2mm}
\begin{tabularx}{\textwidth}{X}
	\\
	\arrayrulecolor{gray}\hline\\
	\hline\\
	\hline\\
	\hline\\
	\hline\\
	\hline\\
	\hline\\
\end{tabularx}
\setlength{\extrarowheight}{0mm}

\begin{itemize}
	\itemWritingHand Можете ли вы как-то укрепить эти связи?
\end{itemize}

\setlength{\extrarowheight}{2mm}
\begin{tabularx}{\textwidth}{X}
	\\
	\arrayrulecolor{gray}\hline\\
	\hline\\
	\hline\\
	\hline\\
	\hline\\
	\hline\\
\end{tabularx}
\setlength{\extrarowheight}{0mm}

\vspace{5ex}

{\large \textbf{Духовно~"--- держитесь своих ценностей}}

\begin{itemize}
	\itemWritingHand Как вы заботитесь о себе духовно (молитва, прогулка в лесу, помощь другим)?
\end{itemize}

\setlength{\extrarowheight}{2mm}
\begin{tabularx}{\textwidth}{X}
	\\
	\arrayrulecolor{gray}\hline\\
	\hline\\
	\hline\\
	\hline\\
	\hline\\
	\hline\\
	\hline\\
\end{tabularx}
\setlength{\extrarowheight}{0mm}

\begin{itemize}
	\itemWritingHand Есть ли еще что-то, о чем вы хотели бы помнить для развития своего внутреннего духовного <<я>>?
\end{itemize}

\setlength{\extrarowheight}{2mm}
\begin{tabularx}{\textwidth}{X}
	\\
	\arrayrulecolor{gray}\hline\\
	\hline\\
	\hline\\
	\hline\\
	\hline\\
	\hline\\
	\hline\\
\end{tabularx}
\setlength{\extrarowheight}{0mm}
