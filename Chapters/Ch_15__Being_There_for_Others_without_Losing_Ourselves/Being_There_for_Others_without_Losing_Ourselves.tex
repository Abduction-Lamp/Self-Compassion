% !TEX root = ../../Self-Compassion.tex

\chapter{Поддерживать других, не теряя себя} \label{Being_There_for_Others_without_Losing_Ourselves}

Одно из изменений, которые приносит в нашу жизнь \textbf{самосострадание~"--- это способность что-то давать другим, не теряя при этом себя}. Когда мы пытаемся быть с кем-то, кто испытывает боль, мы буквально чувствуем эту боль внутри себя.  Некоторые ученые нашли особенный тип нейронов, функция которых~"--- переносить то, что чувствуют или переживают другие, в телесные ощущения. Эти нейроны называются \emph{зеркальными нейронами}\cite{85}.  В мозгу также есть специальные области, которые оценивают социальные ситуации и резонируют с чужими эмоциями\cite{86}. Такой эмпатический резонанс часто происходит на невербальном, висцеральном уровне. 

Эмпатический резонанс выгоден с эволюционной точки зрения, так как он позволяет нам сотрудничать с другими, чтобы лучше воспитать детей или защититься от опасности. Мы, по сути, запрограммированы на социальные взаимодействия. В целом, эмпатия~"--- это обычно достоинство, а не недостаток, но она может и представлять проблему, так как, резонируя с чьей-то болью, мы чувствуем эту боль как свою. Иногда это может быть совершенно ошеломляющим. Когда такое происходит, мы можем попробовать разные маневры для избегания или уменьшения нашего страдания~"--- например, <<отключиться>> от другого человека или мысленно подняться на лифте в свою голову, чтобы устранить проблему (смотри главы \ref{Self-Compassion_in_Relationships} и \ref{Self-Compassion_for_Caregivers} на страницах \pageref{Self-Compassion_in_Relationships} и \pageref{Self-Compassion_for_Caregivers} соответственно). 

Вы когда-нибудь задумывались о том, почему, когда вы пытались рассказать кому-то о какой-то трудной ситуации в вашей жизни, слушатель сразу же вклинивался со своими советами, толком вас не выслушав? А может быть, \emph{вы сами} так поступали с кем-то другим? 

Эта реакция встречается достаточно часто, но почему же мы так делаем? \textbf{Одна из причин~"--- это дискомфорт, который мы испытываем в присутствии чьей-то боли, потому что мы ее тоже ощущаем.} Из-за такой эмпатической боли также могут всплыть на поверхность страхи или неприятные воспоминания из наших собственных жизней. 

\begin{quotation}
	\textit{
		Мария считала себя достаточно чувствительной и всегда хотела кому-то помочь. В один прекрасный день ее близкая подруга Аиша пригласила ее вместе сходить выпить кофе и при встрече рассказала ей о недавнем расставании с молодым человеком, с которым она очень долго встречалась. Но вместо того, чтобы выслушать Аишу, Мария обнаружила, что постоянно ее прерывала, чтобы напомнить ей, что все будет хорошо и она найдет себе кого-то другого. Наконец, очень раздраженная, Аиша выпалила: "Да ты можешь меня выслушать, в конце-то концов? Мне плохо, и мне нужно это выплеснуть. Может быть, когда-нибудь все и будет хорошо, но сегодня у меня все плохо и мне нужна твоя поддержка!>> С этими словами она (по виду очень расстроенная) встала и покинула столик.
	}
\end{quotation}

Хоть Мария и пыталась помочь, ее подход сделал только хуже. Те из нас, кто ориентирован на решение проблем, могут быть особенно склонны пытаться <<устранить>> боль других. Несмотря на хорошие намерения, если прерывать других, не дослушивая их и не валидируя их боль, это может привести к тому, что эмоциональная связь исчезнет. Например, в этой истории Аиша наверняка надеялась, сознательно или неосознанно, получить немного сострадания. \textbf{Сострадание~"--- это ресурс, который позволяет людям побыть с болью, не стремясь от нее тут же избавиться.} Он также позволяет нам нежно заботиться о человеке, который испытывает боль.

Как поддержать эмоциональную связь с кем-то, кому плохо? Для начала нужно оставаться на связи с собой — мы должны \textbf{замечать наш эмпатический дискомфорт и проявлять к себе сострадание}. Когда мы открываемся, принимаем нашу текущую реакцию на слова говорящего и сочувствуем себе (так как слушать иногда бывает нелегко), мы можем просто слушать, не чувствуя нужды прервать говорящего или отвлечься во время разговора. 

\begin{quotation}
	\textit{
		После ухода Аиши у Марии появилась возможность разобраться со своими мыслями. Она поняла, насколько ей было некомфортно видеть подругу такой расстроенной. Ей просто хотелось тут же прогнать ее боль с помощью полезных, как ей казалось, советов. Это, конечно, сработало против нее самой, поскольку Аиша не получила того сочувствующего уха, в котором она так нуждалась. Более того, чувство дискомфорта Марии только увеличивали воспоминания о похожем расставании, через которое она прошла за год до этого.
	}
	
	\textit{
		Мария очень любила свою подругу, поэтому вечером она зашла к ней домой, чтобы извиниться и поговорить. В этот раз, когда Аиша рассказывала свою историю, Мария практиковала подход <<слушать с состраданием>>. Когда ей становилось некомфортно, она делала глубокий успокаивающий вдох. Скоро Мария обнаружила, что ей стало гораздо легче слушать Аишу. Она была рада, что могла поддерживать подругу, поддерживая при этом себя.
	} 
\end{quotation}

\Meditation{Давать и получать сострадание} \label{M:Giving_and_Receiving_Compassion}

В этой медитации есть элементы двух предыдущий медитативных практик~"--- <<любящее дыхание>> (глава \ref{Mindfulness}, стр.\:\pageref{medit:Affectionate_Breathing}) и <<любящая доброта к себе>> (глава \ref{Loving-Kindness_for_Ourselves}, стр.\:\pageref{M:Loving-Kindness_for_Ourselves}). Она включает в себя и осознанность к дыханию, и намеренная культивация в себе доброты и сострадания. Это третья из основных медитативных практик в курсе ОСС.

Мы можем вдыхать за себя, а выдыхать за других. Выдох расширяет поле концентрации медитации так, чтобы туда помещались и другие люди, а вдох напоминает нам о самосострадании.

\vspace{2ex}

\begin{itemize}
	\item Сядьте удобно, закройте глаза и, если хотите, положите руку на сердце в качестве напоминания во время медитации сохранять не просто осознанность, а любящую осознанность ко всему происходящему и к себе.
\end{itemize}

\vspace{2ex}

{\large \textbf{Наслаждение дыханием}}
\begin{itemize}
	\item Сделайте несколько глубоких, успокаивающих вдохов и выдохов, замечая, как дыхание питает ваше тело, когда вы вдыхаете, и расслабляет его, когда вы выдыхаете.
	\item Теперь позвольте своему дыханию найти его естественный ритм. Продолжайте быть внимательным к ощущениям на вдохе и на выдохе. Если хотите, разрешите ритму вашего дыхания вас немного раскачивать и поглаживать.
\end{itemize}
 
\vspace{2ex}
 
{\large \textbf{Разогрев для осознанности}}
\begin{itemize}
	\item Теперь сфокусируйтесь на вдохе, позволяя себе насладиться чувством вдыхания воздуха, замечая, как дыхание питает ваше тело вдох за вдохом… а потом отпустите свое дыхание.
	\item Начните вдыхать доброту и сострадание к себе. Просто прочувствуйте качество этих ощущений или, если вам так удобнее, представьте себе, как на каждом вдохе в вас влетает какое-то слово или изображение.
	\item Начните вдыхать доброту и сострадание к себе. Просто прочувствуйте качество этих ощущений или, если вам так удобнее, представьте себе, как на каждом вдохе в вас влетает какое-то слово или изображение.
	\item А сейчас перебросьте внимание на свои выдохи. Прочувствуйте ощущения, возникающие в теле с каждым выдохом, и особенно легкость, которая с ним появляется.
	\item Вспомните о ком-то, \emph{кого вы любите}, или о ком-то, emph{кто сейчас испытывает трудности} и нуждается в сострадании. Четко визуализируйте этого человека в своем воображении.
	\item Начните направлять свои выдохи на этого человека, как бы делясь с ним легкостью, возникающей при выдохе.
	\item Если хотите, с каждым выдохом посылайте этому человеку добро и сострадание.
\end{itemize}

\vspace{2ex}

{\large \textbf{Входит для меня, выходит для тебя}}
\begin{itemize}
	\item Теперь сфокусируйтесь на ощущениях при вдохе и выдохе, наслаждаясь ими и растягивая их.
	\item Начните вдыхать для себя и выдыхать для другого человека. <<Входит для меня, выходит за тебя>>. <<Один за меня, один за тебя>>.
	\item По мере своего дыхания вбирайте в себя доброту и сострадание себе, а выпускайте доброту и сострадание к другому человеку.
	\item Если хотите, можете уделить больше внимания себе (<<Два за меня, один за тебя>>) или на этом человеке (<<Один за меня, три за тебя>>), а можете оставить поровну — как вам прямо сейчас хочется.
	\item Отпустите любые ненужные старания, делая эту медитацию такой же легкой для себя, как дыхание.
	\item Пусть ваше дыхание то входит, то выходит, подобно плавному движению океана~"--- безграничным и беспредельным потоком. Сами тоже станьте частью этого потока. Океана сострадания.
	\item Мягко и плавно откройте глаза.
\end{itemize}

\newpage
\Reflection{
	Что вы заметили во время медитации? Что вы почувствовали? Что было легче, вдыхать за себя или выдыхать за другого человека? Получилось ли у вас отрегулировать поток, когда это было необходимо, чтобы сосредоточиться на себе или на другом человеке (в зависимости от того, кто больше нуждался в поддержке)?  
	
	Сострадание к себе может принести глубокое облегчение, когда вы проявляете сострадание к другим. Тем не менее, некоторые люди вообще не любят сосредотачивать внимание на себе, особенно если другому человеку очень больно. Важно отрегулировать направление дыхания так, как вам в моменте нужно. Иногда требуется сфокусироваться на выдохах для другого человека, а иногда~"--- на вдохах для себя. Если в поле сострадания включены все, то равновесие в конце концов найдется естественным образом.
	
	Эта медитация составляет основу других практик, цель которых помочь нам поддерживать других, не теряя себя, и может быть отличной составляющей ваших ежедневных 30 минут практики.
}

\newpage
\InformalPractices{Слушать с состраданием} \label{IP:Compassionate_Listening}

Можете опробовать эту практику в следующий раз, когда будете слушать какую-нибудь огорчающую вас историю. Она поможет вам поддержать эмоциональную связь с говорящим, но при этом не перегрузить ваше сознание. 

\vspace{2ex}

{\large \textbf{Олицетворение слушания}}
\begin{itemize}
	\item Первый шаг~"--- это слушать \emph{олицетворенно}. Слушайте всем телом, позволяя себе чувствовать все возникающие в вашем теле ощущения, при этом внимательно слушая и смотря ушами и глазами. Если вам кажется, что сейчас было бы неплохо это сделать, станьте олицетворением любящего, общечеловеческого присутствия (т.\,е. сострадания). Разрешите своему телу испускать эту энергию в обоих направлениях~"--- вашем и говорящего.
	
	\item По мере того, как вы слушаете, у вас будет возникать множество естественных реакций. Например, вы можете почувствовать эмоциональное возбуждение и ошеломление, отвлечься на что-то в вашей жизни, что похоже на то, что вы слышите, или у вас может возникнуть импульс прервать говорящего и <<решить>> его проблему.
\end{itemize}

\vspace{2ex}

{\large \textbf{Давать и получать сострадание}}
\begin{itemize}
	\item Именно в тот момент, когда ваше внимание куда-то улетает, можно тут же начать (неформально) практиковать отдачу и получение сострадания. Просто сосредоточьтесь на какое-то время на своем дыхании, вдыхая сострадание к себе и выдыхая сострадание к говорящему. Вдохи за себя помогут вам восстановить связь с телом, а выдохи~"--- восстановить связь с говорящим, позволяя вам присутствовать при боли другого. Обратить внимание на выдохи также может помочь вам успокоить импульс помочь говорящему, <<решив>> его проблему.
	
	\item Продолжайте вдыхать и выдыхать сострадание, пока не будете снова в состоянии слушать, олицетворяя сострадание. Не нужно \emph{слишком} зацикливаться на дыхании, потому что такая многозадачность может отвлекать. Сострадательное дыхание~"--- это просто защитная сетка, которая ловит нас, когда мы отвлекаемся, и переносит нас обратно в состояние любящего, общечеловеческого присутствия. Другими словами, позволяет нам поддерживать других, не теряя себя.
\end{itemize}

\newpage
\Reflection{
	Попробовав эту практику несколько раз, когда слушаете других, подумайте, как оно влияет на вас как на слушателя и ваш опыт. Если вы понимаете, что такое дыхание вам мешает и отвлекает вас, тогда можете уделять ему меньше внимания. Но, если вы понимаете, что вас все еще одолевает эмпатический дискомфорт или импульс <<устранить>> проблему, хорошим решением было бы обращать больше внимания на ваше тело и дыхание. Поэкспериментируйте, пока не достигнете комфортного для вас баланса.
}