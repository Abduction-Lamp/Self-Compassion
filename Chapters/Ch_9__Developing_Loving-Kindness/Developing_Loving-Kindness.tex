% !TEX root = ../../Self-Compassion.tex

\chapter{Развитие в себе любящей доброты} \label{Developing_Loving-Kindness}

В дополнение к занятиям по углублению практик самосострадания важно развить в себе чувство любящей доброты к себе в целом. \emph{Любящая доброта} (англ. loving-kindness)~"--- это перевод с языка пали слова \emph{metta}\cite{65}. Этот термин также может означать <<\emph{дружелюбие}>>.

Чем отличаются сострадание и любящая доброта? Состраданию можно дать определение \emph{<<сочувствие боли или страданию другого и сильное желание облегчить это страдание>>}. \textbf{Самосострадание}~"--- это сострадание, направленное на самого себя (\emph{внутреннее сострадание})\cite{66}. Любящая доброта включает в себя чувство дружелюбия по отношению к себе и другим и не обязательно связано с каким-то страданием. Очень важно выработать в целом дружелюбное отношение к себе, которое остается даже если все в порядке.

Как сказал Далай-лама, <<любящая доброта~"--- это желание, чтобы все живые существа были \emph{счастливыми}>> \cite{67}. Сострадание, по такому же принципу~"--- <<желание, чтобы ни одно живое существо \emph{не испытывало страдания}>>. Один учитель медитации из Мьянмы сформулировал это так: <<Когда свет любящей доброты встречается со слезами страдания, появляется радуга сострадания>>.

\emph{Любящую доброту} (metta) можно развить с помощью практики, которая называется медитация любящей доброты. В этой практике медитирующий думает о каком-то одном человеке, визуализирует его или ее, и про себя повторяет последовательность фраз, которые нацелены на пробуждение чувства расположения и благожелательности по отношению к этому человеку. Некоторые часто используемые фразы: <<Я желаю тебе быть счастливым>>, <<Я желаю тебе быть здоровым>>, <<Я желаю тебе, чтобы на душе у тебя было мирно и спокойно>>, <<Я желаю тебе жить с легкостью>>. Эти предложения можно отнести к дружественным пожеланиям и выражению благих намерений. Обычно медитирующие начинают с того, что адресуют эти фразы себе, потом учителю или благодетелю, потом кому-то, к кому они нейтрально относятся, потом кому-то, с кем у них сложные отношения, и в завершение расширяют круг своей любящей доброты так, чтобы он включал в себя всех живых существ. Хорошие намерения, которые культивирует медитация любящей доброты, улучшают настроение и помогают добавить во внутренний диалог поддержку и понимание. Исследования показали, что эффект медитации любящей доброты зависит от <<дозировки>>: чем больше ее выполнять, тем ярче он выражен\cite{68}. Одно из самых важных полезных свойств медитации любящей доброты~"--- это снижение негативных эмоций (тревога, депрессия) и повышение позитивных эмоций (радость, счастье)\cite{69,70}.

Для некоторых людей медитация любящей доброты оказывается сложной, потому что процесс повторения фраз им кажется неловким или неуклюжим или сами фразы для них звучат механическими или неискренними. Если с вами такое случится, не переживайте. В иудаизме есть притча, которая иллюстрирует на практике работу медитации:

\begin{quotation}
	Ученик спрашивает раввина: <<Почему в Торе написано <<положите эти слова себе на сердце>>? Почему бы, например, не поместить эти священные слова себе в сердце?>>

	Раввин отвечает: <<Это из-за того, что сейчас наши сердца закрыты, и мы не можем в них положить священные слова. Поэтому мы кладем их себе на сердце и они лежат там, пока в один прекрасный день наше сердце не раскроется и они не смогут туда попасть>>.\cite{71}
\end{quotation}

\newpage
\Meditation{Любящая доброта к близкому человеку} \label{M:Loving-Kindness_for_a_Loved_One}
По традиции, медитация любящей доброты начинается с доброты к себе. <<Возлюби ближнего твоего, как самого себя>>. В более современных методиках порядок меняется: мы начинаем с кого-то, кого мы любим, а потом уже переходим на себя. Многие используют этот вариант медитации любящей доброты в качестве своей основной медитативной практики.

\begin{itemize}
	\item Устройтесь в удобной для вас позе сидя или лежа. Если хотите, можете положить руку на сердце в напоминание того, что нужно не просто осознание, а любящее осознание по отношению к настоящему моменту и к себе. 
\end{itemize}

\vspace{3ex}

{\large \textbf{Живое существо, которое вызывает у вас улыбку}}
\begin{itemize}
	\item Выберите человека или другое живое существо, в присутствии которого (или при мыслях о котором) у вас появляется на лице улыбка~"--- кто-то, с кем у вас простые, естественные и понятные отношения. Это может быть ребенок, ваша бабушка, ваша кошка или собака~"--- тот, кто всегда делает вас счастливым. Если вам приходит на ум много людей или животных, выберите кого-то одного.
	
	\item Вспомните, как вы ощущаете себя в присутствии этого человека или животного. Позвольте себе насладиться хорошей компанией. Воссоздайте в своем воображении это существо.
\end{itemize}

\vspace{3ex}

{\large \textbf{Я желаю тебе ...}}
\begin{itemize}
	\item Теперь признайте, что он или она хочет быть счастливым и свободным от страдания, как и все живые существа, включая вас. Повторяйте про себя, осознавая при этом важность этих слов:
	\begin{itemize}
		\item Я желаю тебе быть счастливым.
		\item Я желаю тебе, чтобы в душе у тебя был мир.
		\item Я желаю тебе быть здоровым.
		\item Я желаю тебе жить с легкостью.
	\end{itemize}
	\emph{(Повторите несколько раз, медленно и ласково.)}
	\item Если у вас есть набор других фраз, которые вы обычно используете, тогда вы можете произносить их. Если нет, продолжайте пользоваться этими.
	\item Когда вы замечаете, что ваше сознание куда-то улетело, вернитесь к словам и к мысленному образу этого любимого человека или животного. Если у вас возникают какие-то теплые чувства, насладитесь ими. Не спешите.
\end{itemize}

\vspace{3ex}

{\large \textbf{Я желаю тебе и себе (нам) ...}}
\begin{itemize}
	\item Теперь включите в свой круг любящей доброты себя. Добавьте себя и в ваш мысленный образ любимого человека или животного.
	\begin{itemize}
		\item Я желаю нам быть счастливыми.
		\item Я желаю нам, чтобы в душе у нас был мир.
		\item Я желаю нам быть здоровыми.
		\item Я желаю нам жить с легкостью.
	\end{itemize}
	\emph{(Повторите несколько раз.)}
	\item Теперь мысленно попрощайтесь с другим человеком или животным и поблагодарите его, если вам хочется, а потом полностью сосредоточьте внимание на себе.
\end{itemize}

\vspace{3ex}

{\large \textbf{Я желаю себе ...}}
\begin{itemize}
	\item Положите свою руку на сердце и почувствуйте ее тепло и легкое давление. Визуализируйте в воображении все ваше тело, замечая любое напряжение или неудобство, которое в вас остались, и адресуйте себе все те же фразы.
	\begin{itemize}
		\item Я желаю себе быть счастливым.
		\item Я желаю себе, чтобы в душе у меня был мир.
		\item Я желаю себе быть здоровым.
		\item Я желаю себе жить с легкостью
	\end{itemize}
	\emph{(Повторите несколько раз, тепло.)}
	\item В завершение сделайте несколько глубоких вдохов и выдохов и просто тихо расслабьтесь, принимая свой опыт точно таким, каким он был. 
\end{itemize}

\newpage
\Reflection{Что вы заметили в ходе этой медитации? Что поняли для себя? К кому было легче почувствовать любящую доброту: к себе или близкому человеку? Какие у вас были ощущения, когда вы направляли чувство любящей доброты на вас обоих одновременно? Были ли в этой медитации сложные аспекты? 

Часто люди обнаруживают, что им намного легче почувствовать любящую доброту к другому, чем к себе. В этой медитации мы как раз и начинаем с самого легкого, а потом приобщаем туда себя, чтобы легче было поддерживать поток любящей доброты к более <<сложному>> человеку~"--- себе.

Тем не менее, у многих людей возникают трудности во время медитации любящей доброты. Либо фразы не до конца верно отображают их чувства, либо формулировка <<я желаю>> звучит наигранно или неестественно. В следующей главе мы поможем вам найти ваши собственные фразы для медитации любящей доброты~"--- фразы, которые будут для вас звучать значимее и искреннее.}

\newpage
\InformalPractices{Прогулка с любящей добротой} \label{IP:Walking_in_Loving-Kindness}
Мы можем практиковать отношение с любящей добротой на протяжении своего обычного дня, направляя фразы на себя или всех, с кем встречаетесь. \emph{Примечание}: В этой практике используется процесс ходьбы, чтобы закрепить наше осознание, но люди с ограниченной мобильностью (например, на инвалидной коляске) могут выбрать любую другую точку телесного контакта с чем-то.

\begin{itemize}
	\item Когда вы гуляете по улице или ходите по людному месту, как, например, торговый центр, вы можете попробовать выполнить эту практику.
	\item Сначала обратите внимание на свои ноги, отмечая ощущения при ходьбе в ногах и стопах (замедлять шаг при этом нет необходимости).
	\item Пока вы идете, начните повторять про себя фразу <<я желаю себе быть счастливым и свободным от страдания>>.
	\item Потом, когда заметите или пройдете мимо другого человека, про себя адресуя ему или ей добрую фразу вроде <<желаю вам быть счастливым и свободным от страдания>>. Попробуйте при этом почувствовать теплоту или расположение к этому человеку.
	\item Если ситуация подходящая, можете кивать или слегка улыбаться проходящим людям, повторяя про себя: "Желаю вам быть счастливым и свободным от страданий>>.
	\item Когда вы отвлекаетесь или вам некомфортно, перенаправьте свое внимание на ощущения в ногах и стопах и скажите себе: <<Я желаю себе быть счастливым и свободным от страдания>>. Когда будете готовы, снова сконцентрируйтесь на других.
	\item Наконец, попробуйте расширить свой круг дружеских пожеланий так, чтобы он включал всех людей (и других живых существ) в вашем поле зрения~"--- и, конечно, не забудьте себя! Про себя повторите несколько раз: <<Я желаю всем живым существам быть счастливыми и свободными от страдания>>.
\end{itemize}

\newpage
\Reflection{Что вы заметили во время выполнения практики? Изменилось ли ваше восприятие других людей? А их реакция на вас?

Эта практика может быть очень хорошим способом почувствовать связь со всеми живыми существами. Ее можно повторять в магазине или ресторане, по дороге на работу в машине или на автобусе~"--- в любом месте, где есть другие люди.

Если предложенная здесь фраза не вызывает у вас искреннего чувства доброты и сострадания, отложите выполнение этой практики до того, пока вы не найдете свои собственные, искренние фразы (см. следующую главу).}