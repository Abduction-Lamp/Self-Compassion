% !TEX root = ../../Self-Compassion.tex

\chapter{Самосострадание и гнев в отношениях} \label{Self-Compassion_and_Anger_in_Relationships}

Один из типов боли, которые могут возникнуть в отношениях с другими людьми~"--- боль \emph{разобщения}, которая появляется при потере или разрыве отношений. Гнев~"--- это распространенная реакция на разобщение. Мы можем разозлиться, когда чувствуем себя отверженными и игнорируемыми, но и также в случаях, когда разобщение неизбежно~"--- например, когда кто-то умирает. Эта реакция, конечно, иррациональна, но она тем не менее возникает. Гнев иногда может оставаться на долгие годы после того, как отношения были разорваны.

Хотя у гнева и не лучшая репутация, он совсем не обязательно <<плохой>>. Как и у всех эмоций, у гнева есть и положительные функции\cite{105}. Например, он может информировать нас о том, что кто-то переступил наши личные границы или чем-то нас задел, и поэтому он бывает мощным сигналом, что пора что-то менять. Гнев может дать нам энергию и решимость для того, чтобы защититься от какой-то угрозы, предпринять действия, чтобы положить конец нежелательному поведению, или завершить токсичные отношения.

Сам по себе гнев не является проблемой, но у нас часто развиваются проблемные, нездоровые с ним отношения. Например, бывает, что мы запрещаем себе чувствовать наш гнев и подавляем его. (Особенно часто это встречается у женщин, потому что всю жизнь общество требует от них быть <<хорошими>>, т.\,е. не злыми.) Когда мы пытаемся запихнуть гнев поглубже, чтобы его не чувствовать, это может привести к тревоге, эмоциональной ограниченности и чувству онемения\cite{106}. Иногда мы обращаем свой гнев против себя же в форме жесткой самокритики, которая на самом деле является надежной дорогой к депрессии\cite{107}. А если мы застреваем в гневном обдумывании одного и того же~"--- кто кому что сделал и что за это заслуживает~"--- наш мозг приходит в возбужденное состояние и мы сердимся на других без очевидных причин\cite{108}. 

\begin{quotation}
	\textit{
		Натан работал электриком в Чикаго. На тот момент он уже пять лет был в разводе с женой, но мысли о ней все еще вызывали у него гнев и ярость. Оказалось, что Лила, его жена, в течение почти года изменяла ему с их близким другом, с которым они часто общались. Когда Натан об этом узнал, он чуть не взорвался от гнева. Каким-то чудом он смог не обозвать ее всеми ругательствами, которые он знал, но каждый раз при мысли о произошедшем его начинало тошнить. Он почти сразу подал на развод~"--- к счастью, детей у них не было, поэтому все прошло достаточно легко и быстро. Хотя он уже несколько лет с ней никак не контактировал, гнев на бывшую жену у Натана никуда не делся. А травма, которую ему нанесла та ситуация, мешала ему вступить в новые отношения, потому что ему было сложно кому-то довериться.
	}
\end{quotation}

Если мы будем постоянно ожесточаться, пытаясь защититься от людей, на которых мы злимся, с течением времени у нас может появиться чувство \emph{горечи} и \emph{обиды}. Гнев, горечь и обида~"--- это <<жесткие>> эмоции\cite{109}. Жесткие эмоции сопротивляются изменениям и часто остаются с нами надолго после ситуации, в которой они были полезны. (Сколько из нас сердятся на кого-то, кого мы, скорее всего, больше никогда не увидим?) Боле того, хронический гнев вызывает хронический стресс, который вредит всем системам органов~"--- сердечно--сосудистой, эндокринной, нервной и даже репродуктивной\cite{110}. Как говорится, <<гнев разъедает сосуд, в котором содержится>> или <<гнев~"--- это яд, который мы пьем, чтобы убить кого-то другого>>. Когда гнев уже нам не приносит пользы, нужно отнестись к себе с состраданием и изменить наши отношения с ним, прибегая к ресурсам осознанности и самосострадания.

Как? Первый шаг~"--- это определение \emph{<<мягких>> эмоций}, стоящих за жесткой эмоцией гнева. Часто гнев защищает более нежные и чувствительные эмоции, такие как боль, страх, одиночество, уязвимость или чувство, что вас никто не любит. Когда мы проникаем за наружный слой гнева, мы часто удивляемся полноте и сложности наших чувств. С жесткими чувствами тяжело работать напрямую, так как они обычно сопротивляются и, кроме того, направлены на внешний мир. Когда мы находим свои мягкие чувства, мы направляемся внутрь себя и можем начинать процесс трансформации.

Но для того, чтобы по-настоящему вылечиться, мы \emph{должны снять} еще один слой чувств и найти неудовлетворенные потребности, которые прячутся за нашими мягкими чувствами. Речь идет об универсальных человеческих потребностях~"--- о том, что лежит в основе любой человеческой жизни\cite{111}. У Центра Ненасильственного Общения есть на сайте полный список этих потребностей: \url{http://cnvc.org/training/needs-inventory}. Приведем несколько примеров: это может быть потребность в безопасности, валидации, независимости, уважении, связи с другими людьми, в том, чтобы быть услышанным. \textbf{А самая наша глубокая потребность~"--- потребность быть любимыми.}

Если мы найдем в себе смелость заглянуть поглубже внутрь, ощутить наши настоящие чувства и потребности и открыться им, мы сможем наконец начать понимать, что с нами происходит. Когда мы вступаем в контакт с болью и отвечаем ей самосостраданием, начинается трансформация на более глубоком уровне. Мы можем применять самосострадание, чтобы удовлетворять наши потребности напрямую. 

Как было объяснено в главе \ref{Self-Compassion_in_Relationships} на стр.\:\pageref{Self-Compassion_in_Relationships}, отвечая самосостраданием на неудовлетворенные потребности, мы можем дать себе то, что мы хотели~"--- возможно, в течение многих лет~"--- получить от других. Мы можем сами для себя стать источником поддержки, уважения, любви, валидации или безопасности. Конечно, мы не роботы, и нам нужны отношения и связь с другими людьми. Но когда другие люди не могут по какой-то причине удовлетворить наши потребности, да еще и приносят нам боль в процессе, мы можем оправиться, заключив боль и <<мягкие>> чувства в объятия сострадания и заполнить душевную пустоту любящим, общечеловеческим присутствием.

\begin{quotation}
	\textit{
		Натан упорно работал над трансформацией своего гнева, так как он понял, что это ему мешает. Он попробовал <<\textbf{очищение}>>, чтобы выплеснуть его наружу~"--- бил подушки, кричал во весь голос~"--- но это не помогло. В конце концов Натан записался на курс ОСС, потому что один его друг посоветовал ему это с большим энтузиазмом, сказав, что программа поможет ему снизить уровень стресса.
	}
	
	\textit{
		Когда Натан пришёл на занятие в курсе ОСС, посвящённое трансформации гнева для удовлетворения неудовлетворённых потребностей, он слегка нервничал, но все равно выполнил все инструкции. Было легко найти связь со своим гневом и даже с прячущейся за ним болью и прочувствовать его в теле. Самым сложным оказалось определение его неудовлетворённой потребности. Конечно, он чувствовал себя преданным и нелюбимым, но это было не то, что его тормозило. Не отчаиваясь, Натан продолжал упражнение, и наконец неудовлетворенная потребность нашлась, а все его тело расслабилось. Уважение!
	}
	
	\textit{
		Натан происходил из семьи рабочих, а его родители были счастливы в браке вот уже 30 лет. Он пытался все сделать правильно в собственном браке, насколько только мог, и очень серьезно относился к своим обещаниям. Честность и уважение были для него основными ценностями. Зная, что от Лилы он уже никогда не получит уважение, в котором он нуждался~"--- для этого уже было слишком поздно~"--- он отважился и постарался сам подарить себе это уважение. <<Я тебя уважаю>>, "--*сказал он себе. Сначала это прозвучало глуповато и пусто. Поэтому он сделал паузу и убедился, что эти слова~"--- чистая правда, перед тем, чтобы их произнести. Он вспомнил обо всем, что ему пришлось пожертвовать, чтобы получить сертификат мастера--электрика и открыть собственный бизнес, о долгих часах труда ради того, чтобы выплатить ипотеку и отложить деньги на чёрный день. <<Я уважаю тебя>>, "--*повторял он вновь и вновь, но это все равно оставалось для него пустыми словами. Потом он подумал о том, каким честным и трудолюбивым он пытался быть в своём браке, даже несмотря на то, что для Лилы этого оказалось недостаточно. Очень медленно Натан наконец начал вбирать это все в себя. В итоге он положил руку на сердце и сказал: <<Я уважаю тебя>>, и на этот раз эти слова были наполнены для него смыслом. У него даже навернулись слёзы на глаза, потому что теперь он все это действительно чувствовал. После этого его гнев на жену начал потихоньку таять. Он увидел ее неудовлетворенные потребности в близости и привязанности, отличные от своих. Это не оправдывало ее поступок, но Натан понял, что ее поведение не было индикатором его ценности как человека. Он сделал вывод, что не может ни на кого положиться~"--- даже на верного и надежного друга~"--- чтобы получить уважение, которое ему нужно. Оно должно приходить изнутри.
	} 
\end{quotation}

\newpage
\Exercises{Удовлетворение неудовлетворённых потребностей} \label{Ex:Meeting_Unmet_Needs}

Цель этой неформальной практики~"--- подойти к обидам из прошлого с осознанностью и ответить на лежащие в их основе неудовлетворенные потребности самосостраданием. Она направлена на \textbf{изменение отношения} к старым ранам, которые вас раньше злили, а \textbf{не на полное излечение} от этих ран. Отпустите нужду почувствовать себя лучше и просто наблюдайте за происходящим.

Выберите для этой практики отношения низкой или средней сложности, а не травматические отношения, так как сильные эмоции могут затруднить выполнение ее до конца. Если вы чувствуете себя эмоционально уязвимым, пропустите эту практику или начните и отвлекитесь от неё, если вам станет некомфортно.

\begin{itemize}
	\itemWritingHand В данном ниже пространстве запишите прошлые отношения, которые до сих пор у вас вызывают гнев и горечь, а потом вспомните какое-то конкретно событие в этих отношениях, которое вас вызвало у вас беспокойство (опять же, не слишком сильное)~"--- где-то 3 на шкале горечи от 1 до 10. Помните, что нельзя выбирать какой-то опыт, который нанёс вам психологическую травму или оставил психологические же шрамы. 
	
	\item Для этого упражнения важно выбрать отношения, которые уже закончились и по поводу которых ваш гнев уже бесполезен~"--- тогда вы будете готовы его отпустить. Не торопитесь, лучше найдите именно те отношения и то событие, с которым вы хотите поработать.
\end{itemize}

\setlength{\extrarowheight}{2mm}
\begin{tabularx}{0.96\textwidth}{X}
	\\
	\arrayrulecolor{gray}\hline\\
	\hline\\
	\hline\\
	\hline\\
	\hline\\
	\hline\\	
	\hline\\
	\hline\\
	\hline\\
	\hline\\
	\hline\\
\end{tabularx}
\setlength{\extrarowheight}{0mm}
\begin{itemize}
	\item Во время выполнения этого упражнения попробуйте освободить в голове много места для того, что вы испытаете, при этом осознанно относясь к произошедшему, а не теряясь в своей обиде.
	
	\item Если в какой-то момент вам потребуется <<закрыть>> или закончить упражнение, пожалуйста, так и сделайте. Не забудьте позаботиться о себе. Закройте на минуту глаза и вспомните то событие, с которым будете работать.
	
	\item Вспомните все детали так точно, как можете, вступая в контакт со своим гневом и ощущая его проявления телесно.
	
	\item Осознайте, что абсолютно естественно чувствовать себя так, как вы сейчас чувствуете. Можете сказать себе: <<Сердиться~"--- нормально, тебе же сделали больно! Это естественная человеческая реакция на боль>>. Или, например: <<Ты не один. Многие люди бы себя так почувствовали>>.
	
	\item Разрешите себе полностью валидировать свой гнев, при этом пытаясь не застрять в размышлениях о том, что было хорошо, а что плохо.
	
	\item Нет необходимости двигаться дальше, если валидация гнева для вас сейчас то, в чем вы больше всего нуждаетесь. Если это ваш случай, просто пропустите оставшиеся инструкции. Помните, что ваш гнев естественен, и отнеситесь к себе с добротой и сочувствием к боли, которую вы терпели в себе, возможно, на протяжении многих лет.
\end{itemize}
 
\vspace{3ex}
 
\noindent{\large \textbf{Мягкие чувства}}

\vspace{1ex}

\begin{itemize}
	\item Если вам хочется двигаться дальше, начните снимать верхний слой гнева и обиды~"--- жестких чувств~"--- и посмотрите, что скрывается за ними.
	
	\item Есть ли под жесткими чувствами какие-то мягкие? (Боль? Страх? Одиночество? Грусть? Горе?).
	
	\item Когда идентифицируете мягкое чувство, попробуйте для себя произнести его название нежным, понимающим голосом, как будто поддерживаете близкого друга: <<о, это грусть>> или <<это страх>>.
	
	\item Если вам нужно, вы можете остановиться и здесь. Что вам кажется правильным?
\end{itemize}

\newpage
\noindent{\large \textbf{Неудовлетворенные потребности }}

\vspace{1ex}

\begin{itemize}
	\item Если вы готовы двигаться дальше, попробуйте отпустить человека и историю, которые связаны с этой болью, хотя бы на короткое время. У вас могут появляться мысли о том, кто был прав, а кто нет. Попробуйте отложить эти мысли, спрашивая себя: <<Какая базовая человеческая потребность, которая у меня есть или была в то время, не была тогда удовлетворена?>> Потребность в том, чтобы быть... (Услышанным? Увиденным? В безопасности? Связанным с другими? Уважаемым? Особенным? Любимым?)
	
	\item Какова была потребность, которая у вас осталась неудовлетворённой?
	
	\item Снова попробуйте назвать ее нежным, понимающим голосом. 
\end{itemize}

\vspace{3ex}

\noindent{\large \textbf{Отвечая состраданием}}

\vspace{1ex}

\begin{itemize}
	\item Если хотите продолжать, положите руку куда-нибудь на своё тело так, чтобы это вас успокаивало и подарите себе немного тепла и доброты~"--- не для того, чтобы прогнать неприятные чувства, а за то, что эти чувства возникают.
	
	\item Те же самые руки, которые тянулись к другим за заботой и поддержкой~"--- могут теперь стать руками, которые дадут вам ту поддержку и заботу, в которой вы нуждаетесь. Хотя вам и хотелось бы, чтобы ваши потребности удовлетворил другой человек, он не смог этого сделать по ряду причин. Но у нас есть другой ресурс~"--- мы сами, и мы можем попробовать удовлетворить эти потребности напрямую.
	\begin{itemize}
		\item Если вам нужно почувствовать себя увиденным, скажите: <<Я тебя вижу>>, <<Ты для меня важен>>.
		\item Если вам нужно почувствовать связь с другими людьми, скажите: <<Я с тобой>>, <<Ты на своём месте>>.
		\item Если вам нужно почувствовать себя любимые, скажите: <<Я люблю тебя>>, <<Ты мне очень близок>>.
	\end{itemize}
	Другими словами, попробуйте сейчас дать себе то, что вы хотели получить от кого-то ещё~"--- возможно, на протяжении долгого времени.
	
	\item Попробуйте получить и принять эти слова. Вы можете почувствовать разочарование по поводу того, что этот другой человек не смог удовлетворить ваши потребности, но посмотрите, можете ли вы прямо сейчас удовлетворить хотя бы некоторые из них?
	
	\item Если вам сложно это сделать, тоже проявите к этому сострадание~"--- сострадание к боли, которую мы, как и все люди, испытываем, когда наши глубочайшие потребности оказываются неудовлетворёнными.
	
	\item Теперь отпустите упражнение и просто отдохните, разрешив настоящему момент быть таким, какой он есть, а себе~"--- быть таким, какой вы есть.
\end{itemize}

\Reflection{
	Что вы почувствовали, валидируя свой гнев? Получилось ли у вас найти под ним мягкие чувства? А неудовлетворенные потребности? Смогли ли вы почувствовать самосострадание за эту неудовлетворённую потребность и, может быть, даже удовлетворить ее напрямую?
	
	Мы надеемся, что вы проделали ровно такую часть этой практики, которую вам было комфортно делать. После столкновения с неприятными чувствами или гневом, оставшимися после старой раны, некоторые просто не готовы снимать слои этих чувств, чтобы найти под ними мягкие чувства и потребности. В этом случае самая полезная вещь~"--- просто валидировать свой гнев и остановиться на этом. У других людей может получиться идентифицировать мягкие чувства и неудовлетворенные потребности, стоящие за их гневом, но при попытке удовлетворить их напрямую какой-то голос внутри говорит:
	
	<<Но я не хочу сам удовлетворять свои потребности, я хочу, чтобы [другой человек] это сделал!>> Это обычно значит, что есть какая-то боль и обида, которые вы не валидировали. Или это просто может быть вполне естественным желанием получить извинения. Но до того дня, когда этот момент наконец настанет, всё-таки подумайте о том, чтобы дать себе то, что вам так нужно и в чем вы, возможно, нуждаетесь уже долго.
}

\newpage
\Exercises{Яростное сострадание} \label{Ex:Fierce_Compassion}

Когда гнев применяется для уменьшения своих или чужих страданий или для того, чтобы отстоять хорошее дело, его можно назвать <<яростным состраданием>>. Некоторые из величайших исторических личностей~"--- например, Мартин Лютер Кинг~"--- использовали гнев на социальную несправедливость в качестве катализатора общественных реформ, при этом сохраняя пламя уважения и сострадания ко всем. Другими словами, сострадание не делает нас слабыми или пассивными и не отнимает у нас способность отличить добро от зла. Сострадание помогает нам отчётливо увидеть картину происходящего и понять сложные причины, по которым люди ведут себя так, как они себя ведут, а не иначе. Это позволяет нам предпринять действия, чтобы остановить вредоносное или опасное поведение, не разделяя при этом людей на <<хороших>> и <<плохих>>. Таким образом, яростное сострадание (на контрасте с реакционным гневом) помогает нам противостоять несправедливости, не ухудшая ситуацию виной или ненавистью.
\begin{itemize}
	\item Сделайте два--три глубоких вдоха и выдоха и закройте глаза, чтобы почувствовать своё тело и настоящий момент. Можете положить руку на сердце в качестве жеста поддержки и доброты к себе.
	
	\itemWritingHand Подумайте о какой-то конкретной общественной или политической ситуации, с ходом которой вы категорически не согласны. Вместо того чтобы просто злиться на неё, представьте себе, как бы вы думали и чувствовали себя с настроем на яростное сострадание. Можете ли вы описать ситуацию таким образом, чтобы никого не демонизировать? Попробуйте осознать, что люди, которые создали эту ситуацию~"--- это тоже просто люди, которые стараются, как могут, но при этом также признайте нанесённый ситуацией вред и необходимость перемен.
\end{itemize}

\setlength{\extrarowheight}{2mm}
\begin{tabularx}{0.96\textwidth}{X}
	\\
	\arrayrulecolor{gray}\hline\\
	\hline\\
	\hline\\
	\hline\\
	\hline\\
	\hline\\	
	\hline\\
	\hline\\
	\hline\\
\end{tabularx}
\setlength{\extrarowheight}{0mm}
\begin{itemize}
	\itemWritingHand Есть ли что-то, что вы хотели бы сделать (с перспективы яростного сострадания), чтобы изменить ситуацию?
\end{itemize}

\setlength{\extrarowheight}{2mm}
\begin{tabularx}{0.96\textwidth}{X}
	\\
	\arrayrulecolor{gray}\hline\\
	\hline\\
	\hline\\
	\hline\\
	\hline\\
	\hline\\	
	\hline\\
	\hline\\
	\hline\\
	\hline\\
\end{tabularx}
\setlength{\extrarowheight}{0mm}
\begin{itemize}
	\itemWritingHand Теперь подумайте о ситуации в вашей личной жизни, с ходом которой вы тоже не согласны~"--- о ситуации, созданной кем-то, кого вы знаете: партнером, ребёнком, другом, коллегой. И снова вместо того, чтобы злиться на неё, представьте себе, как бы вы думали и чувствовали себя с настроем на яростное сострадание. Можете ли вы описать ситуацию таким образом, чтобы никого не демонизировать? Попробуйте осознать, что люди, которые создали эту ситуацию~"--- это тоже просто люди, которые стараются, как могут, но при этом также признайте нанесённый ситуацией вред и необходимость перемен.
\end{itemize}

\setlength{\extrarowheight}{2mm}
\begin{tabularx}{0.96\textwidth}{X}
	\\
	\arrayrulecolor{gray}\hline\\
	\hline\\
	\hline\\
	\hline\\
	\hline\\
	\hline\\	
	\hline\\
	\hline\\
	\hline\\
	\hline\\
	\hline\\
\end{tabularx}
\setlength{\extrarowheight}{0mm}
\begin{itemize}
	\itemWritingHand Есть ли что-то, что вы можете сделать (с перспективы яростного сострадания), чтобы изменить ситуацию?
\end{itemize}

\setlength{\extrarowheight}{2mm}
\begin{tabularx}{0.96\textwidth}{X}
	\\
	\arrayrulecolor{gray}\hline\\
	\hline\\
	\hline\\
	\hline\\
	\hline\\
	\hline\\	
	\hline\\
	\hline\\
	\hline\\
	\hline\\
	\hline\\
	\hline\\
	\hline\\
	\hline\\
\end{tabularx}
\setlength{\extrarowheight}{0mm}

\Reflection{
	Многие люди говорят, что концепция яростного сострадания даёт им чувство свободы. Эта концепция предлагает способ как-то повлиять на ситуацию и спровоцировать изменения без попадания в ловушки гнева или вины.
	
	Это может быть полезным идеалом, к которому нужно стремиться, также верно, что гнев~"--- это естественная реакция и что мы часто будем ловить себя на том, что возвращаемся к старым реакционным закономерностям. Когда это происходит, не надо злиться на себя за то, что вы злитесь на других! Вместо этого мы можем почувствовать сострадание к себе, найти наш центр любящего, общечеловеческого присутствия и попробовать начать сначала.
}