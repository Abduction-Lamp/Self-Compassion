% !TEX root = ../../Self-Compassion.tex

\chapter{Самосострадание и наши тела} \label{Self-Compassion_and_Our_Bodies}

У нас есть комплексы по поводу более или менее всех аспектов наших жизней, но наши тела~"--- это особенно тяжелый случай. Наше самоощущение тесно связано с телом, и наша внешность очень сильно влияет на самооценку. Это особенно важно для женщин, так как установленные обществом стандарты женской красоты очень высоки\cite{77}. Все больше и больше женщин прибегают к пластической хирургии (<<что-нибудь себе делают>>), чтобы быть похожими на идеальных моделей с обложек журналов. Но даже несмотря на все усилия, большинство женщин так и не достигнут этого <<идеала>>~"--- даже фото моделей обрабатываются перед публикацией!

В среднем, мужчины более довольны своим телом, чем женщины, но им тоже бывает сложно его принять: <<Я достаточно подкачанный, достаточно стройный, достаточно мужественный?>> Мужская озабоченность телом скорее направлена на то, как его тело выполняет свою работу: например, насколько они сильны или насколько хорошо у них получается какой-то спорт. В независимости от того, с какими именно трудностями мы встречаемся, и женщины, и мужчины чаще видят свои тела как врагов, а не как друзей. Вместо того, чтобы говорить <<фу>>, когда тело выглядит или ведет себя не так, как, по нашему мнению, должно, лучше проявить самосострадание. 

Другими словами, можем ли мы признать, как наше тело старается, несмотря на несбалансированное питание, недосып, недостаток физической нагрузки и старение? Это важно для обоих полов. 

\begin{quotation}
	\textit{
		В 52 года Джиллиан уже была <<не первой молодости>>. Она была всегда недовольна своим весом и не считала свое тело привлекательным, но с годами эти проблемы только усилились. Она каждый раз кривила нос от отвращения, когда видела свое отражение в зеркале, и постоянно себя чувствовала неполноценной. Под глазами у нее были мешки, а на ляжках~"--- <<мешки жира>>, как она выражалась. Джиллиан уже начинала себя чувствовать одним огромным мешком. Она пыталась найти утешение в поедании арахисовой пасты и шоколадного мороженого, но чувство облегчения длилось, мягко скажем, недолго. Она пыталась не зацикливаться на своем теле, но не могла. Ей никогда не нравилось ее внешность, потому что в глубине души она не чувствовала себя достаточно счастливой.
	}
\end{quotation}

К счастью, самосострадание~"--- это могущественное противоядие от недовольства своим телом. Исследования показали, что даже \textbf{недолгий период практики самосострадания может уменьшить чувство стыда за свое тело и зависимость нашего самоуважения от внешности и помочь нам больше ценить свое тело таким, какое оно есть}\cite{78,79}. 

Когда мы относимся к себе с добротой, теплотой и принятием~"--- даже если то, что мы видим в зеркале не соответствует идеалу~"--- мы понимаем, что стоим гораздо больше, чем наше отражение. Мы начинаем понимать, что на самом деле то, что в нас ценно, это то, что мы человеческие существа, которые пытаются быть счастливыми и, несмотря на то, что это получается не всегда, продолжают пытаться. Вместо того, чтобы целиком себя отождествлять со своим телом и позволять ему определять вас, мы можем увидеть общею картину и понять, что самое важное~"--- это наши внутренние ресурсы и внутренняя красота. Мы можем всегда прерваться на то, чтобы поблагодарить наше тело за восхитительный подарок жизни, который оно нам дарит, и в глубине души почувствовать себя живыми. Благодаря самосостраданию мы можем ценить свои тела за то, что они делают для нас, а не за то, как они выглядят, и понемногу отходить от этого сумасшествия.  

\begin{quotation}
	\textit{
		Когда Джиллиан научилась практиковать самосострадание, ее отношения с самой собой и со своим телом начали меняться. Она осознала, что ей хотелось получать от других людей комплименты ее внешности и слышать, что она красивая, чтобы почувствовать себя любимой и принятой, но на самом деле она могла сама любить и принимать себя. Да, с возрастом Джиллиан поправилась, но с возрастом же и пришли мудрость и видение своих сильных сторон~"--- того, что она могла дать миру. Она не была <<идеальной>> ни внутри, ни снаружи, но она научилась ценить свои недостатки за то, что делали ее настоящей и уникальной~"--- они были просто доказательством ее человечности. Джиллиан не была ни роботом, ни <<Степфордской женой>>~"--- она состояла из плоти и крови, была наполнена дарящей жизнь энергией.
	}
	
	\textit{
		Когда Джиллиан изменила свои отношения с самой собой, ее отношения с едой тоже изменились. Ей больше не нужно было заполнять душевную пустоту большими количествами еды. Она научилась наслаждаться едой, но останавливаться, когда ее организм давал сигнал, что ему хватило. Но самым большим изменением было то, что Джиллиан наконец-то прекратила чувствовать себя неполноценной, поняла, что достаточно просто быть человеком, и в конце концов начала любить и принимать себя такой, какая она есть.
	}
\end{quotation}

\Exercises{Принять свое тело с самосостраданием} \label{Ex:Embracing_Our_Bodies_with_Self-Compassion}

В культуре конкуренции с нездоровой фиксацией на теле сложно относиться с состраданием к внешним несовершенствам. Мы окружены нереалистичными фотографиями в СМИ и в социальных сетях, из-за чего почти невозможно не быть недовольными тем, как мы выглядим. Единственный вариант, который нам остается~"--- принять тот факт, что мы не идеальны, делать все возможное и любить себя несмотря ни на что. Это упражнение нацелено на то, чтобы помочь вам принять себя и свои недостатки такими, какие вы и они есть, используя три элемента самосострадания.

\begin{itemize}
	\itemWritingHand Для начала запишите в пустом пространстве ниже добрую, но честную оценку своего тела. Будьте внимательны и осознанны к тому, что есть~"--- и к хорошему, и к плохому. Сначала перечислите все, что вам нравится в вашем теле. Может быть, у вас отличное здоровье или красивая улыбка. Не забудьте о том, что обычно не влияет на восприятие себя: например, то, что у вас сильные руки или что ваша пищеварительная система хорошо работает (не стоит это принимать как данность!). Запишите все-все части своего тела, которыми вы довольны. 
\end{itemize}

\setlength{\extrarowheight}{2mm}
\begin{tabularx}{\textwidth}{X}
	\\
	\arrayrulecolor{gray}\hline\\
	\hline\\
	\hline\\
	\hline\\
	\hline\\
	\hline\\	
	\hline\\
	\hline\\
	\hline\\
	\hline\\
	\hline\\
	\hline\\
	\hline\\
	\hline\\
	\hline\\
\end{tabularx}
\setlength{\extrarowheight}{0mm}
\begin{itemize}
	\itemWritingHand Теперь перечислите все, что вам в вашем теле НЕ нравится. Может быть, ваша кожа в пятнах, у вас есть складки на животе или вы уже не можете бегать так быстро и далеко, как в молодости. В процессе может появиться чувство дискомфорта~"--- постарайтесь это тоже признать и принять. <<Мне тяжело видеть, как мой подбородок теряет четкий контур. Это действительно нелегко>>. Попробуйте просто побыть с этими чувствами, искренне признавая и принимая ваши несовершенства, не позволяю мозгу их раздуть и связать с чувством неполноценности. Постарайтесь провести уравновешенную оценку своих <<недостатков>>. Разве то, что ваши волосы начали седеть~"--- это проблема мирового масштаба? Эти лишние 5 килограммов действительно мешают вам чувствовать себя хорошо и быть здоровой? Не пытайтесь преуменьшить свои несовершенства, но и не делайте из мухи слова. 
\end{itemize}

\setlength{\extrarowheight}{2mm}
\begin{tabularx}{\textwidth}{X}
	\\
	\arrayrulecolor{gray}\hline\\
	\hline\\
	\hline\\
	\hline\\
	\hline\\
	\hline\\	
	\hline\\
	\hline\\
	\hline\\
	\hline\\
	\hline\\
	\hline\\
	\hline\\
	\hline\\
	\hline\\	
	\hline\\
	\hline\\
	\hline\\
	\hline\\
	\hline\\
	\hline\\
\end{tabularx}
\setlength{\extrarowheight}{0mm}
\begin{itemize}
	\itemWritingHand Затем попробуйте признать человеческую общность в том, что вы чувствуете. Как вы думаете, другие люди тоже иногда себя так чувствуют? Может быть, недовольство телом~"--- это часть жизни для любого человека в современном обществе?
\end{itemize}

\setlength{\extrarowheight}{2mm}
\begin{tabularx}{\textwidth}{X}
	\\
	\arrayrulecolor{gray}\hline\\
	\hline\\
	\hline\\
	\hline\\
	\hline\\
	\hline\\	
	\hline\\
	\hline\\
	\hline\\
	\hline\\	
\end{tabularx}
\setlength{\extrarowheight}{0mm}
\begin{itemize}
	\itemWritingHand Наконец, постарайтесь одарить себя добротой и состраданием к сложным эмоциям, которые вы чувствуете. Как вы в этот момент можете себя утешить и поддержать? Можете ли вы отнестись к себе с принятием, разрешая себе быть такой, какая вы есть, со всеми вашими недостатками? Если вам сложно придумать добрые слова, представьте себе, что бы вы сказали близкой подруге, у которой такие же проблемы с восприятием своего тела. Как бы вы проявили теплоту и поддержку, чтобы дать ей понять, что вам не все равно? Теперь скажите то же самое себе. 
\end{itemize}

\setlength{\extrarowheight}{2mm}
\begin{tabularx}{\textwidth}{X}
	\\
	\arrayrulecolor{gray}\hline\\
	\hline\\
	\hline\\
	\hline\\
	\hline\\
	\hline\\	
	\hline\\
	\hline\\
	\hline\\
	\hline\\
\end{tabularx}
\setlength{\extrarowheight}{0mm}

\Reflection{
	Каково было для вас осознанно признать то, что в вашем теле вам нравится, а что нет? Изменилось ли что-то, когда вы напомнили себе о человеческой общности? Смогли ли вы отнестись к себе с добротой во время сложных переживаний?
	
	Это упражнение может быть достаточно трудным, поскольку у большинства из нас самооценка очень сильно зависит от внешности. Если в ходе выполнения упражнения у вас возникли сложные эмоции, постарайтесь отнестись к себе и к боли недовольства своим телом с состраданием~"--- может быть, с помощью \emph{\textbf{Успокаивающего прикосновения}} (глава \ref{The_Physiology_of_Self-Criticism_and_Self-Compassion}, стр.\:\pageref{IP:Soothing_Touch}) или \emph{\textbf{Перерыва на самосострадание}} (глава \ref{The_Physiology_of_Self-Criticism_and_Self-Compassion}, стр.\:\pageref{IP:Self-Compassion_Break}).
	
	У некоторых людей могут быть цели, связанные со своим телом~"--- например, здоровое питание или занятия в спортзале, и они беспокоятся, что, проявив к себе сострадание, они потеряют мотивацию что-то менять. Помните, что мы можем любить и принимать себя такими, какие мы есть, и в то же самое время мотивировать себя на изменение поведения, которое принесет нам здоровье и счастье.
}

\newpage
\Meditation{Сканирование тела с состраданием} \label{M:Compassionate_Body_Scan}

В процессе этой медитации мы обратим теплое, сердечное внимание на каждую часть тела разным способами, будем переключаться с одной части тела на другую и учиться просто <<быть>> с каждой частью тела с добротой и состраданием. Мы с любопытством и нежностью направим внимание на тело, как на маленького ребенка. Если в какой-то части тела вы чувствуете комфорт и легкость, можете мысленно поблагодарить эту часть тела. Если у вас возникают неприятные ощущения или негативные суждения о какой-то части тела, позвольте своему сердцу смягчиться, сопереживая этим трудностям. Можете положить руку на эту часть тела в качестве жеста сострадания и поддержки, и представьте себе, как из вашей ладони и пальцев вытекают тепло и доброта и заполняют эту часть тела. 

Если вам из-за неприятных ощущений тяжело удерживать внимание на какой-то части тела, перенесите его на какое-то время на другую часть тела (желательно нейтральную для вас), чтобы сделать медитацию наиболее комфортной для себя. Будьте все время на связи со своими потребностями в моменте.

Можете ознакомиться с инструкциями, а затем закрыть глаза и позволить осознанности с состраданием передвигаться по вашему телу. Для начинающих будет гораздо легче использовать какую-нибудь аудиозапись этой медитации.

\begin{itemize}
	\itemdiamondsuit Найдите удобное положение, лежа на спине с ладонями на расстоянии около 15 сантиметров от вашего тела и руками на ширине плеч. Затем положите одну или две руки на сердце как напоминание себе привнести в это упражнение любящую включенность. Почувствуйте теплое и мягкое прикосновение ваших ладоней. Сделайте три глубоких, расслабляющих вдоха и выдоха и верните руки в исходное положение, если вам того хочется. 
	
	\itemdiamondsuit Начните с \emph{пальцев на левой ноге} и обратите внимание на то, есть ли в них какие-то ощущения. Они теплые или холодные, сухие или влажные? Просто старайтесь прочувствовать все ощущения в пальцах~"--- легкость, дискомфорт или, может быть, вообще ничего~"--- и разрешите каждому ощущению быть таким, какое оно есть. Если ощущения приятные, можете пошевелить пальцами и улыбнуться.
	
	\itemdiamondsuit Теперь переключитесь на \emph{левую ступню}. Можете ли вы обнаружить в ней какие-то ощущения? Ваши стопы занимают такую маленькую площадь, но держат ваше тело на протяжении всего дня. Они так много работают. Если хочется, мысленно пошлите своей левой ступне немного благодарности. Если есть какой-то дискомфорт, медленно и нежно откройтесь ему.
	
	\itemdiamondsuit Прочувствуйте всю \emph{стопу целиком}. Если вам удобно, можете поблагодарить ее за отсутствие дискомфорта. Если дискомфорт есть, позвольте этому месту смягчиться, как будто его завернули в теплое полотенце.
	
	\itemdiamondsuit Если хотите, валидируйте свой дискомфорт добрыми словами, например: <<Я чувствую небольшой дискомфорт в этом месте, но в этом нет ничего страшного>>.
	
	\itemdiamondsuit Постепенно скользите своим вниманием вверх по ноге, останавливаясь на каждой ее отдельной части, замечая все ощущения. Мысленно отправьте каждой части благодарность, если нет неприятных ощущений, или сострадание, если дискомфорт присутствует. Все еще работая с левой стороной тела, фокусируйтесь по очереди на...
	\begin{itemize}
		\item щиколотке
		\item икре и голени
		\item колене
	\end{itemize}
	
	\itemdiamondsuit Когда вы замечаете, что ваше внимание улетело (как оно всегда делает и будет делать), вернитесь к ощущениям в части тела, на которой вы остановились.
	
	\itemdiamondsuit Можете произнести слова доброты или сострадания, например, <<желаю своим коленям легкости, пусть они хорошо себя чувствуют>>, а потом снова сфокусируйтесь на простых ощущениях в каждой части тела.
	
	\itemdiamondsuit Относитесь ко всему процессу как к какому-то исследованию~"--- даже игривому исследованию~"--- и продолжайте работать с телом. Переключитесь на...
	\begin{itemize}
		\item ляжку
		\item бедро
	\end{itemize}
	
	\itemdiamondsuit Если при работе с какой-то частью тела вы ощущаете неудобство или начинаете ее осуждать, положите руку на сердце и спокойно дышите, представляя себе, что доброта и сострадание перетекают из ваших пальцев в ваше тело.
	
	\itemdiamondsuit Или, если вам комфортно, можете внутренне улыбнуться, чтобы показать своему телу, что вы его цените.
	
	\itemdiamondsuit Теперь направьте любящую осознанность на \emph{всю левую ногу}, освобождая в сознании место для всего, что вы ощущаете или чувствуете.
	
	\itemdiamondsuit Переключитесь на правую ногу и по очереди направьте внимание на...
	\begin{itemize}
		\item пальцы правой ноги
		\item правую ступню
		\item всю правую стопу
		\item щиколотку
		\item икру и голень
		\item колено
	\end{itemize}
	
	\itemdiamondsuit Смело пропускайте какие-то части тела, если при мысли о них возникает слишком много физического или морального дискомфорта. Теперь...
	\begin{itemize}
		\item ляжка
		\item бедро
		\item вся правая нога
	\end{itemize}
	
	\itemdiamondsuit Теперь перенесите осознанность в зону таза~"--- на те сильные кости, которые поддерживают ваши ноги, и окружающие их мягкие ткани. Можете ощутить прикосновение к полу или стулу ваших ягодиц~"--- больших мышц, которые помогают вам взбираться по лестницам и сидеть мягко и комфортно.
	
	\itemdiamondsuit Пришла очередь нижней части спины~"--- здесь задерживается очень много напряжения. Если вы ощущаете какой-то дискомфорт или натянутость, представьте себе, как ваши мышцы расслабляются и словно плавятся от нежности.
	
	\itemdiamondsuit Не стесняйтесь менять свою позу, если это будет для вас комфортнее.
	
	\itemdiamondsuit Теперь~"--- \emph{верхняя часть спины}.
	
	\itemdiamondsuit А сейчас перенесите внимание на переднюю часть вашего тела, на ваш живот. Это очень сложная часть тела, которая содержит в себе много внутренних органов и выполняет много функций. Можете и этой части тела отправить немного благодарности. Если вам не нравится ваш живот, попробуйте сказать какие-нибудь слова доброты и принятия.
	
	\itemdiamondsuit Движемся вверх, к вашей грудной клетке. Это центр вашего дыхания и сердцебиения. Это место~"--- источник любви и сострадания. Постарайтесь наполнить грудную клетку осознанностью, признательностью и принятием. Можно положить на ее середину руку и дать себе прочувствовать то, что вы сейчас ощущаете.
	
	\itemdiamondsuit Смело трогайте любую часть тела по мере продвижения, можете даже их нежно погладить, если хотите.
	
	\itemdiamondsuit Продолжайте с теплотой задерживать внимание на своем теле с той же теплотой, с какой вы бы отнеслись к маленькому ребенку, продолжая чувствовать ощущения в...
	\begin{itemize}
		\item левом плече
		\item верхней части левой руки
		\item локте
	\end{itemize}
	
	\itemdiamondsuit Одарите каждую часть тела нежной осознанностью...
	\begin{itemize}
		\item предплечье
		\item запястье
		\item ладонь
		\item пальцы	
	\end{itemize}
	
	\itemdiamondsuit Не стесняйтесь шевелить пальцами, если хотите, наслаждаясь ощущениями, которые при этом возникают. Ваши руки созданы для того, чтобы держать или перемещать объекты, и очень чувствительны к прикосновениям.
	
	\itemdiamondsuit Просканируйте всю левую руку целиком с любящей и сострадающей осознанностью.
	
	\itemdiamondsuit Теперь ваша правая рука...
	\begin{itemize}
		\item правое плечо
		\item верхняя часть правой руки
		\item локоть
		\item предплечье
		\item запястье
		\item ладонь
		\item пальцы
		\item вся рука целиком	
	\end{itemize}
	
	\itemdiamondsuit А сейчас продолжайте и перенесите осознанность на голову, начиная с шеи. Если хотите, потрогайте ее рукой и вспомните, как в течение всего дня она поддерживает голову и отвечает за прилив крови к мозгу и поступление в организм кислорода. Дайте своей шее немного признательности и доброты~"--- мысленно или  физически~"--- если в ней приятные ощущения, и сострадание, если в ней есть напряжение или дискомфорт.
	
	\itemdiamondsuit Напоследок переключитесь на \emph{голову}, начиная с затылка~"--- той твердой поверхности, которая защищает ваш мозг. Мягко прикоснитесь к затылку рукой с любящей осознанностью.
	
	\itemdiamondsuit Затем идут ваши \emph{уши}~"--- эти чувствительные органы восприятие, которые рассказывают нам так много об окружающем нас мире. Если вы довольны своей способностью слышать, позвольте признательности возникнуть в вашем сердце. Если ваш слух вас беспокоит, положите руку на сердце и утешьте себя состраданием.
	
	\itemdiamondsuit С той же любящей или сострадающей осознанностью отнеситесь к остальным органам чувств: к своим...
	\begin{itemize}
		\item глазам
		\item носу
		\item губам 	
	\end{itemize}
	
	\itemdiamondsuit Не забудьте поблагодарить \emph{щеки}, \emph{челюсть} и \emph{подбородок} за то, как они помогают вам есть, говорить и улыбаться.
	
	\itemdiamondsuit И в завершение~"--- ваш \emph{лоб} и \emph{макушка головы}, а под ними... ваш \emph{мозг}. Ваш нежный, хрупкий мозг состоит из миллиардов нервных клеток, которые все время друг с другом взаимодействуют, чтобы помочь вам понять этот прекрасный мир. Если хотите, скажите мозгу <<спасибо>> за то, что он работает на вас 24/7.
		
	\itemdiamondsuit Когда вы закончили одаривать ваше тело добрым и сострадающим вниманием, сделайте ему напоследок душ из признательности, сострадания и уважения.
		
	\itemdiamondsuit А потом медленно откройте глаза.
\end{itemize}

\Reflection{
	Каким был ваш опыт? Что вы заметили? Было ли вам легче прочувствовать ощущения в определенных частях  тела, чем во всех остальных?
	
	Смогли ли вы отнестись с состраданием к частям тела, которыми вы недовольны или которые вызывают дискомфорт? Пробовали ли вы успокаивающе класть на них руку? Каково было послать своему телу признательность?

	Попытайтесь не критиковать себя, если вы часто отвлекались во время медитации и если она вас разочаровала или даже показалась скучной. Некоторых людей не очень интересуют их тела, или они предпочитают долго на них внимание не задерживать. Другие, наоборот, говорят, что, практикуя сканирование тела, они как будто возвращаются домой. Все люди разные. Примите свой опыт и разрешите себе быть такой, какая вы есть, с добротой. Это осознанность и самосострадание.
}