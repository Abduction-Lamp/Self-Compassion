% !TEX root = ../../Self-Compassion.tex

\chapter{Самосострадание и стыд} \label{Self-Compassion_and_Shame}

\textbf{Стыд вытекает из невинного желания быть любимым~"--- заслуживать привязанность и заслуживать быть на своем месте.} Мы все рождаемся с желанием быть любимыми. Когда в младенческом возрасте это желание исполняется, мы получаем все, что нам нужно~"--- еду, одежду, крышу над головой и общение\cite{89}. Когда мы взрослеем, нам все еще нужны для выживания другие люди~"--- чтобы растить детей и защищаться от опасности. \textbf{Стыд~"--- это чувство, что с нами что-то настолько не так, что нас невозможно любить или принимать.} Одна из причин, по которой эмоция стыда так интенсивна~"--- это создающееся у нас чувство, что на кону стоит наше выживание.

\vspace{2ex}

Есть три любопытных парадокса, связанных со стыдом:
\begin{enumerate}
	\item Нам кажется, что стыд заслуживает порицания, но на самом деле это совершенно невинная эмоция.	
	\item Стыд вызывает чувство одиночества и изоляции, но на самом деле это универсальная общечеловеческая эмоция.
	\item Нам кажется, что эмоция стыда постоянна и всеобъемлюща, но на самом деле это всего лишь преходящее эмоциональное состояние, которое отражает только часть наших личностей.
\end{enumerate}

\begin{quotation}
	\textit{
		Арун был менеджером высшего звена в медицинско-страховой компании, но каждый раз, когда ему на работе приходилось давать выступление или говорить речь перед группой людей, его парализовывало чувство стыда. Вне зависимости от того, как долго он готовился или как хорошо знал тему, Аруну постоянно казалось, что он говорит невнятно и запинается. Он был уверен, что все остальные считают его дилетантом, которому нельзя занимать руководящую должность. Арун очень хотел, чтобы его считали хорошим лидером, но чувство несоответствия требованиям постоянно его одолевало. То, что английский не был его родным языком, только ухудшало положение. После своих <<атак стыда>>, как он их называл, Арун часто запирался в своем кабинете и прятался ото всех.
	}
\end{quotation}

Между виной и стыдом есть определенная разница. Испытывая чувство вины, мы стыдимся какого-то своего поступка; \textbf{испытывая чувство стыда, мы стыдимся себя самих}\cite{90}. Вина говорит, что \emph{вы сделали} что-то плохое~"--- стыд говорит, что \emph{вы плохой}. 

Вина может быть очень даже конструктивной эмоцией, потому что она подталкивает нас на то, чтобы улучшить ситуацию. Стыд же обычно неконструктивен, потому что он парализует нас и мешает эффективно действовать. Исследования показывают, что самосострадание позволяет нам ощутить свое чувство грусти, сожаления и вины, не попадаясь при этом в ловушку чувства стыда\cite{91}. 

\vspace{3ex}

\noindent{\large \textbf{Негативные основополагающие убеждения}}

\vspace{1ex}

Это определенные повторяющиеся мысли, приходящие нам на ум, когда жизнь становится 
трудной~"--- надолго задерживающиеся сомнения в себе, которые часто берут начало в детстве и которые в моменты нашей уязвимости кажутся неоспоримыми. Это наши негативные основополагающие убеждения составляют почву для стыда\cite{92}. Вот несколько примеров:
\begin{itemize}
	\item <<Я дефектная>>
	\item <<Меня невозможно полюбить>>
	\item <<Я беспомощная>>
	\item <<Я неполноценная>>
	\item <<Я неудачница>>
\end{itemize}

Количество часто встречающихся негативных основополагающих убеждений ограничено: их только 15--20. Так как на планете больше 7 миллиардов людей, можно сделать вывод, что любое несовершенство, которое, как нам кажется, отделяет нас от остальных, мы делим с половиной миллиарда людей!

Стыд поддерживается молчанием. Негативные основополагающие убеждения упорно остаются с нами, потому что мы скрываем их от всех (и от себя в том числе). Мы боимся, что будем отвергнуты, если люди будут знать об этих сторонах нашей личности. Мы, тем не менее, забываем, что другие люди чувствуют себя так же, как и мы, ненормальными и изолированными. Когда мы перестаем скрывать (хотя бы от себя) эти убеждения, они начинают терять свою силу над нами. 

У всех из нас есть как сильные, так и слабые стороны. Нельзя на основании каких-то из них делать вывод, что мы достойны или недостойны, заслуживаем или не заслуживаем любви~"--- мы как люди слишком многогранные и сложные. Самосострадание заключает все наши стороны в теплые, добросердечные объятия осознанности. Когда мы убеждены, что мы глубоко и безнадежно ущербны, всегда такими были и всегда такими будем, это значились, что мы зациклились на какой-то одной детали и не можем увидеть общую картину себя.  Нужно принять и эту деталь, и соответствующее ей негативное основополагающее убеждение, а также увидеть целостную картину себя как человека, чтобы стать свободными. 

\begin{quotation}
	\textit{
		После нескольких лет постоянных <<атак стыда>> Аруну наконец это надоело. Он решил, что не даст этим чувствам испортить его успех. Он знал, что корни его стыда уходили глубоко в детство, когда любимчиком отца был Дев, старший брат Аруна. Отец хвалил Дева за все достижения, а <<Арун-джи>> (его прозвище в семье) только и слышал, что упреки и призывы что-то изменить, чтобы стать лучше. Поэтому Арун начал выстраивать новые отношения с собой-ребенком~"--- с той частью его личности, которая, как он считал, никогда не будет соответствовать ни общественным, ни чьим бы то ни было стандартам. Когда у него возникали чувства стыда и неполноценности, Арун мысленно обнимал рукой маленького Аруна-джи, говоря ему добрые и ободряющие слова. <<У тебя все получится, а даже если ты сделаешь ошибку, в этом нет ничего страшного. Я тебя принимаю и буду с тобой всегда, не смотря ни на что>>. Арун даже поставил свою детскую фотографию на свой рабочий стол у себя дома и разговаривал с ней так, как ему бы хотелось, чтобы с ним разговаривал отец.
	}
	
	\textit{
		После нескольких месяцев этой практики у Аруна появилось больше уверенности в себе при публичных выступлениях. Его стыд никуда не делся, но он перестал ставить ему палки в колеса, и в итоге Арун смог подружиться с этой частью себя. В конце концов, Арун~"--- взрослый мужчина, у которого много знаний и опыта. Более мудрая и зрелая часть Аруна знает, как дать маленькому Аруну-джи поддержку, в которой он нуждается.
	}
\end{quotation}

\textbf{Самосострадание~"--- самое мощное противоядие от стыда.} Подходя к нашим ошибкам с добротой, а не осуждением, вспоминая о человеческой общности, а не изолируя себя из-за своих промахов, и относясь к своим негативным эмоциям осознанно (<<Мне плохо>>), а не идентифицируя себя с ними (<<Я плохой>>), мы с помощью самосострадания разрушаем фундамент стыда. А относясь ко всем нашим ощущениям (включая стыд) с любящим, общечеловеческим присутствием, мы снова становится целыми. 

\newpage
\Exercises{Работа с нашими негативными основополагающими убеждениями} \label{Ex:Working_with_Our_Negative_Core_Beliefs}

Наши негативные основополагающие убеждения~"--- это именно убеждения, а не реальность. Это мысли, прочно закрепившиеся у нас в голове, часто сформированные в подростковом возрасте и в юности, и правды в них обычно мало. Тем не менее, когда эти мысли остаются в подсознании, они имеют на нас очень большое влияние. Важный первый шаг~"--- это идентификация этих мыслей и осознанность к ним. Когда мы подносим эти убеждения к свету дня, их сила начинает таять. Это похоже на разоблачение Волшебника страны Оз, когда обнаруживается, что он не великий и могущественный правитель, каким он себя называет, а обыкновенный аферист из Канзаса.

Работать с негативными основополагающими убеждениями бывает сложно, особенно для людей с детскими травмами. Прислушайтесь к себе, чтобы понять, готовы ли вы сейчас умственно и эмоционально выполнить это упражнение~"--- если нет, то проявите к себе сострадание, пока пропустив его. Как альтернатива, если вы занимаетесь с психотерапевтом, возможно, имеет смысл сделать это упражнение при поддержке квалифицированного профессионала.

\vspace{3ex}

\noindent{\large \textbf{Инструкции}}

\vspace{1ex}

Ниже приведён список часто встречающихся основополагающих убеждений. Отметьте все, которые у вас иногда появляются, и попробуйте определить контекст, в котором они чаще всего возникают (на работе, в отношениях, в кругу семьи и т.\,д.). Есть ли какие-то ситуации, которые их провоцируют.

\vspace{1ex}

\begin{center}
	\setlength{\extrarowheight}{2mm}
	\noindent\begin{tabular}{lll}
		Я недостаточно хорош & Я везде лишний & Я беспомощный \\
		Я тупой & Я просто дилетант & Я ненормальный \\
		Я дефектный & Я плохой & Я ничего не стою \\
		Я слаб & Я неудачник & Я неспособный \\
		Я не заслуживаю любви & Я бессилен & Я ничтожество \\
		Я неполноценный & Я некрасивый & Я мерзкий \\
		\\
		\\
		\arrayrulecolor{gray}\hline\\
		\hline\\
		\hline\\
		\hline\\
	\end{tabular}
	\setlength{\extrarowheight}{0mm}
\end{center}

\newpage
\noindent{\large Теперь попробуйте пустить в ход три компонента самосострадания.}

\begin{itemize}
	\itemWritingHand \textbf{Осознанность:} пишите объективно, валидируя чувства, которые вызывают у вас ваши негативные убеждения. Например: <<Мне очень больно, когда у меня проскальзывает мысль о том, что меня невозможно полюбить>> или <<Очень тяжело чувствовать себя бессильным>>.
\end{itemize}

\setlength{\extrarowheight}{2mm}
\begin{tabularx}{\textwidth}{X}
	\\
	\arrayrulecolor{gray}\hline\\
	\hline\\
	\hline\\
	\hline\\
	\hline\\
	\hline\\	
	\hline\\
	\hline\\
	\hline\\
	\hline\\
	\hline\\
\end{tabularx}
\setlength{\extrarowheight}{0mm}
\begin{itemize}
	\itemWritingHand \textbf{Человеческая общность:} напишите о том, каким образом ваши убеждения~"--- это естественная часть человеческой жизни. Например: <<Наверняка миллионы людей чувствуют себя так же, как и я>> или <<Я не один так себя чувствую>>.
\end{itemize}

\setlength{\extrarowheight}{2mm}
\begin{tabularx}{\textwidth}{X}
	\\
	\arrayrulecolor{gray}\hline\\
	\hline\\
	\hline\\
	\hline\\
	\hline\\
	\hline\\	
	\hline\\
	\hline\\
	\hline\\
	\hline\\
	\hline\\
\end{tabularx}
\setlength{\extrarowheight}{0mm}
\begin{itemize}
	\itemWritingHand \textbf{Доброта:} теперь напишите себе несколько слов понимания и доброты, выражая беспокойство за ваши страдания, вызванные этим негативным убеждением. Можете попробовать писать, как будто разговариваете с подругой, которая только что вам призналась, что у неё есть такое же убеждение. Например: <<Мне так жаль, что ты себя так чувствуешь. Я вижу, как тебе больно. Пожалуйста, помни, что я о тебе так не думаю>>.
\end{itemize}

\setlength{\extrarowheight}{2mm}
\begin{tabularx}{\textwidth}{X}
	\\
	\arrayrulecolor{gray}\hline\\
	\hline\\
	\hline\\
	\hline\\
	\hline\\
	\hline\\	
	\hline\\
	\hline\\
	\hline\\
	\hline\\
	\hline\\
\end{tabularx}
\setlength{\extrarowheight}{0mm}

\Reflection{
	Как прошло это упражнение? Смогли ли вы идентифицировать одно-два негативных убеждения? Что вы почувствовали, проявив осознанность и доброту и вспомнив о человеческой общности?
	
	Иногда люди обнаруживают, что, когда они применяют к своим убеждениям сострадание, они только усиливаются. Это обратная тяга (смотри главу \ref{Backdraft} на стр.\:\pageref{Backdraft})~"--- любовь входит, а старая боль уходит. Ещё один часто встречающийся эффект~"--- это то, что та часть нас, которая идентифицировала себя с негативными основополагающими убеждениями, чувствует страх, как будто мы пытаемся с ней разделаться. Важно помнить, что мы не пытаемся избавиться от негативных убеждений или прогнать их. Мы просто пытаемся отнестись с ним с осознанность и заботой, чтобы они не имели над нами столько силы.
}

\InformalPractices{Работа со стыдом} \label{IP:Working_with_Shame}

Эта практика похожа на <<Работу с трудными эмоциями>> (смотри главу \ref{Meeting_Difficult_Emotions} на стр.\:\pageref{IP:Working_with_Difficult_Emotions}). Мы можем назвать когнитивный компонент стыда~"--- \textbf{негативное основополагающее убеждение}~"--- и определить, где в теле располагается стыд, а также отнестись ко всему с состраданием. Особенно важно при работе со стыдом помнить, что он рождается из желания быть любимым, он практически универсален и он является эмоцией, а значит, он проходящий. Эти элементы включены в практику.

Повторим ещё раз: выполняйте эту практику только в том случае, если вы готовы. Если вы все-таки решите ее сделать, но вам становится в какой-то момент некомфортно, пожалуйста, позаботьтесь о себе и остановитесь. Например, можете принять тёплую ванну, погладить собаку или просто пойти погулять, фокусируясь на ощущениях в стопах (смотри главу \ref{Backdraft} на стр.\:\pageref{IP:Feeling_the_Soles_of_Your_Feet}).

\vspace{2ex}

\textbf{В этой практике поощряется намерение сфокусироваться скорее на смущении и неловкости, чем на стыде. Мы только начинаем собирать в себе ресурсы, поэтому нужно двигаться медленно.}

\begin{itemize}
	\item Найдите удобную позицию (сидя или лёжа), закройте частично или полностью глаза и сделайте несколько глубоких, расслабляющих вдохов и выдохов. Можете даже немного вздохнуть, если хочется~"--- эхххххх...
	
	\item Положите руку на сердце, напоминая себе, что вы находитесь здесь и сейчас в этой комнате, а, возможно, ещё и разрешая доброте перетечь через вашу руку в ваше тело.
	
	\item Теперь вспомните какое-то событие, которое вас \emph{смутило} и из-за которого вам, возможно, стало немного \emph{стыдно}. Например:
	\begin{itemize}
		\item o	вы слишком бурно на что-то отреагировали
		\item o	вы сказали что-то не очень умное
		\item o	вы провалили какое-то задание на работе
		\item o	вы поняли, что на важном мероприятии у вас была расстегнута ширинка
	\end{itemize}

	\item Выберите событие, которое достаточно вас беспокоит, что вы можете почувствовать это телесно. Если оно не вызывает неудобства в теле, выберите другое, но не превышающее по неудобству отметку 3 на шкале от 1 до 10.
	
	\item Пусть это будет событие, о котором вы бы очень не хотели, чтобы кто-то узнал или вспомнил, иначе, возможно, о вас бы стали хуже думать.
	
	\item Для начала выберите ситуацию, из-за которой переживаете только вы, а не ситуацию, в которой вы нанесли другим какой-то вред и чувствуете, что должны попросить у кого-то прощения.
	
	\item Вспомните это событие в деталях, пытаясь почувствовать себя так же, как тогда. Для этого нужна смелость. Используйте все свои органы чувств, обращая особенное внимание на то, как стыд или смущение выражают себя телесно.
\end{itemize}

\vspace{3ex}

{\large \textbf{Дать название основополагающим убеждениям}}

\vspace{1ex}

\begin{itemize}
	\item Теперь на минуту задумайтесь и попробуйте точно определить, что это такое, что вызывает у вас страх, что это узнают другие. Можете дать этому название? Может быть, <<я дефектный>>, <<я недобрый>>, <<я дилетант>>. Это негативные основополагающие убеждения.
	
	\item Если вы нашли несколько, выберите то, которое, как вам кажется, весит больше.
	
	\item Возможно, вы уже чувствуете себя одиноким. Если это так, постарайтесь понять, что мы все <<одиноки вместе>>~"--- все в какой-то момент чувствуют себя так же, как и вы. Стыд — это универсальная эмоция.
	
	\item Теперь назовите для себя это убеждение так, как бы вы назвали его для друга. Например, <<а, ты думаешь, что ты недостоин любви. Это, наверное, так больно!>> Или просто скажите себе тёплым, полным сострадания голосом: <<Недостоин любви. Я думаю, что я недостоин любви!>>
	
	\item Помните, что, когда мы чувствуем смущение или стыд, только часть нас так себя чувствует. Мы не всегда ощущаем эти эмоции, хотя чувство и можно показаться очень знакомым.
	
	\item И наши негативные убеждения возникают из желания быть любимыми. Мы все невинные существа, которые просто хотят быть любимыми.
	
	\item Напоминаем, что, если вам станет некомфортно во время практики, вы можете в любой момент открыть глаза или как-то по-другому от неё отстраниться.
\end{itemize}

\vspace{3ex}

{\large \textbf{Осознанность к телесным проявлениям стыда}}

\vspace{1ex}

\begin{itemize}
	\item Теперь включите все свое тело в поле вашего внимания и осознанности.
	
	\item Опять вспомните трудную ситуацию и просканируйте свое тело, чтобы найти то место, где вам легче всего почувствовать стыд или смущение. Пробегитесь мысленным взглядом по всему телу с головы до ног, останавливаясь там, где вы чувствуете небольшое напряжение или дискомфорт. 

	\item Теперь, если можете, выберите какое-то одно место, где чувство смущения или стыда наиболее выражено (например, через мышечное напряжение, чувство пустоты или даже сердечную боль). Не нужно быть слишком точным. 

	\item Напоминаем еще раз: пожалуйста, позаботьтесь о себе во время выполнения практики. 
\end{itemize}

\vspace{3ex}

{\large \textbf{Смягчить--Успокоить--Разрешить}}

\vspace{1ex}

\begin{itemize}
	\item Теперь направьте ваше внимание на эту часть тела.
	
	\item Попробуйте \emph{смягчить} это место. Позвольте своим мышцам смягчиться, расслабиться, как в теплой воде. Смягчение... смягчение... смягчение... Помните, что мы не пытаемся изменить это чувство~"--- мы просто поддерживаем его, но нежно, мягко и заботливо.
	
	\item Потом \emph{успокойте} себя в связи с этой трудной ситуацией. Если хотите, положите руку на ту часть тела, в которой вы чувствуете больше всего дискомфорта. Можете представить себе, что тепло и доброта перетекают через вашу руку в это место. Можно даже попробовать вообразить, что ваше тело~"--- это тело любимого ребенка.
	
	\item Есть ли какие-то утешающие слова поддержки, которые вам бы нужно было сейчас услышать? Если есть, представьте себе, что у вас есть друг, у которого аналогичные трудности. Что бы вы этому другу сказали, один на один? (<<Мне так жаль, что ты так себя чувствуешь>>, <<Ты и твое состояние для меня очень важны>>.) Что вы хотите, чтобы ваш друг знал или помнил?
	
	\item Теперь попробуйте донести до себя те же слова. (<<Так тяжело это чувствовать>>, <<Пусть я буду добр к себе>>). Впустите в себя эти слова, насколько можете.
	
	\item Опять вспомните, что, когда мы чувствуем стыд или смущение, только часть нас так себя чувствует. Мы не испытываем эти чувства все время.
	
	\item Наконец, \emph{разрешите} дискомфорту остаться. Освободите для него место и отпустите нужду от него избавиться. Разрешите себе быть точно таким, какой вы есть, хотя бы на один момент.
	
	\item Если хотите, можете повторить этот цикл, каждый раз заходя немного глубже. Смягчите... успокойте... разрешите. Смягчите... успокойте... разрешите.
	
	\item Перед окончанием практики просто вспомните, что вы сейчас связаны со всеми людьми в мире, которые когда-либо испытывали чувство смущения или стыда, и что эти чувства проистекают из желания быть любимым.
	
	\item Теперь отпустите практику и сфокусируйтесь на своем теле. Разрешите себе чувствовать то, что вы чувствуете, и быть точно таким, какой вы сейчас есть.
\end{itemize}

\Reflection{
	Смогли ли вы идентифицировать негативное основополагающее убеждение, которое прячется за чувством смущения или страха? Что вы почувствовали, когда его назвали? Смогли ли вы найти в своем теле стыд? Если да, то где? Изменило ли ваши ощущения смягчение, успокоение или разрешение? 
	
	Работать со стыдом бывает очень трудно. Вам наверняка потребовалась смелость, чтобы зайти так далеко, но если вы не закончили упражнение, потому что заботитесь о себе, поблагодарите себя за это.
	
	По мере выполнения этой практики может возникнуть много препятствий. Например, может быть сложно почувствовать в теле стыд. Стыд может быть предвестником того, что вы как бы <<выпадете>> из реальности, и иногда он ощущается как пустота в теле, особенно в голове. Вы можете сфокусироваться и на чувстве, что вы ничего не чувствуете, но это тяжело. Людям также бывает сложно отнестись к себе с состраданием, когда их охватывает стыд, потому что им кажется, что они того не заслужили. И, конечно, очень вероятно, что во время этой практики вы встретились с нашей давней подругой \textit{обратной тягой} (смотри главу \ref{Backdraft} на стр.\:\pageref{Backdraft}). \textbf{Если это упражнение было для вас по какой-то причине трудным, просто перенесите внимание на нежную признательность себе за вашу борьбу. Это и есть самосострадание.}
}