% !TEX root = ../../Self-Compassion.tex

\chapter{Польза самосострадания} \label{The_Benefits_of_Self-Compassion}

\textit{В первый день следования нашей программе Марион очень скептически к ней относилась. <<Как самосострадание может мне помочь? Я привыкла держать себя в ежовых рукавицах, и, мне кажется, это меньшее из двух зол. В конце концов, так я достигла всего, что я сейчас имею. Зачем  мне меняться, да и разве это вообще возможно? Я не могу быть уверена, что у этого не будет никаких неприятных последствий>>.}

\vspace{2ex}

К счастью, Марион не пришлось просто поверить нам на слово. Более тысячи исследований показали, что самосострадание очень полезно для физического и психического здоровья. У людей, практикующих самосострадание, в жизни:

\begin{center}
	\setlength{\extrarowheight}{2mm}
	\begin{tabular}{p{4cm}p{5cm}}
		\textbf{Меньше} & \textbf{Больше} \\
		\hline \hline 
		Депрессии &	Счастья \\
		Тревожности	& Удовлетворения жизнью \\
		Стресса	& Веры в себя \\
		Стыда & Сил и здоровья \\
	\end{tabular}
	\setlength{\extrarowheight}{0mm}
\end{center}

\vspace{2ex}

Хотя у людей от природы в разной степени развито самосострадание, ему можно научиться. Исследования показали, что людям, которые прошли курс ОСС (программа, на основе которой написана эта книга), удалось стать в среднем на 43~\% сострадательнее к себе. Прохождение программы также помогло им развить осознанность и сострадание к другим, Благодаря нашему курсу его участники стали менее подверженными стрессу, депрессии, тревожности, смогли почувствовать себя счастливее, менее одинокими и более довольными жизнью. После окончания программы они реже пытались справиться со сложными эмоциями путем их избегания. 

Большинство из этого самым прямым образом связано с тем, что они научились быть сострадательнее к себе. Более того, \textbf{через год после прохождения курса положительные эффекты самосострадания сохранились}. То, насколько участникам удалось развить в себе самосострадание, зависело от того, сколько времени они уделяли его практике (либо от того, сколько дней в неделю они медитировали, либо от того, сколько раз в день они выполняли неформальные практики). Эти данные исследований подтверждают, что, выполняя приведенные в этой книге упражнения, можно кардинально изменить отношение к себе и свою жизнь.

\vspace{2ex}

\textit{Со стороны у Марион была жизнь, которой можно позавидовать: двое детей, счастливый брак, успешная карьера. Но, несмотря на это, почти каждый день она ложилась спать в очень нервном и разбитом состоянии, беспокоясь, что она кого-то обидела или ругая себя за то, что она <<плохая мать>>. За этим следовало разочарование, что она не дотягивала до своей собственной высокой планки. Казалось, что ее ничем нельзя было утешить или подбодрить. У Марион была репутация человека, который всегда скажет правильные слова и всегда относится ко всем с добротой и поддержкой, но почему-то ничего из этого не переносилось на ее отношение к себе. Она понимала, что изменения должны прийти изнутри~"--- но как? Узнав о самосострадании, Марион подумала, что с его помощью сможет получить ответ на этот вопрос, и записалась на курс ОСС. До начала программы Марион заполнила <<шкалу самосострадания>> (см. ниже) и поняла, что, по сути, она сама свой злейший враг. На первом занятии Марион обнаружила, что она не одна: на самом деле, когда что-то идет не так, мы все инстинктивно критикуем и изолируем себя, думая только о том, какие мы плохие. Следующий шаг к самосостраданию~"--- признание боли, которую причиняет самокритика~"--- дался Марион достаточно легко. Ее семья и друзья уже начинали уставать от ее постоянной нужды в одобрении, и сама Марион даже слишком хорошо понимала свое отчаянное желание быть идеальной. Это стремление уходило корнями в детство Марион. Она росла в семье успешного, но эмоционально отстраненного отца и матери - бывшей королевой красоты, которую сидение дома с дочерью весь день вгоняло в скуку смертную. Девочка нуждалась в большей теплоте и эмоциональной близости со стороны своих родителей, но это всегда казалось недостижимым. Вырастя, Марион смогла наконец привлечь к себе внимание, добиваясь успеха во всем, что она делала. Но у этого была своя цена: успех никогда не приносил Марион таких чувств, каких бы ей хотелось.} 

\textit{Первое откровение случилось, когда Марион задумалась о том, как сильно и независимо от всего она любит своих детей. Она спросила себя: <<Почему я систематически недодаю себе такой любви?>> Разве она не могла укутать себя в этой любви, как она иногда укутывалась вместе с детьми в одеяло, укладывая их спать? Разве она не могла разговаривать с собой таким же заботливым тоном, как со своими друзьями? <<В конце концов, - подумала Марион, - мне, как и всем остальным людям, нужно чувствовать себя любимой!>>}

\textit{Когда Марион разрешила себе полюбить себя, она начала чувствовать отголоски тоски и одиночества из ее детства. Но к тому времени она уже укрепилась в идее, что она так же, как и любой другой человек, заслуживает сострадания. Она даже почувствовала горе по тем долгим годам, которые она привела в попытках заполучить привязанность других, чтобы заполнить душевную пустоту.}

\textit{Практиковать самострадание было трудно, но Марион упорно продолжала. Она знала, что этим давним чувствам нужно найти какой-то выход, и училась черпать необходимые для встречи с ними ресурсы из осознанности и самосострадания. Она теперь могла начать давать самой себе то, что так хотела получить от других.}

\textit{Скоро друзья и семья Марион заметили, что она изменилась. Сначала это проявлялось в мелочах~"--- например, когда она научилась отказываться пойти с друзьями куда-то, когда она была очень уставшей и не хотела. Марион обнаружила, что она стала легче засыпать~"--- возможно, потому что она больше не вспоминала перед сном все ошибки, совершенные в течение дня. Она все еще иногда просыпалась от кошмаров~"--- например, когда ей приснилось, что она должна была выступить с докладом на работе и не помнила, о чем он должен был быть~"--- но тогда она просто клала себе руку на сердце, говорила что-нибудь ласковое и утешительное и быстро обратно засыпала. Ее муж заметил, полушутя, что ей <<стало нужно меньше ухода>>. К концу восьминедельного курса ОСС Марион и вся ее семья согласились, что она стала счастливее. Но что было действительно поразительным, так это то, что она прекратила ругать себя за ошибки, отпустила свое желание быть идеальной и начала любить и принимать себя такой, какая она есть.}

\Exercises{Насколько сострадательно я отношусь к себе?}

Дорога к самосостраданию часто начинается с объективной оценки того, насколько мы уже сострадательно относимся к себе. <<Шкала Самосострадания>> измеряет, насколько люди добры к себе или самокритичны, насколько у них присутствует чувство человеческой общности или, наоборот, они чувствуют себя изолированными из-за своих недостатков, и насколько они осознанны или слишком идентифицируют себя со своим страданием. Большинство исследований используют эту шкалу, чтобы определить уровень развития самосострадания и определить его связь со здоровьем и благополучием. Пройдите тест ниже, чтобы узнать, насколько в вас развито самосострадание. 

\vspace{3ex}

Это адаптированная и сокращенная версия теста. Если вы хотите попробовать пройти полную версию, перейдите по ссылке \newline \url{http://self-compassion.org/test-how-self-compassionateyou-are/}

\vspace{3ex}

Нижеследующие утверждения описывают ваше поведение по отношению к себе в тяжелые времена. Внимательно прочитайте утверждения перед тем, как ответить, и слева от каждого вопроса отметьте, насколько часто вы так себя ведете на шкале от 1 до 5.
 
\vspace{3ex}

\newpage

\textit{Для первого набора утверждений используйте следующую шкалу:}

\begin{flushright}
	\setlength{\extrarowheight}{2mm}
	\begin{tabularx}{11.5cm}{XXXXX}
		\textbf{Почти никогда} & & & & \textbf{Почти всегда} \\
		\textbf{1} & \textbf{2} & \textbf{3} & \textbf{4} & \textbf{5} \\
	\end{tabularx}
	\setlength{\extrarowheight}{0mm}
\end{flushright}

\begin{center}
	\setlength{\extrarowheight}{2mm}
	\begin{tabularx}{\textwidth}{p{1.7cm}X}
		 \noindent\rule{1.7cm}{0.4pt} & Я пытаюсь понимающе и терпеливо относиться к тем аспектам моего характера, которые мне не нравятся. \\
		 \noindent\rule{1.7cm}{0.4pt} & Когда случается что-то неприятное, я стараюсь составить уравновешенное представление о ситуации.\\
		 \noindent\rule{1.7cm}{0.4pt} & Я стараюсь относиться к своим неудачам как к части человеческой жизни.\\
		 \noindent\rule{1.7cm}{0.4pt} & Во время тяжелых периодов в жизни я даю себе заботу и нежность, в которых нуждаюсь.\\
		 \noindent\rule{1.7cm}{0.4pt} & Когда что-то меня расстраивает, я стараюсь держать эмоции под контролем.\\
		 \noindent\rule{1.7cm}{0.4pt} & Когда я чувствую себя несостоявшимся или некомпетентным, я напоминаю себе, что большинство людей иногда себя так чувствуют.\\
	\end{tabularx}
	\setlength{\extrarowheight}{0mm}	
\end{center}

\newpage

\textit{Для второго набора утверждений используйте следующую шкалу (обратите внимание на то, что ее направление противоположно предыдущей):}

\begin{flushright}
	\setlength{\extrarowheight}{2mm}
	\begin{tabularx}{11.5cm}{XXXXX}
		\textbf{Почти всегда} & & & & \textbf{Почти никогда} \\
		\textbf{1} & \textbf{2} & \textbf{3} & \textbf{4} & \textbf{5} \\
	\end{tabularx}
	\setlength{\extrarowheight}{0mm}
\end{flushright}

\begin{center}
	\setlength{\extrarowheight}{2mm}
	\begin{tabularx}{\textwidth}{p{1.7cm}X}
		\noindent\rule{1.7cm}{0.4pt} & Когда у меня не получается что-то, что для меня важно, я чувствую себя несостоявшимся. \\
		\noindent\rule{1.7cm}{0.4pt} & Когда мне плохо, мне кажется, что большинство других людей счастливее меня. \\
		\noindent\rule{1.7cm}{0.4pt} & Когда у меня не получается что-то, что для меня важно, мне кажется, что это у меня одного ничего не получается. \\
		\noindent\rule{1.7cm}{0.4pt} & Когда мне плохо, я зацикливаюсь на всем, что мне не нравится в ситуации, и только об этом и думаю. \\
		\noindent\rule{1.7cm}{0.4pt} & Я неодобрительно отношусь к своим недостаткам и несовершенствам и осуждаю себя за них. \\
		\noindent\rule{1.7cm}{0.4pt} & Я нетерпим по отношению к аспектам моего характера, которые мне не нравятся. \\
	\end{tabularx}
	\setlength{\extrarowheight}{0mm}	
\end{center}

\vspace{3ex}

\noindent\textbf{Как подсчитать свои результаты:}

\begin{flushright}
	\setlength{\extrarowheight}{5mm}
	\begin{tabularx}{11.5cm}{ll}
		\textbf{Тотал (сумма всех 12 ответов)} & \rule{1.7cm}{0.4pt} \\
		\textbf{Среднее значение = Тотал/12} & \rule{1.7cm}{0.4pt}
	\end{tabularx}
	\setlength{\extrarowheight}{0mm}
\end{flushright}

\vspace{3ex}

Средний результат теста~"--- около 3,0 на шкале от 1 до 5; интерпретируйте свой балл, исходя из этого. 
Если у вас вышло 1---2,5 балла, то у вас мало развито самосострадание, 2,5---3,5~"--- средне развито, 3,5---5,0~"--- очень хорошо развито.

\newpage

\Reflection{
	Если ваш средний балл ниже, чем вам бы хотелось, не беспокойтесь. Хорошая новость: самосострадание~"--- это навык, которому можно научиться. Может быть, вам будет нужно больше времени, но в итоге у вас все получится.
}

\InformalPractices{Ведение дневника самосострадания}

Попробуйте вести дневник самосострадания каждый день в течение недели (или больше, если вам хочется). \textbf{Ведение дневника~"--- это эффективный способ выразить эмоции}, который полезен для психического и физического здоровья.
Вечером, когда у вас есть пара свободных минут, оглянитесь на события дня. Запишите в своем дневнике все, за что вам было стыдно или неловко, все, за что вы себя осуждали, и все неприятные или болезненные события. (Например, вы разозлились на официанта в ресторане, потому что он долго не нес счет. Вы отпустили какое-то грубое замечание и ушли, не оставив чаевых. Потом вам стало за это стыдно.)

Попробуйте прибегнуть к осознанности, человеческой общности и доброте, чтобы сформулировать каждую запись так, чтобы в ней было больше самосострадания. Вот как можно это сделать:
\vspace{3ex}

\textbf{Осознанность}

Здесь речь в основном идет о взвешенном, сбалансированном осознании болезненных эмоций, которые возникли из-за вашего самоосуждения или из-за трудных обстоятельств. Напишите, что вы чувствовали: грусть, стыд, страх, напряжение и т.д. Во время написания попытайтесь принять и не осуждать то, что вы пережили и почувствовали, не обесценивая это, но и не драматизируя. (Например, <<Я был раздражен, потому что официант был очень медленным. В конце концов я разозлился, слишком остро отреагировал и потом почувствовал себя дураком>>.) 
\vspace{3ex}

\textbf{Человеческая общность}

Напишите, что из того, что вы пережили~"--- естественная часть человеческой жизни. Для этого вам может понадобиться признать, что быть человеком значит быть несовершенным и что у всех людей бывают такие неприятные ситуации. (<<Все когда-то перегибают палку и слишком остро реагируют~"--- это человеческая природа>>. <<Люди в такой ситуации обычно чувствуют себя именно так>>.) Для вас может быть полезно подумать об уникальных обстоятельствах, обусловивших это болезненное событие. («Мое раздражение подкреплял тот факт, что я и так на полчаса опаздывал ко врачу, к которому нужно было ехать на другой конец города, и в тот день на дорогах были пробки. При других обстоятельствах я бы наверняка отреагировал по-другому».) 
\vspace{3ex}

\textbf{Доброта к себе}

Напишите себе несколько добрых и понимающих слов, как вы бы написали хорошему другу. Дайте себе знать, что вам важно ваше счастье и здоровье, принимая мягкий, подбадривающий тон. (<<Все в порядке. Ты немного облажался, но это не конец света. Я понимаю, как ты был раздражен, и ты в итоге просто вышел из себя. Можно попробовать в качестве компенсации быть как можно более терпеливым и щедрым по отношению ко всем официантам, которых ты встретишь на этой неделе>>.) 

\Reflection{После как \textbf{минимум недели} ведения дневника, спросите себя, заметили ли вы какие-то изменения в том, как вы разговариваете с самим собой. \textbf{Какие были ваши ощущения, когда вы писали с б\'{о}льшим самосостраданием? Как вы думаете, помогло ли это вам лучше справиться с трудностями?}

Некоторые люди найдут ведение дневника самосострадания отличным способом дополнить практики, но другим это может показаться скучным и утомительным. В любом случае стоит попробовать вести его около недели, но если вам не понравится, можно не делать это письменно. Самое главное~"--- это практиковать все три компонента самосострадания: осознанно посмотреть на свою боль, вспомнить, что несовершенства~"--- это часть жизни каждого человека, и отнестись к себе с добротой и поддержкой, потому что нам трудно.
}


%%% end section %%%